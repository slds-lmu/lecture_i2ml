\documentclass{beamer}
% These are loaded from relative file paths which depend on where the cheatsheets are stored
% Currently these are at two different directory levels, so we can reduce this copypasta 
% once they are always at the same level
\usepackage[orientation=landscape,size=a0,scale=1.4,debug]{../style/beamerposter}
% usepackage{beamerthemeNAME} is equivalent to usetheme{NAME}
% But usepackage works with relative file paths, usetheme does not (as easily at least)
\mode<presentation>{\usepackage{../style/beamerthememlr}}

% ----------------------------------------------------------------------------%
% Preamble for lecture cheatsheets based on I2ML
% Intended to be laded from ./cheatsheets/
% Original copy located in 
%     lecture_service/service/style/preamble-cheatsheets.tex
% ----------------------------------------------------------------------------%

\newcommand \beameritemnestingprefix{}

\usepackage[utf8]{inputenc} % UTF-8
\usepackage[english]{babel} % Language
\usepackage{hyperref} % Hyperlinks
\usepackage{ragged2e} % Text position
\usepackage[export]{adjustbox} % Image position
\usepackage[most]{tcolorbox}
\usepackage{amsmath}
\usepackage{mathtools}
\usepackage{dsfont}
\usepackage{verbatim}
\usepackage{amsmath}
\usepackage{amsfonts}
\usepackage{csquotes}
\usepackage{multirow}
\usepackage{longtable}
\usepackage[absolute,overlay]{textpos}
\usepackage{psfrag}
\usepackage{algorithm}
\usepackage{algpseudocode}
\usepackage{eqnarray}
\usepackage{arydshln}
\usepackage{tabularx}
\usepackage{placeins}
\usepackage{tikz}
\usepackage{setspace}
\usepackage{colortbl}
\usepackage{mathtools}
\usepackage{wrapfig}
\usepackage{bm}
\usepackage{mathdots}


\newlength{\columnheight} % Adjust depending on header height
\setlength{\columnheight}{84cm} 

\newtcolorbox{codebox}{%
	sharp corners,
	leftrule=0pt,
	rightrule=0pt,
	toprule=0pt,
	bottomrule=0pt,
	hbox}

\newtcolorbox{codeboxmultiline}[1][]{%
	sharp corners,
	leftrule=0pt,
	rightrule=0pt,
	toprule=0pt,
	bottomrule=0pt,
	#1}


% math spaces
\newcommand{\N}{\mathds{N}}                                                 % N, naturals
\newcommand{\Z}{\mathds{Z}}                                                 % Z, integers
\newcommand{\Q}{\mathds{Q}}                                                 % Q, rationals
\newcommand{\R}{\mathds{R}}                                                 % R, reals
\newcommand{\C}{\mathds{C}}                                                 % C, complex
\newcommand{\HS}{\mathcal{H}}                                               % H, hilbertspace
\newcommand{\continuous}{\mathcal{C}}                                       % C, space of continuous functions
\newcommand{\M}{\mathcal{M}} 												% machine numbers
\newcommand{\epsm}{\epsilon_m} 												% maximum error


% basic math stuff
\newcommand{\xt}{\tilde x}													% x tilde
\def\argmax{\mathop{\sf arg\,max}}                                          % argmax
\def\argmin{\mathop{\sf arg\,min}}                                          % argmin
\newcommand{\sign}{\operatorname{sign}}                                     % sign, signum
\newcommand{\I}{\mathbb{I}}                                                 % I, indicator
\newcommand{\order}{\mathcal{O}}                                            % O, order
\newcommand{\fp}[2]{\frac{\partial #1}{\partial #2}}                        % partial derivative
\newcommand{\pd}[2]{\frac{\partial{#1}}{\partial #2}}						% partial derivative

% sums and products
\newcommand{\sumin}{\sum_{i=1}^n}											% summation from i=1 to n
\newcommand{\sumkg}{\sum_{k=1}^g}											% summation from k=1 to g
\newcommand{\prodin}{\prod_{i=1}^n}											% product from i=1 to n
\newcommand{\prodkg}{\prod_{k=1}^g}											% product from k=1 to g

% linear algebra
\newcommand{\one}{\boldsymbol{1}}                                           % 1, unitvector
\newcommand{\id}{\mathrm{I}}                                                % I, identity
\newcommand{\diag}{\operatorname{diag}}                                     % diag, diagonal
\newcommand{\trace}{\operatorname{tr}}                                      % tr, trace
\newcommand{\spn}{\operatorname{span}}                                      % span
\newcommand{\scp}[2]{\left\langle #1, #2 \right\rangle}                     % <.,.>, scalarproduct
\newcommand{\mat}[1]{ 														% short pmatrix command
	\begin{pmatrix}
		#1
	\end{pmatrix}
}
\newcommand{\Amat}{\bm{A}}													% matrix A
\newcommand{\xv}{\bm{x}}													% vector x (bold)
\newcommand{\yv}{\bm{y}}														% vector y (bold)
\newcommand{\Deltab}{\bm{\Delta}}											% error term for vectors
															

% basic probability + stats
\renewcommand{\P}{\mathds{P}}                                               % P, probability
\newcommand{\E}{\mathds{E}}                                                 % E, expectation
\newcommand{\var}{\mathsf{Var}}                                             % Var, variance
\newcommand{\cov}{\mathsf{Cov}}                                             % Cov, covariance
\newcommand{\corr}{\mathsf{Corr}}                                           % Corr, correlation
\newcommand{\normal}{\mathcal{N}}                                           % N of the normal distribution
\newcommand{\iid}{\overset{i.i.d}{\sim}}                                    % dist with i.i.d superscript
\newcommand{\distas}[1]{\overset{#1}{\sim}}                                 % ... is distributed as ... 
% machine learning

%%%%%% ml - data
\newcommand{\Xspace}{\mathcal{X}}                                           % X, input space
\newcommand{\Yspace}{\mathcal{Y}}                                           % Y, output space
\newcommand{\nset}{\{1, \ldots, n\}}                                        % set from 1 to n
\newcommand{\pset}{\{1, \ldots, p\}}                                        % set from 1 to p
\newcommand{\gset}{\{1, \ldots, g\}}                                        % set from 1 to g
\newcommand{\Pxy}{\P_{xy}}                                                  % P_xy
\newcommand{\xy}{(x, y)}                                                    % observation (x, y)
\newcommand{\xvec}{(x_1, \ldots, x_p)^T}                                    % (x1, ..., xp) 
\newcommand{\D}{\mathcal{D}}                                                % D, data 
\newcommand{\Dset}{\{ (x^{(1)}, y^{(1)}), \ldots, (x^{(n)},  y^{(n)})\}}    % {(x1,y1)), ..., (xn,yn)}, data
\newcommand{\xdat}{\{ x^{(1)}, \ldots, x^{(n)}\}}   						 % {x1, ..., xn}, input data
\newcommand{\ydat}{\mathbf{y}}                                              % y (bold), vector of outcomes
\newcommand{\yvec}{(y^{(1)}, \hdots, y^{(n)})^T}                            % (y1, ..., yn), vector of outcomes
\renewcommand{\xi}[1][i]{x^{(#1)}}                                          % x^i, i-th observed value of x
\newcommand{\yi}[1][i]{y^{(#1)}}                                            % y^i, i-th observed value of y 
\newcommand{\xyi}{(\xi, \yi)}                                               % (x^i, y^i), i-th observation
\newcommand{\xivec}{(x^{(i)}_1, \ldots, x^{(i)}_p)^T}                       % (x1^i, ..., xp^i), i-th observation vector
\newcommand{\xj}{x_j}                                                       % x_j, j-th feature
\newcommand{\xjb}{\mathbf{x}_j}                                             % x_j (bold), j-th feature vecor
\newcommand{\xjvec}{(x^{(1)}_j, \ldots, x^{(n)}_j)^T}                       % (x^1_j, ..., x^n_j), j-th feature vector
\newcommand{\Dtrain}{\mathcal{D}_{\text{train}}}                            % D_train, training set
\newcommand{\Dtest}{\mathcal{D}_{\text{test}}}                              % D_test, test set

%%%%%% ml - models general

% continuous prediction function f
\newcommand{\fx}{f(x)}                                                      % f(x), continuous prediction function
\newcommand{\Hspace}{H}														% hypothesis space where f is from
\newcommand{\fh}{\hat{f}}                                                   % f hat, estimated prediction function
\newcommand{\fxh}{\fh(x)}                                                   % fhat(x)
\newcommand{\fxt}{f(x | \theta)}                                            % f(x | theta)
\newcommand{\fxi}{f(\xi)}                                                   % f(x^(i))
\newcommand{\fxih}{\hat{f}(\xi)}                                            % f(x^(i))
\newcommand{\fxit}{f(x^{(i)} | \theta)}                                     % f(x^(i) | theta)
\newcommand{\fhD}{\fh_{\D}}                                                 % fhat_D, estimate of f based on D
\newcommand{\fhDtrain}{\fh_{\Dtrain}}                                       % fhat_Dtrain, estimate of f based on D

% discrete prediction function h
\newcommand{\hx}{h(x)}                                                      % h(x), discrete prediction function
\newcommand{\hh}{\hat{h}}                                                   % h hat
\newcommand{\hxh}{\hat{h}(x)}                                               % hhat(x)
\newcommand{\hxt}{h(x | \theta)}                                            % h(x | theta)
\newcommand{\hxi}{h(\xi)}                                                   % h(x^(i))
\newcommand{\hxit}{h(x^{(i)} | \theta)}                                     % h(x^(i) | theta)

% yhat
\newcommand{\yh}{\hat{y}}                                                   % y hat for prediction of target
\newcommand{\yih}{\hat{y}}                                                  % y hat for prediction of target

% theta
\newcommand{\thetah}{\hat{\theta}}                                          % theta hat

% densities + probabilities
% pdf of x 
\newcommand{\pdf}{p}                                                        % p
\newcommand{\pdfx}{p(x)}                                                    % p(x)
\newcommand{\pixt}{\pi(x | \theta)}                                         % pi(x|theta), pdf of x given theta

% pdf of (x, y)
\newcommand{\pdfxy}{p(x,y)}                                                 % p(x, y)
\newcommand{\pdfxyt}{p(x, y | \theta)}                                      % p(x, y | theta)
\newcommand{\pdfxyit}{p(\xi, \yi | \theta)}                                 % p(x^(i), y^(i) | theta)

% pdf of x given y
\newcommand{\pdfxyk}{p(x | y=k)}                                            % p(x | y = k)
\newcommand{\lpdfxyk}{\log \pdfxyk}                                         % log p(x | y = k)
\newcommand{\pdfxiyk}{p(\xi | y=k)}                                         % p(x^i | y = k)

% prior probabilities
\newcommand{\pik}{\pi_k}                                                    % pi_k, prior
\newcommand{\lpik}{\log \pik}                                               % log pi_k, log of the prior

% posterior probabilities
\newcommand{\post}{\P(y = 1 | x)}                                           % P(y = 1 | x), post. prob for y=1
\newcommand{\pix}{\pi(x)}                                                   % pi(x), P(y = 1 | x)
\newcommand{\postk}{\P(y = k | x)}                                          % P(y = k | y), post. prob for y=k
\newcommand{\pikx}{\pi_k(x)}                                                % pi_k(x), P(y = k | x)
\newcommand{\pikxt}{\pi_k(x | \theta)}                                      % pi_k(x | theta), P(y = k | x, theta)
\newcommand{\pijx}{\pi_j(x)}                                                % pi_j(x), P(y = j | x)
\newcommand{\pdfygxt}{p(y |x, \theta)}                                      % p(y | x, theta)
\newcommand{\pdfyigxit}{p(\yi |\xi, \theta)}                                % p(y^i |x^i, theta)
\newcommand{\lpdfygxt}{\log \pdfygxt }                                      % log p(y | x, theta)
\newcommand{\lpdfyigxit}{\log \pdfyigxit}                                   % log p(y^i |x^i, theta)
\newcommand{\pixh}{\hat \pi(x)}                                             % pi(x) hat, P(y = 1 | x) hat
\newcommand{\pikxh}{\hat \pi_k(x)}                                          % pi_k(x) hat, P(y = k | x) hat

% residual and margin
\newcommand{\eps}{\epsilon}                                                 % residual, stochastic
\newcommand{\epsi}{\epsilon^{(i)}}                                          % epsilon^i, residual, stochastic
\newcommand{\epsh}{\hat{\epsilon}}                                          % residual, estimated
\newcommand{\yf}{y \fx}                                                     % y f(x), margin
\newcommand{\yfi}{\yi \fxi}                                                 % y^i f(x^i), margin
\newcommand{\Sigmah}{\hat \Sigma}											% estimated covariance matrix
\newcommand{\Sigmahj}{\hat \Sigma_j}										% estimated covariance matrix for the j-th class

% ml - loss, risk, likelihood
\newcommand{\Lxy}{L(y, f(x))}                                               % L(y, f(x)), loss function
\newcommand{\Lxyi}{L(\yi, \fxi)}                                            % L(y^i, f(x^i))
\newcommand{\Lxyt}{L(y, \fxt)}                                              % L(y, f(x | theta))
\newcommand{\Lxyit}{L(\yi, \fxit)}                                          % L(y^i, f(x^i | theta)
\newcommand{\risk}{\mathcal{R}}                                             % R, risk
\newcommand{\riskf}{\risk(f)}                                               % R(f), risk
\newcommand{\riske}{\mathcal{R}_{\text{emp}}}                               % R_emp, empirical risk
\newcommand{\riskef}{\riske(f)}                                             % R_emp(f)
\newcommand{\risket}{\mathcal{R}_{\text{emp}}(\theta)}                      % R_emp(theta)
\newcommand{\riskr}{\mathcal{R}_{\text{reg}}}                               % R_reg, regularized risk
\newcommand{\riskrt}{\mathcal{R}_{\text{reg}}(\theta)}                      % R_reg(theta)
\newcommand{\riskrf}{\riskr(f)}                                             % R_reg(f)
\newcommand{\LL}{\mathcal{L}}                                               % L, likelihood
\newcommand{\LLt}{\mathcal{L}(\theta)}                                      % L(theta), likelihood
\renewcommand{\ll}{\ell}                                                    % l, log-likelihood
\newcommand{\llt}{\ell(\theta)}                                             % l(theta), log-likelihood
\newcommand{\LS}{\mathfrak{L}}                                              % ????????????
\newcommand{\TS}{\mathfrak{T}}                                              % ??????????????
\newcommand{\errtrain}{\text{err}_{\text{train}}}                           % training error
\newcommand{\errtest}{\text{err}_{\text{test}}}                             % training error
\newcommand{\errexp}{\overline{\text{err}_{\text{test}}}}                   % training error

% resampling
\newcommand{\GE}[1]{GE(\fh_{#1})}                                           % Generalization error GE
\newcommand{\GEh}[1]{\widehat{GE}_{#1}}                                     % Estimated train error
\newcommand{\GED}{\GE{\D}}                                                  % Generalization error GE
\newcommand{\EGEn}{EGE_n}                                                   % Generalization error GE
\newcommand{\EDn}{\E_{|D| = n}}                                             % Generalization error GE


% ml - irace
\newcommand{\costs}{\mathcal{C}} % costs
\newcommand{\Celite}{\theta^*} % elite configurations
\newcommand{\instances}{\mathcal{I}} % sequence of instances
\newcommand{\budget}{\mathcal{B}} % computational budget
% ml - trees, extra trees

\newcommand{\Np}{\mathcal{N}}												% Parent node N
\newcommand{\Nl}{\Np_1}														% Left node N_1
\newcommand{\Nr}{\Np_2}														% Right node N_2


\title{I2ML :\,: CHEAT SHEET} % Package title in header, \, adds thin space between ::
\newcommand{\packagedescription}{ % Package description in header
	The \textbf{I2ML}: Introduction to Machine Learning course offers an introductory and applied overview of "supervised" Machine Learning. It is organized as a digital lecture.
}

\begin{document}
\begin{frame}[fragile]{}
\begin{columns}
	\begin{column}{.31\textwidth}
		\begin{beamercolorbox}[center]{postercolumn}
			\begin{minipage}{.98\textwidth}
				\parbox[t][\columnheight]{\textwidth}{

					\begin{myblock}{ML Overview}

						\begin{codebox}
							\textbf{Machine Learning}
						\end{codebox}
            A computer program is said to learn from experience E with respect to some task T and some performance measure P, 
            if its performance on T, as measured by P, improves with experience E.
            \begin{itemize}[$\bullet$] 
              \setlength{\itemindent}{+.3in}
                          \item Supervised Learning (SL): Learns from labeled data
                          \item Unsupervised Learning: Learns from unlabeled data
                          \item Reinforcement Learning: Learns from feedback
                         \end{itemize}

              \begin{codebox}
                \textbf{Components of SL}
              \end{codebox}
              \textbf{Learning = Hypothesis Space + Risk + Optimization}
              \begin{itemize}[$\bullet$] 
                \setlength{\itemindent}{+.3in}
                \item \textbf{Hypothesis Space:} Defines and restricts what kind of model 
                $f$ can be learned from the data.
                \item \textbf{Risk (Loss + Regularization):} Quantifies how well a specific model performs on a given 
                data set.
                \item \textbf{Optimization:} Defines how to search for the best model in the 
                hypothesis space, i.e., the model with the smallest risk.
              \end{itemize}

\end{myblock}


\begin{myblock}{Data}

$$\D = \Dset \in \defAllDatasetsn$$
\begin{itemize}[$\bullet$] 
  \setlength{\itemindent}{+.3in}
  \item $\greekxi$: features, $y^{(i)}$: target
  \item $\Xspace$: input space, usually $\Xspace \subset \R^p$
  \item $\Yspace$: output / target space
  \item \(\xyi\) $\in \Xspace\times \Yspace$:  \(i\)-th observation
\end{itemize}

\begin{codebox} 
\textbf{Categorical Data Encoding:}
\end{codebox}
Expand the representation of a variable $x$
with $k$ mutually exclusive categories from a scalar 
to a length-$\tilde k$ vector with at most one 
element being 1, and 0 otherwise: $\bm{o}(x)_j =\I(x = j) \in \{0,1\}$.

\begin{itemize}[$\bullet$] 
  \setlength{\itemindent}{+.3in}
    \item \textbf{One-hot encoding}: $\tilde k = k$ dummies, so \textit{exactly 
    one} element is 1. \\
    E.g., $x \in \{ a, b, c\} \mapsto \bm{o}(x) = (x_a, x_b, x_c)$, with 
    $x_a = x_b = 0, x_c = 1$ and $\bm{o}(x) = (0, 0, 1)$ for $x = c$.
    \item \textbf{Dummy encoding}: $\tilde k = k - 1$ dummies, so 
    \textit{at most one} element is 1, removing the redundancy of one-hot 
    encoding. \\
    E.g., $x \in \{ a, b, c\} \mapsto \bm{o}(x) = (x_a, x_b)$ for reference 
    category $c$, with $x_a = x_b = 0$ and $\bm{o}(x) = (0, 0)$ for $x = c$.
\end{itemize} 

\end{myblock}\vfill
				}
			\end{minipage}
		\end{beamercolorbox}
	\end{column}
	
%%%%%%%%%%%%%%%%%%%%%%%%%%%%%%%%%%%%%%%%%%%%%%%%%%%%%%%%%%%%%%%%%%%%%

\begin{column}{.31\textwidth}
\begin{beamercolorbox}[center]{postercolumn}
\begin{minipage}{.98\textwidth}
\parbox[t][\columnheight]{\textwidth}{

\begin{myblock}{ }
\begin{codebox}
  \textbf{Data-Generating Process: }
  \end{codebox}
  \begin{itemize}[$\bullet$] 
    \setlength{\itemindent}{+.3in}
    \item Assume the observed data $\D$ to be generated by a process that can
    be characterized by some probability distribution $\Pxy,$ defined on 
    $\Xspace \times \Yspace$.
    \item Assume data to be drawn \emph{i.i.d.} (\textbf{i}ndependent and \textbf{i}dentically 
      \textbf{d}istributed) from $\Pxy$. 
    \item Often, distributions are characterized by a parameter vector 
      $\thetav \in \Theta$. We then write $\pdfxyt$.
    \end{itemize} 
  \end{myblock}

  \begin{myblock}{Supervised Tasks}

      \begin{itemize}[$\bullet$] 
      \setlength{\itemindent}{+.3in}
                  \item \textbf{Regression:} $\Yspace \subseteq \mathbb{R}$ Numerical target
                  \item \textbf{Classification:} $\Yspace = \{C_1,...,C_g\}$ Categorical target
                 \end{itemize}


  \begin{codebox}
  \textbf{Goal:}
  \end{codebox}
  \begin{itemize}[$\bullet$] 
  \setlength{\itemindent}{+.3in}
  \item \textbf{Learning to predict}. Don't care how
        model is structured. Want an accurate predictor for new data.

    \item \textbf{Learning to explain}. Model is only a mean to 
        better understand the inherent relationship in the data.
        Might not use the learned model on new observations.
\end{itemize}

  \end{myblock}

  \begin{myblock}{Model and Parameter}
    A \textbf{model} (or \textbf{hypothesis}) 
    $f : \Xspace \rightarrow \R^g$
    is a function that maps feature vectors to predicted target values.
  
    The set of functions defining a specific model class is called a 
    \textbf{hypothesis space}: $\Hspace = \{f: f \text{ belongs to a certain functional family}\}$

    Usually construct the space as parametrized family of functions:
    $\Hspace = \{f_{\thetav}: f_{\thetav} \text{ belongs to a certain functional family parameterized by } \thetav\}$

  \end{myblock}

  \begin{myblock}{Learner}
    \textbf{Learner} is the algorithm for finding model $f$, is a mean of picking the best element from the hypothesis space $\Hspace$
  for given training data.
  It maps training data $\D \in \allDatasets$ plus a vector of hyperparameter control settings $\lamv \in \Lam$ to a model:
  $ \ind: \preimageInducerShort \rightarrow \Hspace$. 
  Practically, with parameters $\ind: \preimageInducerShort \rightarrow \Theta$.
\end{myblock}

}
\end{minipage}
\end{beamercolorbox}
\end{column}

%%%%%%%%%%%%%%%%%%%%%%%%%%%%%%%%%%%%%%%%%%%%%%%%%%%%%%%%%%%%%%%%%%%%%%%%%%%%%%%%%%%%%%%%%%%%%%%%%%%%

\begin{column}{.31\textwidth}
\begin{beamercolorbox}[center]{postercolumn}
\begin{minipage}{.98\textwidth}
\parbox[t][\columnheight]{\textwidth}{

  \begin{myblock}{Losses and Risk Minimization}
  
  \begin{codebox}
  \textbf{Loss}	
  \end{codebox}
  \textbf{Loss function} $\Lxy$ quantifies the quality of the prediction $\fx$ of a single observation $\xv$:
    $L: \Yspace \times \R^g \to \R$.


  \begin{codebox}
  \textbf{Risk}
  \end{codebox}
  Theoretical \textbf{risk} associated with a certain hypothesis $f \in \Hspace$ measured by a loss function $\Lxy$ is the \textbf{expected loss}
  $$ \riskf := \E_{xy} [\Lxy] = \int \Lxy \text{d}\Pxy $$
  
  \textbf{Problem}: $\Pxy$ is unknown, thus minimizing the theoretical $\riskf$ directly is not feasible.

  \begin{codebox}
  \textbf{Empirical Risk}
  \end{codebox}
  Simply sum up all pointwise losses: $ \riskef = \sumin \Lxyi $.\\
  Can also be defined as an average loss: $ \riskeb(f) = \frac{1}{n}\sumin \Lxyi $.\\
  With parameters: $\risket = \sumin \Lxyit$.\\
  \textbf{Empirical Risk Minimization} (ERM): $\thetavh = \argmin_{\thetav \in \Theta} \risket$.

  \end{myblock}
  
  \begin{myblock}{Optimization}

  \begin{codebox} 
  \textbf{ERM optimization:}
  \end{codebox}
  $$\thetavh  = \argmin_{\thetav \in \Theta} \risket $$
  
  Global minimum $\thetavh$: 
  \[
  \forall \thetav \in \Theta :\quad \riske(\thetavh) \leq \risket .
  \]

  Local minimum $\thetavh$:
  \[
  \exists \epsilon > 0\; \forall \thetav \text{ with } \left\Vert\thetavh - \thetav\right\Vert < \epsilon: \quad \riske(\thetavh) \leq \risket
  \]

  \end{myblock}
  }
  
  \end{minipage}
  \end{beamercolorbox}
  \end{column}
  
  
  
\end{columns}
\end{frame}
\end{document}
