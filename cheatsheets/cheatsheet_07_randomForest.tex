\documentclass{beamer}
\newcommand \beameritemnestingprefix{}


\usepackage[orientation=landscape,size=a0,scale=1.4,debug]{beamerposter}
\mode<presentation>{\usetheme{mlr}}


\usepackage[utf8]{inputenc} % UTF-8
\usepackage[english]{babel} % Language
\usepackage{hyperref} % Hyperlinks
\usepackage{ragged2e} % Text position
\usepackage[export]{adjustbox} % Image position
\usepackage[most]{tcolorbox}
\usepackage{amsmath}
\usepackage{mathtools}
\usepackage{dsfont}
\usepackage{verbatim}
\usepackage{amsmath}
\usepackage{amsfonts}
\usepackage{csquotes}
\usepackage{multirow}
\usepackage{longtable}
\usepackage[absolute,overlay]{textpos}
\usepackage{psfrag}
\usepackage{algorithm}
\usepackage{algpseudocode}
\usepackage{eqnarray}
\usepackage{arydshln}
\usepackage{tabularx}
\usepackage{placeins}
\usepackage{tikz}
\usepackage{setspace}
\usepackage{colortbl}
\usepackage{mathtools}
\usepackage{wrapfig}
\usepackage{bm}


% math spaces
\newcommand{\N}{\mathds{N}}                                                 % N, naturals
\newcommand{\Z}{\mathds{Z}}                                                 % Z, integers
\newcommand{\Q}{\mathds{Q}}                                                 % Q, rationals
\newcommand{\R}{\mathds{R}}                                                 % R, reals
\newcommand{\C}{\mathds{C}}                                                 % C, complex
\newcommand{\HS}{\mathcal{H}}                                               % H, hilbertspace
\newcommand{\continuous}{\mathcal{C}}                                       % C, space of continuous functions
\newcommand{\M}{\mathcal{M}} 												% machine numbers
\newcommand{\epsm}{\epsilon_m} 												% maximum error


% basic math stuff
\newcommand{\xt}{\tilde x}													% x tilde
\def\argmax{\mathop{\sf arg\,max}}                                          % argmax
\def\argmin{\mathop{\sf arg\,min}}                                          % argmin
\newcommand{\sign}{\operatorname{sign}}                                     % sign, signum
\newcommand{\I}{\mathbb{I}}                                                 % I, indicator
\newcommand{\order}{\mathcal{O}}                                            % O, order
\newcommand{\fp}[2]{\frac{\partial #1}{\partial #2}}                        % partial derivative
\newcommand{\pd}[2]{\frac{\partial{#1}}{\partial #2}}						% partial derivative

% sums and products
\newcommand{\sumin}{\sum_{i=1}^n}											% summation from i=1 to n
\newcommand{\sumkg}{\sum_{k=1}^g}											% summation from k=1 to g
\newcommand{\prodin}{\prod_{i=1}^n}											% product from i=1 to n
\newcommand{\prodkg}{\prod_{k=1}^g}											% product from k=1 to g

% linear algebra
\newcommand{\one}{\boldsymbol{1}}                                           % 1, unitvector
\newcommand{\id}{\mathrm{I}}                                                % I, identity
\newcommand{\diag}{\operatorname{diag}}                                     % diag, diagonal
\newcommand{\trace}{\operatorname{tr}}                                      % tr, trace
\newcommand{\spn}{\operatorname{span}}                                      % span
\newcommand{\scp}[2]{\left\langle #1, #2 \right\rangle}                     % <.,.>, scalarproduct
\newcommand{\mat}[1]{ 														% short pmatrix command
	\begin{pmatrix}
		#1
	\end{pmatrix}
}
\newcommand{\Amat}{\bm{A}}													% matrix A
\newcommand{\xv}{\bm{x}}													% vector x (bold)
\newcommand{\yv}{\bm{y}}														% vector y (bold)
\newcommand{\Deltab}{\bm{\Delta}}											% error term for vectors
															

% basic probability + stats
\renewcommand{\P}{\mathds{P}}                                               % P, probability
\newcommand{\E}{\mathds{E}}                                                 % E, expectation
\newcommand{\var}{\mathsf{Var}}                                             % Var, variance
\newcommand{\cov}{\mathsf{Cov}}                                             % Cov, covariance
\newcommand{\corr}{\mathsf{Corr}}                                           % Corr, correlation
\newcommand{\normal}{\mathcal{N}}                                           % N of the normal distribution
\newcommand{\iid}{\overset{i.i.d}{\sim}}                                    % dist with i.i.d superscript
\newcommand{\distas}[1]{\overset{#1}{\sim}}                                 % ... is distributed as ... 
% machine learning

%%%%%% ml - data
\newcommand{\Xspace}{\mathcal{X}}                                           % X, input space
\newcommand{\Yspace}{\mathcal{Y}}                                           % Y, output space
\newcommand{\nset}{\{1, \ldots, n\}}                                        % set from 1 to n
\newcommand{\pset}{\{1, \ldots, p\}}                                        % set from 1 to p
\newcommand{\gset}{\{1, \ldots, g\}}                                        % set from 1 to g
\newcommand{\Pxy}{\P_{xy}}                                                  % P_xy
\newcommand{\xy}{(x, y)}                                                    % observation (x, y)
\newcommand{\xvec}{(x_1, \ldots, x_p)^T}                                    % (x1, ..., xp) 
\newcommand{\D}{\mathcal{D}}                                                % D, data 
\newcommand{\Dset}{\{ (x^{(1)}, y^{(1)}), \ldots, (x^{(n)},  y^{(n)})\}}    % {(x1,y1)), ..., (xn,yn)}, data
\newcommand{\xdat}{\{ x^{(1)}, \ldots, x^{(n)}\}}   						 % {x1, ..., xn}, input data
\newcommand{\ydat}{\mathbf{y}}                                              % y (bold), vector of outcomes
\newcommand{\yvec}{(y^{(1)}, \hdots, y^{(n)})^T}                            % (y1, ..., yn), vector of outcomes
\renewcommand{\xi}[1][i]{x^{(#1)}}                                          % x^i, i-th observed value of x
\newcommand{\yi}[1][i]{y^{(#1)}}                                            % y^i, i-th observed value of y 
\newcommand{\xyi}{(\xi, \yi)}                                               % (x^i, y^i), i-th observation
\newcommand{\xivec}{(x^{(i)}_1, \ldots, x^{(i)}_p)^T}                       % (x1^i, ..., xp^i), i-th observation vector
\newcommand{\xj}{x_j}                                                       % x_j, j-th feature
\newcommand{\xjb}{\mathbf{x}_j}                                             % x_j (bold), j-th feature vecor
\newcommand{\xjvec}{(x^{(1)}_j, \ldots, x^{(n)}_j)^T}                       % (x^1_j, ..., x^n_j), j-th feature vector
\newcommand{\Dtrain}{\mathcal{D}_{\text{train}}}                            % D_train, training set
\newcommand{\Dtest}{\mathcal{D}_{\text{test}}}                              % D_test, test set

%%%%%% ml - models general

% continuous prediction function f
\newcommand{\fx}{f(x)}                                                      % f(x), continuous prediction function
\newcommand{\Hspace}{H}														% hypothesis space where f is from
\newcommand{\fh}{\hat{f}}                                                   % f hat, estimated prediction function
\newcommand{\fxh}{\fh(x)}                                                   % fhat(x)
\newcommand{\fxt}{f(x | \theta)}                                            % f(x | theta)
\newcommand{\fxi}{f(\xi)}                                                   % f(x^(i))
\newcommand{\fxih}{\hat{f}(\xi)}                                            % f(x^(i))
\newcommand{\fxit}{f(x^{(i)} | \theta)}                                     % f(x^(i) | theta)
\newcommand{\fhD}{\fh_{\D}}                                                 % fhat_D, estimate of f based on D
\newcommand{\fhDtrain}{\fh_{\Dtrain}}                                       % fhat_Dtrain, estimate of f based on D

% discrete prediction function h
\newcommand{\hx}{h(x)}                                                      % h(x), discrete prediction function
\newcommand{\hh}{\hat{h}}                                                   % h hat
\newcommand{\hxh}{\hat{h}(x)}                                               % hhat(x)
\newcommand{\hxt}{h(x | \theta)}                                            % h(x | theta)
\newcommand{\hxi}{h(\xi)}                                                   % h(x^(i))
\newcommand{\hxit}{h(x^{(i)} | \theta)}                                     % h(x^(i) | theta)

% yhat
\newcommand{\yh}{\hat{y}}                                                   % y hat for prediction of target
\newcommand{\yih}{\hat{y}}                                                  % y hat for prediction of target

% theta
\newcommand{\thetah}{\hat{\theta}}                                          % theta hat

% densities + probabilities
% pdf of x 
\newcommand{\pdf}{p}                                                        % p
\newcommand{\pdfx}{p(x)}                                                    % p(x)
\newcommand{\pixt}{\pi(x | \theta)}                                         % pi(x|theta), pdf of x given theta

% pdf of (x, y)
\newcommand{\pdfxy}{p(x,y)}                                                 % p(x, y)
\newcommand{\pdfxyt}{p(x, y | \theta)}                                      % p(x, y | theta)
\newcommand{\pdfxyit}{p(\xi, \yi | \theta)}                                 % p(x^(i), y^(i) | theta)

% pdf of x given y
\newcommand{\pdfxyk}{p(x | y=k)}                                            % p(x | y = k)
\newcommand{\lpdfxyk}{\log \pdfxyk}                                         % log p(x | y = k)
\newcommand{\pdfxiyk}{p(\xi | y=k)}                                         % p(x^i | y = k)

% prior probabilities
\newcommand{\pik}{\pi_k}                                                    % pi_k, prior
\newcommand{\lpik}{\log \pik}                                               % log pi_k, log of the prior

% posterior probabilities
\newcommand{\post}{\P(y = 1 | x)}                                           % P(y = 1 | x), post. prob for y=1
\newcommand{\pix}{\pi(x)}                                                   % pi(x), P(y = 1 | x)
\newcommand{\postk}{\P(y = k | x)}                                          % P(y = k | y), post. prob for y=k
\newcommand{\pikx}{\pi_k(x)}                                                % pi_k(x), P(y = k | x)
\newcommand{\pikxt}{\pi_k(x | \theta)}                                      % pi_k(x | theta), P(y = k | x, theta)
\newcommand{\pijx}{\pi_j(x)}                                                % pi_j(x), P(y = j | x)
\newcommand{\pdfygxt}{p(y |x, \theta)}                                      % p(y | x, theta)
\newcommand{\pdfyigxit}{p(\yi |\xi, \theta)}                                % p(y^i |x^i, theta)
\newcommand{\lpdfygxt}{\log \pdfygxt }                                      % log p(y | x, theta)
\newcommand{\lpdfyigxit}{\log \pdfyigxit}                                   % log p(y^i |x^i, theta)
\newcommand{\pixh}{\hat \pi(x)}                                             % pi(x) hat, P(y = 1 | x) hat
\newcommand{\pikxh}{\hat \pi_k(x)}                                          % pi_k(x) hat, P(y = k | x) hat

% residual and margin
\newcommand{\eps}{\epsilon}                                                 % residual, stochastic
\newcommand{\epsi}{\epsilon^{(i)}}                                          % epsilon^i, residual, stochastic
\newcommand{\epsh}{\hat{\epsilon}}                                          % residual, estimated
\newcommand{\yf}{y \fx}                                                     % y f(x), margin
\newcommand{\yfi}{\yi \fxi}                                                 % y^i f(x^i), margin
\newcommand{\Sigmah}{\hat \Sigma}											% estimated covariance matrix
\newcommand{\Sigmahj}{\hat \Sigma_j}										% estimated covariance matrix for the j-th class

% ml - loss, risk, likelihood
\newcommand{\Lxy}{L(y, f(x))}                                               % L(y, f(x)), loss function
\newcommand{\Lxyi}{L(\yi, \fxi)}                                            % L(y^i, f(x^i))
\newcommand{\Lxyt}{L(y, \fxt)}                                              % L(y, f(x | theta))
\newcommand{\Lxyit}{L(\yi, \fxit)}                                          % L(y^i, f(x^i | theta)
\newcommand{\risk}{\mathcal{R}}                                             % R, risk
\newcommand{\riskf}{\risk(f)}                                               % R(f), risk
\newcommand{\riske}{\mathcal{R}_{\text{emp}}}                               % R_emp, empirical risk
\newcommand{\riskef}{\riske(f)}                                             % R_emp(f)
\newcommand{\risket}{\mathcal{R}_{\text{emp}}(\theta)}                      % R_emp(theta)
\newcommand{\riskr}{\mathcal{R}_{\text{reg}}}                               % R_reg, regularized risk
\newcommand{\riskrt}{\mathcal{R}_{\text{reg}}(\theta)}                      % R_reg(theta)
\newcommand{\riskrf}{\riskr(f)}                                             % R_reg(f)
\newcommand{\LL}{\mathcal{L}}                                               % L, likelihood
\newcommand{\LLt}{\mathcal{L}(\theta)}                                      % L(theta), likelihood
\renewcommand{\ll}{\ell}                                                    % l, log-likelihood
\newcommand{\llt}{\ell(\theta)}                                             % l(theta), log-likelihood
\newcommand{\LS}{\mathfrak{L}}                                              % ????????????
\newcommand{\TS}{\mathfrak{T}}                                              % ??????????????
\newcommand{\errtrain}{\text{err}_{\text{train}}}                           % training error
\newcommand{\errtest}{\text{err}_{\text{test}}}                             % training error
\newcommand{\errexp}{\overline{\text{err}_{\text{test}}}}                   % training error

% resampling
\newcommand{\GE}[1]{GE(\fh_{#1})}                                           % Generalization error GE
\newcommand{\GEh}[1]{\widehat{GE}_{#1}}                                     % Estimated train error
\newcommand{\GED}{\GE{\D}}                                                  % Generalization error GE
\newcommand{\EGEn}{EGE_n}                                                   % Generalization error GE
\newcommand{\EDn}{\E_{|D| = n}}                                             % Generalization error GE


% ml - irace
\newcommand{\costs}{\mathcal{C}} % costs
\newcommand{\Celite}{\theta^*} % elite configurations
\newcommand{\instances}{\mathcal{I}} % sequence of instances
\newcommand{\budget}{\mathcal{B}} % computational budget
\input{../latex-math/ml-eval.tex}
% ml - trees, extra trees

\newcommand{\Np}{\mathcal{N}}												% Parent node N
\newcommand{\Nl}{\Np_1}														% Left node N_1
\newcommand{\Nr}{\Np_2}														% Right node N_2

% ml - bagging, random forest
\newcommand{\bl}[1][m]{b^{[#1]}} % baselearner, default m
\newcommand{\blh}[1][m]{\hat{b}^{[#1]}} % estimated base learner, default m 
\newcommand{\blx}[1][m]{b^{[#1]}(\xv)} % baselearner, default m

\newcommand{\blfmh}{\hat{f}^{[m]}} % estimated baselearner: scores
\newcommand{\blfmhx}{\blfmh(\xv)} % estimated baselearner: scores of x
\newcommand{\blhmh}{\hat{h}^{[m]}} % estimated baselearner: hard labels
\newcommand{\blhmhx}{\blhmh(\xv)} % estimated baselearner: hard labels of x
\newcommand{\blpmh}{\hat \pi^{[m]}} % estimated baselearner: probabilities
\newcommand{\blpmhxk}{\hat \pi_{k}^{[m]}(\xv)} % estimated baselearner: probabilities of x for class k

\newcommand{\fM}{f^{[M]}(\xv)} % ensembled predictor
\newcommand{\fMh}{\hat f^{[M]}(\xv)} % estimated ensembled predictor
\newcommand{\ambifM}{\Delta\left(\fM\right)} % ambiguity/instability of ensemble
\newcommand{\betam}[1][m]{\beta^{[#1]}} % weight of basemodel m
\newcommand{\betamh}[1][m]{\hat{\beta}^{[#1]}} % weight of basemodel m with hat
\newcommand{\betaM}{\beta^{[M]}} % last baselearner
\newcommand{\summM}{\sum_{m=1}^M} % sum over m=1 to M baselearners
\newcommand{\avgmM}{\frac{1}{M} \sum_{m=1}^M} % averaging over m=1 to M baselearners

\newcommand{\ib}{\mathrm{IB}} % In-Bag (IB)
\newcommand{\ibm}{\ib^{[m]}} % In-Bag (IB) for m-th bootstrap
\newcommand{\oob}{\mathrm{OOB}} % Out-of-Bag (OOB)
\newcommand{\oobm}{\oob^{[m]}} % Out-of-Bag (OOB) for m-th bootstrap


% ml - boosting
\newcommand{\fm}[1][m]{f^{[#1]}} % prediction in iteration m
\newcommand{\fmh}[1][m]{\hat{f}^{[#1]}} % prediction in iteration m
\newcommand{\fmd}[1][m]{f^{[#1-1]}} % prediction m-1
\newcommand{\fmdh}[1][m]{\hat{f}^{[#1-1]}} % prediction m-1
\newcommand{\errm}[1][m]{\text{err}^{[#1]}} % weighted in-sample misclassification rate
\newcommand{\wm}[1][m]{w^{[#1]}} % weight vector of basemodel m
\newcommand{\wmi}[1][m]{w^{[#1](i)}} % weight of obs i of basemodel m
\newcommand{\thetam}[1][m]{\thetab^{[#1]}} % parameters of basemodel m
\newcommand{\thetamh}[1][m]{\hat{\thetab}^{[#1]}} % parameters of basemodel m with hat
\newcommand{\blxt}[1][m]{b(\xv, \thetab^{[#1]})} % baselearner, default m
\newcommand{\ens}{\sum_{m=1}^M \betam \blxt} % ensemble
\newcommand{\rmm}[1][m]{\tilde{r}^{[#1]}} % pseudo residuals
\newcommand{\rmi}[1][m]{\tilde{r}^{[#1](i)}} % pseudo residuals
\newcommand{\Rtm}[1][m]{R_{t}^{[#1]}} % terminal-region
\newcommand{\Tm}[1][m]{T^{[#1]}} % terminal-region
\newcommand{\ctm}[1][m]{c_t^{[#1]}} % mean, terminal-regions
\newcommand{\ctmh}[1][m]{\hat{c}_t^{[#1]}} % mean, terminal-regions with hat
\newcommand{\ctmt}[1][m]{\tilde{c}_t^{[#1]}} % mean, terminal-regions
\newcommand{\Lp}{L^\prime}
\newcommand{\Ldp}{L^{\prime\prime}}
\newcommand{\Lpleft}{\Lp_{\text{left}}}

% ml - boosting iml lecture
\newcommand{\ts}{\thetab^{\star}} % theta*
\newcommand{\bljt}{\bl[j](\xv, \thetab)} % BL j with theta
\newcommand{\bljts}{\bl[j](\xv, \ts)} % BL j with theta*


\title{I2ML :\,: CHEAT SHEET} % Package title in header, \, adds thin space between ::
\newcommand{\packagedescription}{ % Package description in header
	The \textbf{I2ML}: Introduction to Machine Learning course offers an introductory and applied overview of "supervised" Machine Learning. It is organized as a digital lecture.
}

\newlength{\columnheight} % Adjust depending on header height
\setlength{\columnheight}{84cm} 

\newtcolorbox{codebox}{%
	sharp corners,
	leftrule=0pt,
	rightrule=0pt,
	toprule=0pt,
	bottomrule=0pt,
	hbox}

\newtcolorbox{codeboxmultiline}[1][]{%
	sharp corners,
	leftrule=0pt,
	rightrule=0pt,
	toprule=0pt,
	bottomrule=0pt,
	#1}

\begin{document}
\begin{frame}[fragile]{}
\begin{columns}
	\begin{column}{.31\textwidth}
		\begin{beamercolorbox}[center]{postercolumn}
			\begin{minipage}{.98\textwidth}
				\parbox[t][\columnheight]{\textwidth}{
					\begin{myblock}{Bagging Ensembles}

						\begin{codebox}
			\textbf{Basic Idea:}
						\end{codebox}
						\begin{itemize}[$\bullet$]     
            \setlength{\itemindent}{+.3in}
                        \item \textbf{Bagging:} \textbf{B}ootstrap \textbf{Agg}regation
                        \item \textbf{Ensemble method}: combines models into large \enquote{meta-model}
                        \item \textbf{Base learners (BLs)}: individual models of an ensemble
                        \item Homogeneous ensembles: all BLs are of the same model type\\ $\rightarrow$  Only training data varies due to bootstrap
                         \item \textbf{Bagging reduces prediction variance} by averaging over BL predictions with a mild \textbf{increase in bias} due to data reuse\\
                         $\rightarrow$ Performs best when BL predictions are only weakly correlated
                        \end{itemize}
            %Bagging improve performance by reducing the variance of predictions, but (slightly) increases the bias: training data is reused, so small mistakes can get amplified. Works best if BLs' predictions are only weakly correlated.
						
				

						\begin{codebox}
			      \textbf{Bagging Training: }
						\end{codebox}
						Train BL $\bl, m = 1, \dots, M$ on $M$ \textbf{bootstrap} samples of $\D$. 
						\begin{algorithm}[H]
              \small
              \caption{Bagging algorithm: Training}
              \begin{algorithmic}[1]
                \State Input: Dataset $\D$, type of BLs, number of bootstraps $M$
                \For {$m = 1 \to M$}
                  \State Draw a bootstrap sample $\D^{[m]}$ from $\D$
                  \State Train BL on $\D^{[m]}$ to obtain model $\blh$
                \EndFor
              \end{algorithmic}
            \end{algorithm}

            \begin{codebox}
              \textbf{Bagging Aggregating: }
              \end{codebox}

            Average predictions of $M$ fitted BLs to obtain the ensemble model:
            \begin{align*}
              \hat{f}(\xv) &= \meanmM \left(\blfhx\right) \text{(regression / decision score, use $\hat{f}_k$ in multi-class)} \\[-1ex]
              \hxh &= \argmax_{k \in \Yspace} \summM \I \left(\bllhx = k\right) \text{(majority voting)} \\[-1ex]
            \pikxh &=
            \left\{
            \begin{aligned}
            & \meanmM \blphxk && \text{(probabilities through averaging)} \\[-1ex]
            & \meanmM \I \left(\bllhx = k \right) && \text{(probabilities through class frequencies)}
            \end{aligned}
            \right.
            \end{align*}

             \end{myblock}

             \begin{myblock}{Random Forests}
              \begin{codebox} \textbf{Introduction: }
              \end{codebox}
          
              \begin{itemize}[$\bullet$]     
                \setlength{\itemindent}{+.3in}
                \item Random Forests (RFs) use \textbf{bagging with CARTs} as BLs.
                \item \textbf{Random feature sampling} (\texttt{mtry}) decorrelates the BLs.
                \item \textbf{Fully expanded trees} further increase variability of trees.
                \end{itemize}

              \end{myblock}
             
				}
			\end{minipage}
		\end{beamercolorbox}
	\end{column}
	
%%%%%%%%%%%%%%%%%%%%%%%%%%%%%%%%%%%%%%%%%%%%%%%%%%%%%%%%%%%%%%%%%%%%%

\begin{column}{.31\textwidth}
\begin{beamercolorbox}[center]{postercolumn}
\begin{minipage}{.98\textwidth}
\parbox[t][\columnheight]{\textwidth}{
  \begin{myblock}{ }      
      \textbf{Advantages:} Can handle missing values / categorical features; Easy to parallelize; Work well on high-dimensional / noisy data\\
      \textbf{Disadvantages:} Extrapolation problem; Harder to interpret; Memory-hungry implementation; Expensive prediction for large ensembles
        
      \begin{codebox} \textbf{Random Feature Sampling: }
      \end{codebox}
      \begin{itemize}[$\bullet$]     
        \setlength{\itemindent}{+.3in}
        \item For each node, randomly draw $\texttt{mtry} \le p$ features to find best split\\
        $\rightarrow$ More randomness $\Rightarrow$ lower correlation between trees

        \item Classification: $\texttt{mtry}$ $ = \lfloor \sqrt{p} \rfloor$; Regression: $\texttt{mtry}$ $ = \lfloor p/3 \rfloor$
        \end{itemize}

        \begin{codebox} \textbf{Note:}
        \end{codebox}
        \begin{itemize}[$\bullet$]     
          \setlength{\itemindent}{+.3in}
          \item Other hyperparameters: minimum node size and tree depth
          %\item RF usually use fully expanded trees, without aggressive early stopping or pruning, to further \textbf{increase variability of each tree}.\\
              \item Performance improves with larger ensembles
    \item May overfit, but less prone than individual CARTs due to averaging
    \item Trees are trained independently $\Rightarrow$ increasing the number of trees reduces variance \textbf{without increasing overfitting}
        \end{itemize}
      \end{myblock}

      \begin{myblock}{Out-of-Bag (OOB)}
       \begin{algorithm}[H]
              \small
              \caption{Out-Of-Bag error estimation}
              \begin{algorithmic}[1]
                \State Input: $\oobm = \{i \in \nset | \xyi \notin \D^{[m]}\}$, $\blh \ \forall m \in \{1, \dots, M\}$
                \For {$i = 1 \to n$}
                  \State Compute the ensemble OOB prediction for observation $i$, e.g., for regression:
                  $$\fh^{(i)}_{\oob} =
                  \frac{1}{S_{\oob}^{(i)}} \summM
                  \I(i \in \oobm) \cdot \blfh(\xi) $$

                  $S_{\mathrm{OOB}}^{(i)} = \summM \I(i \in \oobm)$ is the nr. of trees where $i$-th observation is OOB.
                \EndFor
                \State Average losses over all observations: $\GEh_{\oob} = \meanin L(\yi, \fh^{(i)}_{\oob})$
            
              \end{algorithmic}
            \end{algorithm}

            \textbf{OOB Size:} The probability that an observation is out-of-bag (OOB) is:
            $$\P \left(i \in \oobm\right) = \left(1 - \frac{1}{n}\right)^n \stackrel{n \to \infty}{\longrightarrow} \frac{1}{e} \approx 0.37$$
            $\Rightarrow$ similar to holdout or 3-fold CV (1/3 validation, 2/3 training)

            \textbf{OOB Error Usage:}
Get first impression of RF performance; Select ensemble size; Evaluate different RF hyperparameter configurations.
\end{myblock}
}
\end{minipage}
\end{beamercolorbox}
\end{column}

%%%%%%%%%%%%%%%%%%%%%%%%%%%%%%%%%%%%%%%%%%%%%%%%%%%%%%%%%%%%%%%%%%%%%%%%%%%%%%%%%%%%%%%%%%%%%%%%%%%%

\begin{column}{.31\textwidth}
\begin{beamercolorbox}[center]{postercolumn}
\begin{minipage}{.98\textwidth}
\parbox[t][\columnheight]{\textwidth}{
  \begin{myblock}{Feature Importance}

\begin{itemize}[$\bullet$]     
          \setlength{\itemindent}{+.3in}
    \item RF perform better but are less interpretable compared to a single tree
\item \textbf{Feature importance} mitigates this by quantifying performance drop when a feature is permuted or excluded
\end{itemize}
    %RFs improve accuracy by aggregating multiple decision trees but \textbf{lose interpretability} compared to a single tree. 
    %\textbf{Feature importance} mitigates this problem by asking how much does performance decrease, if feature is removed / rendered useless.

    \begin{algorithm}[H]
      \small
      \caption{Feature importance based on permutations}
      \begin{algorithmic}[1]
        \State Calculate $\GEh_{\oob}$ using set-based metric $\rho$
        \For{features $x_j$, $j = 1 \to p$}
        \For{Some statistical repetitions}
          \State {Distort feature-target relation: permute $x_j$ with $\psi_j$}
          \State Compute all $n$ OOB-predictions for permuted feature data, obtain all $\fh^{(i)}_{\oob,\psi_j}$
          \State Arrange predictions in $\bm{\hat{F}}_{\oob,\psi_j}$;
          Compute $\GEh_{\oob, j} = \rho(\yv, \bm{\hat{F}}_{\oob,\psi_j})$
        \State {Estimate importance of $j$-th feature: $\widehat{\text{FI}_j} = \GEh_{\oob,j} - \GEh_{\oob
        } $}
        \EndFor
        \State Average obtained $\widehat{\text{FI}_j}$ values over reps
        \EndFor
      \end{algorithmic}
      \end{algorithm}

    \begin{algorithm}[H]
      \small
      \caption{Feature importance based on impurity}
      \begin{algorithmic}[1]
        \For{features $x_j$, $j = 1 \to p$}
        \For{all models $\blh$, $m = 1 \to M$}
        \State {Find all splits in $\blh$ on $x_j$}
        \State {Extract improvement / risk reduction for these splits}
        \State {Sum them up}
        \EndFor
        \State {Add up improvements over all trees for FI of $x_j$}
        \EndFor
      \end{algorithmic}
      \end{algorithm}
  \end{myblock}

\begin{myblock}{Proximities}


\begin{itemize}[$\bullet$]     
          \setlength{\itemindent}{+.3in}
 \item \textbf{Proximities:} Similarity measure derived from co-occurrence in terminal nodes across trees.
  \item For observations $\xi$ and $\xi[j]$:
  \[
    \operatorname{prox}(\xi, \xi[j]) = \frac{\text{\# trees where } \xi \text{ and } \xi[j] \text{ share a leaf}}{\text{\# total trees}}
  \]
  \item Defines an intrinsic, forest-induced similarity between instances.
  \item \textbf{Applications:} Missing data imputation, outlier detection, identifying mislabeled samples, forest visualization.
  \end{itemize}

% \textbf{Proximities} measure of similarity ("closeness" or "nearness") of observations derived from random forests.
% The proximity $\operatorname{prox}\left(\xi, \xi[j]\right)$ between two observations $\xi$ and $\xi[j]$ is calculated by measuring the number of times that these two observations are placed in the same terminal node of the same tree of random forest, 
% divided by the number of trees in the forest.
% It forms an intrinsic similarity measure between pairs of observations.

% \textbf{Proximities usage: }Imputing missing data, locating outliers, identifying mislabeled data, visualizing the forest.

\end{myblock}
}

  \end{minipage}
  \end{beamercolorbox}
  \end{column}
  
  
  
\end{columns}
\end{frame}
\end{document}
