\documentclass{beamer}

\usepackage[orientation=landscape,size=a0,scale=1.4,debug]{beamerposter}
\mode<presentation>{\usetheme{mlr}}

\usepackage[utf8]{inputenc} % UTF-8
\usepackage[english]{babel} % Language
\usepackage{hyperref} % Hyperlinks
\usepackage{ragged2e} % Text position
\usepackage[export]{adjustbox} % Image position
\usepackage[most]{tcolorbox}
%\usepackage{nomencl}
%\makenomenclature
\usepackage{amsmath}
\usepackage{mathtools}
\usepackage{dsfont}
\usepackage{verbatim}
\usepackage{amsmath}
\usepackage{amsfonts}
\usepackage{csquotes}
\usepackage{multirow}
\usepackage{longtable}
\usepackage{enumerate}
\usepackage[absolute,overlay]{textpos}
\usepackage{psfrag}
\usepackage{algorithm}
\usepackage{algpseudocode}
\usepackage{eqnarray}
\usepackage{arydshln}
\usepackage{tabularx}
\usepackage{placeins}
\usepackage{tikz}
\usepackage{setspace}
\usepackage{colortbl}
\usepackage{mathtools}
\usepackage{wrapfig}

% math spaces
\newcommand{\N}{\mathds{N}}                                                 % N, naturals
\newcommand{\Z}{\mathds{Z}}                                                 % Z, integers
\newcommand{\Q}{\mathds{Q}}                                                 % Q, rationals
\newcommand{\R}{\mathds{R}}                                                 % R, reals
\newcommand{\C}{\mathds{C}}                                                 % C, complex
\newcommand{\HS}{\mathcal{H}}                                               % H, hilbertspace
\newcommand{\continuous}{\mathcal{C}}                                       % C, space of continuous functions
\newcommand{\M}{\mathcal{M}} 												% machine numbers
\newcommand{\epsm}{\epsilon_m} 												% maximum error


% basic math stuff
\newcommand{\xt}{\tilde x}													% x tilde
\def\argmax{\mathop{\sf arg\,max}}                                          % argmax
\def\argmin{\mathop{\sf arg\,min}}                                          % argmin
\newcommand{\sign}{\operatorname{sign}}                                     % sign, signum
\newcommand{\I}{\mathbb{I}}                                                 % I, indicator
\newcommand{\order}{\mathcal{O}}                                            % O, order
\newcommand{\fp}[2]{\frac{\partial #1}{\partial #2}}                        % partial derivative
\newcommand{\pd}[2]{\frac{\partial{#1}}{\partial #2}}						% partial derivative

% sums and products
\newcommand{\sumin}{\sum_{i=1}^n}											% summation from i=1 to n
\newcommand{\sumkg}{\sum_{k=1}^g}											% summation from k=1 to g
\newcommand{\prodin}{\prod_{i=1}^n}											% product from i=1 to n
\newcommand{\prodkg}{\prod_{k=1}^g}											% product from k=1 to g

% linear algebra
\newcommand{\one}{\boldsymbol{1}}                                           % 1, unitvector
\newcommand{\id}{\mathrm{I}}                                                % I, identity
\newcommand{\diag}{\operatorname{diag}}                                     % diag, diagonal
\newcommand{\trace}{\operatorname{tr}}                                      % tr, trace
\newcommand{\spn}{\operatorname{span}}                                      % span
\newcommand{\scp}[2]{\left\langle #1, #2 \right\rangle}                     % <.,.>, scalarproduct
\newcommand{\mat}[1]{ 														% short pmatrix command
	\begin{pmatrix}
		#1
	\end{pmatrix}
}
\newcommand{\Amat}{\bm{A}}													% matrix A
\newcommand{\xv}{\bm{x}}													% vector x (bold)
\newcommand{\yv}{\bm{y}}														% vector y (bold)
\newcommand{\Deltab}{\bm{\Delta}}											% error term for vectors
															

% basic probability + stats
\renewcommand{\P}{\mathds{P}}                                               % P, probability
\newcommand{\E}{\mathds{E}}                                                 % E, expectation
\newcommand{\var}{\mathsf{Var}}                                             % Var, variance
\newcommand{\cov}{\mathsf{Cov}}                                             % Cov, covariance
\newcommand{\corr}{\mathsf{Corr}}                                           % Corr, correlation
\newcommand{\normal}{\mathcal{N}}                                           % N of the normal distribution
\newcommand{\iid}{\overset{i.i.d}{\sim}}                                    % dist with i.i.d superscript
\newcommand{\distas}[1]{\overset{#1}{\sim}}                                 % ... is distributed as ... 
% machine learning

%%%%%% ml - data
\newcommand{\Xspace}{\mathcal{X}}                                           % X, input space
\newcommand{\Yspace}{\mathcal{Y}}                                           % Y, output space
\newcommand{\nset}{\{1, \ldots, n\}}                                        % set from 1 to n
\newcommand{\pset}{\{1, \ldots, p\}}                                        % set from 1 to p
\newcommand{\gset}{\{1, \ldots, g\}}                                        % set from 1 to g
\newcommand{\Pxy}{\P_{xy}}                                                  % P_xy
\newcommand{\xy}{(x, y)}                                                    % observation (x, y)
\newcommand{\xvec}{(x_1, \ldots, x_p)^T}                                    % (x1, ..., xp) 
\newcommand{\D}{\mathcal{D}}                                                % D, data 
\newcommand{\Dset}{\{ (x^{(1)}, y^{(1)}), \ldots, (x^{(n)},  y^{(n)})\}}    % {(x1,y1)), ..., (xn,yn)}, data
\newcommand{\xdat}{\{ x^{(1)}, \ldots, x^{(n)}\}}   						 % {x1, ..., xn}, input data
\newcommand{\ydat}{\mathbf{y}}                                              % y (bold), vector of outcomes
\newcommand{\yvec}{(y^{(1)}, \hdots, y^{(n)})^T}                            % (y1, ..., yn), vector of outcomes
\renewcommand{\xi}[1][i]{x^{(#1)}}                                          % x^i, i-th observed value of x
\newcommand{\yi}[1][i]{y^{(#1)}}                                            % y^i, i-th observed value of y 
\newcommand{\xyi}{(\xi, \yi)}                                               % (x^i, y^i), i-th observation
\newcommand{\xivec}{(x^{(i)}_1, \ldots, x^{(i)}_p)^T}                       % (x1^i, ..., xp^i), i-th observation vector
\newcommand{\xj}{x_j}                                                       % x_j, j-th feature
\newcommand{\xjb}{\mathbf{x}_j}                                             % x_j (bold), j-th feature vecor
\newcommand{\xjvec}{(x^{(1)}_j, \ldots, x^{(n)}_j)^T}                       % (x^1_j, ..., x^n_j), j-th feature vector
\newcommand{\Dtrain}{\mathcal{D}_{\text{train}}}                            % D_train, training set
\newcommand{\Dtest}{\mathcal{D}_{\text{test}}}                              % D_test, test set

%%%%%% ml - models general

% continuous prediction function f
\newcommand{\fx}{f(x)}                                                      % f(x), continuous prediction function
\newcommand{\Hspace}{H}														% hypothesis space where f is from
\newcommand{\fh}{\hat{f}}                                                   % f hat, estimated prediction function
\newcommand{\fxh}{\fh(x)}                                                   % fhat(x)
\newcommand{\fxt}{f(x | \theta)}                                            % f(x | theta)
\newcommand{\fxi}{f(\xi)}                                                   % f(x^(i))
\newcommand{\fxih}{\hat{f}(\xi)}                                            % f(x^(i))
\newcommand{\fxit}{f(x^{(i)} | \theta)}                                     % f(x^(i) | theta)
\newcommand{\fhD}{\fh_{\D}}                                                 % fhat_D, estimate of f based on D
\newcommand{\fhDtrain}{\fh_{\Dtrain}}                                       % fhat_Dtrain, estimate of f based on D

% discrete prediction function h
\newcommand{\hx}{h(x)}                                                      % h(x), discrete prediction function
\newcommand{\hh}{\hat{h}}                                                   % h hat
\newcommand{\hxh}{\hat{h}(x)}                                               % hhat(x)
\newcommand{\hxt}{h(x | \theta)}                                            % h(x | theta)
\newcommand{\hxi}{h(\xi)}                                                   % h(x^(i))
\newcommand{\hxit}{h(x^{(i)} | \theta)}                                     % h(x^(i) | theta)

% yhat
\newcommand{\yh}{\hat{y}}                                                   % y hat for prediction of target
\newcommand{\yih}{\hat{y}}                                                  % y hat for prediction of target

% theta
\newcommand{\thetah}{\hat{\theta}}                                          % theta hat

% densities + probabilities
% pdf of x 
\newcommand{\pdf}{p}                                                        % p
\newcommand{\pdfx}{p(x)}                                                    % p(x)
\newcommand{\pixt}{\pi(x | \theta)}                                         % pi(x|theta), pdf of x given theta

% pdf of (x, y)
\newcommand{\pdfxy}{p(x,y)}                                                 % p(x, y)
\newcommand{\pdfxyt}{p(x, y | \theta)}                                      % p(x, y | theta)
\newcommand{\pdfxyit}{p(\xi, \yi | \theta)}                                 % p(x^(i), y^(i) | theta)

% pdf of x given y
\newcommand{\pdfxyk}{p(x | y=k)}                                            % p(x | y = k)
\newcommand{\lpdfxyk}{\log \pdfxyk}                                         % log p(x | y = k)
\newcommand{\pdfxiyk}{p(\xi | y=k)}                                         % p(x^i | y = k)

% prior probabilities
\newcommand{\pik}{\pi_k}                                                    % pi_k, prior
\newcommand{\lpik}{\log \pik}                                               % log pi_k, log of the prior

% posterior probabilities
\newcommand{\post}{\P(y = 1 | x)}                                           % P(y = 1 | x), post. prob for y=1
\newcommand{\pix}{\pi(x)}                                                   % pi(x), P(y = 1 | x)
\newcommand{\postk}{\P(y = k | x)}                                          % P(y = k | y), post. prob for y=k
\newcommand{\pikx}{\pi_k(x)}                                                % pi_k(x), P(y = k | x)
\newcommand{\pikxt}{\pi_k(x | \theta)}                                      % pi_k(x | theta), P(y = k | x, theta)
\newcommand{\pijx}{\pi_j(x)}                                                % pi_j(x), P(y = j | x)
\newcommand{\pdfygxt}{p(y |x, \theta)}                                      % p(y | x, theta)
\newcommand{\pdfyigxit}{p(\yi |\xi, \theta)}                                % p(y^i |x^i, theta)
\newcommand{\lpdfygxt}{\log \pdfygxt }                                      % log p(y | x, theta)
\newcommand{\lpdfyigxit}{\log \pdfyigxit}                                   % log p(y^i |x^i, theta)
\newcommand{\pixh}{\hat \pi(x)}                                             % pi(x) hat, P(y = 1 | x) hat
\newcommand{\pikxh}{\hat \pi_k(x)}                                          % pi_k(x) hat, P(y = k | x) hat

% residual and margin
\newcommand{\eps}{\epsilon}                                                 % residual, stochastic
\newcommand{\epsi}{\epsilon^{(i)}}                                          % epsilon^i, residual, stochastic
\newcommand{\epsh}{\hat{\epsilon}}                                          % residual, estimated
\newcommand{\yf}{y \fx}                                                     % y f(x), margin
\newcommand{\yfi}{\yi \fxi}                                                 % y^i f(x^i), margin
\newcommand{\Sigmah}{\hat \Sigma}											% estimated covariance matrix
\newcommand{\Sigmahj}{\hat \Sigma_j}										% estimated covariance matrix for the j-th class

% ml - loss, risk, likelihood
\newcommand{\Lxy}{L(y, f(x))}                                               % L(y, f(x)), loss function
\newcommand{\Lxyi}{L(\yi, \fxi)}                                            % L(y^i, f(x^i))
\newcommand{\Lxyt}{L(y, \fxt)}                                              % L(y, f(x | theta))
\newcommand{\Lxyit}{L(\yi, \fxit)}                                          % L(y^i, f(x^i | theta)
\newcommand{\risk}{\mathcal{R}}                                             % R, risk
\newcommand{\riskf}{\risk(f)}                                               % R(f), risk
\newcommand{\riske}{\mathcal{R}_{\text{emp}}}                               % R_emp, empirical risk
\newcommand{\riskef}{\riske(f)}                                             % R_emp(f)
\newcommand{\risket}{\mathcal{R}_{\text{emp}}(\theta)}                      % R_emp(theta)
\newcommand{\riskr}{\mathcal{R}_{\text{reg}}}                               % R_reg, regularized risk
\newcommand{\riskrt}{\mathcal{R}_{\text{reg}}(\theta)}                      % R_reg(theta)
\newcommand{\riskrf}{\riskr(f)}                                             % R_reg(f)
\newcommand{\LL}{\mathcal{L}}                                               % L, likelihood
\newcommand{\LLt}{\mathcal{L}(\theta)}                                      % L(theta), likelihood
\renewcommand{\ll}{\ell}                                                    % l, log-likelihood
\newcommand{\llt}{\ell(\theta)}                                             % l(theta), log-likelihood
\newcommand{\LS}{\mathfrak{L}}                                              % ????????????
\newcommand{\TS}{\mathfrak{T}}                                              % ??????????????
\newcommand{\errtrain}{\text{err}_{\text{train}}}                           % training error
\newcommand{\errtest}{\text{err}_{\text{test}}}                             % training error
\newcommand{\errexp}{\overline{\text{err}_{\text{test}}}}                   % training error

% resampling
\newcommand{\GE}[1]{GE(\fh_{#1})}                                           % Generalization error GE
\newcommand{\GEh}[1]{\widehat{GE}_{#1}}                                     % Estimated train error
\newcommand{\GED}{\GE{\D}}                                                  % Generalization error GE
\newcommand{\EGEn}{EGE_n}                                                   % Generalization error GE
\newcommand{\EDn}{\E_{|D| = n}}                                             % Generalization error GE


% ml - irace
\newcommand{\costs}{\mathcal{C}} % costs
\newcommand{\Celite}{\theta^*} % elite configurations
\newcommand{\instances}{\mathcal{I}} % sequence of instances
\newcommand{\budget}{\mathcal{B}} % computational budget
% ml - bagging, random forest

\newcommand{\bl}[1]{b^{[#1]}(x)}											% baselearner with argument for m
\newcommand{\blm}{\bl{m}}												    % baselearner without argument for m
\newcommand{\blmh}{\hat{b}^{[m]}(x)}										% estimated base learner 

\input{ml-boosting.tex}
% ml - trees, extra trees

\newcommand{\Np}{\mathcal{N}}												% Parent node N
\newcommand{\Nl}{\Np_1}														% Left node N_1
\newcommand{\Nr}{\Np_2}														% Right node N_2



\title{I2ML :\,: CHEAT SHEET} % Package title in header, \, adds thin space between ::
\newcommand{\packagedescription}{ % Package description in header
	The \textbf{I2ML}: Introduction to Machine Learning course offers an introductory and applied overview of "supervised" Machine Learning. It is organized as a digital lecture.
}

\newlength{\columnheight} % Adjust depending on header height
\setlength{\columnheight}{84cm} 

\newtcolorbox{codebox}{%
	sharp corners,
	leftrule=0pt,
	rightrule=0pt,
	toprule=0pt,
	bottomrule=0pt,
	hbox}

\newtcolorbox{codeboxmultiline}[1][]{%
	sharp corners,
	leftrule=0pt,
	rightrule=0pt,
	toprule=0pt,
	bottomrule=0pt,
	#1}

\begin{document}
\begin{frame}[fragile]{}
\begin{columns}
	\begin{column}{.31\textwidth}
		\begin{beamercolorbox}[center]{postercolumn}
			\begin{minipage}{.98\textwidth}
				\parbox[t][\columnheight]{\textwidth}{				
					\begin{myblock}{Classification}
						We want to assign new observations to known categories according to criteria learned from a training set.  
						\\
						\begin{codebox}
						    Assume we are given a \textbf{classification problem:}
						\end{codebox}
						\hspace*{1ex}\begin{eqnarray*} & x \in \Xspace \quad & \text{feature vector}\\ & y \in \Yspace = \gset \quad & \text{\emph{categorical} output variable (label)}\\ &\D = \Dset & \text{observations of $x$ and $y$} \end{eqnarray*}\\
						\begin{codebox}
							Classification usually means to construct $g$ discriminant functions:
						\end{codebox}
						\hspace*{1ex}$f_1(x), \ldots f_g(x)$, so that we choose our class as $h(x) = \argmax_k f_k(x)$ \hspace*{1ex}for $k = 1, 2,\ldots, g$
						\\
						\begin{codebox}
							\textbf{Linear Classifier:}
						\end{codebox}
						\hspace*{1ex}If the functions $f_k(x)$ can be specified as linear functions, we will call \hspace*{1ex}the classifier a \emph{linear classifier}.
						\\
						
						\hspace*{1ex}\textbf{Note: }All linear classifiers can represent non-linear decision boundaries \hspace*{1ex}in our original input space if we include derived features. For example: \hspace*{1ex}higher order interactions, polynomials or other transformations of x in \hspace*{1ex}the model.
						\\
			
						\begin{codebox}
							 \textbf{Binary classification: }If only 2 classes exist
						\end{codebox}
						\hspace*{1ex}We can use a single discriminant function $f(x) = f_1(x) - f_2(x)$.
						\\

					\end{myblock}
					%%%%%%%%%%%%%%%%%%%%%%%%%%%%%%%%%%%%%%%%%%%
					\begin{myblock}{Generative Approach}
						\begin{codebox}
							\textbf{Generative approach} models $\pdfxyk$, usually by making some
						\end{codebox}
						\begin{codebox}
						    assumptions about the structure of these distributions and employs the
						\end{codebox}
						\begin{codebox}
						    Bayes theorem:
						\end{codebox}
						\hspace*{1ex}$\pikx = \postk \propto \pdfxyk \pik$. It allows the computation of \hspace*{1ex}$\pikx$.
						\\
					\end{myblock}\vfill
				}
			\end{minipage}
		\end{beamercolorbox}
	\end{column}
	\begin{column}{.31\textwidth}
		\begin{beamercolorbox}[center]{postercolumn}
			\begin{minipage}{.98\textwidth}
				\parbox[t][\columnheight]{\textwidth}{
					\begin{myblock}{}
					    \begin{codebox}
							Examples of \textbf{Generative} approach:
					    \end{codebox}
						\hspace*{1ex}Generative Approach models $p(x|y)$ and $p(y)$. \\\hspace*{1ex}Example: 
						\begin{enumerate}
                        \item Linear discriminant analysis (LDA)
                        \item Quadratic discriminant analysis (QDA)
                        \item Naïve Bayes
                        \end{enumerate}
						\hspace*{1ex}
						\begin{codebox}
							 \textbf{Linear Discriminant Analysis (LDA): }follows a generative approach,
						\end{codebox}
						\begin{codebox}
							  each class density is modeled as a \emph{multivariate Gaussian} with equal
						\end{codebox}
						\begin{codebox}
							  covariance, i. e. $\Sigma_k = \Sigma \quad \forall k$.
						\end{codebox}
						\hspace*{1ex}Parameters $\theta$ are estimated in a straight-forward manner by estimating \hspace*{1ex}$\hat{\pi}_k$, $\hat{\mu}_k$, $\hat{\Sigma}$.
						\begin{enumerate}
                        \item Each class fit as a Gaussian distribution over the feature space
                        \item Different means but same covariance for all classes
                        \item Rather restrictive model assumption.
                        \end{enumerate}
                        \hspace*{1ex}
						\begin{codebox}
							 \textbf{Quadratic Discriminant Analysis (QDA): }is a direct generalization of
						\end{codebox}
						\begin{codebox}
							  LDA, where the class densities are now Gaussians with unequal covariances
						\end{codebox}
						\begin{codebox}
							  $\Sigma_k$.
						\end{codebox}
						\hspace*{1ex}Parameters $\theta$ are estimated in a straight-forward manner by estimating \hspace*{1ex}$\hat{\pi}_k$, $\hat{\mu}_k$, $\hat{\Sigma_k}$.
						\begin{enumerate}
                        \item Covariance matrices can differ over classes.
                        \item Yields better data fit but also requires estimation of more parameters.
                        \end{enumerate}
                        \hspace*{1ex}
                        
						\begin{codebox}
							 \textbf{Naive Bayes classifier: }A “naive” conditional independence assumption is
						\end{codebox}
						\begin{codebox}
							  made: the features given the category y are conditionally independent of
						\end{codebox}
						\begin{codebox}
							   each other
						\end{codebox}
						\begin{enumerate}
                        \item Covariance matrices can differ over both classes but assumed to be diagonal.
                        \item Assumption of uncorrelated features. Often performs well despite this usually wrong assumption.
                        \item Easy to deal with mixed features (metric and categorical)
                        \end{enumerate}
						\hspace*{1ex}
						%%%%%%%%%%%%%%%%%%%%%%%%%%%%%%%%%%%
					\begin{myblock}{Discriminant  Approach}
						\begin{codebox}
							\textbf{Discriminant approach: }tries to optimize the discriminant functions
						\end{codebox}
						\begin{codebox}
						    directly, usually via empirical risk minimization:
						\end{codebox}
						\hspace*{1ex}$ \fh = \argmin_{f \in \Hspace} \riske(f) = \argmin_{f \in \Hspace} \sumin \Lxyi.$
						\\
						\begin{codebox}
							Examples of \textbf{Discriminant} approach:
						\end{codebox}
						\end{myblock}
						\\
						\hspace*{1ex}\textbf{Discriminant Approach: }models $p(y|x)$ directly. \\\hspace*{1ex}Example: 
						\begin{enumerate}
                        \item Logistic/Softmax regression
                        \item kNN
                        \end{enumerate}
                        \hspace*{1ex}
                        %\end{myblock}
						\begin{codebox}
				            $\eps = y - \fx$ or $\epsi = \yi - \fxi$
						\end{codebox}
					\end{myblock}
				}
			\end{minipage}
		\end{beamercolorbox}
	\end{column}
	\begin{column}{.31\textwidth}
		\begin{beamercolorbox}[center]{postercolumn}
			\begin{minipage}{.98\textwidth}
				\parbox[t][\columnheight]{\textwidth}{
				    \begin{myblock}{ }
				        \hspace*{1ex}\textbf{Discriminant Approach: }models $p(y|x)$ directly. \\\hspace*{1ex}Example: 
						\begin{enumerate}
                        \item Logistic/Softmax regression
                        \item kNN
                        \end{enumerate}
                        \hspace*{1ex}
                        %\end{myblock}
						\begin{codebox}
				            \textbf{Logistic Regression: }A \emph{discriminant} approach for directly modeling
						\end{codebox}
						\begin{codebox}
				            the posterior probabilities $\pix$ of the labels is \textbf{logistic regression}
						\end{codebox}
						\hspace*{1ex}We focus on the binary case $y \in \{0, 1\}$. We then model: \[ \pix = \post = \theta^T x .\] 
						\hspace*{1ex}and this could result in predicted probabilities $\pix \not\in [0,1]$. To avoid \hspace*{1ex}this, logistic regression \enquote{squashes} the estimated linear scores $\theta^T x$ to \hspace*{1ex}$[0,1]$ through the \textbf{logistic function} $s$:
						\[ \pix = \frac{\exp\left(\theta^Tx\right)}{1+\exp\left(\theta^Tx\right)} = \frac{1}{1+\exp\left(-\theta^Tx\right)} = s\left(\theta^T x\right) \]
						\\
						\begin{codebox}
							\textbf{Cross Entropy loss: }Minimizing it refers to maximizing the
						\end{codebox}
						\begin{codebox}
							probabilities of logistic regression, where the labels are $\Yspace = \{0,1\}$
						\end{codebox}
						\hspace*{1ex}$\Lxy = y \theta^T x - \log[1 + \exp(\theta^T x)]$
						\\
						\begin{codebox}
							 \textbf{Bernoulli Loss: }If we encode the labels with $\Yspace = \{-1,+1\}$ we can
						\end{codebox}
						\begin{codebox}
							 simplify the loss function as: $\Lxy = \log[1 + \exp(-\yf] $
						\end{codebox}
						\hspace*{1ex}Bernoulli loss is equivalent to Cross Entropy loss encoded differently.
						\\
						\begin{codebox}
							 \textbf{Softmax: }is a generalization of the logistic function. It \enquote{squashes}
						\end{codebox}
						\begin{codebox}
							 a g-dimensional real-valued vector $z$ to a vector of the same dimension,
						\end{codebox}
						\begin{codebox}
							  with every entry in the range [0, 1] and all entries adding up to 1.
						\end{codebox}
						\hspace*{1ex}\emph{Softmax} is defined on a numerical vector $z$: $$ s(z)_k = \frac{\exp(z_k)}{\sum_{j}\exp(z_j)} $$
					\end{myblock}
				}
			\end{minipage}
		\end{beamercolorbox}
	\end{column}
\end{columns}
\end{frame}
\end{document}
