\documentclass{beamer}
\newcommand \beameritemnestingprefix{}


\usepackage[orientation=landscape,size=a0,scale=1.4,debug]{beamerposter}
\mode<presentation>{\usetheme{mlr}}


\usepackage[utf8]{inputenc} % UTF-8
\usepackage[english]{babel} % Language
\usepackage{hyperref} % Hyperlinks
\usepackage{ragged2e} % Text position
\usepackage[export]{adjustbox} % Image position
\usepackage[most]{tcolorbox}
\usepackage{amsmath}
\usepackage{mathtools}
\usepackage{dsfont}
\usepackage{verbatim}
\usepackage{amsmath}
\usepackage{amsfonts}
\usepackage{csquotes}
\usepackage{multirow}
\usepackage{longtable}
\usepackage[absolute,overlay]{textpos}
\usepackage{psfrag}
\usepackage{algorithm}
\usepackage{algpseudocode}
\usepackage{eqnarray}
\usepackage{arydshln}
\usepackage{tabularx}
\usepackage{placeins}
\usepackage{tikz}
\usepackage{setspace}
\usepackage{colortbl}
\usepackage{mathtools}
\usepackage{wrapfig}
\usepackage{bm}


% math spaces
\newcommand{\N}{\mathds{N}}                                                 % N, naturals
\newcommand{\Z}{\mathds{Z}}                                                 % Z, integers
\newcommand{\Q}{\mathds{Q}}                                                 % Q, rationals
\newcommand{\R}{\mathds{R}}                                                 % R, reals
\newcommand{\C}{\mathds{C}}                                                 % C, complex
\newcommand{\HS}{\mathcal{H}}                                               % H, hilbertspace
\newcommand{\continuous}{\mathcal{C}}                                       % C, space of continuous functions
\newcommand{\M}{\mathcal{M}} 												% machine numbers
\newcommand{\epsm}{\epsilon_m} 												% maximum error


% basic math stuff
\newcommand{\xt}{\tilde x}													% x tilde
\def\argmax{\mathop{\sf arg\,max}}                                          % argmax
\def\argmin{\mathop{\sf arg\,min}}                                          % argmin
\newcommand{\sign}{\operatorname{sign}}                                     % sign, signum
\newcommand{\I}{\mathbb{I}}                                                 % I, indicator
\newcommand{\order}{\mathcal{O}}                                            % O, order
\newcommand{\fp}[2]{\frac{\partial #1}{\partial #2}}                        % partial derivative
\newcommand{\pd}[2]{\frac{\partial{#1}}{\partial #2}}						% partial derivative

% sums and products
\newcommand{\sumin}{\sum_{i=1}^n}											% summation from i=1 to n
\newcommand{\sumkg}{\sum_{k=1}^g}											% summation from k=1 to g
\newcommand{\prodin}{\prod_{i=1}^n}											% product from i=1 to n
\newcommand{\prodkg}{\prod_{k=1}^g}											% product from k=1 to g

% linear algebra
\newcommand{\one}{\boldsymbol{1}}                                           % 1, unitvector
\newcommand{\id}{\mathrm{I}}                                                % I, identity
\newcommand{\diag}{\operatorname{diag}}                                     % diag, diagonal
\newcommand{\trace}{\operatorname{tr}}                                      % tr, trace
\newcommand{\spn}{\operatorname{span}}                                      % span
\newcommand{\scp}[2]{\left\langle #1, #2 \right\rangle}                     % <.,.>, scalarproduct
\newcommand{\mat}[1]{ 														% short pmatrix command
	\begin{pmatrix}
		#1
	\end{pmatrix}
}
\newcommand{\Amat}{\bm{A}}													% matrix A
\newcommand{\xv}{\bm{x}}													% vector x (bold)
\newcommand{\yv}{\bm{y}}														% vector y (bold)
\newcommand{\Deltab}{\bm{\Delta}}											% error term for vectors
															

% basic probability + stats
\renewcommand{\P}{\mathds{P}}                                               % P, probability
\newcommand{\E}{\mathds{E}}                                                 % E, expectation
\newcommand{\var}{\mathsf{Var}}                                             % Var, variance
\newcommand{\cov}{\mathsf{Cov}}                                             % Cov, covariance
\newcommand{\corr}{\mathsf{Corr}}                                           % Corr, correlation
\newcommand{\normal}{\mathcal{N}}                                           % N of the normal distribution
\newcommand{\iid}{\overset{i.i.d}{\sim}}                                    % dist with i.i.d superscript
\newcommand{\distas}[1]{\overset{#1}{\sim}}                                 % ... is distributed as ... 
% machine learning

%%%%%% ml - data
\newcommand{\Xspace}{\mathcal{X}}                                           % X, input space
\newcommand{\Yspace}{\mathcal{Y}}                                           % Y, output space
\newcommand{\nset}{\{1, \ldots, n\}}                                        % set from 1 to n
\newcommand{\pset}{\{1, \ldots, p\}}                                        % set from 1 to p
\newcommand{\gset}{\{1, \ldots, g\}}                                        % set from 1 to g
\newcommand{\Pxy}{\P_{xy}}                                                  % P_xy
\newcommand{\xy}{(x, y)}                                                    % observation (x, y)
\newcommand{\xvec}{(x_1, \ldots, x_p)^T}                                    % (x1, ..., xp) 
\newcommand{\D}{\mathcal{D}}                                                % D, data 
\newcommand{\Dset}{\{ (x^{(1)}, y^{(1)}), \ldots, (x^{(n)},  y^{(n)})\}}    % {(x1,y1)), ..., (xn,yn)}, data
\newcommand{\xdat}{\{ x^{(1)}, \ldots, x^{(n)}\}}   						 % {x1, ..., xn}, input data
\newcommand{\ydat}{\mathbf{y}}                                              % y (bold), vector of outcomes
\newcommand{\yvec}{(y^{(1)}, \hdots, y^{(n)})^T}                            % (y1, ..., yn), vector of outcomes
\renewcommand{\xi}[1][i]{x^{(#1)}}                                          % x^i, i-th observed value of x
\newcommand{\yi}[1][i]{y^{(#1)}}                                            % y^i, i-th observed value of y 
\newcommand{\xyi}{(\xi, \yi)}                                               % (x^i, y^i), i-th observation
\newcommand{\xivec}{(x^{(i)}_1, \ldots, x^{(i)}_p)^T}                       % (x1^i, ..., xp^i), i-th observation vector
\newcommand{\xj}{x_j}                                                       % x_j, j-th feature
\newcommand{\xjb}{\mathbf{x}_j}                                             % x_j (bold), j-th feature vecor
\newcommand{\xjvec}{(x^{(1)}_j, \ldots, x^{(n)}_j)^T}                       % (x^1_j, ..., x^n_j), j-th feature vector
\newcommand{\Dtrain}{\mathcal{D}_{\text{train}}}                            % D_train, training set
\newcommand{\Dtest}{\mathcal{D}_{\text{test}}}                              % D_test, test set

%%%%%% ml - models general

% continuous prediction function f
\newcommand{\fx}{f(x)}                                                      % f(x), continuous prediction function
\newcommand{\Hspace}{H}														% hypothesis space where f is from
\newcommand{\fh}{\hat{f}}                                                   % f hat, estimated prediction function
\newcommand{\fxh}{\fh(x)}                                                   % fhat(x)
\newcommand{\fxt}{f(x | \theta)}                                            % f(x | theta)
\newcommand{\fxi}{f(\xi)}                                                   % f(x^(i))
\newcommand{\fxih}{\hat{f}(\xi)}                                            % f(x^(i))
\newcommand{\fxit}{f(x^{(i)} | \theta)}                                     % f(x^(i) | theta)
\newcommand{\fhD}{\fh_{\D}}                                                 % fhat_D, estimate of f based on D
\newcommand{\fhDtrain}{\fh_{\Dtrain}}                                       % fhat_Dtrain, estimate of f based on D

% discrete prediction function h
\newcommand{\hx}{h(x)}                                                      % h(x), discrete prediction function
\newcommand{\hh}{\hat{h}}                                                   % h hat
\newcommand{\hxh}{\hat{h}(x)}                                               % hhat(x)
\newcommand{\hxt}{h(x | \theta)}                                            % h(x | theta)
\newcommand{\hxi}{h(\xi)}                                                   % h(x^(i))
\newcommand{\hxit}{h(x^{(i)} | \theta)}                                     % h(x^(i) | theta)

% yhat
\newcommand{\yh}{\hat{y}}                                                   % y hat for prediction of target
\newcommand{\yih}{\hat{y}}                                                  % y hat for prediction of target

% theta
\newcommand{\thetah}{\hat{\theta}}                                          % theta hat

% densities + probabilities
% pdf of x 
\newcommand{\pdf}{p}                                                        % p
\newcommand{\pdfx}{p(x)}                                                    % p(x)
\newcommand{\pixt}{\pi(x | \theta)}                                         % pi(x|theta), pdf of x given theta

% pdf of (x, y)
\newcommand{\pdfxy}{p(x,y)}                                                 % p(x, y)
\newcommand{\pdfxyt}{p(x, y | \theta)}                                      % p(x, y | theta)
\newcommand{\pdfxyit}{p(\xi, \yi | \theta)}                                 % p(x^(i), y^(i) | theta)

% pdf of x given y
\newcommand{\pdfxyk}{p(x | y=k)}                                            % p(x | y = k)
\newcommand{\lpdfxyk}{\log \pdfxyk}                                         % log p(x | y = k)
\newcommand{\pdfxiyk}{p(\xi | y=k)}                                         % p(x^i | y = k)

% prior probabilities
\newcommand{\pik}{\pi_k}                                                    % pi_k, prior
\newcommand{\lpik}{\log \pik}                                               % log pi_k, log of the prior

% posterior probabilities
\newcommand{\post}{\P(y = 1 | x)}                                           % P(y = 1 | x), post. prob for y=1
\newcommand{\pix}{\pi(x)}                                                   % pi(x), P(y = 1 | x)
\newcommand{\postk}{\P(y = k | x)}                                          % P(y = k | y), post. prob for y=k
\newcommand{\pikx}{\pi_k(x)}                                                % pi_k(x), P(y = k | x)
\newcommand{\pikxt}{\pi_k(x | \theta)}                                      % pi_k(x | theta), P(y = k | x, theta)
\newcommand{\pijx}{\pi_j(x)}                                                % pi_j(x), P(y = j | x)
\newcommand{\pdfygxt}{p(y |x, \theta)}                                      % p(y | x, theta)
\newcommand{\pdfyigxit}{p(\yi |\xi, \theta)}                                % p(y^i |x^i, theta)
\newcommand{\lpdfygxt}{\log \pdfygxt }                                      % log p(y | x, theta)
\newcommand{\lpdfyigxit}{\log \pdfyigxit}                                   % log p(y^i |x^i, theta)
\newcommand{\pixh}{\hat \pi(x)}                                             % pi(x) hat, P(y = 1 | x) hat
\newcommand{\pikxh}{\hat \pi_k(x)}                                          % pi_k(x) hat, P(y = k | x) hat

% residual and margin
\newcommand{\eps}{\epsilon}                                                 % residual, stochastic
\newcommand{\epsi}{\epsilon^{(i)}}                                          % epsilon^i, residual, stochastic
\newcommand{\epsh}{\hat{\epsilon}}                                          % residual, estimated
\newcommand{\yf}{y \fx}                                                     % y f(x), margin
\newcommand{\yfi}{\yi \fxi}                                                 % y^i f(x^i), margin
\newcommand{\Sigmah}{\hat \Sigma}											% estimated covariance matrix
\newcommand{\Sigmahj}{\hat \Sigma_j}										% estimated covariance matrix for the j-th class

% ml - loss, risk, likelihood
\newcommand{\Lxy}{L(y, f(x))}                                               % L(y, f(x)), loss function
\newcommand{\Lxyi}{L(\yi, \fxi)}                                            % L(y^i, f(x^i))
\newcommand{\Lxyt}{L(y, \fxt)}                                              % L(y, f(x | theta))
\newcommand{\Lxyit}{L(\yi, \fxit)}                                          % L(y^i, f(x^i | theta)
\newcommand{\risk}{\mathcal{R}}                                             % R, risk
\newcommand{\riskf}{\risk(f)}                                               % R(f), risk
\newcommand{\riske}{\mathcal{R}_{\text{emp}}}                               % R_emp, empirical risk
\newcommand{\riskef}{\riske(f)}                                             % R_emp(f)
\newcommand{\risket}{\mathcal{R}_{\text{emp}}(\theta)}                      % R_emp(theta)
\newcommand{\riskr}{\mathcal{R}_{\text{reg}}}                               % R_reg, regularized risk
\newcommand{\riskrt}{\mathcal{R}_{\text{reg}}(\theta)}                      % R_reg(theta)
\newcommand{\riskrf}{\riskr(f)}                                             % R_reg(f)
\newcommand{\LL}{\mathcal{L}}                                               % L, likelihood
\newcommand{\LLt}{\mathcal{L}(\theta)}                                      % L(theta), likelihood
\renewcommand{\ll}{\ell}                                                    % l, log-likelihood
\newcommand{\llt}{\ell(\theta)}                                             % l(theta), log-likelihood
\newcommand{\LS}{\mathfrak{L}}                                              % ????????????
\newcommand{\TS}{\mathfrak{T}}                                              % ??????????????
\newcommand{\errtrain}{\text{err}_{\text{train}}}                           % training error
\newcommand{\errtest}{\text{err}_{\text{test}}}                             % training error
\newcommand{\errexp}{\overline{\text{err}_{\text{test}}}}                   % training error

% resampling
\newcommand{\GE}[1]{GE(\fh_{#1})}                                           % Generalization error GE
\newcommand{\GEh}[1]{\widehat{GE}_{#1}}                                     % Estimated train error
\newcommand{\GED}{\GE{\D}}                                                  % Generalization error GE
\newcommand{\EGEn}{EGE_n}                                                   % Generalization error GE
\newcommand{\EDn}{\E_{|D| = n}}                                             % Generalization error GE


% ml - irace
\newcommand{\costs}{\mathcal{C}} % costs
\newcommand{\Celite}{\theta^*} % elite configurations
\newcommand{\instances}{\mathcal{I}} % sequence of instances
\newcommand{\budget}{\mathcal{B}} % computational budget
% ml - trees, extra trees

\newcommand{\Np}{\mathcal{N}}												% Parent node N
\newcommand{\Nl}{\Np_1}														% Left node N_1
\newcommand{\Nr}{\Np_2}														% Right node N_2



\title{I2ML :\,: CHEAT SHEET} % Package title in header, \, adds thin space between ::
\newcommand{\packagedescription}{ % Package description in header
	The \textbf{I2ML}: Introduction to Machine Learning course offers an introductory and applied overview of supervised machine learning. It is organized as a digital lecture.
}

\newlength{\columnheight} % Adjust depending on header height
\setlength{\columnheight}{84cm} 

\newtcolorbox{codebox}{%
	sharp corners,
	leftrule=0pt,
	rightrule=0pt,
	toprule=0pt,
	bottomrule=0pt,
	hbox}

\newtcolorbox{codeboxmultiline}[1][]{%
	sharp corners,
	leftrule=0pt,
	rightrule=0pt,
	toprule=0pt,
	bottomrule=0pt,
	#1}

\begin{document}
\begin{frame}[fragile]{}
\begin{columns}
	\begin{column}{.31\textwidth}
		\begin{beamercolorbox}[center]{postercolumn}
			\begin{minipage}{.98\textwidth}
				\parbox[t][\columnheight]{\textwidth}{
					\begin{myblock}{Introduction to CART}

						% \begin{codebox}
							\textbf{CART -- \textbf{C}lassification \textbf{A}nd \textbf{R}egression \textbf{T}rees:}
						% \end{codebox}
						\begin{itemize}[$\bullet$]     
            \setlength{\itemindent}{+.3in}
                        \item Divide feature space into sub-regions.
                        \item Learn best constant prediction from training data for each region.
            \end{itemize}

            \textbf{Advantages:} Automatic feature selection; Fast and scales well for larger data; Less preprocessing; Can model discontinuities and non-linearities.\\
            \textbf{Disadvantages:} Predictions are step functions, not smooth, cannot fit linear trends well; Empirically not the best; High instability (variance). 			
          
          % \begin{codebox}
          \textbf{Binary Trees:}
          % \end{codebox}
          \begin{itemize}[$\bullet$]     
            \setlength{\itemindent}{+.3in}
            \item Represent a top-down hierarchy with binary splits that contains:
                 root node, internal nodes, and terminal nodes (leaves).
            \item Nodes have relative relationships: parent nodes and child nodes. 
            \item Root nodes don't have parents -- leaves don't have children.
          \end{itemize}

      % \begin{codebox}
        \textbf{CARTs:} 
      % \end{codebox}
      \begin{itemize}[$\bullet$]     
        \setlength{\itemindent}{+.3in}
        \item Use the structure of a binary tree
        \item Binary splits are constructed top-down in a \emph{data optimal} way.
        \item Each split is a threshold decision for a single feature.
        \item Each node contains the training points which follow its path.
        \item Each leaf contains a constant prediction.
   \end{itemize}

      % \begin{codebox}
        \textbf{Trees as an additive model:}
      % \end{codebox}
      Divide the feature space $\Xspace$ with $M$ leaf nodes into \textbf{rectangular regions} $Q_m$:
      $$\fx = \sum_{m=1}^M c_m \I(x \in Q_m).$$
      $c_m$: predicted numerical response, class label or class distribution in respective leaf node.
\end{myblock}

\begin{myblock}{Growing a Tree}
  
  \textbf{Greedy optimization}: At each node $\Np$, find the locally optimal split by exhaustively evaluating all possible splits on all thresholds $t$ for all features $x_j$ in terms of empirical risk $\risk(\Np,j,t)$. Training data is then distributed to child nodes according to this split, which is never reconsidered.
  \begin{itemize}[$\bullet$]     
    \setlength{\itemindent}{+.3in}
    \item Start with an empty tree, a root node contains all data.
    \item Search for feature and split point that minimizes the empirical risk in child nodes --  makes label distribution more homogenous.
    \item Proceed recursively for each child node: select best split and divide data from parent node into left and right child nodes.
    \item Repeat until a stop criterion, e.g., until each leaf cannot be split.
    \item Greedy optimization does not guarantee to find the global optimum.
  \end{itemize}

\end{myblock}
				}
			\end{minipage}
		\end{beamercolorbox}
	\end{column}
	
%%%%%%%%%%%%%%%%%%%%%%%%%%%%%%%%%%%%%%%%%%%%%%%%%%%%%%%%%%%%%%%%%%%%%

\begin{column}{.31\textwidth}
\begin{beamercolorbox}[center]{postercolumn}
\begin{minipage}{.98\textwidth}
\parbox[t][\columnheight]{\textwidth}{
  \begin{myblock}{Splitting Criteria}
    Use \textbf{empirical risk minimization:}

    $\Np \subseteq \D$: data contained in this node. $c_{\Np}$: predicted constant for $\Np$.\\
    The risk $\risk(\Np)$ for a node is:
    $\risk(\Np) = \sum\limits_{(\xv, y) \in \Np} L(y, c_{\Np})$.\\
    The optimal constant is:
    $c_{\Np} = \argminlim_c \sum\limits_{(\xv, y) \in \Np} L(y, c)$

    A split w.r.t. feature $x_j$ at split point $t$ divides a parent node $\Np$ into 
    $$\Nl = \{ (\xv, y) \in \Np: x_j < t \} \text{ and } \Nr = \{ (\xv, y) \in \Np: x_j \geq t \}.$$
    Finding the best way to split $\Np$ into $\Nl, \Nr$ means solving
    $$\argmin_{j, t} \risk(\Np, j, t) = \argmin_{j, t} \risk(\Nl) +  \risk(\Nr)$$
    If use averages $\frac{1}{|\Np|}$, we have to reweight the terms to obtain a global average w.r.t. $\Np$ as the children have different sizes
    $$\bar{\risk}(\Np, j, t) = \frac{|\Nl|}{|\Np|} \bar{\risk}(\Nl) + \frac{|\Nr|}{|\Np|} \bar{\risk}(\Nr)$$

    % \begin{codebox}
    \textbf{Splitting criteria:}
    % \end{codebox}
    \begin{itemize}[$\bullet$]
    \setlength{\itemindent}{+.3in}
    \item Regression trees --- $L_2$ loss.
    \item Classification trees --- Brier score.\\
    Minimize Brier score $\iff$ Minimize \textbf{Gini impurity}:
    $$I(\Np) = \sum_{k=1}^g \pikNh(1-\pikNh)$$
    \item Classification trees --- Bernoulli loss.\\
    Minimize Bernoulli loss $\iff$ Minimize \textbf{entropy impurity}:
    $$I(\Np) = -\sum_{k=1}^g \pikNh \log \pikNh$$
    \end{itemize}

    For classification, predicted probabilities in node $\Np$ are the class proportions in the node:
    $$ \pikNh = \frac{1}{|\Np|} \sum\limits_{(x,y) \in \Np} \I(y = k) $$
    Brier score and Bernoulli loss are more sensitive to changes in the node probabilities, and therefore often preferred over misclassification loss.
  \end{myblock}


}
\end{minipage}
\end{beamercolorbox}
\end{column}

%%%%%%%%%%%%%%%%%%%%%%%%%%%%%%%%%%%%%%%%%%%%%%%%%%%%%%%%%%%%%%%%%%%%%%%%%%%%%%%%%%%%%%%%%%%%%%%%%%%%

\begin{column}{.31\textwidth}
\begin{beamercolorbox}[center]{postercolumn}
\begin{minipage}{.98\textwidth}
\parbox[t][\columnheight]{\textwidth}{
  \begin{myblock}{Split computation}
  
    % \begin{codebox}
    \textbf{Monotone feature transformations:}
    % \end{codebox}
    Only change numerical value of the split point, not the distribution into left and right child node.

    % \begin{codebox}
      \textbf{Categorical Features:}	
      % \end{codebox}
      
  % \end{myblock}  
  
  \begin{itemize}[$\bullet$]     
  \setlength{\itemindent}{+.3in}
  \item A split on a categorical feature partitions the feature levels:
    $$x_j \in \{a,c,e\} \leftarrow \Np \rightarrow x_j \in \{b,d\} $$
  \item A feature with $m$ levels results in about $2^m$ different possible binary partitions ($2^{m-1} - 1$ because of symmetry and empty groups).
  \item For L2 regression and binary classification, clever shortcuts are available that treat the feature as ordinal and only require checking $m-1$ splits.
  \end{itemize}

  % % \begin{codebox}
  % \textbf{Continuous responses, in each node: }
  % % \end{codebox}
  
  % \begin{itemize}[$\bullet$]     
  % \setlength{\itemindent}{+.3in}
  % \item Calculate the mean of the outcome in each category.
  % \item Sort the categories by increasing mean of the outcome.
  % \end{itemize}

  % \begin{codebox}	
  \textbf{Missing Feature values: }	
  % \end{codebox}
  Use \textbf{surrogate splits} to define replacement splitting rules with a different feature, 
  that result in almost the same child nodes as the original split.

  \end{myblock}
  
  \begin{myblock}
    {Overfitting}

  \textbf{Reduce overfitting:}
  \begin{itemize}[$\bullet$]     
  \setlength{\itemindent}{+.3in}
  \item Use a less deep tree.
  \item Define different \textbf{stopping criteria}.
  \item \textbf{Pruning}: pre-pruning or post-pruning.
  \end{itemize}
  
  % \begin{codebox}
    \textbf{Cost-complexity pruning (CCP): }
  % \end{codebox}
  Post-pruning method to grow a large tree and remove least informative leaves.
  CCP is steered with regularization parameter $\alpha$ that penalizes number of leaves in a subtree
$$\riskr(T) = \sum_{m=1}^{|T|} \sum_{i: x^{(i)} \in Q_m} L(y^{(i)}, c_m) + \alpha |T|,$$
$|T|$: number of leaves of subtree $T$. $Q_m$: subset of the feature space, $m$-th terminal node. $c_m$: m-th node prediction.

CCP performs a greedy backward search:
\begin{itemize}[$\bullet$]     
  \setlength{\itemindent}{+.3in}
\item Computes $\riskr(T)$ with a fixed $\alpha$ for all possible subtrees that can be created by replacing one internal node with a leaf.
\item By replacing a node we also eliminate all subsequent nodes.
\item Select the subtree with lowest risk and repeat the procedure.
\item Stop if pruning does not further reduce the risk.
\end{itemize}
Hyperparameter $\alpha$ is typically selected via cross-validation.

  \end{myblock}
  }
  
  \end{minipage}
  \end{beamercolorbox}
  \end{column}
  
  
  
\end{columns}
\end{frame}
\end{document}
