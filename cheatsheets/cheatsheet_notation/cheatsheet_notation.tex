\documentclass{beamer}


\usepackage[orientation=landscape,size=a0,scale=1.4,debug]{beamerposter}
\mode<presentation>{\usetheme{mlr}}


\usepackage[utf8]{inputenc} % UTF-8
\usepackage[english]{babel} % Language
\usepackage{hyperref} % Hyperlinks
\usepackage{ragged2e} % Text position
\usepackage[export]{adjustbox} % Image position
\usepackage[most]{tcolorbox}
\usepackage{amsmath}
\usepackage{mathtools}
\usepackage{dsfont}
\usepackage{verbatim}
\usepackage{amsmath}
\usepackage{amsfonts}
\usepackage{csquotes}
\usepackage{multirow}
\usepackage{longtable}
\usepackage{enumerate}
\usepackage[absolute,overlay]{textpos}
\usepackage{psfrag}
\usepackage{algorithm}
\usepackage{algpseudocode}
\usepackage{eqnarray}
\usepackage{arydshln}
\usepackage{tabularx}
\usepackage{placeins}
\usepackage{tikz}
\usepackage{setspace}
\usepackage{colortbl}
\usepackage{mathtools}
\usepackage{wrapfig}
\usepackage{bm}

% math spaces
\newcommand{\N}{\mathds{N}}                                                 % N, naturals
\newcommand{\Z}{\mathds{Z}}                                                 % Z, integers
\newcommand{\Q}{\mathds{Q}}                                                 % Q, rationals
\newcommand{\R}{\mathds{R}}                                                 % R, reals
\newcommand{\C}{\mathds{C}}                                                 % C, complex
\newcommand{\HS}{\mathcal{H}}                                               % H, hilbertspace
\newcommand{\continuous}{\mathcal{C}}                                       % C, space of continuous functions
\newcommand{\M}{\mathcal{M}} 												% machine numbers
\newcommand{\epsm}{\epsilon_m} 												% maximum error


% basic math stuff
\newcommand{\xt}{\tilde x}													% x tilde
\def\argmax{\mathop{\sf arg\,max}}                                          % argmax
\def\argmin{\mathop{\sf arg\,min}}                                          % argmin
\newcommand{\sign}{\operatorname{sign}}                                     % sign, signum
\newcommand{\I}{\mathbb{I}}                                                 % I, indicator
\newcommand{\order}{\mathcal{O}}                                            % O, order
\newcommand{\fp}[2]{\frac{\partial #1}{\partial #2}}                        % partial derivative
\newcommand{\pd}[2]{\frac{\partial{#1}}{\partial #2}}						% partial derivative

% sums and products
\newcommand{\sumin}{\sum_{i=1}^n}											% summation from i=1 to n
\newcommand{\sumkg}{\sum_{k=1}^g}											% summation from k=1 to g
\newcommand{\prodin}{\prod_{i=1}^n}											% product from i=1 to n
\newcommand{\prodkg}{\prod_{k=1}^g}											% product from k=1 to g

% linear algebra
\newcommand{\one}{\boldsymbol{1}}                                           % 1, unitvector
\newcommand{\id}{\mathrm{I}}                                                % I, identity
\newcommand{\diag}{\operatorname{diag}}                                     % diag, diagonal
\newcommand{\trace}{\operatorname{tr}}                                      % tr, trace
\newcommand{\spn}{\operatorname{span}}                                      % span
\newcommand{\scp}[2]{\left\langle #1, #2 \right\rangle}                     % <.,.>, scalarproduct
\newcommand{\mat}[1]{ 														% short pmatrix command
	\begin{pmatrix}
		#1
	\end{pmatrix}
}
\newcommand{\Amat}{\bm{A}}													% matrix A
\newcommand{\xv}{\bm{x}}													% vector x (bold)
\newcommand{\yv}{\bm{y}}														% vector y (bold)
\newcommand{\Deltab}{\bm{\Delta}}											% error term for vectors
															

% basic probability + stats
\renewcommand{\P}{\mathds{P}}                                               % P, probability
\newcommand{\E}{\mathds{E}}                                                 % E, expectation
\newcommand{\var}{\mathsf{Var}}                                             % Var, variance
\newcommand{\cov}{\mathsf{Cov}}                                             % Cov, covariance
\newcommand{\corr}{\mathsf{Corr}}                                           % Corr, correlation
\newcommand{\normal}{\mathcal{N}}                                           % N of the normal distribution
\newcommand{\iid}{\overset{i.i.d}{\sim}}                                    % dist with i.i.d superscript
\newcommand{\distas}[1]{\overset{#1}{\sim}}                                 % ... is distributed as ... 
% machine learning

%%%%%% ml - data
\newcommand{\Xspace}{\mathcal{X}}                                           % X, input space
\newcommand{\Yspace}{\mathcal{Y}}                                           % Y, output space
\newcommand{\nset}{\{1, \ldots, n\}}                                        % set from 1 to n
\newcommand{\pset}{\{1, \ldots, p\}}                                        % set from 1 to p
\newcommand{\gset}{\{1, \ldots, g\}}                                        % set from 1 to g
\newcommand{\Pxy}{\P_{xy}}                                                  % P_xy
\newcommand{\xy}{(x, y)}                                                    % observation (x, y)
\newcommand{\xvec}{(x_1, \ldots, x_p)^T}                                    % (x1, ..., xp) 
\newcommand{\D}{\mathcal{D}}                                                % D, data 
\newcommand{\Dset}{\{ (x^{(1)}, y^{(1)}), \ldots, (x^{(n)},  y^{(n)})\}}    % {(x1,y1)), ..., (xn,yn)}, data
\newcommand{\xdat}{\{ x^{(1)}, \ldots, x^{(n)}\}}   						 % {x1, ..., xn}, input data
\newcommand{\ydat}{\mathbf{y}}                                              % y (bold), vector of outcomes
\newcommand{\yvec}{(y^{(1)}, \hdots, y^{(n)})^T}                            % (y1, ..., yn), vector of outcomes
\renewcommand{\xi}[1][i]{x^{(#1)}}                                          % x^i, i-th observed value of x
\newcommand{\yi}[1][i]{y^{(#1)}}                                            % y^i, i-th observed value of y 
\newcommand{\xyi}{(\xi, \yi)}                                               % (x^i, y^i), i-th observation
\newcommand{\xivec}{(x^{(i)}_1, \ldots, x^{(i)}_p)^T}                       % (x1^i, ..., xp^i), i-th observation vector
\newcommand{\xj}{x_j}                                                       % x_j, j-th feature
\newcommand{\xjb}{\mathbf{x}_j}                                             % x_j (bold), j-th feature vecor
\newcommand{\xjvec}{(x^{(1)}_j, \ldots, x^{(n)}_j)^T}                       % (x^1_j, ..., x^n_j), j-th feature vector
\newcommand{\Dtrain}{\mathcal{D}_{\text{train}}}                            % D_train, training set
\newcommand{\Dtest}{\mathcal{D}_{\text{test}}}                              % D_test, test set

%%%%%% ml - models general

% continuous prediction function f
\newcommand{\fx}{f(x)}                                                      % f(x), continuous prediction function
\newcommand{\Hspace}{H}														% hypothesis space where f is from
\newcommand{\fh}{\hat{f}}                                                   % f hat, estimated prediction function
\newcommand{\fxh}{\fh(x)}                                                   % fhat(x)
\newcommand{\fxt}{f(x | \theta)}                                            % f(x | theta)
\newcommand{\fxi}{f(\xi)}                                                   % f(x^(i))
\newcommand{\fxih}{\hat{f}(\xi)}                                            % f(x^(i))
\newcommand{\fxit}{f(x^{(i)} | \theta)}                                     % f(x^(i) | theta)
\newcommand{\fhD}{\fh_{\D}}                                                 % fhat_D, estimate of f based on D
\newcommand{\fhDtrain}{\fh_{\Dtrain}}                                       % fhat_Dtrain, estimate of f based on D

% discrete prediction function h
\newcommand{\hx}{h(x)}                                                      % h(x), discrete prediction function
\newcommand{\hh}{\hat{h}}                                                   % h hat
\newcommand{\hxh}{\hat{h}(x)}                                               % hhat(x)
\newcommand{\hxt}{h(x | \theta)}                                            % h(x | theta)
\newcommand{\hxi}{h(\xi)}                                                   % h(x^(i))
\newcommand{\hxit}{h(x^{(i)} | \theta)}                                     % h(x^(i) | theta)

% yhat
\newcommand{\yh}{\hat{y}}                                                   % y hat for prediction of target
\newcommand{\yih}{\hat{y}}                                                  % y hat for prediction of target

% theta
\newcommand{\thetah}{\hat{\theta}}                                          % theta hat

% densities + probabilities
% pdf of x 
\newcommand{\pdf}{p}                                                        % p
\newcommand{\pdfx}{p(x)}                                                    % p(x)
\newcommand{\pixt}{\pi(x | \theta)}                                         % pi(x|theta), pdf of x given theta

% pdf of (x, y)
\newcommand{\pdfxy}{p(x,y)}                                                 % p(x, y)
\newcommand{\pdfxyt}{p(x, y | \theta)}                                      % p(x, y | theta)
\newcommand{\pdfxyit}{p(\xi, \yi | \theta)}                                 % p(x^(i), y^(i) | theta)

% pdf of x given y
\newcommand{\pdfxyk}{p(x | y=k)}                                            % p(x | y = k)
\newcommand{\lpdfxyk}{\log \pdfxyk}                                         % log p(x | y = k)
\newcommand{\pdfxiyk}{p(\xi | y=k)}                                         % p(x^i | y = k)

% prior probabilities
\newcommand{\pik}{\pi_k}                                                    % pi_k, prior
\newcommand{\lpik}{\log \pik}                                               % log pi_k, log of the prior

% posterior probabilities
\newcommand{\post}{\P(y = 1 | x)}                                           % P(y = 1 | x), post. prob for y=1
\newcommand{\pix}{\pi(x)}                                                   % pi(x), P(y = 1 | x)
\newcommand{\postk}{\P(y = k | x)}                                          % P(y = k | y), post. prob for y=k
\newcommand{\pikx}{\pi_k(x)}                                                % pi_k(x), P(y = k | x)
\newcommand{\pikxt}{\pi_k(x | \theta)}                                      % pi_k(x | theta), P(y = k | x, theta)
\newcommand{\pijx}{\pi_j(x)}                                                % pi_j(x), P(y = j | x)
\newcommand{\pdfygxt}{p(y |x, \theta)}                                      % p(y | x, theta)
\newcommand{\pdfyigxit}{p(\yi |\xi, \theta)}                                % p(y^i |x^i, theta)
\newcommand{\lpdfygxt}{\log \pdfygxt }                                      % log p(y | x, theta)
\newcommand{\lpdfyigxit}{\log \pdfyigxit}                                   % log p(y^i |x^i, theta)
\newcommand{\pixh}{\hat \pi(x)}                                             % pi(x) hat, P(y = 1 | x) hat
\newcommand{\pikxh}{\hat \pi_k(x)}                                          % pi_k(x) hat, P(y = k | x) hat

% residual and margin
\newcommand{\eps}{\epsilon}                                                 % residual, stochastic
\newcommand{\epsi}{\epsilon^{(i)}}                                          % epsilon^i, residual, stochastic
\newcommand{\epsh}{\hat{\epsilon}}                                          % residual, estimated
\newcommand{\yf}{y \fx}                                                     % y f(x), margin
\newcommand{\yfi}{\yi \fxi}                                                 % y^i f(x^i), margin
\newcommand{\Sigmah}{\hat \Sigma}											% estimated covariance matrix
\newcommand{\Sigmahj}{\hat \Sigma_j}										% estimated covariance matrix for the j-th class

% ml - loss, risk, likelihood
\newcommand{\Lxy}{L(y, f(x))}                                               % L(y, f(x)), loss function
\newcommand{\Lxyi}{L(\yi, \fxi)}                                            % L(y^i, f(x^i))
\newcommand{\Lxyt}{L(y, \fxt)}                                              % L(y, f(x | theta))
\newcommand{\Lxyit}{L(\yi, \fxit)}                                          % L(y^i, f(x^i | theta)
\newcommand{\risk}{\mathcal{R}}                                             % R, risk
\newcommand{\riskf}{\risk(f)}                                               % R(f), risk
\newcommand{\riske}{\mathcal{R}_{\text{emp}}}                               % R_emp, empirical risk
\newcommand{\riskef}{\riske(f)}                                             % R_emp(f)
\newcommand{\risket}{\mathcal{R}_{\text{emp}}(\theta)}                      % R_emp(theta)
\newcommand{\riskr}{\mathcal{R}_{\text{reg}}}                               % R_reg, regularized risk
\newcommand{\riskrt}{\mathcal{R}_{\text{reg}}(\theta)}                      % R_reg(theta)
\newcommand{\riskrf}{\riskr(f)}                                             % R_reg(f)
\newcommand{\LL}{\mathcal{L}}                                               % L, likelihood
\newcommand{\LLt}{\mathcal{L}(\theta)}                                      % L(theta), likelihood
\renewcommand{\ll}{\ell}                                                    % l, log-likelihood
\newcommand{\llt}{\ell(\theta)}                                             % l(theta), log-likelihood
\newcommand{\LS}{\mathfrak{L}}                                              % ????????????
\newcommand{\TS}{\mathfrak{T}}                                              % ??????????????
\newcommand{\errtrain}{\text{err}_{\text{train}}}                           % training error
\newcommand{\errtest}{\text{err}_{\text{test}}}                             % training error
\newcommand{\errexp}{\overline{\text{err}_{\text{test}}}}                   % training error

% resampling
\newcommand{\GE}[1]{GE(\fh_{#1})}                                           % Generalization error GE
\newcommand{\GEh}[1]{\widehat{GE}_{#1}}                                     % Estimated train error
\newcommand{\GED}{\GE{\D}}                                                  % Generalization error GE
\newcommand{\EGEn}{EGE_n}                                                   % Generalization error GE
\newcommand{\EDn}{\E_{|D| = n}}                                             % Generalization error GE


% ml - irace
\newcommand{\costs}{\mathcal{C}} % costs
\newcommand{\Celite}{\theta^*} % elite configurations
\newcommand{\instances}{\mathcal{I}} % sequence of instances
\newcommand{\budget}{\mathcal{B}} % computational budget


\title{I2ML :\,: BASICS} % Package title in header, \, adds thin space between ::
\newcommand{\packagedescription}{ \invisible{x} % Package description in header
	% The \textbf{I2ML}: Introduction to Machine Learning course offers an introductory and applied overview of "supervised" Machine Learning. It is organized as a digital lecture.
}

\newlength{\columnheight} % Adjust depending on header height
\setlength{\columnheight}{84cm} 

\newtcolorbox{codebox}{%
	sharp corners,
	leftrule=0pt,
	rightrule=0pt,
	toprule=0pt,
	bottomrule=0pt,
	hbox}

\newtcolorbox{codeboxmultiline}[1][]{%
	sharp corners,
	leftrule=0pt,
	rightrule=0pt,
	toprule=0pt,
	bottomrule=0pt,
	#1}
	

	
\begin{document}
\begin{frame}[fragile]{}
\vspace{-8ex}
\begin{columns}
	\begin{column}{.31\textwidth}
		\begin{beamercolorbox}[center]{postercolumn}
			\begin{minipage}{.98\textwidth}
				\parbox[t][\columnheight]{\textwidth}{
%%%%%%%%%%%%%%%%%%%%%%%%%%%%%%%%%%%%%%%%%%%%%%%%%%%%%%%%%%%%%%%%%%%%%%%%%%%%%%%%
% First Column begin
%-------------------------------------------------------------------------------
% Data
%-------------------------------------------------------------------------------
\begin{myblock}{Data}
 $\Xspace \subset \R^p$ : $p$-dimensional \textbf{feature / input space}\\ 
Usually we assume $\Xspace \equiv \R^p$, but sometimes, dimensions may be \\  
bounded (e.g., for categorical or non-negative features.)    \\

$\Yspace \subset \R^g$ : \textbf{target space} \\ 
e.g.: $\Yspace = \R$, $\Yspace = \setzo$, $\Yspace = \setmp$, $\Yspace = \gset$ with $g$ classes\\

$\xv = \xvec \in \Xspace$ : \textbf{feature vector} \\ 
 
$y \in \Yspace$ : \textbf{target / label / output} \\
 
$\allDatasetsn = \defAllDatasetsn \subset \allDatasets$ : \textbf{set of all finite data sets of size $n$} \\

$\allDatasets = \defAllDatasets$ : \textbf{set of all finite data sets} \\
 
$\D = \Dset \in \allDatasetsn $ : \textbf{data set} with $n$ observations \\
 
$\Dtrain$, $\Dtest \subset \D$ : \textbf{data for training and testing} \\ 
(often: $\D = \Dtrain ~ \dot{\cup} ~ \Dtest$)\\
 
$\xyi \in \Xspace\times \Yspace$ : $i$ -th \textbf{observation} or \textbf{instance} \\
 
$\P_{xy}$ : \textbf{joint probability distribution on} $\Xspace \times \Yspace$ \\

$\pi_k = \P(y = k)$:\textbf{ prior probability} for class $k$ \\
In case of binary labels we might abbreviate: $\pi = \P(y = 1)$. \\
  
% $\pdfxy: \Xspace \times \Yspace \rightarrow [0, \infty)$ : \textbf{joint probability density function (pdf)}, often parametrized by $\thetab \in \Theta$ \textcolor{blue}{(then: $\pdfxyt$)} \\
\end{myblock}
%-------------------------------------------------------------------------------
% Model and Learner
%-------------------------------------------------------------------------------
\begin{myblock}{Model and Learner}
\textbf{Model / \textbf{hypothesis}: }$f : \Xspace \rightarrow \R^g, \quad \xv \mapsto \fx$ is a function that maps feature vectors to predictions, often parametrized by $\thetab \in \Theta$ \\ (then we write $f_{\thetab}$, or, equivalently, $\fxt$). \\

% $\fx$ or $\fxt \in \R$ or $\R^g$ : prediction function (\textbf{model}) \\ %learned from data
% \hspace*{1ex}We might suppress $\thetab$ in notation. \\

$\Theta \subset \R^d$ : \textbf{parameter space} \\
  
$\thetab = (\theta_1, \theta_2, ..., \theta_d) \in \Theta$: model \textbf{parameters}\\
Some models may traditionally use different symbols. \\

$\Hspace = \{f : \Xspace \rightarrow \R^g ~|~ f \text{ belongs to a certain functional family}\}$ : \\ \textbf{hypothesis space} -- set of functions defining a specific model class to which we restrict our \\
learning task 					
\end{myblock}\vfill
% End First Column
%%%%%%%%%%%%%%%%%%%%%%%%%%%%%%%%%%%%%%%%%%%%%%%%%%%%%%%%%%%%%%%%%%%%%%%%%%%%%%%%
				}
			\end{minipage}
		\end{beamercolorbox}
	\end{column}
	\begin{column}{.31\textwidth}
		\begin{beamercolorbox}[center]{postercolumn}
			\begin{minipage}{.98\textwidth}
				\parbox[t][\columnheight]{\textwidth}{
%%%%%%%%%%%%%%%%%%%%%%%%%%%%%%%%%%%%%%%%%%%%%%%%%%%%%%%%%%%%%%%%%%%%%%%%%%%%%%%%
% Begin Second Column
\begin{myblock}{} \vspace{-4ex}
\textbf{Learner }$\inducer: \preimageInducerShort \rightarrow \Hspace$  takes a training set  $\Dtrain \in \allDatasets$  and produces a model $f : \Xspace \rightarrow \R^g$, its hyperparameters set to $\bm{\lambda} \in \bm{\Lambda}$.\\
For a parametrized model this can be adapted to $\inducer: \preimageInducerShort \rightarrow \Theta$ \\

$\bm{\Lambda} = \bm{\Lambda_1} \times \bm{\Lambda_2} \times ... \times \bm{\Lambda_{\ll}} \subset \R^{\ll}$, where $\bm{\Lambda_j} = (a_j, b_j), \quad a_j, b_j \in \R, \quad j = 1, 2, ... , \ll$: \textbf{hyperparameter space} \\

$\bm{\lambda} = (\lambda_1, \lambda_2, ..., \lambda_{\ll}) \in \bm{\Lambda}$ : \textbf{model hyperparameters} \\

$\pikx = \postk \in [0, 1]$: \textbf{posterior probability} for class $k$, \\ given $\xv$ \\
In case of binary labels we might abbreviate: $\pix = \post$. \\

$\hx: \R^g \rightarrow \Yspace$ : \textbf{prediction function} for classification that maps class scores / posterior probabilities to discrete classes \\

% $\LLt$ and $\llt = \log\LLt$ : \textbf{likelihood} and \textbf{log-likelihood} for \\ parameter $\thetab$ \\
% These are based on a statistical model.\\

$\eps = y - \fx$ or $\epsi = \yi - \fxi$ : ($i$-th) \textbf{residual} in regression\\

$\yf$ or $\yfi$ : \textbf{margin} for ($i$-th) observation in binary \\ classification (with $\Yspace = \{-1, 1\}$). \\
%????????????????????????????????

$\yh$, $\fh$, $\hh$, $\pikxh$, $\pixh$ and $\thetah$ \\
The hat symbol denotes \textbf{learned} functions and parameters.
\end{myblock}
%-------------------------------------------------------------------------------
% Loss and Risk 
%-------------------------------------------------------------------------------
\begin{myblock}{Loss and Risk}
  $L: \Yspace \times \R^g \to \R$ : \textbf{loss function} \\
 $\Lxy$ quantifies the "quality" of the prediction $\fx$ for a single \\
 observation $\xv$.  \\
  
  $\riske:  \Hspace \to \R $ : \textbf{empirical risk } \\
The ability of a model $f$ to reproduce the association between $\xv$ and $y$ \\
that is present in the data $\D$ can be measured by the summed loss:
  
  $$\riskef = \sumin \Lxyi$$ 
  
Learning then amounts to \textbf{empirical risk minimization} -- figuring out which model $\fh$ has the smallest summed loss. \\ 
Since $f$ is usually defined by \textbf{parameters} $\thetab$, this becomes:

$$\fh = \argmin_{\thetab \in \Theta} \risket) = \argmin_{\thetab \in \Theta} \sumin \Lxyit,$$
where $\riske : \Theta \to \R.$

  % $$\riske : \Theta \to \R$$
  % $$\risket  =  \sumin \Lxyit $$
  % Learning then amounts to \textbf{empirical risk minimization} -- figuring out which model $\fh$ has the smallest summed loss:$$\fh = \argmin_{\thetab \in \Theta} \risket).$$
\end{myblock}
% End Second Column					
%%%%%%%%%%%%%%%%%%%%%%%%%%%%%%%%%%%%%%%%%%%%%%%%%%%%%%%%%%%%%%%%%%%%%%%%%%%%%%%%
				}
			\end{minipage}
		\end{beamercolorbox}
	\end{column}
	\begin{column}{.31\textwidth}
		\begin{beamercolorbox}[center]{postercolumn}
			\begin{minipage}{.98\textwidth}
				\parbox[t][\columnheight]{\textwidth}{
%%%%%%%%%%%%%%%%%%%%%%%%%%%%%%%%%%%%%%%%%%%%%%%%%%%%%%%%%%%%%%%%%%%%%%%%%%%%%%%%
% Begin Third Column#
%-------------------------------------------------------------------------------
% HRO - Components of Learning 
%-------------------------------------------------------------------------------          
\begin{myblock}{Components of Learning}

\textbf{Learning = Hypothesis space + Risk + Optimization} \\
\phantom{\textbf{Learning}} \textbf{= }$ \Hspace + \risket + \argmin_{\thetab \in \Theta} 
\risket$

% 
% \textbf{Learning &= Hypothesis space &+ Risk  &+ Optimization} \\
% &= $\Hspace &+ \risket &+ \argmin_{\thetab \in \Theta} \risket$
% 
% \textbf{Hypothesis space: } Defines (and restricts!) what kind of model $f$
% can be learned from the data.
% 
% Examples: linear functions, decision trees
% 
% \vspace*{0.5ex}
% 
% \textbf{Risk: } Quantifies how well a model performs on a given
% data set. This allows us to rank candidate models in order to choose the best one.
% 
% Examples: squared error, negative (log-)likelihood
% 
% \vspace*{0.5ex}
% 
% \textbf{Optimization: } Defines how to search for the best model, i.e., the model with the smallest {risk}, in the hypothesis space.
% 
% Examples: gradient descent, quadratic programming


\end{myblock}

%-------------------------------------------------------------------------------
% Regression Losses 
%------------------------------------------------------------------------------- 
\begin{myblock}{Regression Losses}
  \textbf{Basic idea (L2 loss / squared error):} 
\begin{itemize}    
  \setlength{\itemindent}{+.3in}
  \item $\Lxy = (y-\fx)^2$ or $\Lxy = 0.5 (y-\fx)^2$
  \item Convex and differentiable
  \item Tries to reduce large residuals (loss scaling quadratically)
  \item $\fxh = \text{mean of } y | \bm{x}$
\end{itemize}

\vspace*{1ex}
%        \includegraphics[width=1\columnwidth]{img/reg_loss.PNG}


  \textbf{Basic idea (L1 loss / absolute error):} 
\begin{itemize}
\setlength{\itemindent}{+.3in}
  \item $\Lxy = |y-\fx|$
  \item Convex and more robust
  \item Non-differentiable for $y = \fx$, optimization becomes harder
  \item $\fxh = \text{median of } y | \bm{x}$     
\end{itemize}
  %\includegraphics[width=1.03\columnwidth]{img/reg_loss_2.PNG} 
\end{myblock}

%-------------------------------------------------------------------------------
% Classification Losses 
%------------------------------------------------------------------------------- 
\begin{myblock}{Classification Losses}
  tbd
\end{myblock}
%-------------------------------------------------------------------------------
% Classification 
%------------------------------------------------------------------------------- 

\begin{myblock}{Classification}
% 				    We want to assign new observations to known categories according to criteria learned from a training set.  
%             \vspace*{1ex}
%             
Assume we are given a \textbf{classification problem:}
\begin{eqnarray*} & \xv \in \Xspace \quad & \text{feature vector}\\ & y \in \Yspace = \gset \quad & \text{\emph{categorical} output variable (label)}\\ &\D = \Dset & \text{observations of $\xv$ and $y$} \end{eqnarray*}
\vspace*{1ex}
Classification usually means to construct $g$ discriminant functions:
  
$f_1(\xv), \ldots, \fgx$, so that we choose our class as \\ 
 $h(\xv) = \argmax_{k \in \gset} \fkx$ 
  
  \vspace*{2ex}


\textbf{Linear Classifier:} If the functions $\fkx$ can be specified as linear functions, we will call 
the classifier a \emph{linear classifier}.\\

% \hspace*{1ex}\textbf{Note: }All linear classifiers can represent non-linear decision boundaries \hspace*{1ex}in our original input space if we include derived features. For example: \hspace*{1ex}higher order interactions, polynomials or other transformations of x in \hspace*{1ex}the model.

\textbf{Binary classification: }If only 2 classes exist, we can use a single discriminant function $\fx = f_{1}(\xv) - f_{2}(\xv)$.  

\end{myblock}
% End Third Column
%%%%%%%%%%%%%%%%%%%%%%%%%%%%%%%%%%%%%%%%%%%%%%%%%%%%%%%%%%%%%%%%%%%%%%%%%%%%%%%%
			  }
			\end{minipage}
		\end{beamercolorbox}
	\end{column}
\end{columns}

\end{frame}
\end{document}
