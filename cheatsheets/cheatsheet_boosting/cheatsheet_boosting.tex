\documentclass{beamer}
\newcommand \beameritemnestingprefix{}


\usepackage[orientation=landscape,size=a0,scale=1.4,debug]{beamerposter}
\mode<presentation>{\usetheme{mlr}}


\usepackage[utf8]{inputenc} % UTF-8
\usepackage[english]{babel} % Language
\usepackage{hyperref} % Hyperlinks
\usepackage{ragged2e} % Text position
\usepackage[export]{adjustbox} % Image position
\usepackage[most]{tcolorbox}
\usepackage{amsmath}
\usepackage{mathtools}
\usepackage{dsfont}
\usepackage{verbatim}
\usepackage{amsmath}
\usepackage{amsfonts}
\usepackage{csquotes}
\usepackage{multirow}
\usepackage{longtable}
\usepackage[absolute,overlay]{textpos}
\usepackage{psfrag}
\usepackage{algorithm}
\usepackage{algpseudocode}
\usepackage{eqnarray}
\usepackage{arydshln}
\usepackage{tabularx}
\usepackage{placeins}
\usepackage{tikz}
\usepackage{setspace}
\usepackage{colortbl}
\usepackage{mathtools}
\usepackage{wrapfig}
\usepackage{bm}


% math spaces
\newcommand{\N}{\mathds{N}}                                                 % N, naturals
\newcommand{\Z}{\mathds{Z}}                                                 % Z, integers
\newcommand{\Q}{\mathds{Q}}                                                 % Q, rationals
\newcommand{\R}{\mathds{R}}                                                 % R, reals
\newcommand{\C}{\mathds{C}}                                                 % C, complex
\newcommand{\HS}{\mathcal{H}}                                               % H, hilbertspace
\newcommand{\continuous}{\mathcal{C}}                                       % C, space of continuous functions
\newcommand{\M}{\mathcal{M}} 												% machine numbers
\newcommand{\epsm}{\epsilon_m} 												% maximum error


% basic math stuff
\newcommand{\xt}{\tilde x}													% x tilde
\def\argmax{\mathop{\sf arg\,max}}                                          % argmax
\def\argmin{\mathop{\sf arg\,min}}                                          % argmin
\newcommand{\sign}{\operatorname{sign}}                                     % sign, signum
\newcommand{\I}{\mathbb{I}}                                                 % I, indicator
\newcommand{\order}{\mathcal{O}}                                            % O, order
\newcommand{\fp}[2]{\frac{\partial #1}{\partial #2}}                        % partial derivative
\newcommand{\pd}[2]{\frac{\partial{#1}}{\partial #2}}						% partial derivative

% sums and products
\newcommand{\sumin}{\sum_{i=1}^n}											% summation from i=1 to n
\newcommand{\sumkg}{\sum_{k=1}^g}											% summation from k=1 to g
\newcommand{\prodin}{\prod_{i=1}^n}											% product from i=1 to n
\newcommand{\prodkg}{\prod_{k=1}^g}											% product from k=1 to g

% linear algebra
\newcommand{\one}{\boldsymbol{1}}                                           % 1, unitvector
\newcommand{\id}{\mathrm{I}}                                                % I, identity
\newcommand{\diag}{\operatorname{diag}}                                     % diag, diagonal
\newcommand{\trace}{\operatorname{tr}}                                      % tr, trace
\newcommand{\spn}{\operatorname{span}}                                      % span
\newcommand{\scp}[2]{\left\langle #1, #2 \right\rangle}                     % <.,.>, scalarproduct
\newcommand{\mat}[1]{ 														% short pmatrix command
	\begin{pmatrix}
		#1
	\end{pmatrix}
}
\newcommand{\Amat}{\bm{A}}													% matrix A
\newcommand{\xv}{\bm{x}}													% vector x (bold)
\newcommand{\yv}{\bm{y}}														% vector y (bold)
\newcommand{\Deltab}{\bm{\Delta}}											% error term for vectors
															

% basic probability + stats
\renewcommand{\P}{\mathds{P}}                                               % P, probability
\newcommand{\E}{\mathds{E}}                                                 % E, expectation
\newcommand{\var}{\mathsf{Var}}                                             % Var, variance
\newcommand{\cov}{\mathsf{Cov}}                                             % Cov, covariance
\newcommand{\corr}{\mathsf{Corr}}                                           % Corr, correlation
\newcommand{\normal}{\mathcal{N}}                                           % N of the normal distribution
\newcommand{\iid}{\overset{i.i.d}{\sim}}                                    % dist with i.i.d superscript
\newcommand{\distas}[1]{\overset{#1}{\sim}}                                 % ... is distributed as ... 
% machine learning

%%%%%% ml - data
\newcommand{\Xspace}{\mathcal{X}}                                           % X, input space
\newcommand{\Yspace}{\mathcal{Y}}                                           % Y, output space
\newcommand{\nset}{\{1, \ldots, n\}}                                        % set from 1 to n
\newcommand{\pset}{\{1, \ldots, p\}}                                        % set from 1 to p
\newcommand{\gset}{\{1, \ldots, g\}}                                        % set from 1 to g
\newcommand{\Pxy}{\P_{xy}}                                                  % P_xy
\newcommand{\xy}{(x, y)}                                                    % observation (x, y)
\newcommand{\xvec}{(x_1, \ldots, x_p)^T}                                    % (x1, ..., xp) 
\newcommand{\D}{\mathcal{D}}                                                % D, data 
\newcommand{\Dset}{\{ (x^{(1)}, y^{(1)}), \ldots, (x^{(n)},  y^{(n)})\}}    % {(x1,y1)), ..., (xn,yn)}, data
\newcommand{\xdat}{\{ x^{(1)}, \ldots, x^{(n)}\}}   						 % {x1, ..., xn}, input data
\newcommand{\ydat}{\mathbf{y}}                                              % y (bold), vector of outcomes
\newcommand{\yvec}{(y^{(1)}, \hdots, y^{(n)})^T}                            % (y1, ..., yn), vector of outcomes
\renewcommand{\xi}[1][i]{x^{(#1)}}                                          % x^i, i-th observed value of x
\newcommand{\yi}[1][i]{y^{(#1)}}                                            % y^i, i-th observed value of y 
\newcommand{\xyi}{(\xi, \yi)}                                               % (x^i, y^i), i-th observation
\newcommand{\xivec}{(x^{(i)}_1, \ldots, x^{(i)}_p)^T}                       % (x1^i, ..., xp^i), i-th observation vector
\newcommand{\xj}{x_j}                                                       % x_j, j-th feature
\newcommand{\xjb}{\mathbf{x}_j}                                             % x_j (bold), j-th feature vecor
\newcommand{\xjvec}{(x^{(1)}_j, \ldots, x^{(n)}_j)^T}                       % (x^1_j, ..., x^n_j), j-th feature vector
\newcommand{\Dtrain}{\mathcal{D}_{\text{train}}}                            % D_train, training set
\newcommand{\Dtest}{\mathcal{D}_{\text{test}}}                              % D_test, test set

%%%%%% ml - models general

% continuous prediction function f
\newcommand{\fx}{f(x)}                                                      % f(x), continuous prediction function
\newcommand{\Hspace}{H}														% hypothesis space where f is from
\newcommand{\fh}{\hat{f}}                                                   % f hat, estimated prediction function
\newcommand{\fxh}{\fh(x)}                                                   % fhat(x)
\newcommand{\fxt}{f(x | \theta)}                                            % f(x | theta)
\newcommand{\fxi}{f(\xi)}                                                   % f(x^(i))
\newcommand{\fxih}{\hat{f}(\xi)}                                            % f(x^(i))
\newcommand{\fxit}{f(x^{(i)} | \theta)}                                     % f(x^(i) | theta)
\newcommand{\fhD}{\fh_{\D}}                                                 % fhat_D, estimate of f based on D
\newcommand{\fhDtrain}{\fh_{\Dtrain}}                                       % fhat_Dtrain, estimate of f based on D

% discrete prediction function h
\newcommand{\hx}{h(x)}                                                      % h(x), discrete prediction function
\newcommand{\hh}{\hat{h}}                                                   % h hat
\newcommand{\hxh}{\hat{h}(x)}                                               % hhat(x)
\newcommand{\hxt}{h(x | \theta)}                                            % h(x | theta)
\newcommand{\hxi}{h(\xi)}                                                   % h(x^(i))
\newcommand{\hxit}{h(x^{(i)} | \theta)}                                     % h(x^(i) | theta)

% yhat
\newcommand{\yh}{\hat{y}}                                                   % y hat for prediction of target
\newcommand{\yih}{\hat{y}}                                                  % y hat for prediction of target

% theta
\newcommand{\thetah}{\hat{\theta}}                                          % theta hat

% densities + probabilities
% pdf of x 
\newcommand{\pdf}{p}                                                        % p
\newcommand{\pdfx}{p(x)}                                                    % p(x)
\newcommand{\pixt}{\pi(x | \theta)}                                         % pi(x|theta), pdf of x given theta

% pdf of (x, y)
\newcommand{\pdfxy}{p(x,y)}                                                 % p(x, y)
\newcommand{\pdfxyt}{p(x, y | \theta)}                                      % p(x, y | theta)
\newcommand{\pdfxyit}{p(\xi, \yi | \theta)}                                 % p(x^(i), y^(i) | theta)

% pdf of x given y
\newcommand{\pdfxyk}{p(x | y=k)}                                            % p(x | y = k)
\newcommand{\lpdfxyk}{\log \pdfxyk}                                         % log p(x | y = k)
\newcommand{\pdfxiyk}{p(\xi | y=k)}                                         % p(x^i | y = k)

% prior probabilities
\newcommand{\pik}{\pi_k}                                                    % pi_k, prior
\newcommand{\lpik}{\log \pik}                                               % log pi_k, log of the prior

% posterior probabilities
\newcommand{\post}{\P(y = 1 | x)}                                           % P(y = 1 | x), post. prob for y=1
\newcommand{\pix}{\pi(x)}                                                   % pi(x), P(y = 1 | x)
\newcommand{\postk}{\P(y = k | x)}                                          % P(y = k | y), post. prob for y=k
\newcommand{\pikx}{\pi_k(x)}                                                % pi_k(x), P(y = k | x)
\newcommand{\pikxt}{\pi_k(x | \theta)}                                      % pi_k(x | theta), P(y = k | x, theta)
\newcommand{\pijx}{\pi_j(x)}                                                % pi_j(x), P(y = j | x)
\newcommand{\pdfygxt}{p(y |x, \theta)}                                      % p(y | x, theta)
\newcommand{\pdfyigxit}{p(\yi |\xi, \theta)}                                % p(y^i |x^i, theta)
\newcommand{\lpdfygxt}{\log \pdfygxt }                                      % log p(y | x, theta)
\newcommand{\lpdfyigxit}{\log \pdfyigxit}                                   % log p(y^i |x^i, theta)
\newcommand{\pixh}{\hat \pi(x)}                                             % pi(x) hat, P(y = 1 | x) hat
\newcommand{\pikxh}{\hat \pi_k(x)}                                          % pi_k(x) hat, P(y = k | x) hat

% residual and margin
\newcommand{\eps}{\epsilon}                                                 % residual, stochastic
\newcommand{\epsi}{\epsilon^{(i)}}                                          % epsilon^i, residual, stochastic
\newcommand{\epsh}{\hat{\epsilon}}                                          % residual, estimated
\newcommand{\yf}{y \fx}                                                     % y f(x), margin
\newcommand{\yfi}{\yi \fxi}                                                 % y^i f(x^i), margin
\newcommand{\Sigmah}{\hat \Sigma}											% estimated covariance matrix
\newcommand{\Sigmahj}{\hat \Sigma_j}										% estimated covariance matrix for the j-th class

% ml - loss, risk, likelihood
\newcommand{\Lxy}{L(y, f(x))}                                               % L(y, f(x)), loss function
\newcommand{\Lxyi}{L(\yi, \fxi)}                                            % L(y^i, f(x^i))
\newcommand{\Lxyt}{L(y, \fxt)}                                              % L(y, f(x | theta))
\newcommand{\Lxyit}{L(\yi, \fxit)}                                          % L(y^i, f(x^i | theta)
\newcommand{\risk}{\mathcal{R}}                                             % R, risk
\newcommand{\riskf}{\risk(f)}                                               % R(f), risk
\newcommand{\riske}{\mathcal{R}_{\text{emp}}}                               % R_emp, empirical risk
\newcommand{\riskef}{\riske(f)}                                             % R_emp(f)
\newcommand{\risket}{\mathcal{R}_{\text{emp}}(\theta)}                      % R_emp(theta)
\newcommand{\riskr}{\mathcal{R}_{\text{reg}}}                               % R_reg, regularized risk
\newcommand{\riskrt}{\mathcal{R}_{\text{reg}}(\theta)}                      % R_reg(theta)
\newcommand{\riskrf}{\riskr(f)}                                             % R_reg(f)
\newcommand{\LL}{\mathcal{L}}                                               % L, likelihood
\newcommand{\LLt}{\mathcal{L}(\theta)}                                      % L(theta), likelihood
\renewcommand{\ll}{\ell}                                                    % l, log-likelihood
\newcommand{\llt}{\ell(\theta)}                                             % l(theta), log-likelihood
\newcommand{\LS}{\mathfrak{L}}                                              % ????????????
\newcommand{\TS}{\mathfrak{T}}                                              % ??????????????
\newcommand{\errtrain}{\text{err}_{\text{train}}}                           % training error
\newcommand{\errtest}{\text{err}_{\text{test}}}                             % training error
\newcommand{\errexp}{\overline{\text{err}_{\text{test}}}}                   % training error

% resampling
\newcommand{\GE}[1]{GE(\fh_{#1})}                                           % Generalization error GE
\newcommand{\GEh}[1]{\widehat{GE}_{#1}}                                     % Estimated train error
\newcommand{\GED}{\GE{\D}}                                                  % Generalization error GE
\newcommand{\EGEn}{EGE_n}                                                   % Generalization error GE
\newcommand{\EDn}{\E_{|D| = n}}                                             % Generalization error GE


% ml - irace
\newcommand{\costs}{\mathcal{C}} % costs
\newcommand{\Celite}{\theta^*} % elite configurations
\newcommand{\instances}{\mathcal{I}} % sequence of instances
\newcommand{\budget}{\mathcal{B}} % computational budget
% ml - bagging, random forest

\newcommand{\bl}[1]{b^{[#1]}(x)}											% baselearner with argument for m
\newcommand{\blm}{\bl{m}}												    % baselearner without argument for m
\newcommand{\blmh}{\hat{b}^{[m]}(x)}										% estimated base learner 

\input{../../latex-math/ml-boosting.tex}
% ml - trees, extra trees

\newcommand{\Np}{\mathcal{N}}												% Parent node N
\newcommand{\Nl}{\Np_1}														% Left node N_1
\newcommand{\Nr}{\Np_2}														% Right node N_2




\title{I2ML :\,: CHEAT SHEET} % Package title in header, \, adds thin space between ::
\newcommand{\packagedescription}{ % Package description in header
	The \textbf{I2ML}: Introduction to Machine Learning course offers an introductory and applied overview of "supervised" Machine Learning. It is organized as a digital lecture.
}

\newlength{\columnheight} % Adjust depending on header height
\setlength{\columnheight}{84cm}

\newtcolorbox{codebox}{%
	sharp corners,
	leftrule=0pt,
	rightrule=0pt,
	toprule=0pt,
	bottomrule=0pt,
	hbox}

\newtcolorbox{codeboxmultiline}[1][]{%
	sharp corners,
	leftrule=0pt,
	rightrule=0pt,
	toprule=0pt,
	bottomrule=0pt,
	#1}

\begin{document}
\begin{frame}[fragile]{}
\begin{columns}
	\begin{column}{.31\textwidth}
		\begin{beamercolorbox}[center]{postercolumn}
			\begin{minipage}{.98\textwidth}
				\parbox[t][\columnheight]{\textwidth}{
					\begin{myblock}{Forward Stagewise Additive Modelling}
						\begin{codebox}
			        \textbf{Basic Idea:}
						\end{codebox}
						  \begin{itemize}[$\bullet$]
                \setlength{\itemindent}{+.3in}
                \item
                  Gradient boosting uses the idea of stagewise additive modelling
                \item
                  We want to learn an additive model:
                  $$
                  \fx = \sum_{m=1}^M \betam b(\xv, \thetam).
                  $$
                \item
                  Hence, we minimize the empirical risk:
                  $$
                    \riskef = \sum_{i=1}^n L\left(\yi,\fxi \right)
                            = \sum_{i=1}^n L\left(\yi, \sum_{m=1}^M \betam b\left(\xi, \thetam\right)\right)
                  $$
                \item
                  Because of the additive structure it is difficult to jointly minimize $\riskef$ w.r.t.
                  $\left(\left(\beta^{[1]}, \theta^{[1]}\right), \ldots, \left(\beta^{[M]}, \theta^{[M]}\right)\right)$
                \item
                  Hence, we add additive components in a greedy fashion by sequentially minimizing
                  the risk only w.r.t. the next additive component:
                  $$
                  \min \limits_{\beta, \theta} \sum_{i=1}^n L\left(\yi, \fmdh\left(\xi\right) +
                    \beta b\left(\xi, \theta\right)\right)
                  $$
              \end{itemize}

						  \begin{codebox}
                  \textbf{Boosting vs. Bagging: }
              \end{codebox}
						  In contrast to bagging, boosting fits a model sequentially where each component
              builds on the ones before.
              \begin{center}
                \includegraphics[width=0.5\columnwidth]{img/bagging_vs_boosting.png}
              \end{center}
              \begin{codebox}
                \textbf{Note: }
              \end{codebox}
              Forward stagewise additive modelling is not really an algorithm, but rather an
              abstract principle. We need to find the new additive component
              $b\left(\xv, \thetam\right)$ and its weight coefficient $\betam$ in each iteration
              $m$. This can be done by gradient descent, but in function space.
            \end{myblock}\vfill
				} % parbox end
			\end{minipage}
		\end{beamercolorbox}
	\end{column}

%%%%%%%%%%%%%%%%%%%%%%%%%%%%%%%%%%%%%%%%%%%%%%%%%%%%%%%%%%%%%%%%%%%%%%%%%%%%%%%%%%%%%%%%%%%

\begin{column}{.31\textwidth}
  \begin{beamercolorbox}[center]{postercolumn}
    \begin{minipage}{.98\textwidth}
      \parbox[t][\columnheight]{\textwidth}{
        \begin{myblock}{Gradient Boosting}
          Modification of bagging for trees proposed by Breiman (2001):
          \begin{itemize}[$\bullet$]
            \setlength{\itemindent}{+.3in}
            \item
              Pseudo residuals $\rmi$ tells us how we could \enquote{nudge} our whole
              function $f$ in the direction of the data reduce its empirical risk:
              $$
              \rmi = -\left[\fp{\Lxyi}{f(\xi)}\right]_{f=\fmd}
              $$
            \item
              The additive component (simple model) $b\left(\xv, \thetam\right) \in \mathcal{B}$ is
              fit to the pseudo residuals: $\thetamh = \argmin_{\theta} \sum_{i=1}^n \left(\rmi - b(\xi, \theta)\right)^2$
          \end{itemize}
          \begin{algorithm}[H]
  \begin{footnotesize}
  \begin{center}
  \caption{Gradient Boosting Algorithm.}
    \begin{algorithmic}[1]
      \State Initialize $\hat{f}^{[0]}(\xv) = \argmin_{\theta} \sumin L(\yi, b(\xi, \theta))$
      \For{$m = 1 \to M$}
          \State For all $i$: $\rmi = -\left[\fp{\Lxyi}{\fxi}\right]_{f=\fmdh}$
        \State Fit a regression base learner to the pseudo-residuals $\rmi$:
        \State $\thetamh = \argmin \limits_\theta \sumin (\rmi - b(\xi, \theta))^2$
        \State Line search: $\betamh = \argmin_{\beta} \sumin L(\yi, \fmd(\xv) + \beta b(\xv, \thetamh))$
        \State Update $\fmh(\xv) = \fmdh(\xv) + \betamh b(\xv, \thetamh)$
      \EndFor
      \State Output $\fh(\xv) = \hat{f}^{[M]}(\xv)$
    \end{algorithmic}
    \end{center}
    \end{footnotesize}
\end{algorithm}


          \begin{codebox}
            \textbf{In a nutshell}
          \end{codebox}
            One boosting iteration is exactly one approximated gradient step in
            function space, which minimizes the empirical risk as much as possible.

          \begin{codebox}
            \textbf{Gradient boosting with trees}
          \end{codebox}
          \begin{itemize}[$\bullet$]
            \setlength{\itemindent}{+.3in}
            \item
              Trees can be written as: $ b(\xv) = \sum_{t=1}^{T} c_t \mathds{1}_{\{\xv \in R_t\}} $
            \item
              Instead of finding $\ctm$ and $\betam$ in two separate steps we do:
              $$
              \fm(\xv) = \fmd(\xv) +  \sum_{t=1}^{\Tm} \ctmt \mathds{1}_{\{\xv \in \Rtm\}}.
              $$
            \item
              Constants $\ctm$ are calculated individually and directly loss-optimally:
              $$
              \ctmt = \argmin_{c} \sum_{\xi \in \Rtm} L(\yi, \fmd(\xi) + c)
              $$
             \item
               Regions $\Rtm$ are calculated not loss-optimal but with the squared loss
               against the pseudo-residuals.
          \end{itemize}
          \begin{codebox}
            \textbf{Classification}
          \end{codebox}
          \begin{itemize}[$\bullet$]
            \setlength{\itemindent}{+.3in}
            \item
              For binary classification $\Yspace = \{0, 1\}$, an appropriate loss
              function has to be used, e.g. Bernoulli loss $\Lxy = - y \cdot \fx + \log(1 + \exp(\fx))$
            \item
              Using this loss function, we can simply run GB as for regression.
             \item
               For multiclass problems $\Yspace = \{1, \ldots, g\}$ we create $g$
               discriminant functions $\fxk$, one for each class, each an additive
               model of base learners.
          \end{itemize}
        \end{myblock}
      } % parbox end
    \end{minipage}
  \end{beamercolorbox}
\end{column}

%%%%%%%%%%%%%%%%%%%%%%%%%%%%%%%%%%%%%%%%%%%%%%%%%%%%%%%%%%%%%%%%%%%%%%%%%%%%%%%%%%%%%%%%%%%%%%%%%%%%%%%%

\begin{column}{.31\textwidth}
  \begin{beamercolorbox}[center]{postercolumn}
    \begin{minipage}{.98\textwidth}
      \parbox[t][\columnheight]{\textwidth}{
        \begin{myblock}{Regularization and Shrinkage}
          If GB runs for a large number of iterations, it can overfit due to its aggressive
          loss minimization. Options for regularization:
          \begin{itemize}[$\bullet$]
            \setlength{\itemindent}{+.3in}
              \item
                Limit the number of boosting iterations $M$ (\enquote{early stopping}), i.e.,
                limit the number of additive components.
              \item
                Limit the depth of the trees. This can also be interpreted as choosing the
                order of interaction.
              \item
                Shorten the step length $\betam$ of each iteration.
          \end{itemize}
        \end{myblock}

        \begin{myblock}{Componentwise Gradient Boosting}
          Aim of componentwise gradient boosting (model-based boosting) to find a model that
          has strong predictive performance, interpretable components, does automatic selection
          of components, and is sparser than a model fitted with maximum-likelihood estimation.
          \begin{itemize}[$\bullet$]
            \setlength{\itemindent}{+.3in}
              \item
                The base learner is now chosen from a set of base learner
                $\bj(\xv,\pmb\theta) \quad j = 1,\dots, J$
              \item
                Base learners are often familiar statistical models
              \item
                In each iteration, only the best base learner is selected and used for
                updating the model
          \end{itemize}
        \end{myblock}

        \begin{myblock}{Synopsis}
        \textbf{Hypothesis Space:}
          Models build with boosting are weighted sums of base learners $b(\xv, \thetam)$.
          \vspace*{1ex}

        \textbf{Risk:}
          Models build with boosting can use any kind of loss function as long as the derivative exists.
          \vspace*{1ex}

        \textbf{Optimization:}
          Gradient descent in function space.
          \end{myblock}
      }
    \end{minipage}
  \end{beamercolorbox}
\end{column}


\end{columns}
\end{frame}
\end{document}
