\documentclass[11pt, t, aspectratio=169]{beamer}
% \usepackage[]{graphicx}
\usepackage[]{color}
% maxwidth is the original width if it is less than linewidth
% otherwise use linewidth (to make sure the graphics do not exceed the margin)
\makeatletter
\def\maxwidth{ %
  \ifdim\Gin@nat@width>\linewidth
    \linewidth
  \else
    \Gin@nat@width
  \fi
}
\makeatother

% ---------------------------------%
% latex-math dependencies, do not remove:
% - \usepackage{mathtools}
% - \usepackage{bm}
% - \usepackage{siunitx}
% - \usepackage{dsfont}
% - \usepackage{xspace}
% ---------------------------------%

%--------------------------------------------------------%
%       Language, encoding, typography
%--------------------------------------------------------%

\usepackage[english]{babel}
\usepackage[utf8]{inputenc} % Enables inputting UTF-8 symbols
% Standard AMS suite
\usepackage{amsmath,amsfonts,amssymb}

% Font four double-stroke / blackboard letters for sets of numbers (N, R, ...)
% Distribution name is "doublestroke"
% According to https://mirror.physik.tu-berlin.de/pub/CTAN/fonts/doublestroke/dsdoc.pdf
% the "bbm" package does a similar thing and may be superfluous.
% Required for latex-math
\usepackage{dsfont}

% bbm – "Blackboard-style" cm fonts (https://www.ctan.org/pkg/bbm)
% Used to be in common.tex, loaded directly after this file
% Maybe superfluous given dsfont is loaded
% TODO: Check if really unused?
% \usepackage{bbm}

% bm – Access bold symbols in maths mode - https://ctan.org/pkg/bm
% Required for latex-math
% https://tex.stackexchange.com/questions/3238/bm-package-versus-boldsymbol
\usepackage{bm}

% pifont – Access to PostScript standard Symbol and Dingbats fonts
% Used for \newcommand{\xmark}{\ding{55}, which is never used
% aside from lecture_advml/attic/xx-automl/slides.Rnw
% \usepackage{pifont}

% Quotes (inline and display), provdes \enquote
% https://ctan.org/pkg/csquotes
\usepackage{csquotes}

% Adds arg to enumerate env, technically superseded by enumitem according
% to https://ctan.org/pkg/enumerate
% Replace with https://ctan.org/pkg/enumitem ?
\usepackage{enumerate}

% Line spacing - provides \singlespacing \doublespacing \onehalfspacing
% https://ctan.org/pkg/setspace
% TODO: Check if really unused?
%\usepackage{setspace}

% mathtools – Mathematical tools to use with amsmath
% https://ctan.org/pkg/mathtools?lang=en
% latex-math dependency according to latex-math repo
\usepackage{mathtools}

%--------------------------------------------------------%
%       Displaying code and algorithms
%--------------------------------------------------------%
\usepackage{verbatim}
\usepackage{algorithm}
\usepackage{algpseudocode}

%--------------------------------------------------------%
%       Tables
%--------------------------------------------------------%

% multi-row table cells: https://www.namsu.de/Extra/pakete/Multirow.html
\usepackage{multirow}

% long/multi-page tables: https://texdoc.org/serve/longtable.pdf/0
% TODO: Check if really unused?

\usepackage{longtable}

% pretty table env: https://ctan.org/pkg/booktabs?lang=en
% TODO: Check if really unused?
\usepackage{booktabs}

%--------------------------------------------------------%
%       Figures: Creating, placing, verbing
%--------------------------------------------------------%

% wrapfig - Wrapping text around figures https://de.overleaf.com/learn/latex/Wrapping_text_around_figures
\usepackage{wrapfig}

% Sub figures in figures and tables
% https://ctan.org/pkg/subfig -- supersedes subfigure package
% TODO: Check if really unused?
\usepackage{subfig}

% Actually it's pronounced PGF https://en.wikibooks.org/wiki/LaTeX/PGF/TikZ
\usepackage{tikz}

\usetikzlibrary{shapes,arrows,automata,positioning,calc,chains,trees, shadows}
\tikzset{
  %Define standard arrow tip
  >=stealth',
  %Define style for boxes
  punkt/.style={
    rectangle,
    rounded corners,
    draw=black, very thick,
    text width=6.5em,
    minimum height=2em,
    text centered},
  % Define arrow style
  pil/.style={
    ->,
    thick,
    shorten <=2pt,
    shorten >=2pt,}
}


% Unsorted
% textpos – Place boxes at arbitrary positions on the LATEX page
% https://ctan.org/pkg/textpos?lang=en
% Provides \begin{textblock}
 % TODO: Check if really unused?
\usepackage[absolute,overlay]{textpos}

% psfrag – Replace strings in encapsulated PostScript figures
% https://www.overleaf.com/latex/examples/psfrag-example/tggxhgzwrzhn
% https://ftp.mpi-inf.mpg.de/pub/tex/mirror/ftp.dante.de/pub/tex/macros/latex/contrib/psfrag/pfgguide.pdf
% Can't tell if this is needed
% TODO: Check if really unused?
\usepackage{psfrag}

% Maybe not great to use this https://tex.stackexchange.com/a/197/19093
% Use align instead -- TODO: Global search & replace to check
\usepackage{eqnarray}

\usepackage{colortbl}

% arydshln – Draw dash-lines in array/tabular
% https://www.ctan.org/pkg/arydshln
% !! "arydshln has to be loaded after array, longtable, colortab and/or colortbl"
% Provides \hdashline and \cdashline
% TODO: Check if really unused?
% \usepackage{arydshln}

% tabularx – Tabulars with adjustable-width columns
% https://ctan.org/pkg/tabularx
% Provides \begin{tabularx}
% TODO: Check if really unused?
% \usepackage{tabularx}

% placeins – Control float placement
% https://ctan.org/pkg/placeins
% Defines a \FloatBarrier command
% TODO: Check if really unused?
% \usepackage{placeins}


% framed – Framed or shaded regions that can break across pages
% https://ctan.org/pkg/framed
% Provides \begin{framed} which uses \colorbox{shadecolor} relying on \definecolor{shadecolor}.
% TODO: Check if really unused?
% \usepackage{framed}

% Used often in conjunction with \definecolor{shadecolor}{rgb}{0.969, 0.969, 0.969}
% Might be able to be removed or at least redefined to only have shadecolor (if needed)
\definecolor{fgcolor}{rgb}{0.345, 0.345, 0.345}
\definecolor{shadecolor}{rgb}{0.969, 0.969, 0.969}
\newenvironment{knitrout}{}{} % an empty environment to be redefined in TeX


% Defines macros and environments
\usepackage{../../style/lmu-lecture}

\let\code=\texttt % Used regularly
\let\proglang=\textsf % Unused?

% Not sure what/why this does
\setkeys{Gin}{width=0.9\textwidth}

\setbeamertemplate{frametitle}{\expandafter\uppercase\expandafter\insertframetitle}

% Can't find a reason why common.tex is not just part of this file?

% basic latex stuff
\newcommand{\pkg}[1]{{\fontseries{b}\selectfont #1}} %fontstyle for R packages
\newcommand{\lz}{\vspace{0.5cm}} %vertical space
\newcommand{\dlz}{\vspace{1cm}} %double vertical space
\newcommand{\oneliner}[1] % Oneliner for important statements
{\begin{block}{}\begin{center}\begin{Large}#1\end{Large}\end{center}\end{block}}


%new environments
\newenvironment{vbframe}  %frame with breaks and verbatim
{
 \begin{frame}[containsverbatim,allowframebreaks]
}
{
\end{frame}
}

\newenvironment{vframe}  %frame with verbatim without breaks (to avoid numbering one slided frames)
{
 \begin{frame}[containsverbatim]
}
{
\end{frame}
}

\newenvironment{blocki}[1]   % itemize block
{
 \begin{block}{#1}\begin{itemize}
}
{
\end{itemize}\end{block}
}

\newenvironment{fragileframe}[2]{  %fragile frame with framebreaks
\begin{frame}[allowframebreaks, fragile, environment = fragileframe]
\frametitle{#1}
#2}
{\end{frame}}


\newcommand{\myframe}[2]{  %short for frame with framebreaks
\begin{frame}[allowframebreaks]
\frametitle{#1}
#2
\end{frame}}

\newcommand{\remark}[1]{
  \textbf{Remark:} #1
}


\newenvironment{deleteframe}
{
\begingroup
\usebackgroundtemplate{\includegraphics[width=\paperwidth,height=\paperheight]{../style/color/red.png}}
 \begin{frame}
}
{
\end{frame}
\endgroup
}
\newenvironment{simplifyframe}
{
\begingroup
\usebackgroundtemplate{\includegraphics[width=\paperwidth,height=\paperheight]{../style/color/yellow.png}}
 \begin{frame}
}
{
\end{frame}
\endgroup
}\newenvironment{draftframe}
{
\begingroup
\usebackgroundtemplate{\includegraphics[width=\paperwidth,height=\paperheight]{../style/color/green.jpg}}
 \begin{frame}
}
{
\end{frame}
\endgroup
}
% https://tex.stackexchange.com/a/261480: textcolor that works in mathmode
\makeatletter
\renewcommand*{\@textcolor}[3]{%
  \protect\leavevmode
  \begingroup
    \color#1{#2}#3%
  \endgroup
}
\makeatother


%-------------------------------------------------------------------------------------------------------%
%  Unused stuff that needs to go but is kept here currently juuuust in case it was important after all  %
%-------------------------------------------------------------------------------------------------------%

% \newcommand{\hlnum}[1]{\textcolor[rgb]{0.686,0.059,0.569}{#1}}%
% \newcommand{\hlstr}[1]{\textcolor[rgb]{0.192,0.494,0.8}{#1}}%
% \newcommand{\hlcom}[1]{\textcolor[rgb]{0.678,0.584,0.686}{\textit{#1}}}%
% \newcommand{\hlopt}[1]{\textcolor[rgb]{0,0,0}{#1}}%
% \newcommand{\hlstd}[1]{\textcolor[rgb]{0.345,0.345,0.345}{#1}}%
% \newcommand{\hlkwa}[1]{\textcolor[rgb]{0.161,0.373,0.58}{\textbf{#1}}}%
% \newcommand{\hlkwb}[1]{\textcolor[rgb]{0.69,0.353,0.396}{#1}}%
% \newcommand{\hlkwc}[1]{\textcolor[rgb]{0.333,0.667,0.333}{#1}}%
% \newcommand{\hlkwd}[1]{\textcolor[rgb]{0.737,0.353,0.396}{\textbf{#1}}}%
% \let\hlipl\hlkwb

% \makeatletter
% \newenvironment{kframe}{%
%  \def\at@end@of@kframe{}%
%  \ifinner\ifhmode%
%   \def\at@end@of@kframe{\end{minipage}}%
%   \begin{minipage}{\columnwidth}%
%  \fi\fi%
%  \def\FrameCommand##1{\hskip\@totalleftmargin \hskip-\fboxsep
%  \colorbox{shadecolor}{##1}\hskip-\fboxsep
%      % There is no \\@totalrightmargin, so:
%      \hskip-\linewidth \hskip-\@totalleftmargin \hskip\columnwidth}%
%  \MakeFramed {\advance\hsize-\width
%    \@totalleftmargin\z@ \linewidth\hsize
%    \@setminipage}}%
%  {\par\unskip\endMakeFramed%
%  \at@end@of@kframe}
% \makeatother

% \definecolor{shadecolor}{rgb}{.97, .97, .97}
% \definecolor{messagecolor}{rgb}{0, 0, 0}
% \definecolor{warningcolor}{rgb}{1, 0, 1}
% \definecolor{errorcolor}{rgb}{1, 0, 0}
% \newenvironment{knitrout}{}{} % an empty environment to be redefined in TeX

% \usepackage{alltt}
% \newcommand{\SweaveOpts}[1]{}  % do not interfere with LaTeX
% \newcommand{\SweaveInput}[1]{} % because they are not real TeX commands
% \newcommand{\Sexpr}[1]{}       % will only be parsed by R
% \newcommand{\xmark}{\ding{55}}%
 % use this if compiling with make file
\usepackage[]{graphicx}
\usepackage[]{color}
% maxwidth is the original width if it is less than linewidth
% otherwise use linewidth (to make sure the graphics do not exceed the margin)
\makeatletter
\def\maxwidth{ %
  \ifdim\Gin@nat@width>\linewidth
    \linewidth
  \else
    \Gin@nat@width
  \fi
}
\makeatother

% ---------------------------------%
% latex-math dependencies, do not remove:
% - \usepackage{mathtools}
% - \usepackage{bm}
% - \usepackage{siunitx}
% - \usepackage{dsfont}
% - \usepackage{xspace}
% ---------------------------------%

%--------------------------------------------------------%
%       Language, encoding, typography
%--------------------------------------------------------%

\usepackage[english]{babel}
\usepackage[utf8]{inputenc} % Enables inputting UTF-8 symbols
% Standard AMS suite
\usepackage{amsmath,amsfonts,amssymb}

% Font four double-stroke / blackboard letters for sets of numbers (N, R, ...)
% Distribution name is "doublestroke"
% According to https://mirror.physik.tu-berlin.de/pub/CTAN/fonts/doublestroke/dsdoc.pdf
% the "bbm" package does a similar thing and may be superfluous.
% Required for latex-math
\usepackage{dsfont}

% bbm – "Blackboard-style" cm fonts (https://www.ctan.org/pkg/bbm)
% Used to be in common.tex, loaded directly after this file
% Maybe superfluous given dsfont is loaded
% TODO: Check if really unused?
% \usepackage{bbm}

% bm – Access bold symbols in maths mode - https://ctan.org/pkg/bm
% Required for latex-math
% https://tex.stackexchange.com/questions/3238/bm-package-versus-boldsymbol
\usepackage{bm}

% pifont – Access to PostScript standard Symbol and Dingbats fonts
% Used for \newcommand{\xmark}{\ding{55}, which is never used
% aside from lecture_advml/attic/xx-automl/slides.Rnw
% \usepackage{pifont}

% Quotes (inline and display), provdes \enquote
% https://ctan.org/pkg/csquotes
\usepackage{csquotes}

% Adds arg to enumerate env, technically superseded by enumitem according
% to https://ctan.org/pkg/enumerate
% Replace with https://ctan.org/pkg/enumitem ?
\usepackage{enumerate}

% Line spacing - provides \singlespacing \doublespacing \onehalfspacing
% https://ctan.org/pkg/setspace
% TODO: Check if really unused?
%\usepackage{setspace}

% mathtools – Mathematical tools to use with amsmath
% https://ctan.org/pkg/mathtools?lang=en
% latex-math dependency according to latex-math repo
\usepackage{mathtools}

%--------------------------------------------------------%
%       Displaying code and algorithms
%--------------------------------------------------------%
\usepackage{verbatim}
\usepackage{algorithm}
\usepackage{algpseudocode}

%--------------------------------------------------------%
%       Tables
%--------------------------------------------------------%

% multi-row table cells: https://www.namsu.de/Extra/pakete/Multirow.html
\usepackage{multirow}

% long/multi-page tables: https://texdoc.org/serve/longtable.pdf/0
% TODO: Check if really unused?

\usepackage{longtable}

% pretty table env: https://ctan.org/pkg/booktabs?lang=en
% TODO: Check if really unused?
\usepackage{booktabs}

%--------------------------------------------------------%
%       Figures: Creating, placing, verbing
%--------------------------------------------------------%

% wrapfig - Wrapping text around figures https://de.overleaf.com/learn/latex/Wrapping_text_around_figures
\usepackage{wrapfig}

% Sub figures in figures and tables
% https://ctan.org/pkg/subfig -- supersedes subfigure package
% TODO: Check if really unused?
\usepackage{subfig}

% Actually it's pronounced PGF https://en.wikibooks.org/wiki/LaTeX/PGF/TikZ
\usepackage{tikz}

\usetikzlibrary{shapes,arrows,automata,positioning,calc,chains,trees, shadows}
\tikzset{
  %Define standard arrow tip
  >=stealth',
  %Define style for boxes
  punkt/.style={
    rectangle,
    rounded corners,
    draw=black, very thick,
    text width=6.5em,
    minimum height=2em,
    text centered},
  % Define arrow style
  pil/.style={
    ->,
    thick,
    shorten <=2pt,
    shorten >=2pt,}
}


% Unsorted
% textpos – Place boxes at arbitrary positions on the LATEX page
% https://ctan.org/pkg/textpos?lang=en
% Provides \begin{textblock}
 % TODO: Check if really unused?
\usepackage[absolute,overlay]{textpos}

% psfrag – Replace strings in encapsulated PostScript figures
% https://www.overleaf.com/latex/examples/psfrag-example/tggxhgzwrzhn
% https://ftp.mpi-inf.mpg.de/pub/tex/mirror/ftp.dante.de/pub/tex/macros/latex/contrib/psfrag/pfgguide.pdf
% Can't tell if this is needed
% TODO: Check if really unused?
\usepackage{psfrag}

% Maybe not great to use this https://tex.stackexchange.com/a/197/19093
% Use align instead -- TODO: Global search & replace to check
\usepackage{eqnarray}

\usepackage{colortbl}

% arydshln – Draw dash-lines in array/tabular
% https://www.ctan.org/pkg/arydshln
% !! "arydshln has to be loaded after array, longtable, colortab and/or colortbl"
% Provides \hdashline and \cdashline
% TODO: Check if really unused?
% \usepackage{arydshln}

% tabularx – Tabulars with adjustable-width columns
% https://ctan.org/pkg/tabularx
% Provides \begin{tabularx}
% TODO: Check if really unused?
% \usepackage{tabularx}

% placeins – Control float placement
% https://ctan.org/pkg/placeins
% Defines a \FloatBarrier command
% TODO: Check if really unused?
% \usepackage{placeins}


% framed – Framed or shaded regions that can break across pages
% https://ctan.org/pkg/framed
% Provides \begin{framed} which uses \colorbox{shadecolor} relying on \definecolor{shadecolor}.
% TODO: Check if really unused?
% \usepackage{framed}

% Used often in conjunction with \definecolor{shadecolor}{rgb}{0.969, 0.969, 0.969}
% Might be able to be removed or at least redefined to only have shadecolor (if needed)
\definecolor{fgcolor}{rgb}{0.345, 0.345, 0.345}
\definecolor{shadecolor}{rgb}{0.969, 0.969, 0.969}
\newenvironment{knitrout}{}{} % an empty environment to be redefined in TeX


% Defines macros and environments
\usepackage{../../style/lmu-lecture}

\let\code=\texttt % Used regularly
\let\proglang=\textsf % Unused?

% Not sure what/why this does
\setkeys{Gin}{width=0.9\textwidth}

\setbeamertemplate{frametitle}{\expandafter\uppercase\expandafter\insertframetitle}

% Can't find a reason why common.tex is not just part of this file?

% basic latex stuff
\newcommand{\pkg}[1]{{\fontseries{b}\selectfont #1}} %fontstyle for R packages
\newcommand{\lz}{\vspace{0.5cm}} %vertical space
\newcommand{\dlz}{\vspace{1cm}} %double vertical space
\newcommand{\oneliner}[1] % Oneliner for important statements
{\begin{block}{}\begin{center}\begin{Large}#1\end{Large}\end{center}\end{block}}


%new environments
\newenvironment{vbframe}  %frame with breaks and verbatim
{
 \begin{frame}[containsverbatim,allowframebreaks]
}
{
\end{frame}
}

\newenvironment{vframe}  %frame with verbatim without breaks (to avoid numbering one slided frames)
{
 \begin{frame}[containsverbatim]
}
{
\end{frame}
}

\newenvironment{blocki}[1]   % itemize block
{
 \begin{block}{#1}\begin{itemize}
}
{
\end{itemize}\end{block}
}

\newenvironment{fragileframe}[2]{  %fragile frame with framebreaks
\begin{frame}[allowframebreaks, fragile, environment = fragileframe]
\frametitle{#1}
#2}
{\end{frame}}


\newcommand{\myframe}[2]{  %short for frame with framebreaks
\begin{frame}[allowframebreaks]
\frametitle{#1}
#2
\end{frame}}

\newcommand{\remark}[1]{
  \textbf{Remark:} #1
}


\newenvironment{deleteframe}
{
\begingroup
\usebackgroundtemplate{\includegraphics[width=\paperwidth,height=\paperheight]{../style/color/red.png}}
 \begin{frame}
}
{
\end{frame}
\endgroup
}
\newenvironment{simplifyframe}
{
\begingroup
\usebackgroundtemplate{\includegraphics[width=\paperwidth,height=\paperheight]{../style/color/yellow.png}}
 \begin{frame}
}
{
\end{frame}
\endgroup
}\newenvironment{draftframe}
{
\begingroup
\usebackgroundtemplate{\includegraphics[width=\paperwidth,height=\paperheight]{../style/color/green.jpg}}
 \begin{frame}
}
{
\end{frame}
\endgroup
}
% https://tex.stackexchange.com/a/261480: textcolor that works in mathmode
\makeatletter
\renewcommand*{\@textcolor}[3]{%
  \protect\leavevmode
  \begingroup
    \color#1{#2}#3%
  \endgroup
}
\makeatother


%-------------------------------------------------------------------------------------------------------%
%  Unused stuff that needs to go but is kept here currently juuuust in case it was important after all  %
%-------------------------------------------------------------------------------------------------------%

% \newcommand{\hlnum}[1]{\textcolor[rgb]{0.686,0.059,0.569}{#1}}%
% \newcommand{\hlstr}[1]{\textcolor[rgb]{0.192,0.494,0.8}{#1}}%
% \newcommand{\hlcom}[1]{\textcolor[rgb]{0.678,0.584,0.686}{\textit{#1}}}%
% \newcommand{\hlopt}[1]{\textcolor[rgb]{0,0,0}{#1}}%
% \newcommand{\hlstd}[1]{\textcolor[rgb]{0.345,0.345,0.345}{#1}}%
% \newcommand{\hlkwa}[1]{\textcolor[rgb]{0.161,0.373,0.58}{\textbf{#1}}}%
% \newcommand{\hlkwb}[1]{\textcolor[rgb]{0.69,0.353,0.396}{#1}}%
% \newcommand{\hlkwc}[1]{\textcolor[rgb]{0.333,0.667,0.333}{#1}}%
% \newcommand{\hlkwd}[1]{\textcolor[rgb]{0.737,0.353,0.396}{\textbf{#1}}}%
% \let\hlipl\hlkwb

% \makeatletter
% \newenvironment{kframe}{%
%  \def\at@end@of@kframe{}%
%  \ifinner\ifhmode%
%   \def\at@end@of@kframe{\end{minipage}}%
%   \begin{minipage}{\columnwidth}%
%  \fi\fi%
%  \def\FrameCommand##1{\hskip\@totalleftmargin \hskip-\fboxsep
%  \colorbox{shadecolor}{##1}\hskip-\fboxsep
%      % There is no \\@totalrightmargin, so:
%      \hskip-\linewidth \hskip-\@totalleftmargin \hskip\columnwidth}%
%  \MakeFramed {\advance\hsize-\width
%    \@totalleftmargin\z@ \linewidth\hsize
%    \@setminipage}}%
%  {\par\unskip\endMakeFramed%
%  \at@end@of@kframe}
% \makeatother

% \definecolor{shadecolor}{rgb}{.97, .97, .97}
% \definecolor{messagecolor}{rgb}{0, 0, 0}
% \definecolor{warningcolor}{rgb}{1, 0, 1}
% \definecolor{errorcolor}{rgb}{1, 0, 0}
% \newenvironment{knitrout}{}{} % an empty environment to be redefined in TeX

% \usepackage{alltt}
% \newcommand{\SweaveOpts}[1]{}  % do not interfere with LaTeX
% \newcommand{\SweaveInput}[1]{} % because they are not real TeX commands
% \newcommand{\Sexpr}[1]{}       % will only be parsed by R
% \newcommand{\xmark}{\ding{55}}%
 % use this if compiling in overleaf
% math spaces
\newcommand{\N}{\mathds{N}}                                                 % N, naturals
\newcommand{\Z}{\mathds{Z}}                                                 % Z, integers
\newcommand{\Q}{\mathds{Q}}                                                 % Q, rationals
\newcommand{\R}{\mathds{R}}                                                 % R, reals
\newcommand{\C}{\mathds{C}}                                                 % C, complex
\newcommand{\HS}{\mathcal{H}}                                               % H, hilbertspace
\newcommand{\continuous}{\mathcal{C}}                                       % C, space of continuous functions
\newcommand{\M}{\mathcal{M}} 												% machine numbers
\newcommand{\epsm}{\epsilon_m} 												% maximum error


% basic math stuff
\newcommand{\xt}{\tilde x}													% x tilde
\def\argmax{\mathop{\sf arg\,max}}                                          % argmax
\def\argmin{\mathop{\sf arg\,min}}                                          % argmin
\newcommand{\sign}{\operatorname{sign}}                                     % sign, signum
\newcommand{\I}{\mathbb{I}}                                                 % I, indicator
\newcommand{\order}{\mathcal{O}}                                            % O, order
\newcommand{\fp}[2]{\frac{\partial #1}{\partial #2}}                        % partial derivative
\newcommand{\pd}[2]{\frac{\partial{#1}}{\partial #2}}						% partial derivative

% sums and products
\newcommand{\sumin}{\sum_{i=1}^n}											% summation from i=1 to n
\newcommand{\sumkg}{\sum_{k=1}^g}											% summation from k=1 to g
\newcommand{\prodin}{\prod_{i=1}^n}											% product from i=1 to n
\newcommand{\prodkg}{\prod_{k=1}^g}											% product from k=1 to g

% linear algebra
\newcommand{\one}{\boldsymbol{1}}                                           % 1, unitvector
\newcommand{\id}{\mathrm{I}}                                                % I, identity
\newcommand{\diag}{\operatorname{diag}}                                     % diag, diagonal
\newcommand{\trace}{\operatorname{tr}}                                      % tr, trace
\newcommand{\spn}{\operatorname{span}}                                      % span
\newcommand{\scp}[2]{\left\langle #1, #2 \right\rangle}                     % <.,.>, scalarproduct
\newcommand{\mat}[1]{ 														% short pmatrix command
	\begin{pmatrix}
		#1
	\end{pmatrix}
}
\newcommand{\Amat}{\bm{A}}													% matrix A
\newcommand{\xv}{\bm{x}}													% vector x (bold)
\newcommand{\yv}{\bm{y}}														% vector y (bold)
\newcommand{\Deltab}{\bm{\Delta}}											% error term for vectors
															

% basic probability + stats
\renewcommand{\P}{\mathds{P}}                                               % P, probability
\newcommand{\E}{\mathds{E}}                                                 % E, expectation
\newcommand{\var}{\mathsf{Var}}                                             % Var, variance
\newcommand{\cov}{\mathsf{Cov}}                                             % Cov, covariance
\newcommand{\corr}{\mathsf{Corr}}                                           % Corr, correlation
\newcommand{\normal}{\mathcal{N}}                                           % N of the normal distribution
\newcommand{\iid}{\overset{i.i.d}{\sim}}                                    % dist with i.i.d superscript
\newcommand{\distas}[1]{\overset{#1}{\sim}}                                 % ... is distributed as ... 
% machine learning

%%%%%% ml - data
\newcommand{\Xspace}{\mathcal{X}}                                           % X, input space
\newcommand{\Yspace}{\mathcal{Y}}                                           % Y, output space
\newcommand{\nset}{\{1, \ldots, n\}}                                        % set from 1 to n
\newcommand{\pset}{\{1, \ldots, p\}}                                        % set from 1 to p
\newcommand{\gset}{\{1, \ldots, g\}}                                        % set from 1 to g
\newcommand{\Pxy}{\P_{xy}}                                                  % P_xy
\newcommand{\xy}{(x, y)}                                                    % observation (x, y)
\newcommand{\xvec}{(x_1, \ldots, x_p)^T}                                    % (x1, ..., xp) 
\newcommand{\D}{\mathcal{D}}                                                % D, data 
\newcommand{\Dset}{\{ (x^{(1)}, y^{(1)}), \ldots, (x^{(n)},  y^{(n)})\}}    % {(x1,y1)), ..., (xn,yn)}, data
\newcommand{\xdat}{\{ x^{(1)}, \ldots, x^{(n)}\}}   						 % {x1, ..., xn}, input data
\newcommand{\ydat}{\mathbf{y}}                                              % y (bold), vector of outcomes
\newcommand{\yvec}{(y^{(1)}, \hdots, y^{(n)})^T}                            % (y1, ..., yn), vector of outcomes
\renewcommand{\xi}[1][i]{x^{(#1)}}                                          % x^i, i-th observed value of x
\newcommand{\yi}[1][i]{y^{(#1)}}                                            % y^i, i-th observed value of y 
\newcommand{\xyi}{(\xi, \yi)}                                               % (x^i, y^i), i-th observation
\newcommand{\xivec}{(x^{(i)}_1, \ldots, x^{(i)}_p)^T}                       % (x1^i, ..., xp^i), i-th observation vector
\newcommand{\xj}{x_j}                                                       % x_j, j-th feature
\newcommand{\xjb}{\mathbf{x}_j}                                             % x_j (bold), j-th feature vecor
\newcommand{\xjvec}{(x^{(1)}_j, \ldots, x^{(n)}_j)^T}                       % (x^1_j, ..., x^n_j), j-th feature vector
\newcommand{\Dtrain}{\mathcal{D}_{\text{train}}}                            % D_train, training set
\newcommand{\Dtest}{\mathcal{D}_{\text{test}}}                              % D_test, test set

%%%%%% ml - models general

% continuous prediction function f
\newcommand{\fx}{f(x)}                                                      % f(x), continuous prediction function
\newcommand{\Hspace}{H}														% hypothesis space where f is from
\newcommand{\fh}{\hat{f}}                                                   % f hat, estimated prediction function
\newcommand{\fxh}{\fh(x)}                                                   % fhat(x)
\newcommand{\fxt}{f(x | \theta)}                                            % f(x | theta)
\newcommand{\fxi}{f(\xi)}                                                   % f(x^(i))
\newcommand{\fxih}{\hat{f}(\xi)}                                            % f(x^(i))
\newcommand{\fxit}{f(x^{(i)} | \theta)}                                     % f(x^(i) | theta)
\newcommand{\fhD}{\fh_{\D}}                                                 % fhat_D, estimate of f based on D
\newcommand{\fhDtrain}{\fh_{\Dtrain}}                                       % fhat_Dtrain, estimate of f based on D

% discrete prediction function h
\newcommand{\hx}{h(x)}                                                      % h(x), discrete prediction function
\newcommand{\hh}{\hat{h}}                                                   % h hat
\newcommand{\hxh}{\hat{h}(x)}                                               % hhat(x)
\newcommand{\hxt}{h(x | \theta)}                                            % h(x | theta)
\newcommand{\hxi}{h(\xi)}                                                   % h(x^(i))
\newcommand{\hxit}{h(x^{(i)} | \theta)}                                     % h(x^(i) | theta)

% yhat
\newcommand{\yh}{\hat{y}}                                                   % y hat for prediction of target
\newcommand{\yih}{\hat{y}}                                                  % y hat for prediction of target

% theta
\newcommand{\thetah}{\hat{\theta}}                                          % theta hat

% densities + probabilities
% pdf of x 
\newcommand{\pdf}{p}                                                        % p
\newcommand{\pdfx}{p(x)}                                                    % p(x)
\newcommand{\pixt}{\pi(x | \theta)}                                         % pi(x|theta), pdf of x given theta

% pdf of (x, y)
\newcommand{\pdfxy}{p(x,y)}                                                 % p(x, y)
\newcommand{\pdfxyt}{p(x, y | \theta)}                                      % p(x, y | theta)
\newcommand{\pdfxyit}{p(\xi, \yi | \theta)}                                 % p(x^(i), y^(i) | theta)

% pdf of x given y
\newcommand{\pdfxyk}{p(x | y=k)}                                            % p(x | y = k)
\newcommand{\lpdfxyk}{\log \pdfxyk}                                         % log p(x | y = k)
\newcommand{\pdfxiyk}{p(\xi | y=k)}                                         % p(x^i | y = k)

% prior probabilities
\newcommand{\pik}{\pi_k}                                                    % pi_k, prior
\newcommand{\lpik}{\log \pik}                                               % log pi_k, log of the prior

% posterior probabilities
\newcommand{\post}{\P(y = 1 | x)}                                           % P(y = 1 | x), post. prob for y=1
\newcommand{\pix}{\pi(x)}                                                   % pi(x), P(y = 1 | x)
\newcommand{\postk}{\P(y = k | x)}                                          % P(y = k | y), post. prob for y=k
\newcommand{\pikx}{\pi_k(x)}                                                % pi_k(x), P(y = k | x)
\newcommand{\pikxt}{\pi_k(x | \theta)}                                      % pi_k(x | theta), P(y = k | x, theta)
\newcommand{\pijx}{\pi_j(x)}                                                % pi_j(x), P(y = j | x)
\newcommand{\pdfygxt}{p(y |x, \theta)}                                      % p(y | x, theta)
\newcommand{\pdfyigxit}{p(\yi |\xi, \theta)}                                % p(y^i |x^i, theta)
\newcommand{\lpdfygxt}{\log \pdfygxt }                                      % log p(y | x, theta)
\newcommand{\lpdfyigxit}{\log \pdfyigxit}                                   % log p(y^i |x^i, theta)
\newcommand{\pixh}{\hat \pi(x)}                                             % pi(x) hat, P(y = 1 | x) hat
\newcommand{\pikxh}{\hat \pi_k(x)}                                          % pi_k(x) hat, P(y = k | x) hat

% residual and margin
\newcommand{\eps}{\epsilon}                                                 % residual, stochastic
\newcommand{\epsi}{\epsilon^{(i)}}                                          % epsilon^i, residual, stochastic
\newcommand{\epsh}{\hat{\epsilon}}                                          % residual, estimated
\newcommand{\yf}{y \fx}                                                     % y f(x), margin
\newcommand{\yfi}{\yi \fxi}                                                 % y^i f(x^i), margin
\newcommand{\Sigmah}{\hat \Sigma}											% estimated covariance matrix
\newcommand{\Sigmahj}{\hat \Sigma_j}										% estimated covariance matrix for the j-th class

% ml - loss, risk, likelihood
\newcommand{\Lxy}{L(y, f(x))}                                               % L(y, f(x)), loss function
\newcommand{\Lxyi}{L(\yi, \fxi)}                                            % L(y^i, f(x^i))
\newcommand{\Lxyt}{L(y, \fxt)}                                              % L(y, f(x | theta))
\newcommand{\Lxyit}{L(\yi, \fxit)}                                          % L(y^i, f(x^i | theta)
\newcommand{\risk}{\mathcal{R}}                                             % R, risk
\newcommand{\riskf}{\risk(f)}                                               % R(f), risk
\newcommand{\riske}{\mathcal{R}_{\text{emp}}}                               % R_emp, empirical risk
\newcommand{\riskef}{\riske(f)}                                             % R_emp(f)
\newcommand{\risket}{\mathcal{R}_{\text{emp}}(\theta)}                      % R_emp(theta)
\newcommand{\riskr}{\mathcal{R}_{\text{reg}}}                               % R_reg, regularized risk
\newcommand{\riskrt}{\mathcal{R}_{\text{reg}}(\theta)}                      % R_reg(theta)
\newcommand{\riskrf}{\riskr(f)}                                             % R_reg(f)
\newcommand{\LL}{\mathcal{L}}                                               % L, likelihood
\newcommand{\LLt}{\mathcal{L}(\theta)}                                      % L(theta), likelihood
\renewcommand{\ll}{\ell}                                                    % l, log-likelihood
\newcommand{\llt}{\ell(\theta)}                                             % l(theta), log-likelihood
\newcommand{\LS}{\mathfrak{L}}                                              % ????????????
\newcommand{\TS}{\mathfrak{T}}                                              % ??????????????
\newcommand{\errtrain}{\text{err}_{\text{train}}}                           % training error
\newcommand{\errtest}{\text{err}_{\text{test}}}                             % training error
\newcommand{\errexp}{\overline{\text{err}_{\text{test}}}}                   % training error

% resampling
\newcommand{\GE}[1]{GE(\fh_{#1})}                                           % Generalization error GE
\newcommand{\GEh}[1]{\widehat{GE}_{#1}}                                     % Estimated train error
\newcommand{\GED}{\GE{\D}}                                                  % Generalization error GE
\newcommand{\EGEn}{EGE_n}                                                   % Generalization error GE
\newcommand{\EDn}{\E_{|D| = n}}                                             % Generalization error GE


% ml - irace
\newcommand{\costs}{\mathcal{C}} % costs
\newcommand{\Celite}{\theta^*} % elite configurations
\newcommand{\instances}{\mathcal{I}} % sequence of instances
\newcommand{\budget}{\mathcal{B}} % computational budget
% ml - svms
\newcommand{\sv}{\operatorname{SV}}                                         % supportvectors
\newcommand{\HS}{\mathcal{H}}                                               % H, hilbertspace
\let\myxi\xi																% xi, slack variable (SVM)

% ml - trees, extra trees

\newcommand{\Np}{\mathcal{N}}												% Parent node N
\newcommand{\Nl}{\Np_1}														% Left node N_1
\newcommand{\Nr}{\Np_2}														% Right node N_2

% ml - bagging, random forest
\newcommand{\bl}[1][m]{b^{[#1]}} % baselearner, default m
\newcommand{\blh}[1][m]{\hat{b}^{[#1]}} % estimated base learner, default m 
\newcommand{\blx}[1][m]{b^{[#1]}(\xv)} % baselearner, default m

\newcommand{\blfmh}{\hat{f}^{[m]}} % estimated baselearner: scores
\newcommand{\blfmhx}{\blfmh(\xv)} % estimated baselearner: scores of x
\newcommand{\blhmh}{\hat{h}^{[m]}} % estimated baselearner: hard labels
\newcommand{\blhmhx}{\blhmh(\xv)} % estimated baselearner: hard labels of x
\newcommand{\blpmh}{\hat \pi^{[m]}} % estimated baselearner: probabilities
\newcommand{\blpmhxk}{\hat \pi_{k}^{[m]}(\xv)} % estimated baselearner: probabilities of x for class k

\newcommand{\fM}{f^{[M]}(\xv)} % ensembled predictor
\newcommand{\fMh}{\hat f^{[M]}(\xv)} % estimated ensembled predictor
\newcommand{\ambifM}{\Delta\left(\fM\right)} % ambiguity/instability of ensemble
\newcommand{\betam}[1][m]{\beta^{[#1]}} % weight of basemodel m
\newcommand{\betamh}[1][m]{\hat{\beta}^{[#1]}} % weight of basemodel m with hat
\newcommand{\betaM}{\beta^{[M]}} % last baselearner
\newcommand{\summM}{\sum_{m=1}^M} % sum over m=1 to M baselearners
\newcommand{\avgmM}{\frac{1}{M} \sum_{m=1}^M} % averaging over m=1 to M baselearners

\newcommand{\ib}{\mathrm{IB}} % In-Bag (IB)
\newcommand{\ibm}{\ib^{[m]}} % In-Bag (IB) for m-th bootstrap
\newcommand{\oob}{\mathrm{OOB}} % Out-of-Bag (OOB)
\newcommand{\oobm}{\oob^{[m]}} % Out-of-Bag (OOB) for m-th bootstrap


% ml - boosting
\newcommand{\fm}[1][m]{f^{[#1]}} % prediction in iteration m
\newcommand{\fmh}[1][m]{\hat{f}^{[#1]}} % prediction in iteration m
\newcommand{\fmd}[1][m]{f^{[#1-1]}} % prediction m-1
\newcommand{\fmdh}[1][m]{\hat{f}^{[#1-1]}} % prediction m-1
\newcommand{\errm}[1][m]{\text{err}^{[#1]}} % weighted in-sample misclassification rate
\newcommand{\wm}[1][m]{w^{[#1]}} % weight vector of basemodel m
\newcommand{\wmi}[1][m]{w^{[#1](i)}} % weight of obs i of basemodel m
\newcommand{\thetam}[1][m]{\thetab^{[#1]}} % parameters of basemodel m
\newcommand{\thetamh}[1][m]{\hat{\thetab}^{[#1]}} % parameters of basemodel m with hat
\newcommand{\blxt}[1][m]{b(\xv, \thetab^{[#1]})} % baselearner, default m
\newcommand{\ens}{\sum_{m=1}^M \betam \blxt} % ensemble
\newcommand{\rmm}[1][m]{\tilde{r}^{[#1]}} % pseudo residuals
\newcommand{\rmi}[1][m]{\tilde{r}^{[#1](i)}} % pseudo residuals
\newcommand{\Rtm}[1][m]{R_{t}^{[#1]}} % terminal-region
\newcommand{\Tm}[1][m]{T^{[#1]}} % terminal-region
\newcommand{\ctm}[1][m]{c_t^{[#1]}} % mean, terminal-regions
\newcommand{\ctmh}[1][m]{\hat{c}_t^{[#1]}} % mean, terminal-regions with hat
\newcommand{\ctmt}[1][m]{\tilde{c}_t^{[#1]}} % mean, terminal-regions
\newcommand{\Lp}{L^\prime}
\newcommand{\Ldp}{L^{\prime\prime}}
\newcommand{\Lpleft}{\Lp_{\text{left}}}

% ml - boosting iml lecture
\newcommand{\ts}{\thetab^{\star}} % theta*
\newcommand{\bljt}{\bl[j](\xv, \thetab)} % BL j with theta
\newcommand{\bljts}{\bl[j](\xv, \ts)} % BL j with theta*

\title{Important Learning Algorithms in ML}
\institute{\href{https://compstat-lmu.github.io/lecture_i2ml/}{
compstat-lmu.github.io/lecture\_i2ml}}
\author{LMU SLDS}
\date{}

% Define: \titlefigure, \titlefiguresize (as share of textwidth), 
% \titlefiguresource
\newcommand{\titlefigure}{figure_man/titlefig}
\newcommand{\titlefiguresize}{1}
\newcommand{\titlefiguresource}{https://www.keysight.com/}
\newcommand{\learninggoals}{
  \item General idea of important ML algorithms
  \item Overview of strengths and weaknesses
  % \item Understand general idea of most important ML algorithms
  % \item Learn to choose best-suited algorithm by weighing strengths and 
  % weaknesses
  % \item Apply algorithms more effectively
}

% ------------------------------------------------------------------------------

\begin{document}

\lecturechapter{}
\lecture{Introduction to Machine Learning}

\begin{frame}{Contents}
  \tableofcontents
\end{frame}

\footnotesize

%\setdraft

\section{$k$-Nearest Neighbors ($k$-NN)}
\begin{frame}{$k$-NN -- method summary}

% \maketag{Supervised} 
\maketag{regression} \maketag{classification}
\maketag{Nonparametric} \maketag[50]{White-box}

\medskip

\highlight{General idea}
\begin{itemize}
  \item Rationale: \textbf{similarity} in feature space (w.r.t. certain \textbf{distance metric}) $\leadsto$ similarity in target space 
  % \item The \textbf{$k$-nearest neighbors ($k$-NN)} model is based on 
  % inter-observational \textbf{distances}, thus heavily depending on the chosen 
  % \textbf{distance measure}.
  \item \textbf{Prediction} for $\xv$: construct \textbf{$k$-neighborhood} 
  $N_k(\xv)$ from $k$ points closest to $\xv$ in $\Xspace$, then 
  predict
  \begin{itemize}
    \footnotesize
    \item (weighted) mean target for \textbf{regression}: 
    $\yh = \tfrac{1}{\sum\limits_{i: \xi \in N_k(\xv)} w_i}  
    \sum\limits_{i: \xi \in N_k(\xv)} w_i \yi $ with $w_i = \tfrac{1}{d(\xi,\xv)}$\\
    $\rightarrow$ optional: higher weights $w_i$ for close neighbors
    \item most frequent class for \textbf{classification}: 
    $\yh = \underset{\ell \in \gset}{\mathrm{\argmax}} \sum\limits_{i: \xi \in N_k(\xv)} \I(\yi = \ell)$\\
    $\Rightarrow$ Estimating posterior probabilities as $\hat{\pi}_{\ell}(\xi)= \frac{1}{k} \sum\limits_{i: \xi \in N_k(\xv)} \I(\yi = \ell)$
  \end{itemize}
  %\item No distributional or functional \textbf{assumptions}
  \item \textbf{Nonparametric} behavior: parameters = training data; no 
  compression of information
  \item Not immediately interpretable, but inspection of neighborhoods can be revealing
\end{itemize}
\end{frame}

% ------------------------------------------------------------------------------

\begin{frame}{$k$-NN -- method summary}

\vspace{1.5cm}

\highlight{Hyperparameters} ~~ Neighborhood \textbf{size} $k$ (locality), 
\textbf{distance} metric (next page)

%\vfill

\begin{minipage}{0.7\textwidth}
  \includegraphics[width=\textwidth]{figure/knn-neighborhood.pdf}
\end{minipage}%
\hfill
\begin{minipage}{0.25\textwidth}
  \tiny
  \raggedright
  \textit{Left}: Neighborhood for exemplary observation in \texttt{iris}, 
  $k = 50$ \\
  \textit{Right}: Prediction surfaces for $k \in \{1, 50\}$
\end{minipage}

\medskip

\begin{itemize}
    \item Small $k$ $\Rightarrow$ very local, "wiggly" decision boundaries
    \item Large $k$ $\Rightarrow$ rather gobal, smooth decision boundaries
\end{itemize}

\end{frame}

% ------------------------------------------------------------------------------

\begin{frame}{$k$-NN -- method summary}

\highlight{Popular distance metrics}

\begin{itemize}
  \item Numerical feature space:\\
  \begin{minipage}{0.7\textwidth}
  $\Rightarrow$ Typically, \textbf{Minkowski} distances
  $d(\xv, \xtil) = \|\xv - \xtil \|_q = 
  \left( \sum_j | x_j - \tilde{x_j} |^q
  \right)^{\tfrac{1}{q}}$
  \begin{itemize}
    \item $q = 1$: \textbf{Manhattan} distance $\rightarrow d(\xv, \xtil) =
    \sum_j | x_j - \tilde{x_j} |$
  \item $q = 2$: \textbf{Euclidean} distance $\rightarrow d(\xv, \xtil) =
  \sqrt{\sum_j (x_j - \tilde{x_j})^2}$
  \item Visualization: Manhatten (red, blue, yellow) vs. Euclidean (green)
  \end{itemize}
  %\item In presence of categorical features: \textbf{Gower} distance
\end{minipage}%
\begin{minipage}{0.25\textwidth}
 \begin{center}
  \includegraphics[width=.6\textwidth]{learners-overview/figure/manhattan_distance.png} %https://es.m.wikipedia.org/wiki/Archivo:Manhattan_distance.svg
 \end{center}
\end{minipage}
    \medskip
  \item Mixed feature space: 
  \begin{itemize}
%      \item Minkowski distances not applicable anymore
      \item \textbf{Gower distance} (defined as \textit{similarity} by Gower, 1971):\\ %https://www.jstor.org/stable/2528823?seq=3#metadata_info_tab_contents
            - numerical features: $s_{ij}(x_i,x_j) =  1 - \dfrac{|x_i-x_j|}{R_j}$\\
            - categoricals: $s_{ij}(x_i,x_j) =
            \begin{cases}
              1, \text{if\,} x_i = x_j\\
              0, \text{if\,} x_i \neq x_j
            \end{cases}$\\
            - Gower distance as average over individual scores
  \end{itemize}
  %\item \textbf{Custom} distance measures applicable
  \item Optional \textbf{weighting} to account for beliefs about varying feature
  importance
\end{itemize}
\vfill
  {\tiny Figure Source: \href{https://es.m.wikipedia.org/wiki/Archivo:Manhattan_distance.svg}{https://es.m.wikipedia.org/wiki/Archivo:Manhattan\_distance.svg}}
\end{frame}

% ------------------------------------------------------------------------------

\begin{frame}{$k$-NN -- Practical hints}

\highlight{Implementation}
\begin{itemize}
  \item \textbf{R:} \texttt{mlr3} learners (calling \texttt{kknn::kknn()})
  \begin{itemize}
    \item \textbf{Classification:}\\ 
    - \texttt{LearnerClassifKKNN}\\
    - \texttt{fnn::knn()}
    \item \textbf{Regression:}\\
    - \texttt{LearnerRegrKKNN}\\
    - \texttt{fnn::knn.reg()}
    \item Nearest Neighbour Search in $\order(N \log N)$: \texttt{RANN::nn2()}
  \end{itemize}
  \item \textbf{Python:} From package \texttt{sklearn.neighbors} 
  \begin{itemize}
    \item \textbf{Classification:}\\ 
    - \texttt{KNeighborsClassifier()}\\
    - \texttt{RadiusNeighborsClassifier()} as alternative if data not uniformly sampled
    \item \textbf{Regression:}\\
    - \texttt{KNeighborsRegressor()} \\
    - \texttt{RadiusNeighborsRegressor()} as alternative if data not uniformly sampled
  \end{itemize}
\end{itemize}

\end{frame}
% ------------------------------------------------------------------------------

\begin{frame}{$k$-NN -- Pros \& Cons}

\footnotesize

\begin{columns}[onlytextwidth]
  \begin{column}{0.5\textwidth}
    \highlight{Advantages}
    \footnotesize
    \begin{itemize}
      \positem Algorithm \textbf{easy} to explain and implement
      % \positem Applicable to both regression and classification
      \positem No distributional or functional \textbf{assumptions}\\
      $\rightarrow$ able to model data of \textbf{arbitrary complexity} %(in theory) 
      \positem No \textbf{training} or \textbf{optimization} required 
      %\positem Constant evolvement with \textbf{new data}
      \positem \textbf{local model} $\rightarrow$ \textbf{nonlinear} decision boundaries
      \positem Easy to \textbf{tune} (few hyperparameters)\\
      $\rightarrow$ e.g. number of neighbors $k$, distance metric
      % \positem Only one \textbf{hyperparameter} to tune
      \positem \textbf{Custom} distance metrics can often be easily designed to incorporate domain knowledge
    \end{itemize}
  \end{column}
  \begin{column}{0.5\textwidth}
    \highlight{Disadvantages}
    \footnotesize
    \begin{itemize}
      \negitem Sensitivity w.r.t. \textbf{noisy} or \textbf{irrelevant} features and outliers due to dependency on distance measure
      \negitem Heavily affected by \textbf{curse of dimensionality}
      \negitem Bad performance when feature \textbf{scales} are not consistent with feature relevance
      \negitem Poor handling of data \textbf{imbalances} (worse for more global model, i.e., large $k$)
      %\negitem High \textbf{memory} consumption of distance computation
    \end{itemize}
  \end{column}
\end{columns}

\vfill

\small

\conclbox{Easy and intuitive for small, well-behaved datasets with meaningful 
feature space distances}

\end{frame}


\section{Generalized Linear Models (GLM)}
\begin{vbframe}{(Generalized) Linear Models -- method summary}

% \maketag{SUPERVISED} 
\maketag{regression} \maketag{classification} \maketag{PARAMETRIC} 
\maketag{WHITE-BOX} \maketag[50]{Feature selection}
\medskip

\highlight{General idea} ~~ Represent target as function of linear predictor
$\thx$ (weighted sum of features)\\
$\rightarrow$ \textbf{Interpretation:} if feature $x_j$ increases by 1 unit, the linear predictor changes by $\theta_j$ units


\medskip

\highlight{Hypothesis space} ~~
$\Hspace = \left\{f: \Xspace \to \R ~|~\fx = \phi(\thetab^\top \xv)\right\}$, 
with suitable transformation $\phi(\cdot)$, e.g.,

\begin{itemize}
  \item \textbf{Linear Regression}: $\Yspace = \R$, $\phi$ identity
  % $\rightarrow$ continous output
  \item \textbf{Logistic Regression}: $\Yspace = \{0, 1\}$, logistic sigmoid $\phi(\thx) = \frac{1}{1 + \exp(- \thx)} 
  =: \pixt$\\
  $\Rightarrow$ Decision rule: Linear hyperplane
\end{itemize}
\begin{minipage}[b]{0.24\textwidth}
  \begin{center}
    \includegraphics[width=0.8\textwidth, trim=0 40 0 0, clip]{
    figure/lm_3d.png} \\
    \tiny{Linear regression hyperplane}
  \end{center}
\end{minipage}
\begin{minipage}[b]{0.24\textwidth}
  \begin{center}
     \includegraphics[width=0.7\textwidth, trim=0 40 0 0, clip]{../slides/supervised-classification/figure/reg_class_log_1}\\
    \tiny{Logistic sigmoid function}
  \end{center}
\end{minipage}
\begin{minipage}[b]{0.24\textwidth}
  \begin{center}
    % \includegraphics[width=0.7\textwidth, trim=0 40 0 0, clip]{
    % ../slides/supervised-classification/figure_man/logreg-2vars-surface}\\
    \includegraphics[width=0.65\textwidth, trim=30 50 0 0, clip]{
    figure/logreg_3d}\\
    \tiny{Logistic function for bivariate input and loss-minimal $\thetab$}
  \end{center}
\end{minipage}
\begin{minipage}[b]{0.24\textwidth}
  \begin{center}
    \includegraphics[width=0.5\textwidth]{figure/logreg_2d}\\
    % \includegraphics[width=0.6\textwidth]{
    % ../slides/supervised-classification/figure_man/logreg-2vars-data} \\
    \tiny{Corresponding separating hyperplane}
\end{center}
\end{minipage}

\framebreak

\highlight{Loss functions}

\begin{itemize}
  \item \textbf{Lin. Regr.}:
  \begin{itemize}
    
    \item Typically, based on \textbf{quadratic} loss (OLS estimation): 
    $$L\left(\yi, \fxi\right) = \left(\yi - \fxi \right)^2$$ %~~
    %$\Rightarrow$ OLS estimation
    % \item Alternatives: e.g., \textbf{absolute} or \textbf{Huber} loss (both 
    % improving robustness)
  \end{itemize}
  \item \textbf{Log. Regr.}: Based on \textbf{bernoulli / log / cross-entropy} loss ~ 
  %$\Rightarrow \risket = \sumin -\yi \log \left(\pixii\right) - (1 - \yi) \log \left(1 - \pixii \right)$
  \begin{itemize}
      \item Loss based on scores
      \begin{eqnarray*}
    L\left(\yi, \fxi\right) &=& \ln\left(1+\exp\left(-\yi \cdot \fxi\right)\right) \quad \text{for } \yi \in \setmp\\
    L\left(\yi, \fxi\right) &=& - \yi \cdot \fxi + \log\left(1 + \exp\left(\fxi\right)\right) \quad \text{for } \yi \in \setzo 
    \end{eqnarray*}
    \item Loss based on probabilities:
      \begin{eqnarray*}
    L\left(\yi, \pixii\right) &=& \ln\left(1+\exp\left(-\yi \cdot \log \left(\pixii\right)\right)\right) \quad \text{for } \yi \in \setmp\\
    L\left(\yi, \pixii\right) &=& -\yi \log \left(\pixii\right) - (1 - \yi) \log \left(1 - \pixii \right)  \quad \text{for } \yi \in \setzo 
    \end{eqnarray*}
  \end{itemize}
\end{itemize}

\framebreak

\medskip

\highlight{Optimization} 

\begin{itemize}
    \item Minimization of the empirical risk
    \item For \textbf{OLS}: analytical solution $\thetabh = \olsest$
    \item For other loss functions: 
    \begin{itemize}
        \item \textbf{Log. Regr.}: Convex problem, solvable via second-order optimization methods (e.g. BFGS)
        \item \textbf{Else}: Numerical optimization 
    \end{itemize}
    
\end{itemize}

\medskip

% \highlight{Hyperparameters} ~~ None

%\medskip

\highlight{Multi-class extension of logistic regression}

\begin{itemize}
  \item Estimate \textbf{class-wise} scoring functions:
  $\Rightarrow \pi: \Xspace \rightarrow \unitint^g, ~
  \pix = (\pikx[1], \dots, \pikx[g]), ~\sumkg \pikx = 1$
  \item Achieved through \textbf{softmax} transformation: 
  $\pikx = \exp(\thetab_k^\top \xv) \big/ \sum_{j=1}^g \exp(\thetab_j^\top 
  \xv)  $
  \item Multi-class log-loss: $\Lpixy = - \sumkg \I_{\{y = k\}} \log(\pikx)$
  \item Predict class with maximum score (or use thresholding variant)
\end{itemize}

% \highlight{Runtime behavior} ~~ $\mathcal{O}(p^2 \cdot n + p^3)$ for $n$ 
% observations and $p$ features

\end{vbframe}


% ------------------------------------------------------------------------------

\begin{frame}{(Generalized) Linear Models -- regularization}

\highlight{General idea}

\begin{itemize}
  \item Unregularized LM: risk of \textbf{overfitting} in high-dimensional 
  space with only few observations
  %\item \textbf{Goal}: find compromise between model fit and generalization by adding \textbf{penalty term}
  \item \textbf{Goal}: avoidance of overfitting by adding \textbf{penalty term}
%  \item Regularization ubiquitous in ML, with similar techniques
\end{itemize}

%\medskip

% \highlight{Hypothesis space} ~~\\
% $\Hspace = \left\{f: \Xspace \to \R ~|~\fx = \phi(\thetab^\top \xv)\right\}$, 
% where $\phi(\cdot)$ is a transformation function.


%$\Hspace = \{ \theta_0 + \thx\ |\ (\theta_0, \thetab) \in \R^{p+1} \} $

\medskip
\begin{columns}[T, totalwidth=\textwidth]
    \begin{column}{0.55\textwidth}
        
\highlight{Regularized empirical risk}

\begin{itemize}
  \item Empirical risk function \textbf{plus complexity penalty} 
  $J(\thetab)$, controlled by shrinkage parameter $\lambda > 0$:
  $\riskrt := \risket + \lambda \cdot J(\thetab)$
  %\item Popular regularizers
  %\begin{itemize} 
    \item \textbf{Ridge} regression: L2 penalty $J(\thetab) = \|\thetab\|_2^2 $
    \item \textbf{LASSO} regression: L1 penalty $J(\thetab) = \|\thetab\|_1 $
  %\end{itemize}
\end{itemize}

\medskip

\highlight{Optimization under regularization}
\begin{itemize}
  \item \textbf{Ridge}: analytically with 
  $\thetabh_{\text{Ridge}} = (\Xmat^\top \Xmat  + \lambda \id)^{-1} \Xmat^\top 
  \yv$
  \item \textbf{LASSO}: numerically with, e.g., (sub-)gradient descent
\end{itemize}



    \end{column}
        \begin{column}{0.45\textwidth}
          \includegraphics[width=\textwidth]{figure/l1_l2_hat.png}
    \end{column}
\end{columns}

\medskip
\highlight{Choice of regularization parameter}

\begin{itemize}
  \item Standard hyperparameter optimization problem
  \item E.g., choose $\lambda$ with minimum mean cross-validated error 
  %(default in R package \texttt{glmnet})
\end{itemize}
\end{frame}

% ------------------------------------------------------------------------------

\begin{frame}{(Generalized) Linear Models -- regularization}

  \begin{columns}[T, totalwidth=\textwidth]
  \begin{column}{0.62\textwidth}

\highlight{Ridge vs. LASSO} 

\begin{itemize}
  \item \textbf{Ridge}
  \begin{itemize} 
    \item Global shrinkage $\Rightarrow$ overall smaller but still dense $\thetab$
    \item Applicable with large number of influential features, handling 
    correlated variables by shrinking their coefficients by equal amount
  \end{itemize}
  \item \textbf{LASSO}
  \begin{itemize} 
    \item Actual variable selection by shrinking coefficients for irrelevant 
    features all the way to zero
    \item Suitable for sparse problems, ineffective with correlated 
    features (randomly selecting one)
  \end{itemize}  
  \item Neither overall better $\Rightarrow$ compromise: \textbf{elastic net}
  \begin{itemize} 
    \item Weighted combination of Ridge and LASSO
    \item Introducing additional penalization coefficient: %$\riskrt = \risket + \lambda_1 \cdot \|\thetab\|_1 + \lambda_2 \cdot \|\thetab\|_2^2$\\
    $\riskrt = \risket 
    + \lambda \cdot P_{\alpha}(\thetab)$, with\\
    $P_{\alpha}(\thetab) = [\alpha \cdot \|\thetab\|_1 + (1 - \alpha) \cdot \frac{1}{2} \cdot \|\thetab\|_2^2]$
  \end{itemize}  
\end{itemize}

\end{column}

\begin{column}{0.38\textwidth}
\tiny
\centering
\begin{columns}[T, totalwidth=\textwidth]
\begin{column}{0.5\textwidth}
\tiny
\begin{center}
\textbf{Ridge} performs better for correlated features: \\ 
\medskip
$\boldsymbol{\beta}=(\underbrace{2,\ldots,2}_{5},\underbrace{0,\ldots,0}_{5})$\\
$ \operatorname{cor}(\Xmat_{i},\Xmat_{j})=0.8^{|i-j|}, \forall i, j$
  \end{center}
\end{column}
\begin{column}{0.5\textwidth} \tiny

\begin{center}
\textbf{Lasso} performs better for uncorrelated features: \\
\medskip
$\boldsymbol{\beta}=(2, 2, 2,\underbrace{0,\ldots,0}_{7})$ \\
$\operatorname{cor}(\Xmat_{i},\Xmat_{j})= 0, \forall i \neq j$
\end{center}
\end{column}
\end{columns}



          \includegraphics[width=\textwidth]{figure/enet_lasso_ridge_mse.png}
          
          \includegraphics[width=\textwidth]{figure/enet_tradeoff.png}
\end{column}

\end{columns}

\end{frame}

% ------------------------------------------------------------------------------

\begin{frame}{(Generalized) Linear Models -- Implementation}

\highlight{Implementation}

\begin{itemize}
  \item \textbf{R:}
  \begin{itemize}
    \item \textbf{Unregularized:} \texttt{mlr3} learner \texttt{LearnerRegrLM}, 
    calling \texttt{stats::lm()} / \texttt{mlr3} learner 
    \texttt{LearnerClassifLogReg}, calling \texttt{stats::glm()}
    \item \textbf{Regularized / ElasticNet:} \texttt{mlr3} learners 
    \texttt{LearnerClassifGlmnet} / 
    \texttt{LearnerRegrGlmnet}, calling \texttt{glmnet::glmnet()}
    \item For \textbf{large classification} data: \texttt{mlr3} learner     
    \texttt{LearnerClassifLiblineaR}, calling \texttt{LiblineaR::LiblineaR()} uses fast coordinate descent
  \end{itemize}
  \item \textbf{Python:} From package \texttt{sklearn.linear\_model} 
  \begin{itemize}
    \item \textbf{Unregularized:}\\ 
    - \texttt{LinearRegression()}\\
    - \texttt{LogisticRegression(penalty = None)}
    \item \textbf{Regularized:}\\
    - \textit{Linear regression:} \texttt{Lasso(),Ridge(),ElasticNet()} \\
    - \textit{Logistic regression:} \texttt{LogisticRegression(penalty = \{‘l1’, ‘l2’, ‘elasticnet’\})}
    \item Package for advanced \textbf{statistical} models: \texttt{statsmodels.api} 
  \end{itemize}
\end{itemize}

%% WOULD DELTETE THIS!
%  \highlight{\textcolor{blue}{Check assumptions??} }\\
% Linear models are effective if the following assumptions are fulfilled:
%  \begin{itemize}
%   \item \textbf{linearity}: The expected response is a linear combination of the features.
%   \item \textbf{homoscedasticity}: The variance of residuals is equal for all features.
%   \item \textbf{independence}: All observations are independent of each other.
%   \item \textbf{normality}: Y is normally distributed for any fixed value of the features
% \end{itemize}

\end{frame}


% ------------------------------------------------------------------------------

\begin{frame}{(Generalized) Linear Models -- Pros \& Cons}



\begin{columns}[onlytextwidth]
  \begin{column}{0.5\textwidth}
    \highlight{Advantages}
    
    \begin{procon}
      \setlength{\itemsep}{1pt}
      \setlength{\parskip}{1pt}
      \positem \textbf{Simple and fast} implementation
      \positem \textbf{Analytical} solution for L2 loss
      %\positem \textbf{Cheap} computation
      \positem Applicable for any \textbf{dataset size}, as long as number of 
      observations $\gg$ number of features
      \positem Flexibility \textbf{beyond linearity} with polynomials, 
      trigonometric transformations, interaction terms etc.
      \positem Intuitive \textbf{interpretability} via feature effects
      % \item fits \textbf{linearly} separable data sets very well
      \positem Statistical hypothesis \textbf{tests} for effects available
    \end{procon}
  \end{column}

  \begin{column}{0.5\textwidth}
    \highlight{Disadvantages}
    
    \begin{itemize}
      \negitem \textbf{Nonlinearity} of many real-world problems
      \negitem Further restrictive \textbf{assumptions}: linearly independent 
      features, homoskedastic residuals, normality of conditional response 
      %\textcolor{blue}{actually relevant in ML?}
      \negitem \textbf{Sensitivity} w.r.t. outliers and noisy data (especially 
      with L2 loss)
      \negitem Also a LM can \textbf{overfit} (e.g., many features and few observations) 
      \negitem Feature \textbf{interactions} must be handcrafted\\
      $\rightarrow$ practically infeasible for higher orders
      %\negitem No handling of \textbf{missing} data
    \end{itemize}
  \end{column}
\end{columns}

\end{frame}

% \section{Regularized Linear Models}
% \begin{frame}{Regularized LM -- Functionality}

\highlight{General idea}

\begin{itemize}
  \item Unregularized LM: risk of \textbf{overfitting} in high-dimensional 
  space with only few observations
  \item \textbf{Goal}: find compromise between model fit and generalization
\end{itemize}

\medskip

% \highlight{Hypothesis space} ~~\\
% $\Hspace = \left\{f: \Xspace \to \R ~|~\fx = \phi(\thetab^\top \xv)\right\}$, 
% where $\phi(\cdot)$ is a transformation function.


%$\Hspace = \{ \theta_0 + \thx\ |\ (\theta_0, \thetab) \in \R^{p+1} \} $

\medskip

\highlight{Empirical risk}

\begin{itemize}
  \item Empirical risk function \textbf{plus complexity penalty} 
  $J(\thetab)$, controlled by shrinkage parameter $\lambda > 0$: \\
  $\riskrt := \risket + \lambda \cdot J(\thetab).$ 
  \item Popular regularizers
  \begin{itemize} 
    \item \textbf{Ridge} regression: L2 penalty $J(\thetab) = \|\thetab\|_2^2 $
    \item \textbf{LASSO} regression: L1 penalty $J(\thetab) = \|\thetab\|_1 $
  \end{itemize}
\end{itemize}

\medskip

\highlight{Optimization}
\begin{itemize}
  \item \textbf{Ridge}: analytically with 
  $\thetabh_{\text{Ridge}} = (\Xmat^\top \Xmat  + \lambda \id)^{-1} \Xmat^\top 
  \yv$
  \item \textbf{LASSO}: numerically with, e.g., (sub-)gradient descent
\end{itemize}

\medskip

\highlight{Hyperparameters} ~~ Shrinkage parameter $\lambda$

\medskip

% \highlight{Runtime behavior} ~~ $\mathcal{O}(p^2 \cdot n + p^3)$ for $n$ 
% observations and $p$ features 

\end{frame}

% ------------------------------------------------------------------------------

\begin{frame}{Regularized LM -- Practical hints}

\highlight{Choice of regularization parameter}

\begin{itemize}
  \item Standard hyperparameter optimization problem
  \item E.g., choose $\lambda$ with minimum mean cross-validated error 
  (default in R package \texttt{glmnet})
\end{itemize}

\medskip

\highlight{Ridge vs. LASSO} 

\begin{itemize}
  \item \textbf{Ridge}
  \begin{itemize} 
    \item Overall smaller, but still dense $\thetab$
    \item Suitable with many influential features present, handling correlated 
    features by shrinking their coefficients equally
  \end{itemize}
  \item \textbf{LASSO}
  \begin{itemize} 
    \item Actual variable selection
    \item Suitable for sparse problems, ineffective with correlated 
    features (randomly selecting one)
  \end{itemize}  
  \item Neither overall better -- compromise: \textbf{elastic net} \\
  $\rightarrow$ weighted 
  combination of Ridge and LASSO regularizers
\end{itemize}

\medskip

\highlight{Implementation}

\begin{itemize}
    \item \textbf{R:} \texttt{mlr3} learners \texttt{LearnerClassifGlmnet} / 
    \texttt{LearnerRegrGlmnet}, calling \texttt{glmnet::glmnet()}
    \item \textbf{Python:} \texttt{LinearRegression} from package 
    \texttt{sklearn.linear\_model}, package for advanced statistical parameters 
    \texttt{statsmodels.api} 
  \end{itemize}

\end{frame}


\section{Generalized Additive Models (GAM)}
\begin{vbframe}{Generalized Additive Models -- method summary}

% \maketag{SUPERVISED} 
\maketag{regression} \maketag{classification} \maketag[50]{(NON)PARAMETRIC}
\maketag[50]{WHITE-BOX} \maketag[50]{Feature selection}
\medskip

\highlight{General idea}
\begin{itemize}
  \item Same as GLM, but introduce \textbf{flexibility} through
  \textbf{nonlinear (smooth)} effects $f_j(x_j)$
  \item Typically, combination of linear \& smooth effects
  \item Smooth effects also conceivable for feature interactions
\end{itemize}
\medskip

\highlight{Hypothesis space} ~~
$\Hspace = \left\{f: \Xspace \to \R ~|~\fx = \phi \left(\theta_0 + \textstyle \sumjp
f_j(x_j) \right) \right\}$,
with suitable transformation $\phi(\cdot)$, intercept term $\theta_0$, and smooth
functions $f_j(\cdot)$

\includegraphics[width=0.28\textwidth]{figure/gam_bike_partial_effect}
\includegraphics[width=0.35\textwidth, trim=0 0 0 80, clip]{figure/gam_bike_pred}

\tiny
Prediction of \texttt{bike rentals} from smooth term of \texttt{humidity}
(left: partial effect) and linear term of \texttt{temperature} (right: bivariate
prediction).
\normalsize

\framebreak

\highlight{Smooth functions}

\begin{itemize}
  \item Nonparametric/semiparametric/parametric approaches conceivable
  \item Frequently: express $f_j$ as weighted sum of \textbf{basis functions}
  $\rightsquigarrow$ model \textbf{linear} in weight coefficients again
  \begin{itemize}
      \item Use fixed basis of functions $b_1, \dots, b_K$ and estimate
      associated coefficients $\gamma_1, \dots, \gamma_K$ \\ $\rightsquigarrow$
      $f_j(x_j) = \textstyle \sum_{k=1}^{K_j} \gamma_{j, k} b_k(x_j)$
      \item Popular types of basis functions
      \begin{itemize}
        \footnotesize
        \item Polynomial $\rightsquigarrow$ smoothing/TP-/B-\textbf{splines}
        \item Radial $\rightsquigarrow$ \textbf{Kriging}
        \item Trigonometric $\rightsquigarrow$ \textbf{Fourier/wavelet} forms
      \end{itemize}
    \end{itemize}
    \item Alternatives: \textbf{local regression (LOESS)}, other
    kernel-smoothing approaches, \dots
\end{itemize}

\includegraphics[width=0.7\textwidth]{figure/gam_bike_spline_basis}

\tiny
Left: B-spline basis with 9 basis functions.
Middle: BFs weighted with coefficients estimated for
\texttt{humidity}.
Right: sum of weighted BFs in black (= partial effect).

\end{vbframe}

% ------------------------------------------------------------------------------

\begin{frame}{Generalized Additive Models -- method summary}

\highlight{Regularization}
\begin{itemize}
    \item Smooth functions possibly very flexible $\rightsquigarrow$
    regularization vital to prevent overfitting
    \item Control \textbf{smoothness}
    \begin{itemize}
      \item \textbf{Basis-function approaches}: control number; impose penalty
      on coefficients
      (e.g., magnitude or differences between coefficients of neighboring
      components) \& control associated hyperparameter
      \item \textbf{Local smoothers}: control width of smoothing window
      (larger $\rightsquigarrow$ smoother)
    \end{itemize}
\end{itemize}

\begin{minipage}{0.65\textwidth}
\includegraphics[width=0.45\textwidth, trim=0 0 80 80, clip]{
figure/gam_bike_pred}
\includegraphics[width=0.45\textwidth, trim=80 0 0 80, clip]{
figure/gam_bike_pred_wiggly}
\hfill
\end{minipage}
\begin{minipage}{0.3\textwidth}
\tiny \raggedright
Prediction surfaces for \texttt{bike rentals} with 9 (left) and 500 (right)
basis functions in smooth \texttt{humidity} term.
Higher number of basis functions yields more local, less smooth model.
\end{minipage}

\medskip

\highlight{Loss functions} ~~ Same as in GLM $\rightsquigarrow$ essentially:
use \textbf{negative log-likelihood}

\medskip

\highlight{Optimization}
\begin{itemize}
  \item \textbf{Coefficients} (of smooth + linear terms):
  penalized MLE, Bayesian inference
  \item \textbf{Smoothing hyperparameters}: typically, generalized
  cross-validation
\end{itemize}

\end{frame}

% ------------------------------------------------------------------------------

\begin{frame}{Generalized Additive Models -- implementation}

\highlight{Implementation}

\begin{itemize}
  \item \textbf{R:} \texttt{mlr3} learner \texttt{LearnerRegrGam},
    calling \texttt{mgcv::gam()}
  \begin{itemize}
      \item Smooth terms: \texttt{s(\dots, bs="<basis>")} or \texttt{te(\dots)}
      for multivariate (tensorproduct) effects
      \item Link functions: \texttt{family=$\{$Gamma, Binomial, \dots $\}$}
  \end{itemize}
    \item \textbf{Python}: \texttt{GLMGam} from package \texttt{statsmodels};
    package \texttt{pygam}
\end{itemize}

\medskip
\begin{columns}[onlytextwidth]
  \begin{column}{0.5\textwidth}
    \highlight{Advantages}

%    Strengths of GLMs, plus \dots
    \begin{itemize}
      \positem \textbf{Simple and fast} implementation
%      \positem \textbf{Analytical} solution for L2 loss
      %\positem \textbf{Cheap} computation
      \positem Applicable for any \textbf{dataset size}, as long as number of
      observations $\gg$ number of features
      \positem High \textbf{flexibility} via smooth effects
      \positem Easy to \textbf{combine} linear \& nonlinear effects
      \positem Intuitive \textbf{interpretability} via feature effects (though
      not quite as straightforward as in GLM)
      % \item fits \textbf{linearly} separable data sets very well
      \positem Statistical hypothesis \textbf{tests} for effects available
    \end{itemize}
  \end{column}

  \begin{column}{0.5\textwidth}
    \highlight{Disadvantages}

%    Shortcomings of GLMs, plus \dots
    \begin{itemize}
      \negitem \textbf{Sensitivity} w.r.t. outliers and noisy data
      \negitem Feature \textbf{interactions} must be handcrafted\\
      $\rightarrow$ practically infeasible for higher orders
      \negitem Harder to \textbf{optimize} than GLM
      \negitem Additional \textbf{hyperparameters} (type of smooth functions,
      smoothness degree, \dots)
    \end{itemize}
  \end{column}
\end{columns}

\end{frame}


\section{Classification \& Regression Trees (CART)}
\begin{frame}{CART -- Functionality}

\footnotesize

% \maketag{Supervised} 
\maketag{regression | classification} 
\maketag{Nonparametric} \maketag{White-box} \maketag{Feature selection}

\medskip

\highlight{General idea}
\begin{itemize}
  \item Starting from root node containing all data, perform repeated 
  \textbf{binary splits}, thereby subsequently dividing input space into $T$ 
  \textbf{rectangular partitions} $Q_t$
  \begin{itemize}
    \item In each step, find \textbf{optimal split} (feature-threshold 
    combination) $\rightarrow$ greedy search
    \item Assign same response $c_t$ to all observations in terminal region 
    $Q_t$
  \end{itemize}
  \item Splits based on node \textbf{impurity}, equivalently interpretable as 
  \textbf{ERM}
  % \item Unless interrupted, splitting continues until each observation ends up 
  % in its own leaf node $\rightarrow$ \textbf{control complexity}
\end{itemize}

\medskip
 
\highlight{Hypothesis space} ~~
$\Hspace = \left\{ \fx: \fx = \sum_{t = 1}^T c_t \I(\xv \in Q_t) 
\right\}$

\medskip

\begin{minipage}[b]{0.5\textwidth}
  \includegraphics[width=\textwidth]{../slides/trees/figure/cart_treegrow_32} \\
  \tiny{Classification tree for \texttt{iris} data after 3 splits}
\end{minipage}
\begin{minipage}[b]{0.49\textwidth}
  \includegraphics[width=\textwidth]{
  ../slides/trees/figure/cart_splitcriteria_1} \\
  \tiny{Corresponging prediction surface with axis-aligned boundaries}
\end{minipage}%

\end{frame}

% ------------------------------------------------------------------------------

\begin{frame}{CART -- Functionality}

\footnotesize

\highlight{Empirical risk} \\

\begin{itemize}
  \item Calculated for each potential terminal node $\Np_t$
  of a split
  \item In general, compatible with arbitrary losses -- typical choices:
  \begin{itemize}
    \footnotesize
    \item $g$-way classification:
    \begin{itemize}
      \footnotesize
      \item \textbf{Brier score} ~~
      $\risk(\Np_t) = \sum\limits_{(\xv,y) \in \Np_t} \sumkg \left( \I(y = k)
      - \pikx \right)^2$ ~~ $\rightarrow$ \textbf{Gini} impurity
      \item \textbf{Bernoulli} loss ~~
      $\risk(\Np_t) = \sum\limits_{(\xv,y) \in \Np_t} \sumkg \I(y = k) \cdot
      \log(\pikx)$ ~~ $\rightarrow$ \textbf{entropy} impurity
    \end{itemize}
    \item Regression: \textbf{quadratic} loss ~~
    $\risk(\Np_t) = \sum\limits_{(\xv,y) \in \Np_t} (y - c_t)^2$
  \end{itemize}
\end{itemize}

\medskip

\highlight{Optimization}

\begin{itemize}
  \item \textbf{Exhaustive} search over all split candidates, choice of 
  risk-minimal split
  \item In practice: limit number of candidates, use tricks to avoid 
  combinatorial explosion
\end{itemize}

\medskip

\highlight{Hyperparameters} ~~ \textbf{Complexity}, i.e., 
number of leaves $T$ (controlled indirectly, see \textit{Implementation}) 

\end{frame}

% ------------------------------------------------------------------------------

\begin{frame}{CART -- Pro's \& Con's}

\begin{columns}[onlytextwidth]
  \begin{column}{0.5\textwidth}
    \highlight{Advantages}
    \footnotesize
    \begin{itemize}
      \positem \textbf{Easy} to understand \& visualize
      \positem Highly \textbf{interpretable}
      \positem Built-in \textbf{feature selection}
      \positem Applicable to \textbf{non-numerical} features
      \positem Automatic handling of \textbf{missings} 
      \positem \textbf{Interaction} effects between features naturally included, 
      even of higher orders
      \positem \textbf{Fast} computation and good scalability
      \positem High \textbf{flexibility} (custom split criteria or leaf-node 
      prediction rules)   
    \end{itemize}
  \end{column}
  \begin{column}{0.5\textwidth}
    \highlight{Disadvantages}
    \footnotesize
    \begin{itemize}
      \negitem Rather \textbf{poor generalization} when used stand-alone 
      \negitem High \textbf{variance/instability}: strong dependence on training 
      data
      \negitem Substantial risk of \textbf{overfitting}
      \negitem Not well-suited for modeling \textbf{linear} relationships
      \negitem \textbf{Bias} toward features with \textbf{many categories}
    \end{itemize}
  \end{column}
\end{columns}

\vfill

\small

\conclbox{Simple, good with feature selection and highly interpretable, but not 
the most performant learner}

\end{frame}

% ------------------------------------------------------------------------------

\begin{frame}{CART -- Practical hints}

\footnotesize

\highlight{Complexity control}

\begin{itemize}
  \item Unless interrupted, splitting continues until we have one observation per 
  leaf node (costly + overfitting)
  \item Limit tree growth via
  \begin{itemize}
    \item \textbf{Early stopping:} stop growth prematurely \\ $\rightarrow$ hard 
    to determine good stopping point before actually trying all combinations
    \item \textbf{Pruning:} grow to large size and cut back in risk-optimal 
    manner
  \end{itemize}
\end{itemize}

\medskip

\highlight{Bagging / boosting} ~~ 
As CART are highly \textbf{instable} predictors on their own, they are typically 
used as base learners in bagging (random forest) or boosting ensembles.

\medskip

\highlight{Implementation}
\begin{itemize}
  \item \textbf{R:} \texttt{mlr3} learners \texttt{LearnerClassifRpart} / 
    \texttt{LearnerRegrRpart}, calling \texttt{rpart::rpart()}
  \item \textbf{Python:} \texttt{DecisionTreeClassifier} / 
  \texttt{DecisionTreeRegressor} from package \texttt{scikit-learn}
  \item Complexity controlled via tree depth, minimum number of observations 
  per node, maximum number of leaves, minimum risk reduction per split, ...
\end{itemize}

\end{frame}


\section{Random Forests}
\begin{frame}{Random Forests -- Functionality}

% \maketag{SUPERVISED} 
\maketag{regression | classification}  
\maketag{NONPARAMETRIC} \maketag{BLACK-BOX} \maketag{FEATURE SELECTION}

\medskip

\highlight{General idea} 
\begin{itemize}
  \item Combine $M$ tree \textbf{base learners} into 
  \textbf{bagging ensemble}, fitting same learner on \textbf{bootstrap} data
  samples
   \begin{itemize}
    \item Use unstable, \textbf{high-variance} base learners $\rightarrow$
    let trees grow to full size
    \item Mitigate invididual trees' bias by promoting \textbf{decorrelation} 
    $\rightarrow$ use random subset of 
    candidate features for each split
  \end{itemize}
  \item \textbf{Prediction} via averaging (regression) or majority vote 
  (classification)
\end{itemize}

\medskip

\highlight{Hypothesis space} ~~
$\Hspace = \left\{ \fx: \fx = \frac{1}{M} \sum_{m = 1}^M \sum_{t = 1}^{T^{[m]}} 
c_t^{[m]} \I(\xv \in Q_t^{[m]}) \right\}$

\medskip

\begin{minipage}[b]{0.5\textwidth}
  % FIGURE SOURCE: https://docs.google.com/presentation/d/1xodP6ayu1Gay6mMKgzVWYEFmSoeG5kNuqsaTkFFmd78  /edit
  \centering
  \includegraphics[width=0.7\textwidth]{figure/rf-bagging} \\
  \tiny Schematic depiction of bagging process
\end{minipage}%
\begin{minipage}[b]{0.5\textwidth}
\centering
  \includegraphics[width=0.7\textwidth]{
  ../slides/forests/figure/cart_forest_intro_3} \\
  \tiny Prediction surface for \texttt{iris} data with 500-tree ensemble
\end{minipage}

\end{frame}

% ------------------------------------------------------------------------------

\begin{frame}{Random Forests -- Functionality}

\highlight{Empirical risk}

\begin{itemize}
  \item Applicable with \textbf{any} kind of loss function (just like tree base 
  learners)
  \item Computation of empirical risk for all potential child nodes in all trees
\end{itemize}

\medskip

\highlight{Optimization} ~~
\textbf{Exhaustive} search over all split candidates in each node of each tree
to minimize empirical risk in child nodes (greedy optimization) \\

\medskip

\highlight{Hyperparameters}

\begin{itemize}
  \item \textbf{Ensemble size}, i.e., number of trees
  \item \textbf{Complexity} of base learners
  \item \textbf{Number of split candidates}, i.e., number of features to be
  considered at each split \\
  $\rightarrow$ frequently used heuristics with total of $p$ features: 
  $\left \lfloor{\sqrt{p}}\right \rfloor$ for classification,
  $\left \lfloor{p/3}\right \rfloor$ for regression
\end{itemize}

\medskip

\highlight{Out-of-bag (OOB) error}
\begin{itemize}
  \item Compute ensemble prediction for observations outside individual 
  trees' bootstrap training sample \\ $\rightarrow$ unseen test points
  \item Use resulting loss as unbiased estimate of \textbf{generalization error}
\end{itemize}

% \highlight{Runtime behavior} ~~
% $\mathcal{O}(M \cdot n^2 \cdot p)$ for $M$ trees, $n$ observations and $p$ 
% features
  
\end{frame}

% ------------------------------------------------------------------------------

\begin{frame}{Random Forests -- Pro's \& Con's}

\begin{columns}[onlytextwidth]
  \begin{column}{0.5\textwidth}
    \highlight{Advantages}
    \footnotesize
    \begin{itemize}
      \positem Translation of most of \textbf{trees'} advantages (e.g., 
      feature selection, feature interactions)
      \positem Fairly good \textbf{good predictors}: mitigating base learners' 
      weakness through bagging
      \positem Quite \textbf{stable} w.r.t. changes in data
      \positem Good with \textbf{high-dimensional} data, even in presence of 
      noisy covariates
      % \positem Applicable to \textbf{unbalanced} data
      \positem Easy to \textbf{parallelize}
      \positem Rather easy to \textbf{tune}
      \positem Intuitive measures of \textbf{feature importance}
    \end{itemize}
  \end{column}
  \begin{column}{0.5\textwidth}
    \highlight{Disadvantages}
    \footnotesize
    \begin{itemize}
      \negitem Loss of trees' \textbf{interpretability} -- black-box 
      method
      \negitem Hard to \textbf{visualize}
      \negitem Often suboptimal for \textbf{regression}
      \negitem \textbf{Bias} toward features with \textbf{many categories}
      \negitem Often still inferior in \textbf{performance} to other methods 
      (e.g., boosting)
    \end{itemize}
  \end{column}
\end{columns}

\vfill

\small

\conclbox{Fairly good and stable predictor with built-in feature selection, but 
black-box method}

\end{frame}

% ------------------------------------------------------------------------------

\begin{frame}{Random Forests -- Practical hints}

\highlight{Pre-processing} ~~ Inherent feature selection, but high 
\textbf{computational cost} for large number of features \\
$\rightarrow$ upstream feature selection (e.g., via PCA) might be advisable

\medskip

\highlight{Feature importance}

\begin{itemize}
  \item Based on \textbf{improvement in split criterion:} aggregate improvements 
  by all splits using $j$-th feature
  \item Based on \textbf{permutation:} permute $j$-th feature in 
  OOB observations and compute impact on OOB error
\end{itemize}

\medskip

\highlight{Tuning} ~~ Number of split candidates often more impactful than 
number of trees

\medskip

\highlight{Implementation}

\begin{itemize}
  \item \textbf{R:} \texttt{mlr3} learners \texttt{LearnerClassifRanger} / 
    \texttt{LearnerRanger}, calling \texttt{ranger::ranger()}
  \item \textbf{Python:} \texttt{RandomForestClassifier} / 
  \texttt{RandomForestRegressor} from package \texttt{scikit-learn}
\end{itemize}

\end{frame}

\section{Gradient Boosting}
\begin{frame}{Gradient Boosting -- method summary}

% \maketag{supervised} 
\maketag{regression} \maketag{classification}
\maketag[50]{(NON)PARAMETRIC}
\maketag{BLACK-BOX}
\maketag{FEATURE SELECTION}

\medskip

\highlight{General idea}

\begin{itemize}
  \item \textbf{Sequential ensemble} of $M$ \textbf{base learners} by greedy forward stagewise additive modeling
  \begin{itemize}
      \item In each iteration a base learner is fitted to current \textbf{pseudo residuals} $\Rightarrow$ one boosting iteration is one approximate \textbf{gradient step in function space}
      \item Base learners are either \textbf{trees} or \textbf{linear regressions}
  \end{itemize}
  \item \textbf{Predict} via (weighted) sum of base learners
  
\end{itemize}

\medskip

\highlight{Hypothesis space} ~~
$\Hspace = \left\{ \fx: \fx = \sum_{m = 1}^M \betam b(\xv, \thetam) \right\}$

\begin{minipage}{0.45\textwidth}
  \centering
  \includegraphics[width=\textwidth, trim=0 0 450 0, clip]{
  figure/illustration_gaussian_huber_2_10} \\
  \tiny{Boosting prediction function with GAM base learners for univariate 
  regression problem after 10 iterations}
\end{minipage}%
\hfill
\begin{minipage}{0.45\textwidth}
  % FIGURE SOURCE: http://arogozhnikov.github.io/2016/06/24/gradient_boosting_explained.html
  % \includegraphics[width=\textwidth]{figure/gb-3d} \\
  \centering
  \includegraphics[width=\textwidth]{
  figure/boosting_multiclass_100} \\
  \tiny{Boosting prediction surface with tree base learners for \texttt{iris} 
  data after 100 iterations (\textit{right:} contour lines of discriminant 
  functions)}
\end{minipage}

\end{frame}

% ------------------------------------------------------------------------------

\begin{frame}{Gradient Boosting -- method summary}

\footnotesize

\highlight{Empirical risk}

\begin{itemize}
  \item In general, compatible with any \textbf{differentiable} loss
  \item Base learner in iteration $m$ is fitted on \textbf{Pseudo residuals}: $\tilde{r}^{(i)} = - \pd{\Lxyi}{\fxi}$ by minimizing the \textbf{L2-loss}: $\sumin (\rmi - b(\xi, \bm{\theta}))^2$
\end{itemize}

\medskip

\highlight{Optimization} ~~
\begin{itemize}
    \item Same optimization procedure as base learner, while keeping the current ensemble $\fmdh$ fixed
    \item $\betam$ is found via \textbf{line search} or fixed to a \textbf{small constant value} and combined with the leaf values $\ctm$ for tree base learners: $\ctmt = \betam \cdot \ctm$
\end{itemize}

\medskip

\highlight{Hyperparameters}

\begin{itemize}
  \item \textbf{Ensemble size}, i.e., number of base learners
  \item \textbf{Complexity} of base learners (depending on type used)
  \item \textbf{Learning rate}, i.e., impact of next base learner
\end{itemize}

\medskip

% \highlight{Runtime behavior} ~~ $\mathcal{O}(M \cdot n \cdot p)$ 
% for $M$ base learners, $n$ observations and $p$ features

\end{frame}

% ------------------------------------------------------------------------------

\begin{frame}{Gradient Boosting -- Pro's \& Con's}

\footnotesize

\begin{columns}[onlytextwidth]
  \begin{column}{0.5\textwidth}
    \highlight{Advantages}
    \footnotesize
    \begin{itemize}
      \positem Retains of most of \textbf{base learners'} advantages 
      \positem Very \textbf{good predctor}: mitigating base learners' weakness through ensembling
      \positem Often state-of-the-art results, only outperformed by heterogenous \textbf{stacking ensembles}
      \positem High \textbf{flexibility} via custom loss functions and choice of base learner
      % \positem Applicable to \textbf{unbalanced} data
    \end{itemize}
  \end{column}
  \begin{column}{0.5\textwidth}
    \highlight{Disadvantages}
    \footnotesize
    \begin{itemize}
      \negitem Loss of base learners' potential \textbf{interpretability}
      \negitem \textbf{Many hyperparameters} that need to be tuned carefully
      \negitem Hard to \textbf{parallelize}
    \end{itemize}
  \end{column}
\end{columns}

\vfill

\small

\conclbox{High-performing and flexible predictor, but rather delicate to handle}

\end{frame}

% ------------------------------------------------------------------------------

\begin{frame}{Gradient Boosting -- Practical hints}

\footnotesize

\highlight{Scalable Gradient Boosting} 

\begin{itemize}
  \item \textbf{Feature and data subsampling} for each base learner fit
  \item \textbf{Parallelization} and \textbf{approximate split finding} for tree base learners
  \item GPU accelaration
\end{itemize}

\medskip

\highlight{Explainable / Componentwise Gradient Boosting}
\begin{itemize}
    \item Base learners of \textbf{simple linear regression} models or \textbf{splines}, selecting a single feature in each iteration
    \item Allows \textbf{feature selection} and creates an \textbf{interpretable} model
    \item Feature interactions can be learned via ranking techniques (e.g., GA$^2$M FAST)
\end{itemize}

\medskip

\highlight{Tuning}
\begin{itemize}
    \item Use \textbf{early-stopping} to determine ensemble size
    \item Tune learning rate and base learner complexity hyperparameters on \textbf{log-scale}
\end{itemize}

\medskip

\highlight{Implementation}

\begin{itemize}
  \item \textbf{R:} \texttt{mlr3} learners \texttt{LearnerClassifXgboost} / 
  \texttt{LearnerRegrXgboost}, calling \texttt{xgboost::xgb.train()}
  \item \textbf{Python:} \texttt{GradientBoostingClassifier} / 
  \texttt{GradientBoostingRegressor} from package \texttt{scikit-learn}, 
  \texttt{XGBClassifier} / \texttt{XGBRegressor} from package \texttt{xgboost}
\end{itemize}

\end{frame}

\section{Linear Support Vector Machines (SVM)}
\begin{vbframe}{Linear SVM -- method summary}

% \maketag{SUPERVISED} 
\maketag{CLASSIFICATION} \maketag[50]{REGRESSION} \maketag{PARAMETRIC} 
\maketag{WHITE-BOX} 
\medskip

\highlight{General idea}

\begin{itemize}
  \item Find linear decision boundary (\textbf{separating hyperplane}) that 
  best discriminates classes
  \begin{itemize}
    \item \textbf{Hard-margin} SVM: maximize distance (\textbf{margin} 
    $\gamma$ > 0) to closest points (\textbf{support vectors, SVs}) on each 
    side of decision boundary
    \item \textbf{Soft-margin} SVM: relax separation to allow for margin 
    violations $\Rightarrow$ maximize margin while minimizing violations
  \end{itemize}
\end{itemize}

\begin{minipage}{0.7\textwidth}

\begin{itemize}
  \item 3 types of training points
  \begin{itemize}
    \item \textbf{non-SVs} with no impact on decision boundary
    \item \textbf{SVs} located exactly on decision boundary
    \item \textbf{margin violators}
  \end{itemize}
  \item \textbf{Interpretable} weighted sum of basis functions with positive 
  coefficients for support vectors
  \item Extension to \textbf{regression} is possible but requires modifications 
  \\ $\Rightarrow$ here: only classification case
\end{itemize}

\medskip

\highlight{Hypothesis space} ~~
\textcolor{blue}{
$
% \Hspace = \left\{f: \Xspace \to \R ~|~\fx = \thetab^\top \xv + \theta_0
% \right\} 
\left \{ \fx = \sumin \beta_i \yi \langle \xi, \xv \rangle  + \theta_0 ~|~
\theta_0, \beta_i \in \R ~ \forall i \right \}$
}
\end{minipage}
\begin{minipage}{0.25\textwidth}
  \includegraphics[width=1.3\textwidth]{figure/svm_wording.png} \\
  \tiny{Soft-margin SVM with margin violations}
\end{minipage}

\framebreak

% \medskip
% \footnotesize
% \begin{minipage}{0.6\textwidth}
%   \centering
%   \includegraphics[width=0.9\textwidth]{
%   ../slides/linear-svm/figure/linear_classif_1.png}  \\
%   \tiny{Hard-margin SVM: margin is maximized by boundary on the right}
% \end{minipage}
% \hfill
% \begin{minipage}{0.3\textwidth}
%   \centering
%     %https://docs.google.com/presentation/d/1g7q1hbTNmQeuRWQIM8SF9l6iKWmJyuhyhm3s9QjA0jM/edit?usp=sharing
%   \includegraphics[width=1.1\textwidth]{figure/svm_wording.png} \\
%   \tiny{Soft-margin SVM with margin violations}
% \end{minipage}

\medskip

\highlight{Dual problem} ~~ %lecture_cim2\2020\08-linear-svm\slides-3-soft-margin-svm.Rnw
\textcolor{blue}{Motivation: \dots}

\begin{eqnarray*}
    & \max\limits_{\alphav \in \R^n} & \dualobj \\
    & \text{s.t. } & 0 \le \alpha_i \le C ~~ \forall i \in \nset ~~ (C = \infty
    \text{ for hard-margin SVM)}, \\
    & \quad & \sum_{i=1}^n \alpha_i \yi = 0
\end{eqnarray*}

\medskip

\highlight{Empirical risk} ~~ Soft-margin SVM also interpretable as 
\textbf{L2-regularized ERM}: 

\begin{minipage}[b]{0.58\textwidth}
  $$ \frac{1}{2} \|\thetab\|_2^2 + C \sumin \Lxyi$$ 
  with  
  \begin{itemize}
    \item $\|\thetab\| = 1 / \gamma$,\\
    \item $C > 0$: penalization for missclassified data points
    \item $\Lyf = \max(1-yf, 0)$: \textbf{hinge} loss \\
    $\Rightarrow$ other loss functions applicable (e.g., \textbf{Huber} loss)
  \end{itemize}
\end{minipage}
\begin{minipage}[b]{0.4\textwidth}
  \centering
  \includegraphics[height=0.4\textwidth, keepaspectratio=true]{
  figure/plot-hinge-loss.png}
\end{minipage}

\framebreak

\highlight{Optimization}

\begin{itemize}
  \item Typically, tackling \textbf{dual} problem (though feasible 
  in corresponding primal) via \textbf{quadratic programming}
  \item Popular: \textbf{sequential minimal optimization} $\Rightarrow$ 
  iterative algorithm based on breaking down objective into bivariate quadratic 
  problems with analytical solutions
\end{itemize}
\medskip

\highlight{Hyperparameters} ~~ Cost parameter \textbf{$C$}

\hfill

\includegraphics[width=0.55\textwidth]{
  figure/linear_classif_1.png}  \\
  \tiny{Hard-margin SVM: margin is maximized by boundary on the right}
  \normalsize

\end{vbframe}

% ------------------------------------------------------------------------------

\begin{frame}{Linear SVM -- Pro's \& Con's}

\begin{columns}[onlytextwidth]
  \begin{column}{0.5\textwidth}
    \highlight{Advantages}
    \footnotesize
    \begin{itemize}
      % \positem High \textbf{accuracy}
      \positem Often \textbf{sparse} solution (w.r.t. observations)
      \positem Robust against overfitting (\textbf{regularized}); especially in 
      high-dimensional space
      \positem \textbf{Stable} solutions, as non-SV do not influence decision 
      boundary
      %\positem \textbf{memory efficient} (only use non-SVs)
    \end{itemize}
  \end{column}

  \begin{column}{0.5\textwidth}
    \highlight{Disadvantages}
    \footnotesize
    \begin{itemize}
      \negitem \textbf{Costly} implementation; long training times
      \negitem \textbf{Limited scalability} to larger data sets 
      \textcolor{blue}{\textbf{??}}
      \negitem Confined to \textbf{linear separation}
      % \negitem Poor \textbf{interpretability}
      \negitem No handling of \textbf{missing} data
    \end{itemize}
  \end{column}
\end{columns}

\vfill

\small

\conclbox{Very accurate solution for high-dimensional data that is linearly 
separable}

\end{frame}

% ------------------------------------------------------------------------------

\begin{frame}{Linear SVM -- Practical hints}

\footnotesize

\highlight{Preprocessing} ~~
Features must be rescaled before applying SVMs (true in general for regularized 
models).

\medskip

\highlight{Tuning}

\begin{itemize}
  \item Tuning of cost parameter $C$ advisable ~~ $\Rightarrow$ strong influence 
  on resulting separating hyperplane
  \item Frequently, tuned on log-scale grid
\end{itemize}

\medskip

\highlight{Implementation} 
\begin{itemize}
  \item \textbf{R:} \texttt{mlr3} learners \texttt{LearnerClassifSVM} /
  \texttt{LearnerRegrSVM}, calling \texttt{e1071::svm()} (interface to 
  \texttt{libSVM}), with linear kernel
  \item \textbf{Python:} \texttt{sklearn.svm.SVC} from package 
  \texttt{scikit-learn} / package \texttt{libSVM}
\end{itemize}

\end{frame}


\section{Nonlinear Support Vector Machines}
\begin{frame}{nonlinear SVM -- method summary}

\footnotesize

% \maketag{SUPERVISED} 
\maketag{CLASSIFICATION} \maketag[50]{REGRESSION} \maketag{NONPARAMETRIC} 
\maketag{BLACK-BOX}

\medskip

\highlight{General idea}
\begin{itemize}
  \item Move \textbf{beyond linearity} by mapping data to 
  transformed space where they are linearly separable
  \item \textbf{Kernel trick} \textcolor{blue}{(based on Mercer's theorem, 
  existence of reproducing kernel Hilbert space)}: 
  \begin{itemize}
    \item Replace two-step operation feature map $\phi$ $\leadsto$ inner product 
    by \textbf{kernel} $k: \Xspace \times \Xspace \rightarrow \R$, s.t.
    $\scp{\phix}{\phixt} = \kxxt$
    \item No need for explicit construction of feature maps; very fast and 
    flexible
  \end{itemize}
  \item Loss of interpretability through nonlinear feature map
\end{itemize}

\medskip

% \operatorname{sign}(\mathbf{w} \cdot \Phi(\mathbf{x})+b)

\highlight{Hypothesis space} ~~
% $\Hspace = \left \{ \fx = \sumin \alpha_i \yi k(\xi, \xv)  + \theta_0 ~|~
% \theta_0, \alpha_i \in \R ~ \forall i \right \} $
%\textcolor{blue}{$\Hspace = \{ \operatorname{sign}(\sumin \alpha_i \yi k(\xi, \xv)  + \theta_0) |\ (\theta_0, \thetab) \in \R^{p+1} \} $}
$\left \{ \fx = \sumin \beta_i \yi \langle \phi \left( \xi \right), 
    \phi(\xv) \rangle  + 
    \theta_0 ~|~ \theta_0, \beta_i \in \R ~ \forall i \right \}$

\begin{minipage}[b]{0.33\textwidth}
  \centering
  \includegraphics[width=0.5\textwidth]{
  ../slides/nonlinear-svm/figure/circles_ds.png} \\
  \tiny{Nonlinear problem in original space} 
\end{minipage}
\begin{minipage}[b]{0.66\textwidth}
  \centering
  \includegraphics[width=0.9\textwidth, trim=0 30 0 0, clip]{
  ../slides/nonlinear-svm/figure/circles_feature_map.png} \\
  \tiny{Mapping to 3D space and subsequent linear separation -- implicitly 
  handled by kernel in nonlinear SVM}
\end{minipage}

\end{frame}

% ------------------------------------------------------------------------------

\begin{frame}{nonlinear SVM -- method summary}

\footnotesize

\highlight{Dual problem} ~~ \textbf{Kernelize} dual (soft-margin) SVM problem, 
replacing all inner products by kernels:
$$\max_{\alphav} \sumin \alpha_i - \frac{1}{2}\sumin \sumjn
\alpha_i\alpha_j\yi y^{(j)} \textcolor{blue}{k(\xi, \xi[j])}, ~~ \text{s.t. } ~~ 
0 \le \alpha_i \le C, ~~ \sumin \alpha_i \yi = 0.
$$

\medskip

\highlight{Hyperparameters} ~~ Cost $C$ of margin violations, kernel 
hyperparameters (e.g., width of RBF kernel)

\medskip

\highlight{Interpretation as basis function approach}

\begin{minipage}[c]{0.5\textwidth}

  \begin{itemize}
    \item \textbf{Representer theorem:} dual soft-margin SVM problem expressible 
    through 
    $\thetab = \sumjn \beta_j \phi \left(\xi[j] \right)$ \\
    \item Sparse, weighted sum of \textbf{basis functions} with $\beta_j = 0$ 
    for non-SVs
    \item Result: \textbf{local} model with smoothness depending on kernel 
    properties
  \end{itemize}
\end{minipage}
\hfill
\begin{minipage}[c]{0.4\textwidth}
  \centering
  \includegraphics[width=0.9\textwidth, trim=0 70 0 100, clip]{
  ../slides/nonlinear-svm/figure/svm_rbf_as_basis.png} \\
  \tiny{RBF kernel as mixture of Gaussian basis functions, forming
  bumpy, nonlinear decision surface to discern red and green points}
\end{minipage}

\end{frame}

% ------------------------------------------------------------------------------

\begin{frame}{nonlinear SVM -- Pro's \& Con's}

\footnotesize

\begin{columns}[onlytextwidth]
  \begin{column}{0.5\textwidth}
    \highlight{Advantages}
    \footnotesize
    \begin{itemize}
      \positem high \textbf{accuaracy}
      \positem can learn \textbf{nonlinear decision boundaries}
      \positem often \textbf{sparse} solution
      \positem robust against overfitting (\textbf{regularized}); especially in 
      high-dimensional space 
      \positem \textbf{stable} solutions, as the non-SV do not influence the 
      separating hyperplane
    \end{itemize}
  \end{column}

  \begin{column}{0.5\textwidth}
    \highlight{Disadvantages}
    \footnotesize
    \begin{itemize}
      \negitem \textbf{costly implementation}; long training times
      \negitem does not scale well to \textbf{larger data sets} 
      \textcolor{blue}{\textbf{??}}
      \negitem poor \textbf{interpretability}
      %\item[$\textbf{\textcolor{gray!80}{-}}$] very memory-intensive
      \negitem \textbf{not easy tunable} as it is highly important to choose the 
      right kernel
      \negitem No handling of \textbf{missing} data
      
    \end{itemize}
  \end{column}
\end{columns}

\vfill

\small

\conclbox{nonlinear SVMs perform very well for nonlinear separable data, but are 
hard to interpret and need a lot of tuning.}

\end{frame}

% ------------------------------------------------------------------------------

\begin{frame}{nonlinear SVM -- Practical hints}

\footnotesize

\highlight{Common kernels}

\begin{itemize}
  \item \textbf{Linear} kernel: dot product of given observations ~~ 
  $\Rightarrow \kxxt = \xv^\top \xtil$ ~~ $\Rightarrow$ linear SVM
  \item \textbf{Polynomial} kernel of degree $d \in \N$: monomials (i.e., 
  feature interactions) up to $d$-th 
  order ~~$\Rightarrow 
  \kxxt = \left(\xv^\top \xtil + b \right)^d, ~ b \geq 0$
  \item \textbf{Radial basis function (RBF)} kernel: infinite-dimensional 
  feature space, allowing for perfect separation of all finite 
  datasets ~~ $\Rightarrow \kxxt = \exp \left( -\gamma \| \xv - \xtil \|_2^2 
  \right )$ with 
  bandwidth parameter $\gamma > 0$
\end{itemize}
 
\medskip

 \highlight{Tuning}
 
 \begin{itemize}
  \item High sensitivity w.r.t. hyperparameters, especially those of kernel
  ~~ $\Rightarrow$ \textbf{tuning} very important
  \item For RBF kernels, use \textbf{RBF sigma heuristic} to determine 
  bandwidth
\end{itemize}

  \medskip

\highlight{Implementation} 
\begin{itemize}
  \item \textbf{R:} \texttt{mlr3} learners \texttt{LearnerClassifSVM} /
  \texttt{LearnerRegrSVM}, calling \texttt{e1071::svm()} (interface to 
  \texttt{libSVM}), with nonlinear kernel
  \item \textbf{Python:} \texttt{sklearn.svm.SVC} from package 
  \texttt{scikit-learn} / package \texttt{libSVM}
\end{itemize}

\end{frame}

% \section{Gaussian Processes (GP)}
% %! Author = Son Trinh
%! Date = 10/13/2022

\begin{frame}{Gaussian Processes (GP) -- method summary}

\maketag{regression} \maketag{classification} \maketag{nonparametric} \maketag{probabilistic}

\medskip

\highlight{General idea}
\begin{itemize}
  \item GP approach determines a distribution across the potential functions $\bm{f}$ that suit the observed data.
  \item It is based on the "prior" assumption that neighboring observations should be correlated with each other.
  \item It assumes that the observations are normally distributed, and that the coupling between them occurs through the use of a normal distribution's covariance matrix.
  \item \textbf{Predict} via the maximum a-posteriori (MAP) estimate.
\end{itemize}

\medskip

\highlight{Hypothesis space} ~~
$\Hspace = \left\{ \bm{f} = \left[f\left(\xi[1]\right), \dots, f\left(\xi[n]\right)\right] \sim \mathcal{N}\left(\bm{m}, \bm{K}\right) ~|~ \bm{m} \in \R^n, \bm{K} \in \R^{n\times n} \right\}$

\medskip

\begin{minipage}[b]{0.5\textwidth}
  % FIGURE SOURCE: https://docs.google.com/presentation/d/1xodP6ayu1Gay6mMKgzVWYEFmSoeG5kNuqsaTkFFmd78  /edit
  \centering
  \includegraphics[width=0.9\textwidth]{figure/gp-prior} \\
\end{minipage}%
\begin{minipage}[b]{0.5\textwidth}
\centering
  \includegraphics[width=0.9\textwidth]{figure/gp-posterior}
\end{minipage}

\end{frame}

% ------------------------------------------------------------------------------

\begin{frame}{Gaussian Processes (GP) -- Pro's \& Con's}

\begin{columns}[onlytextwidth]
  \begin{column}{0.5\textwidth}
    \highlight{Advantages}
    \footnotesize
    \begin{itemize}
      \positem GP allows to \textbf{quantify prediction uncertainty} induced by both intrinsic noise in the problem and errors in the parameter estimation process.
      \positem GP is a function \textbf{interpolator}. It can "predict" the exact value of a training point.
      \positem GP is \textbf{non-parametric} and can model virtually any functions of observations.
    \end{itemize}
  \end{column}
  \begin{column}{0.5\textwidth}
    \highlight{Disadvantages}
    \footnotesize
    \begin{itemize}
      \negitem GP is \textbf{not sparse}, i.e., it uses the whole training set for prediction.
      \negitem GP is \textbf{not particularly easy to understand} conceptually at first sight.
    \end{itemize}
  \end{column}
\end{columns}

\vfill

\small

\conclbox{Powerful predictor with built-in measurement for uncertainty, suitable for small data sets}

\end{frame}

% ------------------------------------------------------------------------------

\begin{frame}{Gaussian Processes (GP) -- Practical hints}

\highlight{Sparse Gaussian Processes}
\begin{itemize}
	\item The sparse version of the original Gaussian Processes
	\item Suitable for large sample size
\end{itemize}

\medskip

\highlight{Implementation}

\begin{itemize}
  \item \textbf{R:} \texttt{mlr3} learners \texttt{LearnerClassifGausspr} /
    \texttt{LearnerRegrGausspr}, calling \texttt{kernlab::gausspr()}
  \item \textbf{Python:} \texttt{GaussianProcessClassifier} /
  \texttt{GaussianProcessRegressor} from package \texttt{scikit-learn}
\end{itemize}

\end{frame}

\section{Neural Networks (NN)}


\begin{frame}{Neural Networks -- method summary}

% \maketag{un/SUPERVISED} 
\maketag{regression} \maketag{classification}
\maketag[50]{(non)parametric}
\maketag{BLACK-BOX} \maketag{feature selection}

\medskip

\highlight{General idea}
\begin{itemize}
  \item Learn \textbf{composite function} through series of nonlinear feature 
  transformations, represented as \textbf{neurons}, organized hierarchically 
  in \textbf{layers}
  \begin{itemize}
    \item Basic neuron operation: 1) affine \textbf{transformation} $\phi$ (weighted sum of inputs), 
    % multiplying inputs with weights (possibly including bias term), 
    2) nonlinear \textbf{activation} $\sigma$
    % , applying (nonlinear) function to transformed inputs
    \item Combinations of simple building 
    blocks to create a complex model
  \end{itemize}
  \item Optimize via \textbf{mini-batch gradient descent} variants:
  \begin{itemize}
    \item Gradient of each weight can be infered from the \textbf{computational graph} of the network\\
    $\rightarrow$ \textit{automated differentiation} (AutoDiff)
    \item Training progress is measured in full passes over the full training data, called \textbf{epochs}
    %\textbf{Forward pass}: predict result with current parameters and 
    %compute empirical risk 
    %\item \textbf{Backward pass}: update each parameter in proportion to its 
    %error contribution $\Rightarrow$ gradients
  \end{itemize}
\end{itemize}

\medskip
 
\highlight{Hypothesis space} ~~
$\Hspace = \left\{ \fx: \fx = \tau \circ \phi \circ \sigma^{(h)} \circ
\phi^{(h)} \circ \sigma^{(h - 1)} \circ \phi^{(h - 1)} \circ ... \circ 
\sigma^{(1)} \circ \phi^{(1)} (\xv) \right\}$

\smallskip
\begin{center}
\begin{minipage}[b]{0.24\textwidth}
  \includegraphics[width=0.9\textwidth]{figure/nn-single-neuron} \\
  %\tiny{Single neuron}
\end{minipage}%
\begin{minipage}[b]{0.24\textwidth}
  \includegraphics[width=0.9\textwidth]{figure/nn-feedforward} \\
  %\tiny{(Fully-connected) Feedforward network, 1 hidden layer}
\end{minipage}%
\end{center}


\end{frame}

% ------------------------------------------------------------------------------

\begin{frame}{Neural Networks -- method summary}

\footnotesize

\highlight{Architecture}

\begin{itemize}
    \item Input layer: original features $\xv$
    \item Hidden layers: nonlinear transformation of previous layer $\phi^{(h)} = \sigma^{(h - 1)}(\phi^{(h-1)})$
    \item Output layer: number of output neurons and activation depends on problem $\tau(\phi)$
    \begin{itemize}
    \item Regression: one output neuron, $\tau = $ identity
    \item Binary classification: one output neuron, $\tau = \frac{1}{1 + \exp(- \thx)}$ (logistic sigmoid)
    \item Multiclass Classification: $g$ output neurons, $\tau_j = \frac{\exp(f_j)}{\sum_{j=1}^g \exp(f_j)}$ (softmax)
\end{itemize}
\end{itemize}


\highlight{Empirical risk} \\% ~~
In principle: Any \textbf{differentiable} loss function
\begin{itemize}
    \item \textbf{Regression}: Quadratic loss (cf. Lin. Regr.)
    \item \textbf{Classification}: (Binary) cross-entropy (cf. Log. Regr.)
\end{itemize}

\medskip

\highlight{Optimization}

\begin{itemize}
  \item Variety of different optimizers, mostly based on some form of 
  \textbf{stochastic gradient descent (SGD)}\\
  $\Rightarrow$ ADAM optimizer oder SGD with Momentum
  \item Backbone: 
    \begin{itemize}
        \item Gradient computation for arbitrary functions via 
  \textbf{computational graphs}
        \item \textbf{Automatic differentiation (autodiff)} calculates derivatives automatically
    \end{itemize}
  
  % \item Crucial role of \textbf{regularization} due to high expressivity of 
  % NNs' hypothesis spaces
\end{itemize}

\end{frame}

% ------------------------------------------------------------------------------

\begin{frame}{Neural Networks -- method summary}

\highlight{NN types} ~~ Large variety of architectures for different purposes
\begin{itemize}
  \item \textbf{Feedforward NNs / multi-layer perceptrons (MLPs):} sequence of 
  \textbf{fully-connected} layers
  \item \textbf{Convolutional NNs (CNNs):} sequence of feature map extractors 
  with spatial awareness $\Rightarrow$ images
  \item \textbf{Recurrent NNs (RNNs):} handling of sequential, variable-length 
  information $\Rightarrow$ times series, text, audio
  \item \textbf{Transformer networks:} de-facto standard for handling text data or data of multiple modalities
  \item Unsupervised: \textbf{autoencoders}, \textbf{generative adversarial 
  networks (GANs)}, \dots
  % \item \textbf{Autoencoders:} learning unsupervised embeddings
  % \item \textbf{Generative adversarial networks (GANs):} learning to generate 
  % artificial samples
\end{itemize}

\begin{minipage}[b]{0.49\textwidth}
\begin{center}
  \includegraphics[width=.9\textwidth]{figure/nn-cnn-1} \\
  \tiny{Convolutional network architecture}\\ \medskip
  \includegraphics[width=.9\textwidth]{learners-overview/figure/one_to_one.png} \\
  \tiny{Recurrent network architecture}
\end{center}
\end{minipage}
\begin{minipage}[b]{0.49\textwidth}
\begin{center}
  \includegraphics[width=.4\textwidth]{learners-overview/figure/transformer.png} \\
  \tiny{Transformer network architecture}
\end{center}
\end{minipage}

\end{frame}

% ------------------------------------------------------------------------------

\begin{frame}{Neural Networks -- method summary}

\footnotesize


\highlight{Hyperparameters}

\begin{itemize}
  \item Regarding \textbf{architecture}
  \begin{itemize}
    \item Lots of design choices ~~$\Rightarrow$ tuning problem of its own: 
    \textbf{neural architecture search (NAS)}
    \item E.g., network depth, layer types, activation functions, \dots
  \end{itemize}
  \item Regarding \textbf{optimization \& regularization}
  \begin{itemize}
    \item Crucial due to \textbf{overparameterization} and strong 
    \textbf{nonconvexity} 
    \item E.g., weight initialization, choice of optimizer, learning rate, 
    batch size, number of epochs, \dots
  \end{itemize}
\end{itemize}

\medskip

\highlight{Transfer Learning}

\begin{itemize}
    \item Makes choice of the architecture dispensable\\
          $\Rightarrow$ Pre-defined architecture with pre-trained weights is used
    \item Reduces training cost a lot, since pre-trained weights are only adapted during fine-tuning
    \item \textbf{Pre-training} done in a self-supervised fashion on ubiquituous amount of data\\
          $\Rightarrow$ In self-supervised learning, labels are generated from the data itself, no human labeling effort needed
\end{itemize}

% \highlight{Runtime behavior} ~~ \textcolor{blue}{???}

\end{frame}


% ------------------------------------------------------------------------------

\begin{frame}{Neural Networks -- Practical hints}

\highlight{Some options for regularization} 
\begin{itemize}
  \item Control weight magnitude with \textbf{weight decay} (L2 
  regularization)
  \item Interrupt training when validation error starts to pick up 
  $\Rightarrow$ \textbf{early stopping}
  \item Use \textbf{dropout} to deactivate neurons at random, thus down-sizing 
  network
  \item Expand training data and enforce invariances via \textbf{augmentation}
  \item \dots
\end{itemize}

\highlight{Optimization tricks}
\begin{itemize}
  \item Accelerate training via optimizer (ADAM, Momentum)
  \item Control learning rate with \textbf{schedulers}, or keep it 
  \textbf{adaptive}
  \item Use \textbf{batch normalization} for stability by keeping input distributions fixed throughout transformations
  \item \dots
\end{itemize}

% \highlight{Types of neural networks (RNNs, CNNs)}
% 
% \begin{itemize}
%   \item \textbf{Recurrent neural networks (RNNs}: Deep NN that make use of 
%   \textbf{sequential} information with temporal \textbf{dependencies} \\
%   $\rightarrow$ Frequently applied to \textbf{natural language processing}
%   \item \textbf{Convolutional neural networks (CNNs)}: Regularized version of the 
%   fully connected feed-forward NN (where each neuron is connected to all 
%   neurons of the subsequent layer) that abstracts inputs to feature maps via 
%   \textbf{convolution} \\
%   $\rightarrow$ Frequently applied to \textbf{image recognition}
% 
% \end{itemize}
% 
% \medskip
% 
% \highlight{Problem of neural architecture search (NAS)}
% 
% NN are \textbf{not off-the-shelf} methods -- the network architecture needs to 
% be tailored to each problem anew \\
% $\rightarrow$ Automated machine learning attempts to learn architectures

\medskip
 
\highlight{Implementation}

\begin{itemize}
  \item \textbf{R:} packages \texttt{reticulate}, \texttt{neuralnet}
  \item \textbf{Python libraries:} 
  \begin{itemize}
      \item \texttt{PyTorch} and \texttt{PyTorch Lightning}
      \item \texttt{TensorFlow} (high-level API: \texttt{keras})
      \item \texttt{huggingface}
  \end{itemize}
\end{itemize}

\end{frame}


% ------------------------------------------------------------------------------

\begin{frame}{Neural Networks -- Pros \& Cons}

\footnotesize

\begin{columns}[onlytextwidth]
  \begin{column}{0.5\textwidth}
    \highlight{Advantages}
    \footnotesize
    \begin{itemize}
      \positem Applicable to \textbf{complex, nonlinear} problems
      \positem Very \textbf{versatile} w.r.t. architectures
      \positem Suitable for \textbf{unstructured} data (e.g., images)
      \positem Strong \textbf{performance} if done right
      \positem Built-in \textbf{feature extraction}, obtained by intermediate
      representations
      \positem Easy handling of \textbf{high-dimensional} data
      \positem \textbf{Parallelizable} training 
    \end{itemize}
  \end{column}

  \begin{column}{0.5\textwidth}
    \highlight{Disadvantages}
    \footnotesize
    \begin{itemize}
      \negitem Typically, high computational \textbf{cost}
      \negitem High demand for \textbf{training data} 
      \negitem Strong tendency to \textbf{overfit}
      \negitem Requiring lots of \textbf{tuning expertise} 
      \negitem \textbf{Black-box} model -- hard to interpret or explain
    \end{itemize}
  \end{column}
\end{columns}

\vfill

\small

\conclbox{Able to solve extremely complex tasks, but computationally 
expensive and hard to get right}

\end{frame}

\endlecture

\end{document}
