\usepackage[]{graphicx}
\usepackage[]{color}
% maxwidth is the original width if it is less than linewidth
% otherwise use linewidth (to make sure the graphics do not exceed the margin)
\makeatletter
\def\maxwidth{ %
  \ifdim\Gin@nat@width>\linewidth
    \linewidth
  \else
    \Gin@nat@width
  \fi
}
\makeatother

% ---------------------------------%
% latex-math dependencies, do not remove:
% - \usepackage{mathtools}
% - \usepackage{bm}
% - \usepackage{siunitx}
% - \usepackage{dsfont}
% - \usepackage{xspace}
% ---------------------------------%

%--------------------------------------------------------%
%       Language, encoding, typography
%--------------------------------------------------------%

\usepackage[english]{babel}
\usepackage[utf8]{inputenc} % Enables inputting UTF-8 symbols
% Standard AMS suite
\usepackage{amsmath,amsfonts,amssymb}

% Font four double-stroke / blackboard letters for sets of numbers (N, R, ...)
% Distribution name is "doublestroke"
% According to https://mirror.physik.tu-berlin.de/pub/CTAN/fonts/doublestroke/dsdoc.pdf
% the "bbm" package does a similar thing and may be superfluous.
% Required for latex-math
\usepackage{dsfont}

% bbm – "Blackboard-style" cm fonts (https://www.ctan.org/pkg/bbm)
% Used to be in common.tex, loaded directly after this file
% Maybe superfluous given dsfont is loaded
% TODO: Check if really unused?
% \usepackage{bbm}

% bm – Access bold symbols in maths mode - https://ctan.org/pkg/bm
% Required for latex-math
% https://tex.stackexchange.com/questions/3238/bm-package-versus-boldsymbol
\usepackage{bm}

% pifont – Access to PostScript standard Symbol and Dingbats fonts
% Used for \newcommand{\xmark}{\ding{55}, which is never used
% aside from lecture_advml/attic/xx-automl/slides.Rnw
% \usepackage{pifont}

% Quotes (inline and display), provdes \enquote
% https://ctan.org/pkg/csquotes
\usepackage{csquotes}

% Adds arg to enumerate env, technically superseded by enumitem according
% to https://ctan.org/pkg/enumerate
% Replace with https://ctan.org/pkg/enumitem ?
\usepackage{enumerate}

% Line spacing - provides \singlespacing \doublespacing \onehalfspacing
% https://ctan.org/pkg/setspace
% TODO: Check if really unused?
%\usepackage{setspace}

% mathtools – Mathematical tools to use with amsmath
% https://ctan.org/pkg/mathtools?lang=en
% latex-math dependency according to latex-math repo
\usepackage{mathtools}

%--------------------------------------------------------%
%       Displaying code and algorithms
%--------------------------------------------------------%
\usepackage{verbatim}
\usepackage{algorithm}
\usepackage{algpseudocode}

%--------------------------------------------------------%
%       Tables
%--------------------------------------------------------%

% multi-row table cells: https://www.namsu.de/Extra/pakete/Multirow.html
\usepackage{multirow}

% long/multi-page tables: https://texdoc.org/serve/longtable.pdf/0
% TODO: Check if really unused?

\usepackage{longtable}

% pretty table env: https://ctan.org/pkg/booktabs?lang=en
% TODO: Check if really unused?
\usepackage{booktabs}

%--------------------------------------------------------%
%       Figures: Creating, placing, verbing
%--------------------------------------------------------%

% wrapfig - Wrapping text around figures https://de.overleaf.com/learn/latex/Wrapping_text_around_figures
\usepackage{wrapfig}

% Sub figures in figures and tables
% https://ctan.org/pkg/subfig -- supersedes subfigure package
% TODO: Check if really unused?
\usepackage{subfig}

% Actually it's pronounced PGF https://en.wikibooks.org/wiki/LaTeX/PGF/TikZ
\usepackage{tikz}

\usetikzlibrary{shapes,arrows,automata,positioning,calc,chains,trees, shadows}
\tikzset{
  %Define standard arrow tip
  >=stealth',
  %Define style for boxes
  punkt/.style={
    rectangle,
    rounded corners,
    draw=black, very thick,
    text width=6.5em,
    minimum height=2em,
    text centered},
  % Define arrow style
  pil/.style={
    ->,
    thick,
    shorten <=2pt,
    shorten >=2pt,}
}


% Unsorted
% textpos – Place boxes at arbitrary positions on the LATEX page
% https://ctan.org/pkg/textpos?lang=en
% Provides \begin{textblock}
 % TODO: Check if really unused?
\usepackage[absolute,overlay]{textpos}

% psfrag – Replace strings in encapsulated PostScript figures
% https://www.overleaf.com/latex/examples/psfrag-example/tggxhgzwrzhn
% https://ftp.mpi-inf.mpg.de/pub/tex/mirror/ftp.dante.de/pub/tex/macros/latex/contrib/psfrag/pfgguide.pdf
% Can't tell if this is needed
% TODO: Check if really unused?
\usepackage{psfrag}

% Maybe not great to use this https://tex.stackexchange.com/a/197/19093
% Use align instead -- TODO: Global search & replace to check
\usepackage{eqnarray}

\usepackage{colortbl}

% arydshln – Draw dash-lines in array/tabular
% https://www.ctan.org/pkg/arydshln
% !! "arydshln has to be loaded after array, longtable, colortab and/or colortbl"
% Provides \hdashline and \cdashline
% TODO: Check if really unused?
% \usepackage{arydshln}

% tabularx – Tabulars with adjustable-width columns
% https://ctan.org/pkg/tabularx
% Provides \begin{tabularx}
% TODO: Check if really unused?
% \usepackage{tabularx}

% placeins – Control float placement
% https://ctan.org/pkg/placeins
% Defines a \FloatBarrier command
% TODO: Check if really unused?
% \usepackage{placeins}


% framed – Framed or shaded regions that can break across pages
% https://ctan.org/pkg/framed
% Provides \begin{framed} which uses \colorbox{shadecolor} relying on \definecolor{shadecolor}.
% TODO: Check if really unused?
% \usepackage{framed}

% Used often in conjunction with \definecolor{shadecolor}{rgb}{0.969, 0.969, 0.969}
% Might be able to be removed or at least redefined to only have shadecolor (if needed)
\definecolor{fgcolor}{rgb}{0.345, 0.345, 0.345}
\definecolor{shadecolor}{rgb}{0.969, 0.969, 0.969}
\newenvironment{knitrout}{}{} % an empty environment to be redefined in TeX


% Defines macros and environments
\usepackage{../../style/lmu-lecture}

\let\code=\texttt % Used regularly
\let\proglang=\textsf % Unused?

% Not sure what/why this does
\setkeys{Gin}{width=0.9\textwidth}

\setbeamertemplate{frametitle}{\expandafter\uppercase\expandafter\insertframetitle}

% Can't find a reason why common.tex is not just part of this file?

% basic latex stuff
\newcommand{\pkg}[1]{{\fontseries{b}\selectfont #1}} %fontstyle for R packages
\newcommand{\lz}{\vspace{0.5cm}} %vertical space
\newcommand{\dlz}{\vspace{1cm}} %double vertical space
\newcommand{\oneliner}[1] % Oneliner for important statements
{\begin{block}{}\begin{center}\begin{Large}#1\end{Large}\end{center}\end{block}}


%new environments
\newenvironment{vbframe}  %frame with breaks and verbatim
{
 \begin{frame}[containsverbatim,allowframebreaks]
}
{
\end{frame}
}

\newenvironment{vframe}  %frame with verbatim without breaks (to avoid numbering one slided frames)
{
 \begin{frame}[containsverbatim]
}
{
\end{frame}
}

\newenvironment{blocki}[1]   % itemize block
{
 \begin{block}{#1}\begin{itemize}
}
{
\end{itemize}\end{block}
}

\newenvironment{fragileframe}[2]{  %fragile frame with framebreaks
\begin{frame}[allowframebreaks, fragile, environment = fragileframe]
\frametitle{#1}
#2}
{\end{frame}}


\newcommand{\myframe}[2]{  %short for frame with framebreaks
\begin{frame}[allowframebreaks]
\frametitle{#1}
#2
\end{frame}}

\newcommand{\remark}[1]{
  \textbf{Remark:} #1
}


\newenvironment{deleteframe}
{
\begingroup
\usebackgroundtemplate{\includegraphics[width=\paperwidth,height=\paperheight]{../style/color/red.png}}
 \begin{frame}
}
{
\end{frame}
\endgroup
}
\newenvironment{simplifyframe}
{
\begingroup
\usebackgroundtemplate{\includegraphics[width=\paperwidth,height=\paperheight]{../style/color/yellow.png}}
 \begin{frame}
}
{
\end{frame}
\endgroup
}\newenvironment{draftframe}
{
\begingroup
\usebackgroundtemplate{\includegraphics[width=\paperwidth,height=\paperheight]{../style/color/green.jpg}}
 \begin{frame}
}
{
\end{frame}
\endgroup
}
% https://tex.stackexchange.com/a/261480: textcolor that works in mathmode
\makeatletter
\renewcommand*{\@textcolor}[3]{%
  \protect\leavevmode
  \begingroup
    \color#1{#2}#3%
  \endgroup
}
\makeatother


%-------------------------------------------------------------------------------------------------------%
%  Unused stuff that needs to go but is kept here currently juuuust in case it was important after all  %
%-------------------------------------------------------------------------------------------------------%

% \newcommand{\hlnum}[1]{\textcolor[rgb]{0.686,0.059,0.569}{#1}}%
% \newcommand{\hlstr}[1]{\textcolor[rgb]{0.192,0.494,0.8}{#1}}%
% \newcommand{\hlcom}[1]{\textcolor[rgb]{0.678,0.584,0.686}{\textit{#1}}}%
% \newcommand{\hlopt}[1]{\textcolor[rgb]{0,0,0}{#1}}%
% \newcommand{\hlstd}[1]{\textcolor[rgb]{0.345,0.345,0.345}{#1}}%
% \newcommand{\hlkwa}[1]{\textcolor[rgb]{0.161,0.373,0.58}{\textbf{#1}}}%
% \newcommand{\hlkwb}[1]{\textcolor[rgb]{0.69,0.353,0.396}{#1}}%
% \newcommand{\hlkwc}[1]{\textcolor[rgb]{0.333,0.667,0.333}{#1}}%
% \newcommand{\hlkwd}[1]{\textcolor[rgb]{0.737,0.353,0.396}{\textbf{#1}}}%
% \let\hlipl\hlkwb

% \makeatletter
% \newenvironment{kframe}{%
%  \def\at@end@of@kframe{}%
%  \ifinner\ifhmode%
%   \def\at@end@of@kframe{\end{minipage}}%
%   \begin{minipage}{\columnwidth}%
%  \fi\fi%
%  \def\FrameCommand##1{\hskip\@totalleftmargin \hskip-\fboxsep
%  \colorbox{shadecolor}{##1}\hskip-\fboxsep
%      % There is no \\@totalrightmargin, so:
%      \hskip-\linewidth \hskip-\@totalleftmargin \hskip\columnwidth}%
%  \MakeFramed {\advance\hsize-\width
%    \@totalleftmargin\z@ \linewidth\hsize
%    \@setminipage}}%
%  {\par\unskip\endMakeFramed%
%  \at@end@of@kframe}
% \makeatother

% \definecolor{shadecolor}{rgb}{.97, .97, .97}
% \definecolor{messagecolor}{rgb}{0, 0, 0}
% \definecolor{warningcolor}{rgb}{1, 0, 1}
% \definecolor{errorcolor}{rgb}{1, 0, 0}
% \newenvironment{knitrout}{}{} % an empty environment to be redefined in TeX

% \usepackage{alltt}
% \newcommand{\SweaveOpts}[1]{}  % do not interfere with LaTeX
% \newcommand{\SweaveInput}[1]{} % because they are not real TeX commands
% \newcommand{\Sexpr}[1]{}       % will only be parsed by R
% \newcommand{\xmark}{\ding{55}}%

% math spaces
\newcommand{\N}{\mathds{N}}                                                 % N, naturals
\newcommand{\Z}{\mathds{Z}}                                                 % Z, integers
\newcommand{\Q}{\mathds{Q}}                                                 % Q, rationals
\newcommand{\R}{\mathds{R}}                                                 % R, reals
\newcommand{\C}{\mathds{C}}                                                 % C, complex
\newcommand{\HS}{\mathcal{H}}                                               % H, hilbertspace
\newcommand{\continuous}{\mathcal{C}}                                       % C, space of continuous functions
\newcommand{\M}{\mathcal{M}} 												% machine numbers
\newcommand{\epsm}{\epsilon_m} 												% maximum error


% basic math stuff
\newcommand{\xt}{\tilde x}													% x tilde
\def\argmax{\mathop{\sf arg\,max}}                                          % argmax
\def\argmin{\mathop{\sf arg\,min}}                                          % argmin
\newcommand{\sign}{\operatorname{sign}}                                     % sign, signum
\newcommand{\I}{\mathbb{I}}                                                 % I, indicator
\newcommand{\order}{\mathcal{O}}                                            % O, order
\newcommand{\fp}[2]{\frac{\partial #1}{\partial #2}}                        % partial derivative
\newcommand{\pd}[2]{\frac{\partial{#1}}{\partial #2}}						% partial derivative

% sums and products
\newcommand{\sumin}{\sum_{i=1}^n}											% summation from i=1 to n
\newcommand{\sumkg}{\sum_{k=1}^g}											% summation from k=1 to g
\newcommand{\prodin}{\prod_{i=1}^n}											% product from i=1 to n
\newcommand{\prodkg}{\prod_{k=1}^g}											% product from k=1 to g

% linear algebra
\newcommand{\one}{\boldsymbol{1}}                                           % 1, unitvector
\newcommand{\id}{\mathrm{I}}                                                % I, identity
\newcommand{\diag}{\operatorname{diag}}                                     % diag, diagonal
\newcommand{\trace}{\operatorname{tr}}                                      % tr, trace
\newcommand{\spn}{\operatorname{span}}                                      % span
\newcommand{\scp}[2]{\left\langle #1, #2 \right\rangle}                     % <.,.>, scalarproduct
\newcommand{\mat}[1]{ 														% short pmatrix command
	\begin{pmatrix}
		#1
	\end{pmatrix}
}
\newcommand{\Amat}{\bm{A}}													% matrix A
\newcommand{\xv}{\bm{x}}													% vector x (bold)
\newcommand{\yv}{\bm{y}}														% vector y (bold)
\newcommand{\Deltab}{\bm{\Delta}}											% error term for vectors
															

% basic probability + stats
\renewcommand{\P}{\mathds{P}}                                               % P, probability
\newcommand{\E}{\mathds{E}}                                                 % E, expectation
\newcommand{\var}{\mathsf{Var}}                                             % Var, variance
\newcommand{\cov}{\mathsf{Cov}}                                             % Cov, covariance
\newcommand{\corr}{\mathsf{Corr}}                                           % Corr, correlation
\newcommand{\normal}{\mathcal{N}}                                           % N of the normal distribution
\newcommand{\iid}{\overset{i.i.d}{\sim}}                                    % dist with i.i.d superscript
\newcommand{\distas}[1]{\overset{#1}{\sim}}                                 % ... is distributed as ... 
% machine learning

%%%%%% ml - data
\newcommand{\Xspace}{\mathcal{X}}                                           % X, input space
\newcommand{\Yspace}{\mathcal{Y}}                                           % Y, output space
\newcommand{\nset}{\{1, \ldots, n\}}                                        % set from 1 to n
\newcommand{\pset}{\{1, \ldots, p\}}                                        % set from 1 to p
\newcommand{\gset}{\{1, \ldots, g\}}                                        % set from 1 to g
\newcommand{\Pxy}{\P_{xy}}                                                  % P_xy
\newcommand{\xy}{(x, y)}                                                    % observation (x, y)
\newcommand{\xvec}{(x_1, \ldots, x_p)^T}                                    % (x1, ..., xp) 
\newcommand{\D}{\mathcal{D}}                                                % D, data 
\newcommand{\Dset}{\{ (x^{(1)}, y^{(1)}), \ldots, (x^{(n)},  y^{(n)})\}}    % {(x1,y1)), ..., (xn,yn)}, data
\newcommand{\xdat}{\{ x^{(1)}, \ldots, x^{(n)}\}}   						 % {x1, ..., xn}, input data
\newcommand{\ydat}{\mathbf{y}}                                              % y (bold), vector of outcomes
\newcommand{\yvec}{(y^{(1)}, \hdots, y^{(n)})^T}                            % (y1, ..., yn), vector of outcomes
\renewcommand{\xi}[1][i]{x^{(#1)}}                                          % x^i, i-th observed value of x
\newcommand{\yi}[1][i]{y^{(#1)}}                                            % y^i, i-th observed value of y 
\newcommand{\xyi}{(\xi, \yi)}                                               % (x^i, y^i), i-th observation
\newcommand{\xivec}{(x^{(i)}_1, \ldots, x^{(i)}_p)^T}                       % (x1^i, ..., xp^i), i-th observation vector
\newcommand{\xj}{x_j}                                                       % x_j, j-th feature
\newcommand{\xjb}{\mathbf{x}_j}                                             % x_j (bold), j-th feature vecor
\newcommand{\xjvec}{(x^{(1)}_j, \ldots, x^{(n)}_j)^T}                       % (x^1_j, ..., x^n_j), j-th feature vector
\newcommand{\Dtrain}{\mathcal{D}_{\text{train}}}                            % D_train, training set
\newcommand{\Dtest}{\mathcal{D}_{\text{test}}}                              % D_test, test set

%%%%%% ml - models general

% continuous prediction function f
\newcommand{\fx}{f(x)}                                                      % f(x), continuous prediction function
\newcommand{\Hspace}{H}														% hypothesis space where f is from
\newcommand{\fh}{\hat{f}}                                                   % f hat, estimated prediction function
\newcommand{\fxh}{\fh(x)}                                                   % fhat(x)
\newcommand{\fxt}{f(x | \theta)}                                            % f(x | theta)
\newcommand{\fxi}{f(\xi)}                                                   % f(x^(i))
\newcommand{\fxih}{\hat{f}(\xi)}                                            % f(x^(i))
\newcommand{\fxit}{f(x^{(i)} | \theta)}                                     % f(x^(i) | theta)
\newcommand{\fhD}{\fh_{\D}}                                                 % fhat_D, estimate of f based on D
\newcommand{\fhDtrain}{\fh_{\Dtrain}}                                       % fhat_Dtrain, estimate of f based on D

% discrete prediction function h
\newcommand{\hx}{h(x)}                                                      % h(x), discrete prediction function
\newcommand{\hh}{\hat{h}}                                                   % h hat
\newcommand{\hxh}{\hat{h}(x)}                                               % hhat(x)
\newcommand{\hxt}{h(x | \theta)}                                            % h(x | theta)
\newcommand{\hxi}{h(\xi)}                                                   % h(x^(i))
\newcommand{\hxit}{h(x^{(i)} | \theta)}                                     % h(x^(i) | theta)

% yhat
\newcommand{\yh}{\hat{y}}                                                   % y hat for prediction of target
\newcommand{\yih}{\hat{y}}                                                  % y hat for prediction of target

% theta
\newcommand{\thetah}{\hat{\theta}}                                          % theta hat

% densities + probabilities
% pdf of x 
\newcommand{\pdf}{p}                                                        % p
\newcommand{\pdfx}{p(x)}                                                    % p(x)
\newcommand{\pixt}{\pi(x | \theta)}                                         % pi(x|theta), pdf of x given theta

% pdf of (x, y)
\newcommand{\pdfxy}{p(x,y)}                                                 % p(x, y)
\newcommand{\pdfxyt}{p(x, y | \theta)}                                      % p(x, y | theta)
\newcommand{\pdfxyit}{p(\xi, \yi | \theta)}                                 % p(x^(i), y^(i) | theta)

% pdf of x given y
\newcommand{\pdfxyk}{p(x | y=k)}                                            % p(x | y = k)
\newcommand{\lpdfxyk}{\log \pdfxyk}                                         % log p(x | y = k)
\newcommand{\pdfxiyk}{p(\xi | y=k)}                                         % p(x^i | y = k)

% prior probabilities
\newcommand{\pik}{\pi_k}                                                    % pi_k, prior
\newcommand{\lpik}{\log \pik}                                               % log pi_k, log of the prior

% posterior probabilities
\newcommand{\post}{\P(y = 1 | x)}                                           % P(y = 1 | x), post. prob for y=1
\newcommand{\pix}{\pi(x)}                                                   % pi(x), P(y = 1 | x)
\newcommand{\postk}{\P(y = k | x)}                                          % P(y = k | y), post. prob for y=k
\newcommand{\pikx}{\pi_k(x)}                                                % pi_k(x), P(y = k | x)
\newcommand{\pikxt}{\pi_k(x | \theta)}                                      % pi_k(x | theta), P(y = k | x, theta)
\newcommand{\pijx}{\pi_j(x)}                                                % pi_j(x), P(y = j | x)
\newcommand{\pdfygxt}{p(y |x, \theta)}                                      % p(y | x, theta)
\newcommand{\pdfyigxit}{p(\yi |\xi, \theta)}                                % p(y^i |x^i, theta)
\newcommand{\lpdfygxt}{\log \pdfygxt }                                      % log p(y | x, theta)
\newcommand{\lpdfyigxit}{\log \pdfyigxit}                                   % log p(y^i |x^i, theta)
\newcommand{\pixh}{\hat \pi(x)}                                             % pi(x) hat, P(y = 1 | x) hat
\newcommand{\pikxh}{\hat \pi_k(x)}                                          % pi_k(x) hat, P(y = k | x) hat

% residual and margin
\newcommand{\eps}{\epsilon}                                                 % residual, stochastic
\newcommand{\epsi}{\epsilon^{(i)}}                                          % epsilon^i, residual, stochastic
\newcommand{\epsh}{\hat{\epsilon}}                                          % residual, estimated
\newcommand{\yf}{y \fx}                                                     % y f(x), margin
\newcommand{\yfi}{\yi \fxi}                                                 % y^i f(x^i), margin
\newcommand{\Sigmah}{\hat \Sigma}											% estimated covariance matrix
\newcommand{\Sigmahj}{\hat \Sigma_j}										% estimated covariance matrix for the j-th class

% ml - loss, risk, likelihood
\newcommand{\Lxy}{L(y, f(x))}                                               % L(y, f(x)), loss function
\newcommand{\Lxyi}{L(\yi, \fxi)}                                            % L(y^i, f(x^i))
\newcommand{\Lxyt}{L(y, \fxt)}                                              % L(y, f(x | theta))
\newcommand{\Lxyit}{L(\yi, \fxit)}                                          % L(y^i, f(x^i | theta)
\newcommand{\risk}{\mathcal{R}}                                             % R, risk
\newcommand{\riskf}{\risk(f)}                                               % R(f), risk
\newcommand{\riske}{\mathcal{R}_{\text{emp}}}                               % R_emp, empirical risk
\newcommand{\riskef}{\riske(f)}                                             % R_emp(f)
\newcommand{\risket}{\mathcal{R}_{\text{emp}}(\theta)}                      % R_emp(theta)
\newcommand{\riskr}{\mathcal{R}_{\text{reg}}}                               % R_reg, regularized risk
\newcommand{\riskrt}{\mathcal{R}_{\text{reg}}(\theta)}                      % R_reg(theta)
\newcommand{\riskrf}{\riskr(f)}                                             % R_reg(f)
\newcommand{\LL}{\mathcal{L}}                                               % L, likelihood
\newcommand{\LLt}{\mathcal{L}(\theta)}                                      % L(theta), likelihood
\renewcommand{\ll}{\ell}                                                    % l, log-likelihood
\newcommand{\llt}{\ell(\theta)}                                             % l(theta), log-likelihood
\newcommand{\LS}{\mathfrak{L}}                                              % ????????????
\newcommand{\TS}{\mathfrak{T}}                                              % ??????????????
\newcommand{\errtrain}{\text{err}_{\text{train}}}                           % training error
\newcommand{\errtest}{\text{err}_{\text{test}}}                             % training error
\newcommand{\errexp}{\overline{\text{err}_{\text{test}}}}                   % training error

% resampling
\newcommand{\GE}[1]{GE(\fh_{#1})}                                           % Generalization error GE
\newcommand{\GEh}[1]{\widehat{GE}_{#1}}                                     % Estimated train error
\newcommand{\GED}{\GE{\D}}                                                  % Generalization error GE
\newcommand{\EGEn}{EGE_n}                                                   % Generalization error GE
\newcommand{\EDn}{\E_{|D| = n}}                                             % Generalization error GE


% ml - irace
\newcommand{\costs}{\mathcal{C}} % costs
\newcommand{\Celite}{\theta^*} % elite configurations
\newcommand{\instances}{\mathcal{I}} % sequence of instances
\newcommand{\budget}{\mathcal{B}} % computational budget
% ml - Gaussian Process

\newcommand{\fvec}{[f(\xi[1]), \dots, f(\xi[n])]} % {f(x1), ..., f(xn)}
\newcommand{\fv}{\mathbf{f}} % bold f, function vector 
\newcommand{\mv}{\mathbf{m}} % bold m, GP mean vector 
\newcommand{\kv}{\mathbf{k}} % bold k, kernel mat partition 
\newcommand{\kcc}{k(\cdot, \cdot)} % k(.,.), kernel for arbitrary inputs 
\newcommand{\kxij}[2]{k(\xi, \xi[j])} % k(xi, xj), cov of x_i, x_j
\newcommand{\Kmat}{\mathbf{K}} % K, kernel mat
\newcommand{\Nmk}{\normal(\mv, \Kmat)} % n(m,K), Gaussian w/ mean vec, cov mat
\newcommand{\Nzk}{\normal(\zero, \Kmat)} % n(0,K), zero-mean Gaussian
\newcommand{\GPmk}{\mathcal{GP}(m(\cdot), \kcc)} % GP(m(.), k(.,.)), GP definition
\newcommand{\GPzk}{\mathcal{GP}(\zero, \kcc)} % GP(0, k(.,.)), zero-mean GP
\newcommand{\Xsubset}{\bm{X}} % bold X, finite subset from xspace
\newcommand{\fX}{f(\Xsubset)} % f(X), function vector of finite subset
\newcommand{\kXX}{k(\Xsubset, \Xsubset)} % k(X,X), cov for finite subset
\newcommand{\mX}{m(\Xsubset)} % m(X), mean for finite subset
\newcommand{\ls}{\ell} % length-scale
\newcommand{\xxtnorm}{\| \xv - \xtil\|} % ||x - xtilde||

% GP prediction
\newcommand{\xs}{\xv_\ast} % x_*, test obs features
\newcommand{\ys}{\yv_\ast} % y_*, test obs target
\newcommand{\fs}{\fv_\ast} % f_*, test obs fun vector
\newcommand{\Xs}{\Xmat_\ast} % X_*, test design matrix
\newcommand{\ks}{\kv_{\ast}} % k_*, cov vec of new obs with x
\newcommand{\kss}{\kv_{\ast \ast}} % k_**, cov vec of new obs
\newcommand{\Ks}{\Kmat_{\ast}} % K_*, cov mat of new obs with x
\newcommand{\Kss}{\Kmat_{\ast \ast}} % K_**, cov mat of new obs
\newcommand{\Kinv}{\Kmat^{-1}} % K^-1, inverse cov mat
\newcommand{\Ky}{\Kmat_y} % K_y, cov mat of y



\newcommand{\titlefigure}{figure_man/post-variance.png}
\newcommand{\learninggoals}{
  \item \textcolor{blue}{XXX}
  \item \textcolor{blue}{XXX}
}

\title{Introduction to Machine Learning}
\date{}

\begin{document}

\lecturechapter{Gaussian Process Prediction}
\lecture{Introduction to Machine Learning}

\begin{vbframe}{Gaussian Posterior Process And Prediction}

\vspace*{1cm}

\begin{itemize}
  \item So far, we have learned how to \textbf{sample} from a GP prior.
\item However, most of the time, we are not interested in drawing random functions from the prior. Instead, we usually like to use the knowledge provided by the training data to predict values of $f$ at a new test point $\xv_*$. 
\item In what follows, we will investigate how to update the Gaussian process prior ($\to$ posterior process) and how to make predictions.
\end{itemize}

\end{vbframe}

\section{Gaussian Posterior Process and Prediction}


\begin{vbframe}{Posterior process}

% \textbf{Noise-free Case:}
% 
% In the noise-free case, $y^{(i)} = f(\xi)$ ($\epsi \equiv 0$, no additive noise). The targets correspond to the true function values $\yv = \bm{f}$ (training observations) and $\ydat^* = \bm{f}^*$ (test observations).
% 
% \lz 

\begin{itemize}
  \item Let us now distinguish between observed training inputs, also denote by a design matrix $\Xmat$, and the corresponding observed values
  $$
    \bm{f} = \left[f\left(\xi[1]\right), ..., f\left(\xi[n]\right)\right]
  $$ 

and one single \textbf{unobserved test point} $\xv_*$ with $f_* = f\left(\xv_*\right).$

\item We now want to infer the distribution of $f_* | \xv_*, \bm{X}, \bm{f}$.
  $$
    f_* = f\left(\xv_*\right)
  $$  
  \item Assuming a zero-mean GP prior $\mathcal{GP}\left(\bm{0}, k(\xv, \xv^\prime)\right)$ we know

$$
\begin{bmatrix}
\bm{f} \\
f_*
\end{bmatrix} \sim  
\mathcal{N}\biggl(\bm{0}, \begin{bmatrix} \Kmat & \bm{k}_* \\ \bm{k}_*^T & \bm{k}_{**}\end{bmatrix}\biggr).
$$

Here, $\Kmat = \left(k\left(\xi, \xv^{(j)}\right)\right)_{i,j}$, $\bm{k}_* = \left[k\left(\xv_*, \xi[1]\right), ..., k\left(\xv_*, \xi[n]\right)\right]$ and $ \bm{k}_{**}\ = k(\xv_*, \xv_*)$. 

\framebreak 

\item Given that $\bm{f}$ is observed, we can apply the general rule for condition $^{(*)}$ of Gaussian random variables and obtain the following formula: 

\begin{eqnarray*}
f_* ~|~ \xv_*, \Xmat, \bm{f} \sim \mathcal{N}(\bm{k}_{*}^{T}\Kmat^{-1}\bm{f}, \bm{k}_{**} - \bm{k}_*^T \Kmat ^{-1}\bm{k}_*).
\end{eqnarray*}

% \begin{eqnarray*}
% \bm{f}_* | \Xmat_*, \Xmat, \bm{f} \sim \mathcal{N}(\Kmat_{*}^{T}\Kmat^{-1}\bm{f}, \Kmat_{**} - \Kmat_*^T \Kmat ^{-1}\Kmat_*).
% \end{eqnarray*}
\item As the posterior is a Gaussian, the maximum a-posteriori estimate, i.e. the mode of the posterior distribution, is $\bm{k}_{*}^{T}\Kmat^{-1}\bm{f}. $
\end{itemize}

\framebreak 

$^{(*)}$ General rule for condition of Gaussian random variables: 

\lz 

  If the $m$-dimensional Gaussian vector $\bm{z} \sim \mathcal{N}(\mu, \Sigma)$ can be partitioned with $\bm{z} = \left(\bm{z}_1, \bm{z}_2\right)$ where $\bm{z}_1$ is $m_1$-dimensional and $\bm{z}_2$ is $m_2$-dimensional, and:
$$\left(\mu_1, \mu_2\right), \quad \Sigma = \begin{pmatrix} \Sigma_{11} & \Sigma_{12} \\ \Sigma_{21} & \Sigma_{22} \end{pmatrix},$$

then the conditioned distribution of $\bm{z}_2 ~|~ \bm{z}_1 = \bm{a}$ is a multivariate normal  

$$
  \mathcal{N}\left(\mu_2 + \Sigma_{21} \Sigma_{11}^{-1}\left(\bm{a} - \mu_1\right), \Sigma_{22} - \Sigma_{21}\Sigma_{11}^{-1}\Sigma_{12} \right)
$$

\end{vbframe} 

\begin{frame}{GP prediction: Two points}

Let us visualize this by a simple example: 
\begin{itemize}
  \item Assume we observed a single training point $\xv = - 0.5$, and want to make a prediction at a test point $\xv_* = 0.5$. 
  \item Under a zero-mean GP with $k(\xv, \xv^\prime) = \exp(-\frac{1}{2}\|\xv - \xv^\prime\|^2)$, we compute the cov-matrix:
  $$
    \begin{bmatrix} f \\ f_* \end{bmatrix} \sim \mathcal{N}\biggl(\bm{0}, \begin{bmatrix} 1 & 0.61 \\ 0.61 & 1\end{bmatrix}\biggr). 
  $$ 
  \item Assume that we observe the point $\fx = 1$. 
  \item We compute the posterior distribution: 
  \begin{eqnarray*}
    f_* ~|~ \xv_*, \xv, f &\sim& \mathcal{N}(\bm{k}_{*}^{T}\Kmat^{-1}f, k_{**} - \bm{k}_*^T \Kmat^{-1}\bm{k}_*) \\
    &\sim& \mathcal{N}(0.61 \cdot 1 \cdot 1, 1 - 0.61 \cdot 1 \cdot 0.61) \\
    &\sim& \mathcal{N}\left(0.61, 0.6279\right) 
  \end{eqnarray*}
  \item The MAP-estimate for $\xv_*$ is $f(\xv_*) = 0.61$, and the uncertainty estimate is $0.6279$. 
  % what can we say about the function value at a new point $\xv_* = -0.5$? 
  % \item<+-> We compute the covariance function and have

  % \item<+-> After observing $\fx = 1$ we want to predict a value for $f(\xv_*)$. 
  % \item<+-> We compute the posterior distribution conditioned on $\fx = 1$
  % \begin{eqnarray*}

  % \end{eqnarray*}
\end{itemize}

\end{frame} 


\begin{vbframe}{GP prediction: Two points}

\begin{footnotesize}
  Shown is the bivariate normal density, and the respective marginals. 
\end{footnotesize}\vspace*{0.2cm}

\begin{figure}
  \includegraphics[width=0.8\textwidth]{figure_man/GP01.png}
\end{figure}


\end{vbframe}

\begin{frame}{GP prediction: Two points}

\begin{footnotesize}
  Assume we observed $\fx = 1$ for the training point $\xv = -0.5$.  
\end{footnotesize}\vspace*{0.2cm}

\begin{figure}
  \includegraphics[width=0.8\textwidth]{figure_man/GP02.png}
\end{figure}

\end{frame}
\begin{frame}{GP prediction: Two points}

\begin{footnotesize}
  We condition the Gaussian on $\fx = 1$.
\end{footnotesize}\vspace*{0.2cm}

\begin{figure}
  \includegraphics[width=0.8\textwidth]{figure_man/GP03.png}
\end{figure}

\end{frame}


\begin{frame}{GP prediction: Two points}

\begin{footnotesize}
  We compute the posterior distribution of $f(\xv_*)$ given that $\fx = 1$. 
\end{footnotesize}\vspace*{0.2cm}


\begin{figure}
  \includegraphics[width=0.8\textwidth]{figure_man/GP04.png}
\end{figure}


\end{frame}
\begin{frame}{GP prediction: Two points}

\begin{footnotesize}
  A possible predictor for $f$ at $\xv_*$ is the MAP of the posterior distribution.
\end{footnotesize}\vspace*{0.2cm}

\begin{figure}
  \includegraphics[width=0.8\textwidth]{figure_man/GP05.png}
\end{figure}


\end{frame} 

\begin{frame}{GP prediction: Two points}

\begin{footnotesize}
  We can do this for different values $\xv_*$, and show the respective mean (grey line) and standard deviations (grey area is mean $\pm 2 \cdot $ posterior standard deviation). 
\end{footnotesize}\vspace*{0.2cm}


\begin{figure}
  \includegraphics[width=0.8\textwidth]{figure_man/GP06.png}
\end{figure}

\end{frame}

\begin{vbframe}{Posterior Process}

\begin{itemize}
  \item We can generalize the formula for the posterior process for multiple unobserved test points: 

$$
  \bm{f}_* = \left[f\left(\xi[1]_*\right), ..., f\left(\xi[m]_*\right)\right]. 
$$
  \item Under a zero-mean Gaussian process, we have
  $$
    \begin{bmatrix}
    \bm{f} \\
    \bm{f}_*
    \end{bmatrix} \sim  
    \mathcal{N}\biggl(\bm{0}, \begin{bmatrix} \Kmat & \Kmat_* \\ \Kmat_*^T & \Kmat_{**} \end{bmatrix}\biggr),
  $$
    with $\Kmat_* = \left(k\left(\xi, \xv_*^{(j)}\right)\right)_{i,j}$, $\Kmat_{**} = \left(k\left(\xi[i]_*, \xi[j]_*\right)\right)_{i,j}$.
  
  \framebreak 
  
  \item Similar to the single test point situation, to get the posterior distribution, we exploit the general rule of conditioning for Gaussians:
  \begin{eqnarray*}
    \bm{f}_* ~|~ \Xmat_*, \Xmat, \bm{f} \sim \mathcal{N}(\Kmat_{*}^{T}\Kmat^{-1}\bm{f}, \Kmat_{**} - \Kmat_*^T \Kmat ^{-1}\Kmat_*).
  \end{eqnarray*}  
  \item This formula enables us to talk about correlations among different test points and sample functions from the posterior process. 
\end{itemize}

\end{vbframe}


\section{Properties of a Gaussian Process}

\begin{vbframe}{GP as interpolator}

The \enquote{prediction} for a training point $\xi$ is the exact function value $\fxi$

\vspace*{-0.8cm}

\begin{eqnarray*}
\bm{f} ~|~ \Xmat, \bm{f} \sim \mathcal{N}(\Kmat\Kmat^{-1}\bm{f}, \Kmat - \Kmat^T \Kmat^{-1} \Kmat) = \mathcal{N}(\bm{f}, \bm{0}).
\end{eqnarray*}

Thus, a Gaussian process is a function \textbf{interpolator}.

\begin{center}
\includegraphics[width=0.8\textwidth]{figure_man/gp-interpolator.png}
\end{center}
% \begin{footnotesize}
% A the posterior process (black) after observing the training points (red) interpolates the training points. 
% \end{footnotesize}

\end{vbframe}


\begin{vbframe}{GP as a spatial model}

\vspace*{-0.3cm}

\begin{itemize}
  \begin{footnotesize}
  \item The correlation among two outputs depends on distance of  the corresponding input points  $\xv$ and $\xv^\prime$ (e.g. Gaussian covariance kernel $k(\xv, \xv^\prime) = \exp \left(\frac{- \|\xv - \xv^\prime\|^2}{2 l^2}\right)$ )
  \item Hence, close data points with high spatial similarity $k(\xv, \xv^\prime)$ enter into more strongly correlated predictions: $\bm{k}_*^\top \bm{K}^{-1} \bm{f}$ ($\bm{k}_* := \left(k(\xv, \xv^{(1)}), ..., k(\xv, \xv^{(n)})\right)$).
  \end{footnotesize}  


\begin{center}
\includegraphics[width=0.4\textwidth]{figure_man/post-mean.png}
\end{center}


\begin{footnotesize}
Example: Posterior mean of a GP that was fitted with the Gaussian covariance kernel with $l = 1$. 
\end{footnotesize}


\framebreak 

\item Posterior uncertainty increases if the new data points are far from the design points.
\item The uncertainty is minimal at the design points, since the posterior variance is zero at these points.
\end{itemize}


\begin{center}
\includegraphics[width=0.4\textwidth]{figure_man/post-variance.png}
\end{center}

\begin{footnotesize}
Example (continued): Posterior variance. 
\end{footnotesize}


\end{vbframe}


\section{Noisy Gaussian Process}

\begin{vbframe}{Noisy Gaussian Process}

\begin{itemize}
  \item So far, we implicitly assumed that we had access to the true function value $\fx$.
  \item For the squared exponential kernel, for example, we have
  $$
    \cov\left(f(\xi), f(\xi[j])\right) = 1.
  $$
  \item As a result, the posterior Gaussian process is an interpolator: 
  \begin{center}
    \includegraphics[width=0.8\textwidth]{figure_man/gp-interpolator.png}
  \end{center}

\framebreak 

  \item In reality, however, this is often not the case. 
  \item We often only have access to a noisy version of the true function value
  $$
    y = \fx + \eps, \eps \sim\mathcal{N}\left(0, \sigma^2\right).
  $$
  \item Let us still assume that $\fx$ is a Gaussian process.
  \item Then,
  \begin{footnotesize} 
  \begin{eqnarray*}
    &&\cov(y^{(i)}, y^{(j)}) = \cov\left(f\left(\xi\right) + \epsilon^{(i)}, f\left(\xi[j]\right) + \epsilon^{(j)}\right) \\
    &=& \cov\left(f\left(\xi\right), f\left(\xi[j]\right)\right) + 2 \cdot \cov\left(f\left(\xi\right), \epsilon^{(j)}\right) + \cov\left(\epsilon^{(i)}, \epsilon^{(j)}\right) 
    \\ &=& k\left(\xi, \xi[j]\right) + \sigma^2 \delta_{ij}. 
  \end{eqnarray*}
  \end{footnotesize}
  \item $\sigma^2$ is called \textbf{nugget}. 
\end{itemize}

\framebreak 

\begin{itemize}
  \item Let us now derive the predictive distribution for the case of noisy observations. 
  \item The prior distribution of $y$, assuming that $f$ is modeled by a Gaussian process is then
  $$
    \bm{y} = \begin{pmatrix} \yi[1] \\ \yi[2] \\ \vdots \\ \yi[n] \end{pmatrix} \sim \mathcal{N}\left(\bm{m}, \bm{K} + \sigma^2 \bm{I}_n \right),
  $$
  with 
  \begin{eqnarray*}
    \textbf{m} &:=& \left(m\left(\xi\right)\right)_{i}, \quad
    \textbf{K} := \left(k\left(\xi, \xv^{(j)}\right)\right)_{i,j}. 
  \end{eqnarray*}

  \framebreak 
  
  \item We distinguish again between 
  \begin{itemize}
    \item observed training points $\Xmat, \yv$, and 
    \item unobserved test inputs $\Xmat_*$ with unobserved values $\bm{f}_*$
  \end{itemize} 
  and get
  $$
  \begin{bmatrix}
  \bm{y} \\
  \bm{f}_*
  \end{bmatrix} \sim  
    \mathcal{N}\biggl(\bm{0}, \begin{bmatrix} \Kmat + \sigma^2 \bm{I}_n & \Kmat_* \\ \Kmat_*^T & \Kmat_{**} \end{bmatrix}\biggr).
  $$

\framebreak

  \item Similarly to the noise-free case, we condition according to the rule of conditioning for Gaussians to get the posterior distribution for the test outputs $\bm{f}_*$ at $\Xmat_*$: 

  \begin{eqnarray*}
    \bm{f}_* ~|~ \Xmat_*, \Xmat, \bm{y} \sim \mathcal{N}(\bm{m}_{\text{post}}, \bm{K}_\text{post}).
\end{eqnarray*}
  with 
  \begin{eqnarray*}
    \bm{m}_{\text{post}} &=& \Kmat_{*}^{T} \left(\Kmat+ \sigma^2 \cdot \id\right)^{-1}\bm{y} \\
    \bm{K}_\text{post} &=& \Kmat_{**} - \Kmat_*^T \left(\Kmat ^{-1} + \sigma^2 \cdot \id\right)\Kmat_*,
  \end{eqnarray*}
\item This converts back to the noise-free formula if $\sigma^2 = 0$.

\framebreak 

\item The noisy Gaussian process is not an interpolator any more.
\item A larger nugget term leads to a wider ``band'' around the observed training points.
\item The nugget term is estimated during training.


\begin{center}
    \includegraphics[width=0.6\textwidth]{figure_man/gp-regression.png}
\end{center}
\end{itemize}

\end{vbframe}



\section{Decision Theory for Gaussian Processes}

\begin{vbframe}{Risk Minimization for Gaussian Processes}

In machine learning, we learned about risk minimization. We usually choose a loss function and minimize the empirical risk  

$$
  \riske(f) := \sumin \Lxyi
$$
as an approximation to the theoretical risk

$$ 
  \riskf := \E_{xy} [\Lxy] = \int \Lxy \text{d}\Pxy. 
$$

\begin{itemize}
  \item How does the theory of Gaussian processes fit into this theory? 
  \item What if we want to make a prediction which is optimal w.r.t. a certain loss function? 
\end{itemize}

\framebreak 

\begin{itemize}
  \item The theory of Gaussian process gives us a posterior distribution 
  $$
    p(y ~|~\D)
  $$
  \item If we now want to make a prediction at a test point $\bm{x}_*$, we approximate the theoretical risk in a different way, by using the posterior distribution: 
  $$
    \mathcal{R}(y_* ~|~ \bm{x}_*) \approx \int L(\tilde y_*, y_*) p(\tilde y_*~|~\bm{x}_*, \D)d\tilde y_*. 
  $$
  \item The optimal prediciton w.r.t the loss function is then: 
  $$
    \hat y_* | \bm{x}_* = \argmin_{y_*} \mathcal{R}(y_*~|~ \bm{x}_*).
  $$
\end{itemize}


% In practical applications, we are often forced to make predictions. We need a point-like prediction that is \enquote{optimal} in some sense. 

% \lz

% We define \enquote{optimality} with respect to some loss function

% $$
% L(y_\text{true}, y_\text{guess}). 
% $$

% \vfill

% \begin{footnotesize}
% Notice that we computed the predictive distribution without reference to the loss function. In non-Bayesian paradigms, the model is typically trained by minimizing the empirical risk (or loss). In contrast, in the Bayesian setting there is a clear separation between the likelihood function (used for training in addition to the prior) and the loss function.
% \end{footnotesize}

% \framebreak 

% As we do not know the true value $y_\text{true}$ for our test input $\bm{x}_*$, we minimize w.r.t. to the expected loss called \textbf{risk} w.r.t. our model's opinion as to what the truth might be

% $$
% \mathcal{R}(y_\text{guess} | \bm{x}_*) = \int L(y_*, y_\text{guess}) p(y_*|\bm{x}_*, \D)dy_*. 
% $$

% Our best guess w.r.t. $L$ is then

% $$
% $$

% For quadratic loss $L(y, y^\prime) = (y - y^\prime)^2$ this corresponds to the posterior mean. 

\end{vbframe}


\endlecture
\end{document}
