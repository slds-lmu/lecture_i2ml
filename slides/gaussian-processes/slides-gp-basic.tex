\usepackage[]{graphicx}
\usepackage[]{color}
% maxwidth is the original width if it is less than linewidth
% otherwise use linewidth (to make sure the graphics do not exceed the margin)
\makeatletter
\def\maxwidth{ %
  \ifdim\Gin@nat@width>\linewidth
    \linewidth
  \else
    \Gin@nat@width
  \fi
}
\makeatother

% ---------------------------------%
% latex-math dependencies, do not remove:
% - \usepackage{mathtools}
% - \usepackage{bm}
% - \usepackage{siunitx}
% - \usepackage{dsfont}
% - \usepackage{xspace}
% ---------------------------------%

%--------------------------------------------------------%
%       Language, encoding, typography
%--------------------------------------------------------%

\usepackage[english]{babel}
\usepackage[utf8]{inputenc} % Enables inputting UTF-8 symbols
% Standard AMS suite
\usepackage{amsmath,amsfonts,amssymb}

% Font four double-stroke / blackboard letters for sets of numbers (N, R, ...)
% Distribution name is "doublestroke"
% According to https://mirror.physik.tu-berlin.de/pub/CTAN/fonts/doublestroke/dsdoc.pdf
% the "bbm" package does a similar thing and may be superfluous.
% Required for latex-math
\usepackage{dsfont}

% bbm – "Blackboard-style" cm fonts (https://www.ctan.org/pkg/bbm)
% Used to be in common.tex, loaded directly after this file
% Maybe superfluous given dsfont is loaded
% TODO: Check if really unused?
% \usepackage{bbm}

% bm – Access bold symbols in maths mode - https://ctan.org/pkg/bm
% Required for latex-math
% https://tex.stackexchange.com/questions/3238/bm-package-versus-boldsymbol
\usepackage{bm}

% pifont – Access to PostScript standard Symbol and Dingbats fonts
% Used for \newcommand{\xmark}{\ding{55}, which is never used
% aside from lecture_advml/attic/xx-automl/slides.Rnw
% \usepackage{pifont}

% Quotes (inline and display), provdes \enquote
% https://ctan.org/pkg/csquotes
\usepackage{csquotes}

% Adds arg to enumerate env, technically superseded by enumitem according
% to https://ctan.org/pkg/enumerate
% Replace with https://ctan.org/pkg/enumitem ?
\usepackage{enumerate}

% Line spacing - provides \singlespacing \doublespacing \onehalfspacing
% https://ctan.org/pkg/setspace
% TODO: Check if really unused?
%\usepackage{setspace}

% mathtools – Mathematical tools to use with amsmath
% https://ctan.org/pkg/mathtools?lang=en
% latex-math dependency according to latex-math repo
\usepackage{mathtools}

%--------------------------------------------------------%
%       Displaying code and algorithms
%--------------------------------------------------------%
\usepackage{verbatim}
\usepackage{algorithm}
\usepackage{algpseudocode}

%--------------------------------------------------------%
%       Tables
%--------------------------------------------------------%

% multi-row table cells: https://www.namsu.de/Extra/pakete/Multirow.html
\usepackage{multirow}

% long/multi-page tables: https://texdoc.org/serve/longtable.pdf/0
% TODO: Check if really unused?

\usepackage{longtable}

% pretty table env: https://ctan.org/pkg/booktabs?lang=en
% TODO: Check if really unused?
\usepackage{booktabs}

%--------------------------------------------------------%
%       Figures: Creating, placing, verbing
%--------------------------------------------------------%

% wrapfig - Wrapping text around figures https://de.overleaf.com/learn/latex/Wrapping_text_around_figures
\usepackage{wrapfig}

% Sub figures in figures and tables
% https://ctan.org/pkg/subfig -- supersedes subfigure package
% TODO: Check if really unused?
\usepackage{subfig}

% Actually it's pronounced PGF https://en.wikibooks.org/wiki/LaTeX/PGF/TikZ
\usepackage{tikz}

\usetikzlibrary{shapes,arrows,automata,positioning,calc,chains,trees, shadows}
\tikzset{
  %Define standard arrow tip
  >=stealth',
  %Define style for boxes
  punkt/.style={
    rectangle,
    rounded corners,
    draw=black, very thick,
    text width=6.5em,
    minimum height=2em,
    text centered},
  % Define arrow style
  pil/.style={
    ->,
    thick,
    shorten <=2pt,
    shorten >=2pt,}
}


% Unsorted
% textpos – Place boxes at arbitrary positions on the LATEX page
% https://ctan.org/pkg/textpos?lang=en
% Provides \begin{textblock}
 % TODO: Check if really unused?
\usepackage[absolute,overlay]{textpos}

% psfrag – Replace strings in encapsulated PostScript figures
% https://www.overleaf.com/latex/examples/psfrag-example/tggxhgzwrzhn
% https://ftp.mpi-inf.mpg.de/pub/tex/mirror/ftp.dante.de/pub/tex/macros/latex/contrib/psfrag/pfgguide.pdf
% Can't tell if this is needed
% TODO: Check if really unused?
\usepackage{psfrag}

% Maybe not great to use this https://tex.stackexchange.com/a/197/19093
% Use align instead -- TODO: Global search & replace to check
\usepackage{eqnarray}

\usepackage{colortbl}

% arydshln – Draw dash-lines in array/tabular
% https://www.ctan.org/pkg/arydshln
% !! "arydshln has to be loaded after array, longtable, colortab and/or colortbl"
% Provides \hdashline and \cdashline
% TODO: Check if really unused?
% \usepackage{arydshln}

% tabularx – Tabulars with adjustable-width columns
% https://ctan.org/pkg/tabularx
% Provides \begin{tabularx}
% TODO: Check if really unused?
% \usepackage{tabularx}

% placeins – Control float placement
% https://ctan.org/pkg/placeins
% Defines a \FloatBarrier command
% TODO: Check if really unused?
% \usepackage{placeins}


% framed – Framed or shaded regions that can break across pages
% https://ctan.org/pkg/framed
% Provides \begin{framed} which uses \colorbox{shadecolor} relying on \definecolor{shadecolor}.
% TODO: Check if really unused?
% \usepackage{framed}

% Used often in conjunction with \definecolor{shadecolor}{rgb}{0.969, 0.969, 0.969}
% Might be able to be removed or at least redefined to only have shadecolor (if needed)
\definecolor{fgcolor}{rgb}{0.345, 0.345, 0.345}
\definecolor{shadecolor}{rgb}{0.969, 0.969, 0.969}
\newenvironment{knitrout}{}{} % an empty environment to be redefined in TeX


% Defines macros and environments
\usepackage{../../style/lmu-lecture}

\let\code=\texttt % Used regularly
\let\proglang=\textsf % Unused?

% Not sure what/why this does
\setkeys{Gin}{width=0.9\textwidth}

\setbeamertemplate{frametitle}{\expandafter\uppercase\expandafter\insertframetitle}

% Can't find a reason why common.tex is not just part of this file?

% basic latex stuff
\newcommand{\pkg}[1]{{\fontseries{b}\selectfont #1}} %fontstyle for R packages
\newcommand{\lz}{\vspace{0.5cm}} %vertical space
\newcommand{\dlz}{\vspace{1cm}} %double vertical space
\newcommand{\oneliner}[1] % Oneliner for important statements
{\begin{block}{}\begin{center}\begin{Large}#1\end{Large}\end{center}\end{block}}


%new environments
\newenvironment{vbframe}  %frame with breaks and verbatim
{
 \begin{frame}[containsverbatim,allowframebreaks]
}
{
\end{frame}
}

\newenvironment{vframe}  %frame with verbatim without breaks (to avoid numbering one slided frames)
{
 \begin{frame}[containsverbatim]
}
{
\end{frame}
}

\newenvironment{blocki}[1]   % itemize block
{
 \begin{block}{#1}\begin{itemize}
}
{
\end{itemize}\end{block}
}

\newenvironment{fragileframe}[2]{  %fragile frame with framebreaks
\begin{frame}[allowframebreaks, fragile, environment = fragileframe]
\frametitle{#1}
#2}
{\end{frame}}


\newcommand{\myframe}[2]{  %short for frame with framebreaks
\begin{frame}[allowframebreaks]
\frametitle{#1}
#2
\end{frame}}

\newcommand{\remark}[1]{
  \textbf{Remark:} #1
}


\newenvironment{deleteframe}
{
\begingroup
\usebackgroundtemplate{\includegraphics[width=\paperwidth,height=\paperheight]{../style/color/red.png}}
 \begin{frame}
}
{
\end{frame}
\endgroup
}
\newenvironment{simplifyframe}
{
\begingroup
\usebackgroundtemplate{\includegraphics[width=\paperwidth,height=\paperheight]{../style/color/yellow.png}}
 \begin{frame}
}
{
\end{frame}
\endgroup
}\newenvironment{draftframe}
{
\begingroup
\usebackgroundtemplate{\includegraphics[width=\paperwidth,height=\paperheight]{../style/color/green.jpg}}
 \begin{frame}
}
{
\end{frame}
\endgroup
}
% https://tex.stackexchange.com/a/261480: textcolor that works in mathmode
\makeatletter
\renewcommand*{\@textcolor}[3]{%
  \protect\leavevmode
  \begingroup
    \color#1{#2}#3%
  \endgroup
}
\makeatother


%-------------------------------------------------------------------------------------------------------%
%  Unused stuff that needs to go but is kept here currently juuuust in case it was important after all  %
%-------------------------------------------------------------------------------------------------------%

% \newcommand{\hlnum}[1]{\textcolor[rgb]{0.686,0.059,0.569}{#1}}%
% \newcommand{\hlstr}[1]{\textcolor[rgb]{0.192,0.494,0.8}{#1}}%
% \newcommand{\hlcom}[1]{\textcolor[rgb]{0.678,0.584,0.686}{\textit{#1}}}%
% \newcommand{\hlopt}[1]{\textcolor[rgb]{0,0,0}{#1}}%
% \newcommand{\hlstd}[1]{\textcolor[rgb]{0.345,0.345,0.345}{#1}}%
% \newcommand{\hlkwa}[1]{\textcolor[rgb]{0.161,0.373,0.58}{\textbf{#1}}}%
% \newcommand{\hlkwb}[1]{\textcolor[rgb]{0.69,0.353,0.396}{#1}}%
% \newcommand{\hlkwc}[1]{\textcolor[rgb]{0.333,0.667,0.333}{#1}}%
% \newcommand{\hlkwd}[1]{\textcolor[rgb]{0.737,0.353,0.396}{\textbf{#1}}}%
% \let\hlipl\hlkwb

% \makeatletter
% \newenvironment{kframe}{%
%  \def\at@end@of@kframe{}%
%  \ifinner\ifhmode%
%   \def\at@end@of@kframe{\end{minipage}}%
%   \begin{minipage}{\columnwidth}%
%  \fi\fi%
%  \def\FrameCommand##1{\hskip\@totalleftmargin \hskip-\fboxsep
%  \colorbox{shadecolor}{##1}\hskip-\fboxsep
%      % There is no \\@totalrightmargin, so:
%      \hskip-\linewidth \hskip-\@totalleftmargin \hskip\columnwidth}%
%  \MakeFramed {\advance\hsize-\width
%    \@totalleftmargin\z@ \linewidth\hsize
%    \@setminipage}}%
%  {\par\unskip\endMakeFramed%
%  \at@end@of@kframe}
% \makeatother

% \definecolor{shadecolor}{rgb}{.97, .97, .97}
% \definecolor{messagecolor}{rgb}{0, 0, 0}
% \definecolor{warningcolor}{rgb}{1, 0, 1}
% \definecolor{errorcolor}{rgb}{1, 0, 0}
% \newenvironment{knitrout}{}{} % an empty environment to be redefined in TeX

% \usepackage{alltt}
% \newcommand{\SweaveOpts}[1]{}  % do not interfere with LaTeX
% \newcommand{\SweaveInput}[1]{} % because they are not real TeX commands
% \newcommand{\Sexpr}[1]{}       % will only be parsed by R
% \newcommand{\xmark}{\ding{55}}%

% math spaces
\newcommand{\N}{\mathds{N}}                                                 % N, naturals
\newcommand{\Z}{\mathds{Z}}                                                 % Z, integers
\newcommand{\Q}{\mathds{Q}}                                                 % Q, rationals
\newcommand{\R}{\mathds{R}}                                                 % R, reals
\newcommand{\C}{\mathds{C}}                                                 % C, complex
\newcommand{\HS}{\mathcal{H}}                                               % H, hilbertspace
\newcommand{\continuous}{\mathcal{C}}                                       % C, space of continuous functions
\newcommand{\M}{\mathcal{M}} 												% machine numbers
\newcommand{\epsm}{\epsilon_m} 												% maximum error


% basic math stuff
\newcommand{\xt}{\tilde x}													% x tilde
\def\argmax{\mathop{\sf arg\,max}}                                          % argmax
\def\argmin{\mathop{\sf arg\,min}}                                          % argmin
\newcommand{\sign}{\operatorname{sign}}                                     % sign, signum
\newcommand{\I}{\mathbb{I}}                                                 % I, indicator
\newcommand{\order}{\mathcal{O}}                                            % O, order
\newcommand{\fp}[2]{\frac{\partial #1}{\partial #2}}                        % partial derivative
\newcommand{\pd}[2]{\frac{\partial{#1}}{\partial #2}}						% partial derivative

% sums and products
\newcommand{\sumin}{\sum_{i=1}^n}											% summation from i=1 to n
\newcommand{\sumkg}{\sum_{k=1}^g}											% summation from k=1 to g
\newcommand{\prodin}{\prod_{i=1}^n}											% product from i=1 to n
\newcommand{\prodkg}{\prod_{k=1}^g}											% product from k=1 to g

% linear algebra
\newcommand{\one}{\boldsymbol{1}}                                           % 1, unitvector
\newcommand{\id}{\mathrm{I}}                                                % I, identity
\newcommand{\diag}{\operatorname{diag}}                                     % diag, diagonal
\newcommand{\trace}{\operatorname{tr}}                                      % tr, trace
\newcommand{\spn}{\operatorname{span}}                                      % span
\newcommand{\scp}[2]{\left\langle #1, #2 \right\rangle}                     % <.,.>, scalarproduct
\newcommand{\mat}[1]{ 														% short pmatrix command
	\begin{pmatrix}
		#1
	\end{pmatrix}
}
\newcommand{\Amat}{\bm{A}}													% matrix A
\newcommand{\xv}{\bm{x}}													% vector x (bold)
\newcommand{\yv}{\bm{y}}														% vector y (bold)
\newcommand{\Deltab}{\bm{\Delta}}											% error term for vectors
															

% basic probability + stats
\renewcommand{\P}{\mathds{P}}                                               % P, probability
\newcommand{\E}{\mathds{E}}                                                 % E, expectation
\newcommand{\var}{\mathsf{Var}}                                             % Var, variance
\newcommand{\cov}{\mathsf{Cov}}                                             % Cov, covariance
\newcommand{\corr}{\mathsf{Corr}}                                           % Corr, correlation
\newcommand{\normal}{\mathcal{N}}                                           % N of the normal distribution
\newcommand{\iid}{\overset{i.i.d}{\sim}}                                    % dist with i.i.d superscript
\newcommand{\distas}[1]{\overset{#1}{\sim}}                                 % ... is distributed as ... 
% machine learning

%%%%%% ml - data
\newcommand{\Xspace}{\mathcal{X}}                                           % X, input space
\newcommand{\Yspace}{\mathcal{Y}}                                           % Y, output space
\newcommand{\nset}{\{1, \ldots, n\}}                                        % set from 1 to n
\newcommand{\pset}{\{1, \ldots, p\}}                                        % set from 1 to p
\newcommand{\gset}{\{1, \ldots, g\}}                                        % set from 1 to g
\newcommand{\Pxy}{\P_{xy}}                                                  % P_xy
\newcommand{\xy}{(x, y)}                                                    % observation (x, y)
\newcommand{\xvec}{(x_1, \ldots, x_p)^T}                                    % (x1, ..., xp) 
\newcommand{\D}{\mathcal{D}}                                                % D, data 
\newcommand{\Dset}{\{ (x^{(1)}, y^{(1)}), \ldots, (x^{(n)},  y^{(n)})\}}    % {(x1,y1)), ..., (xn,yn)}, data
\newcommand{\xdat}{\{ x^{(1)}, \ldots, x^{(n)}\}}   						 % {x1, ..., xn}, input data
\newcommand{\ydat}{\mathbf{y}}                                              % y (bold), vector of outcomes
\newcommand{\yvec}{(y^{(1)}, \hdots, y^{(n)})^T}                            % (y1, ..., yn), vector of outcomes
\renewcommand{\xi}[1][i]{x^{(#1)}}                                          % x^i, i-th observed value of x
\newcommand{\yi}[1][i]{y^{(#1)}}                                            % y^i, i-th observed value of y 
\newcommand{\xyi}{(\xi, \yi)}                                               % (x^i, y^i), i-th observation
\newcommand{\xivec}{(x^{(i)}_1, \ldots, x^{(i)}_p)^T}                       % (x1^i, ..., xp^i), i-th observation vector
\newcommand{\xj}{x_j}                                                       % x_j, j-th feature
\newcommand{\xjb}{\mathbf{x}_j}                                             % x_j (bold), j-th feature vecor
\newcommand{\xjvec}{(x^{(1)}_j, \ldots, x^{(n)}_j)^T}                       % (x^1_j, ..., x^n_j), j-th feature vector
\newcommand{\Dtrain}{\mathcal{D}_{\text{train}}}                            % D_train, training set
\newcommand{\Dtest}{\mathcal{D}_{\text{test}}}                              % D_test, test set

%%%%%% ml - models general

% continuous prediction function f
\newcommand{\fx}{f(x)}                                                      % f(x), continuous prediction function
\newcommand{\Hspace}{H}														% hypothesis space where f is from
\newcommand{\fh}{\hat{f}}                                                   % f hat, estimated prediction function
\newcommand{\fxh}{\fh(x)}                                                   % fhat(x)
\newcommand{\fxt}{f(x | \theta)}                                            % f(x | theta)
\newcommand{\fxi}{f(\xi)}                                                   % f(x^(i))
\newcommand{\fxih}{\hat{f}(\xi)}                                            % f(x^(i))
\newcommand{\fxit}{f(x^{(i)} | \theta)}                                     % f(x^(i) | theta)
\newcommand{\fhD}{\fh_{\D}}                                                 % fhat_D, estimate of f based on D
\newcommand{\fhDtrain}{\fh_{\Dtrain}}                                       % fhat_Dtrain, estimate of f based on D

% discrete prediction function h
\newcommand{\hx}{h(x)}                                                      % h(x), discrete prediction function
\newcommand{\hh}{\hat{h}}                                                   % h hat
\newcommand{\hxh}{\hat{h}(x)}                                               % hhat(x)
\newcommand{\hxt}{h(x | \theta)}                                            % h(x | theta)
\newcommand{\hxi}{h(\xi)}                                                   % h(x^(i))
\newcommand{\hxit}{h(x^{(i)} | \theta)}                                     % h(x^(i) | theta)

% yhat
\newcommand{\yh}{\hat{y}}                                                   % y hat for prediction of target
\newcommand{\yih}{\hat{y}}                                                  % y hat for prediction of target

% theta
\newcommand{\thetah}{\hat{\theta}}                                          % theta hat

% densities + probabilities
% pdf of x 
\newcommand{\pdf}{p}                                                        % p
\newcommand{\pdfx}{p(x)}                                                    % p(x)
\newcommand{\pixt}{\pi(x | \theta)}                                         % pi(x|theta), pdf of x given theta

% pdf of (x, y)
\newcommand{\pdfxy}{p(x,y)}                                                 % p(x, y)
\newcommand{\pdfxyt}{p(x, y | \theta)}                                      % p(x, y | theta)
\newcommand{\pdfxyit}{p(\xi, \yi | \theta)}                                 % p(x^(i), y^(i) | theta)

% pdf of x given y
\newcommand{\pdfxyk}{p(x | y=k)}                                            % p(x | y = k)
\newcommand{\lpdfxyk}{\log \pdfxyk}                                         % log p(x | y = k)
\newcommand{\pdfxiyk}{p(\xi | y=k)}                                         % p(x^i | y = k)

% prior probabilities
\newcommand{\pik}{\pi_k}                                                    % pi_k, prior
\newcommand{\lpik}{\log \pik}                                               % log pi_k, log of the prior

% posterior probabilities
\newcommand{\post}{\P(y = 1 | x)}                                           % P(y = 1 | x), post. prob for y=1
\newcommand{\pix}{\pi(x)}                                                   % pi(x), P(y = 1 | x)
\newcommand{\postk}{\P(y = k | x)}                                          % P(y = k | y), post. prob for y=k
\newcommand{\pikx}{\pi_k(x)}                                                % pi_k(x), P(y = k | x)
\newcommand{\pikxt}{\pi_k(x | \theta)}                                      % pi_k(x | theta), P(y = k | x, theta)
\newcommand{\pijx}{\pi_j(x)}                                                % pi_j(x), P(y = j | x)
\newcommand{\pdfygxt}{p(y |x, \theta)}                                      % p(y | x, theta)
\newcommand{\pdfyigxit}{p(\yi |\xi, \theta)}                                % p(y^i |x^i, theta)
\newcommand{\lpdfygxt}{\log \pdfygxt }                                      % log p(y | x, theta)
\newcommand{\lpdfyigxit}{\log \pdfyigxit}                                   % log p(y^i |x^i, theta)
\newcommand{\pixh}{\hat \pi(x)}                                             % pi(x) hat, P(y = 1 | x) hat
\newcommand{\pikxh}{\hat \pi_k(x)}                                          % pi_k(x) hat, P(y = k | x) hat

% residual and margin
\newcommand{\eps}{\epsilon}                                                 % residual, stochastic
\newcommand{\epsi}{\epsilon^{(i)}}                                          % epsilon^i, residual, stochastic
\newcommand{\epsh}{\hat{\epsilon}}                                          % residual, estimated
\newcommand{\yf}{y \fx}                                                     % y f(x), margin
\newcommand{\yfi}{\yi \fxi}                                                 % y^i f(x^i), margin
\newcommand{\Sigmah}{\hat \Sigma}											% estimated covariance matrix
\newcommand{\Sigmahj}{\hat \Sigma_j}										% estimated covariance matrix for the j-th class

% ml - loss, risk, likelihood
\newcommand{\Lxy}{L(y, f(x))}                                               % L(y, f(x)), loss function
\newcommand{\Lxyi}{L(\yi, \fxi)}                                            % L(y^i, f(x^i))
\newcommand{\Lxyt}{L(y, \fxt)}                                              % L(y, f(x | theta))
\newcommand{\Lxyit}{L(\yi, \fxit)}                                          % L(y^i, f(x^i | theta)
\newcommand{\risk}{\mathcal{R}}                                             % R, risk
\newcommand{\riskf}{\risk(f)}                                               % R(f), risk
\newcommand{\riske}{\mathcal{R}_{\text{emp}}}                               % R_emp, empirical risk
\newcommand{\riskef}{\riske(f)}                                             % R_emp(f)
\newcommand{\risket}{\mathcal{R}_{\text{emp}}(\theta)}                      % R_emp(theta)
\newcommand{\riskr}{\mathcal{R}_{\text{reg}}}                               % R_reg, regularized risk
\newcommand{\riskrt}{\mathcal{R}_{\text{reg}}(\theta)}                      % R_reg(theta)
\newcommand{\riskrf}{\riskr(f)}                                             % R_reg(f)
\newcommand{\LL}{\mathcal{L}}                                               % L, likelihood
\newcommand{\LLt}{\mathcal{L}(\theta)}                                      % L(theta), likelihood
\renewcommand{\ll}{\ell}                                                    % l, log-likelihood
\newcommand{\llt}{\ell(\theta)}                                             % l(theta), log-likelihood
\newcommand{\LS}{\mathfrak{L}}                                              % ????????????
\newcommand{\TS}{\mathfrak{T}}                                              % ??????????????
\newcommand{\errtrain}{\text{err}_{\text{train}}}                           % training error
\newcommand{\errtest}{\text{err}_{\text{test}}}                             % training error
\newcommand{\errexp}{\overline{\text{err}_{\text{test}}}}                   % training error

% resampling
\newcommand{\GE}[1]{GE(\fh_{#1})}                                           % Generalization error GE
\newcommand{\GEh}[1]{\widehat{GE}_{#1}}                                     % Estimated train error
\newcommand{\GED}{\GE{\D}}                                                  % Generalization error GE
\newcommand{\EGEn}{EGE_n}                                                   % Generalization error GE
\newcommand{\EDn}{\E_{|D| = n}}                                             % Generalization error GE


% ml - irace
\newcommand{\costs}{\mathcal{C}} % costs
\newcommand{\Celite}{\theta^*} % elite configurations
\newcommand{\instances}{\mathcal{I}} % sequence of instances
\newcommand{\budget}{\mathcal{B}} % computational budget
% ml - Gaussian Process

\newcommand{\fvec}{[f(\xi[1]), \dots, f(\xi[n])]} % {f(x1), ..., f(xn)}
\newcommand{\fv}{\mathbf{f}} % bold f, function vector 
\newcommand{\mv}{\mathbf{m}} % bold m, GP mean vector 
\newcommand{\kv}{\mathbf{k}} % bold k, kernel mat partition 
\newcommand{\kcc}{k(\cdot, \cdot)} % k(.,.), kernel for arbitrary inputs 
\newcommand{\kxij}[2]{k(\xi, \xi[j])} % k(xi, xj), cov of x_i, x_j
\newcommand{\Kmat}{\mathbf{K}} % K, kernel mat
\newcommand{\Nmk}{\normal(\mv, \Kmat)} % n(m,K), Gaussian w/ mean vec, cov mat
\newcommand{\Nzk}{\normal(\zero, \Kmat)} % n(0,K), zero-mean Gaussian
\newcommand{\GPmk}{\mathcal{GP}(m(\cdot), \kcc)} % GP(m(.), k(.,.)), GP definition
\newcommand{\GPzk}{\mathcal{GP}(\zero, \kcc)} % GP(0, k(.,.)), zero-mean GP
\newcommand{\Xsubset}{\bm{X}} % bold X, finite subset from xspace
\newcommand{\fX}{f(\Xsubset)} % f(X), function vector of finite subset
\newcommand{\kXX}{k(\Xsubset, \Xsubset)} % k(X,X), cov for finite subset
\newcommand{\mX}{m(\Xsubset)} % m(X), mean for finite subset
\newcommand{\ls}{\ell} % length-scale
\newcommand{\xxtnorm}{\| \xv - \xtil\|} % ||x - xtilde||

% GP prediction
\newcommand{\xs}{\xv_\ast} % x_*, test obs features
\newcommand{\ys}{\yv_\ast} % y_*, test obs target
\newcommand{\fs}{\fv_\ast} % f_*, test obs fun vector
\newcommand{\Xs}{\Xmat_\ast} % X_*, test design matrix
\newcommand{\ks}{\kv_{\ast}} % k_*, cov vec of new obs with x
\newcommand{\kss}{\kv_{\ast \ast}} % k_**, cov vec of new obs
\newcommand{\Ks}{\Kmat_{\ast}} % K_*, cov mat of new obs with x
\newcommand{\Kss}{\Kmat_{\ast \ast}} % K_**, cov mat of new obs
\newcommand{\Kinv}{\Kmat^{-1}} % K^-1, inverse cov mat
\newcommand{\Ky}{\Kmat_y} % K_y, cov mat of y



\newcommand{\titlefigure}{figure_man/discrete/marginalization-more.png} %not best picture
\newcommand{\learninggoals}{
  \item \textcolor{blue}{XXX}
  \item \textcolor{blue}{XXX}
}

\title{Introduction to Machine Learning}
\date{}

\begin{document}

\lecturechapter{Gaussian Processes}
\lecture{Introduction to Machine Learning}


\begin{vbframe}{Weight-Space View}

\begin{itemize}
  \item Until now we considered a hypothesis space $\Hspace$ of parameterized functions $\fxt$ (in particular, the space of linear functions). 
  \item Using Bayesian inference, we derived distributions for $\thetab$ after having observed data $\D$. 
  \item Prior believes about the parameter are expressed via a prior distribution $q(\thetab)$, which is updated according to Bayes' rule 

  $$
  \underbrace{p(\thetab | \Xmat, \yv)}_{\text{posterior}} = \frac{\overbrace{p(\yv | \Xmat, \thetab)}^{\text{likelihood}}\overbrace{q(\thetab)}^{\text{prior}}}{\underbrace{p(\yv|\Xmat)}_{\text{marginal}}}. 
  $$
\end{itemize}

\end{vbframe}


\begin{vbframe}{Function-space View}

Let us change our point of view: 

\begin{itemize}
  \item Instead of \enquote{searching} for a parameter  $\thetab$ in the parameter space, we directly search in a space of \enquote{allowed} functions $\Hspace$.  
  \item We still use Bayesian inference, but instead specifying a prior distribution over a parameter, we specify a prior distribution \textbf{over functions} and update it according to the data points we have observed. 
\end{itemize}

\framebreak 

Intuitively, imagine we could draw a huge number of functions from some prior distribution over functions $^{(*)}$. 

\begin{figure}
  \includegraphics[width=0.8\textwidth]{figure_man/gp-sample/gp-sample-1-1.pdf}
\end{figure}

\vspace*{-0.5cm}

\begin{footnotesize}
  $^{(*)}$ We will see in a minute how distributions over functions can be specified. 
\end{footnotesize}

\framebreak 

\foreach \x in{1,2,3} {
    After observing some data points, we are only allowed to sample those functions, that are consistent with the data. \\
  \begin{figure}
    \includegraphics[width=0.8\textwidth]{figure_man/gp-sample/gp-sample-2-\x.pdf}
  \end{figure}
}

\framebreak 

As we observe more and more data points, the variety of functions consistent with the data shrinks. 
  \begin{figure}
    \includegraphics[width=0.8\textwidth]{figure_man/gp-sample/gp-sample-2-4.pdf}
  \end{figure}

\framebreak 

Inutitively, there is something like \enquote{mean} and a \enquote{variance} of a distribution over functions. 

  \begin{figure}
    \includegraphics[width=0.8\textwidth]{figure_man/gp-sample/gp-sample-2-4.pdf}
  \end{figure}

\end{vbframe}

\begin{frame}{Weight-space vs. Function-space View}

\begin{table}
  \begin{tabular}{cc}
  \textbf{Weight-Space View} & \textbf{Function-Space View} \vspace{4mm}\\ 
  Parameterize functions & \vspace{1mm}\\
  \footnotesize Example: $\fxt = \thetab^\top \xv$ & \vspace{3mm}\\
  Define distributions on $\thetab$ & Define distributions on $f$ \vspace{4mm}\\
  Inference in parameter space $\Theta$ & Inference in function space $\Hspace$
  \end{tabular}
\end{table}  

\lz

Next, we will see how we can define distributions over functions mathematically. 


\end{frame}

\section{Distributions on Functions}

\begin{vbframe}{Discrete Functions}

For simplicity, let us consider functions with finite domains first. 

\lz 


Let $\mathcal{X} = \left\{\xv^{(1)}, \dots , \xv^{(n)}\right\}$ be a finite set of elements and $\Hspace$ the set of all functions from $\mathcal{X} \to \R$.

\lz

Since the domain of any $h(.) \in \Hspace$ has only $n$ elements, we can represent the function $h(.)$ compactly as a $n$-dimensional vector $$\bm{h} = \left[h\left(\xv^{(1)}\right), \dots, h\left(\xv^{(n)}\right)\right].$$
\end{vbframe}


\begin{frame}{Discrete Functions}

\textbf{Example 1:} Let us consider $h: \Xspace \to \Yspace$ where the input space consists of \textbf{two} points $\Xspace = \{0, 1\}$. 

\lz 

Examples for functions that live in this space: 

\begin{figure}[h]
\foreach \x in{1,2,3} {
  \includegraphics<\x>[width=0.7\linewidth]{figure_man/discrete/example_2_\x.pdf} \par
}
\end{figure}


\end{frame}

\begin{frame}{Discrete Functions}

\textbf{Example 2:} Let us consider $h: \Xspace \to \Yspace$ where the input space consists of \textbf{five} points $\Xspace = \{0, 0.25, 0.5, 0.75, 1\}$.

\lz 

Examples for functions that live in this space: 

\begin{figure}[h]
\foreach \x in{1,2,3} {
  \includegraphics<\x>[width=0.7\linewidth]{figure_man/discrete/example_5_\x.pdf}\par
}
\end{figure}

\end{frame}


\begin{frame}{Discrete Functions}

\textbf{Example 3:} Let us consider $h: \Xspace \to \Yspace$ where the input space consists of \textbf{ten} points. 

\lz 

Examples for functions that live in this space: 

\begin{figure}[h]
\foreach \x in{1,2,3} {
  \includegraphics<\x>[width=0.7\linewidth]{figure_man/discrete/example_10_\x.pdf}\par
}
\end{figure}

\end{frame}


\begin{vbframe}{Distributions on Discrete Functions}

\vspace*{0.5cm}

One natural way to specify a probability function on discrete function $h \in \Hspace$ is to use the vector representation 
$$
  \bm{h} = \left[h\left(\xi[1]\right), h\left(\xi[2]\right), \dots, h\left(\xi[n]\right)\right]
$$ 


of the function.

\lz

Let us see $\bm{h}$ as a $n$-dimensional random variable. We will further assume the following normal distribution: 

$$
  \bm{h} \sim \mathcal{N}\left(\bm{m}, \bm{K}\right).
$$ 

\textbf{Note: } For now, we set $\bm{m} = \bm{0}$ and take the covariance matrix $\bm{K}$ as given. We will see later how they are chosen / estimated. 

\end{vbframe}

\begin{frame}{Discrete Functions}

\textbf{Example 1 (continued):} Let $h: \Xspace \to \Yspace$ be a function that is defined on \textbf{two} points $\Xspace$. We sample functions by sampling from a two-dimensional normal variable

$$
\bm{h} = [h(1), h(2)] \sim \mathcal{N}(\bm{m}, \bm{K})
$$


\begin{figure}[H]
\foreach \x in{1,2,3} {
  \includegraphics<\x>[width=0.4\linewidth]{figure_man/discrete/example_norm_2_\x-a.pdf} ~  \includegraphics<\x>[width=0.4\linewidth]{figure_man/discrete/example_norm_2_\x-b.pdf} 
} \par
\begin{footnotesize}
In this example, $m = (0, 0)$ and $K = \begin{pmatrix} 1 & 0.5 \\ 0.5 & 1 \end{pmatrix}$. 
\end{footnotesize}
\end{figure}

\end{frame}


\begin{frame}{Discrete Functions}

\textbf{Example 2 (continued):} Let us consider $h: \Xspace \to \Yspace$ where the input space consists of \textbf{five} points. We sample functions by sampling from a five-dimensional normal variable


$$
\bm{h} = [h(1), h(2), h(3), h(4), h(5)] \sim \mathcal{N}(\bm{m}, \bm{K})
$$

\begin{figure}[h]
\foreach \x in{1,2,3} {
  \includegraphics<\x>[width=0.4\linewidth]{figure_man/discrete/example_norm_5_\x-a.pdf} ~  \includegraphics<\x>[width=0.4\linewidth]{figure_man/discrete/example_norm_5_\x-b.pdf}
}
\end{figure}

\end{frame}

\begin{frame}{Discrete Functions}

\textbf{Example 3 (continued):} Let us consider $h: \Xspace \to \Yspace$ where the input space consists of \textbf{ten} points. We sample functions by sampling from ten-dimensional normal variable

$$
\bm{h} = [h(1), h(2), \dots, h(10)] \sim \mathcal{N}(\bm{m}, \bm{K})
$$

\begin{figure}[h]
\foreach \x in{1,2,3} {
  \includegraphics<\x>[width=0.4\linewidth]{figure_man/discrete/example_norm_10_\x-a.pdf} ~  \includegraphics<\x>[width=0.4\linewidth]{figure_man/discrete/example_norm_10_\x-b.pdf}
}
\end{figure}

\end{frame}


\begin{vbframe}{Role of the Covariance Function}

Note that the covariance controls the \enquote{shape} of the drawn function. Consider two extreme cases where function values are

\begin{enumerate}
  \item[a)] strongly correlated: $\bm{K} = \begin{footnotesize}\begin{pmatrix} 1 & 0.99 & \dots & 0.99 \\
  0.99 & 1 & \dots & 0.99 \\
  0.99 & 0.99 & \ddots & 0.99 \\
  0.99 & \dots & 0.99 & 1 \end{pmatrix}\end{footnotesize}$
  \item[b)] uncorrelated: $\bm{K} = \id$
\end{enumerate}

\begin{figure}
  \includegraphics[width=0.35\linewidth]{figure_man/discrete/example_extreme_50-1.pdf} ~~  \includegraphics[width=0.35\linewidth]{figure_man/discrete/example_extreme_50-2.pdf}
\end{figure}


\framebreak 

\begin{itemize}
  \item \enquote{Meaningful} functions (on a numeric space $\Xspace$) may be characterized by a spatial property: \vspace*{0.2cm}
  \begin{itemize}
    \item[] If two points $\xi, \xi[j]$ are close in $\Xspace$-space, their function values $f(\xi), f(\xi[j])$ should be close in $\Yspace$-space. 
  \end{itemize} \vspace*{0.2cm}
  In other words: If they are close in $\Xspace$-space, their functions values should be \textbf{correlated}! \vspace*{0.4cm}
  \item We can enforce that by choosing a covariance function with  
  $$
    \bm{K}_{ij} \text{ high, if } \xi[i], \xi[j] \text{ close.}
  $$

  \framebreak 

  \item We can compute the entries of the covariance matrix by a function that is based on the distance between $\xi, \xi[j]$, for example: 
  
  \vspace*{0.2cm}
  \begin{enumerate}
    \item[c)] Spatial correlation: \begin{footnotesize}$K_{ij} = k(\xi[i], \xi[j]) = \exp\left(-\frac{1}{2}\left|\xi - \xi[j]\right|^2\right)$\end{footnotesize}
  \end{enumerate}
  
\begin{figure}
  \includegraphics[width=0.45\linewidth]{figure_man/discrete/example_extreme_50-4.pdf} ~~  \includegraphics[width=0.45\linewidth]{figure_man/discrete/example_extreme_50-3.pdf}
\end{figure}

\end{itemize}

\begin{footnotesize}
\textbf{Note}: $k(\cdot,\cdot)$ is known as the \textbf{covariance function} or \textbf{kernel}. It will be studied in more detail later on.
\end{footnotesize}

\end{vbframe}




% \begin{vbframe}
% \begin{figure}
% 	\centering
% 	\includegraphics{figure_man/discrete/sample2.png} \\
% 	\begin{footnotesize} If we sample again, we get another function. 
% 	\end{footnotesize}
% \end{figure}


% However, we are usually interested in functions with infinite domain size. 

% \lz 

% This idea is extended to infinite domain size via \textbf{Gaussian processes}. 

% \end{vbframe}


\section{Gaussian Processes}

\begin{vbframe}{From Discrete to Continuous Functions}

\begin{itemize}
  \item We defined distributions on functions with discrete domain by defining a Gaussian on the vector of the respective function values 
  $$
    \mathbf{h} = [h(\xi[1]), h(\xi[2]), \dots, h(\xi[n])] \sim \mathcal{N}(\bm{m}, \bm{K})
  $$

  \item We can do this for $n \to \infty$ (as \enquote{granular} as we want)
  \begin{figure}
    \includegraphics[width = 0.9\textwidth]{figure_man/discrete/example_limit.pdf}
  \end{figure}
\end{itemize}

\end{vbframe}

\begin{frame}{From Discrete to Continuous Functions}


\begin{itemize}
  \item No matter how large $n$ is, we are still considering a function over a discrete domain. 
  \item How can we extend our definition to functions with \textbf{continuous domain} $\Xspace \subset \R$?
\end{itemize}

\end{frame}


\begin{frame}{Gaussian Processes: Intuition}

\begin{itemize}
  \only<1>{
    \item Intuitively, a function $f$ drawn from \textbf{Gaussian process} can be understood as an \enquote{infinite} long Gaussian random vector. 
    \item It is unclear how to handle an \enquote{infinite} long Gaussian random vector!
  \lz 
  \begin{figure}
    \includegraphics[width=0.3\textwidth]{figure_man/question.png}
  \end{figure}
  }
  \only<2-4>{
    \item Thus, it is required that for \textbf{any finite set} of inputs $\{\xi[1], \dots, \xi[n]\} \subset \Xspace$, the vector $\mathbf{f}$ has a Gaussian distribution
    $$
      \bm{f} = \left[f\left(\xi[1]\right), \dots, f\left(\xi[n]\right)\right] \sim \mathcal{N}\left(\bm{m}, \bm{K}\right),
    $$ 
    with $\bm{m}$ and $\bm{K}$ being calculated by a mean function $m(.)$ / covariance function $k(.,.)$.
    \item This property is called \textbf{Marginalization Property}. 
    \begin{figure}
      \only<2>{\includegraphics[width=0.4\textwidth]{figure_man/discrete/example_marginalization_5.pdf}\includegraphics[width=0.5\textwidth]{figure_man/discrete/marginalization-5.png}}
      \only<3>{\includegraphics[width=0.4\textwidth]{figure_man/discrete/example_marginalization_10.pdf}\includegraphics[width=0.5\textwidth]{figure_man/discrete/marginalization-more.png}}
      \only<4>{\includegraphics[width=0.4\textwidth]{figure_man/discrete/example_marginalization_50.pdf}\includegraphics[width=0.5\textwidth]{figure_man/discrete/marginalization-more.png}}
   \end{figure}
    }
\end{itemize}

\end{frame}


\begin{vbframe}{Gaussian Processes}

This intuitive explanation is formally defined as follows: 

\lz 

A function $\fx$ is generated by a GP $\gp$ if for \textbf{any finite} set of inputs $\left\{\xv^{(1)}, \dots, \xv^{(n)}\right\}$, the associated vector of function values $\bm{f} = \left(f(\xv^{(1)}), \dots, f(\xv^{(n)})\right)$ has a Gaussian distribution

$$
\bm{f} = \left[f\left(\xi[1]\right),\dots, f\left(\xi[n]\right)\right] \sim \mathcal{N}\left(\bm{m}, \bm{K}\right),
$$

with 


\begin{eqnarray*}
\textbf{m} &:=& \left(m\left(\xi\right)\right)_{i}, \quad
\textbf{K} := \left(k\left(\xi, \xv^{(j)}\right)\right)_{i,j}, 
\end{eqnarray*}
 
where $m(\xv)$ is called mean function and $k(\xv, \xv^\prime)$ is called covariance function. 


\framebreak 

\vspace*{0.5cm} 

A GP is thus \textbf{completely specified} by its mean and covariance function

\vspace*{-0.2cm}
\begin{eqnarray*}
m(\xv) &=& \E[f(\xv)] \\
k(\xv, \xv^\prime) &=& \E\biggl[\left( f(\xv) - \E[f(\xv)] \right) \left( f(\xv^\prime) - \E[f(\xv^\prime)] \right)\biggr]
\end{eqnarray*}

\vfill

\textbf{Note}: For now, we assume $m(\xv) \equiv 0$. This is not necessarily a drastic limitation - thus it is common to consider GPs with a zero mean function. 

% \framebreak

% \vspace*{0.5cm}

% Intuitively, one can think of a function $f$ drawn from a Gaussian process prior as a Gaussian distribution with an \enquote{infinitely} long mean vector and an \enquote{infinite by infinite} covariance matrix.

% \lz

% Each dimension of the Gaussian corresponds to an element $\xv$ from the domain $\mathcal{X}$. The corresponding component of the random vector represents the value of $f(\xv)$.

% \lz

% The \textbf{marginalization property} makes it possible to handle this \enquote{infinite} representation: evaluations of the process on any finite number of points follow a multivariate normal distribution.

\end{vbframe}

\begin{vbframe}{Sampling from a Gaussian process Prior}

We can draw functions from a Gaussian process prior. Let us consider $\fx \sim \mathcal{GP}\left(0, k(\xv, \xv^\prime)\right)$ with the squared exponential covariance function $^{(*)}$

$$
k(\xv, \xv^\prime) = \exp\left(-\frac{1}{2\ls^2}\|\xv - \xv^\prime\|^2\right), ~~ \ls = 1.
$$
\vspace{-4cm}
This specifies the Gaussian process completely. 

\vspace{8cm}
\footnotesize
$^{(*)}$ We will talk later about different choices of covariance functions. 

\normalsize

\framebreak 

To visualize a sample function, we 

\begin{itemize}
  \item choose a high number $n$ (equidistant) points $\left\{\xv^{(1)}, \dots, \xv^{(n)}\right\}$
  \item compute the corresponding covariance matrix $\Kmat = \left(k\left(\xi, \xv^{(j)}\right)\right)_{i,j}$ by plugging in all pairs $\xv^{(i)}, \xv^{(j)}$ 
  \item sample from a Gaussian $\bm{f} \sim \mathcal{N}(\bm{0}, \bm{K})$. 
\end{itemize}

We draw $10$ times from the Gaussian, to get $10$ different samples.  

% Using $100$ equidistant points, we repeat the process of generating the Gaussian $10$ times ($10$ different functions) and draw each function by connecting the sampled values. 

% \lz

\begin{figure}
  \includegraphics[width=0.9\textwidth]{figure_man/different-samples.png}
\end{figure}

\vspace{-0.2cm}
Since we specified the mean function to be zero $m(\xv) \equiv 0$, the drawn functions have zero mean.

\end{vbframe}


\section{Gaussian Processes as Indexed Family}




\begin{vbframe}{Gaussian processes as an Indexed Family}

% \begin{block}{Definition}
% A \textbf{Gaussian process} is a (infinite) collection of random variables, any \textbf{finite} number of which have a \textbf{joint Gaussian distribution}.
% \end{block}

% \lz

A Gaussian process is a special case of a \textbf{stochastic process} which is defined as a collection of random variables indexed by some index set (also called an \textbf{indexed family}). What does it mean? 

\lz 

An \textbf{indexed family} is a mathematical function (or \enquote{rule}) to map indices $t \in T$ to objects in $\mathcal{S}$. 

\begin{block}{Definition}
A \textbf{family of elements in $\mathcal{S}$ indexed by $T$} (indexed family) is a surjective function 
\vspace*{-0.3cm}
\begin{eqnarray*}
s: T &\to& \mathcal{S} \\
   t &\mapsto& s_t = s(t) 
\end{eqnarray*}
\end{block}

\end{vbframe}

\begin{vbframe}{Indexed Family}

Some simple examples for indexed families are:

\vspace*{0.3cm}

\begin{minipage}{0.43\linewidth}
  \begin{itemize}
  \item finite sequences (lists): $T = \{1, 2, \dots, n\}$ and $\left(s_t\right)_{t \in T} \in \R$ \vspace{1cm}
  \item infinite sequences: $T = \N$ and $\left(s_t\right)_{t \in T} \in \R$
  \end{itemize}
\end{minipage}
\begin{minipage}{0.55\linewidth}
\includegraphics{figure_man/indexed_family/indexed_family_1.png} \\
\includegraphics{figure_man/indexed_family/indexed_family_2.png}
\end{minipage}


\framebreak

But the indexed set $\mathcal{S}$ can be something more complicated, for example functions or \textbf{random variables} (RV):

\begin{minipage}{0.43\linewidth}
  \vspace*{0.5cm}
  \begin{itemize}
    \item $T = \{1, \dots, m\}$, $Y_t$'s are RVs: Indexed family is a random vector. \vspace*{0.2cm}
    \item $T = \{1, \dots, m\}$, $Y_t$'s are RVs: Indexed family is a stochastic process in discrete time \vspace*{0.2cm}
    \item $T = \Z^2$, $Y_t$'s are RVs: Indexed family is a 2D-random walk.
  \end{itemize}
\end{minipage}\hfill
\begin{minipage}{0.5\linewidth}
\includegraphics{figure_man/indexed_family/indexed_family_4.png} \\
\includegraphics{figure_man/indexed_family/indexed_family_3.png}
\end{minipage}

\end{vbframe}

\begin{frame}{Indexed Family}

\begin{itemize}
  \item A Gaussian process is also an indexed family, where the random variables $f(\xv)$ are indexed by the input values $\xv \in \Xspace$. 
  \item Their special feature: Any indexed (finite) random vector has a multivariate Gaussian distribution (which comes with all the nice properties of Gaussianity!). 
\end{itemize}

\begin{figure}
  \includegraphics<1>[width=0.7\textwidth]{figure_man/indexed_family/indexed_family_5.png} \par
  \only<1>{\begin{footnotesize} Visualization for a one-dimensional $\Xspace$. \end{footnotesize}}
  \includegraphics<2>[width=0.6\textwidth]{figure_man/indexed_family/indexed_family_6.png}\par
  \only<2>{\begin{footnotesize} Visualization for a two-dimensional $\Xspace$. \end{footnotesize}}
\end{figure}

\end{frame}


\endlecture
\end{document}
