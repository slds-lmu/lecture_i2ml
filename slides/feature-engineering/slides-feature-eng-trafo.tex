\usepackage[]{graphicx}
\usepackage[]{color}
% maxwidth is the original width if it is less than linewidth
% otherwise use linewidth (to make sure the graphics do not exceed the margin)
\makeatletter
\def\maxwidth{ %
  \ifdim\Gin@nat@width>\linewidth
    \linewidth
  \else
    \Gin@nat@width
  \fi
}
\makeatother

% ---------------------------------%
% latex-math dependencies, do not remove:
% - \usepackage{mathtools}
% - \usepackage{bm}
% - \usepackage{siunitx}
% - \usepackage{dsfont}
% - \usepackage{xspace}
% ---------------------------------%

%--------------------------------------------------------%
%       Language, encoding, typography
%--------------------------------------------------------%

\usepackage[english]{babel}
\usepackage[utf8]{inputenc} % Enables inputting UTF-8 symbols
% Standard AMS suite
\usepackage{amsmath,amsfonts,amssymb}

% Font four double-stroke / blackboard letters for sets of numbers (N, R, ...)
% Distribution name is "doublestroke"
% According to https://mirror.physik.tu-berlin.de/pub/CTAN/fonts/doublestroke/dsdoc.pdf
% the "bbm" package does a similar thing and may be superfluous.
% Required for latex-math
\usepackage{dsfont}

% bbm – "Blackboard-style" cm fonts (https://www.ctan.org/pkg/bbm)
% Used to be in common.tex, loaded directly after this file
% Maybe superfluous given dsfont is loaded
% TODO: Check if really unused?
% \usepackage{bbm}

% bm – Access bold symbols in maths mode - https://ctan.org/pkg/bm
% Required for latex-math
% https://tex.stackexchange.com/questions/3238/bm-package-versus-boldsymbol
\usepackage{bm}

% pifont – Access to PostScript standard Symbol and Dingbats fonts
% Used for \newcommand{\xmark}{\ding{55}, which is never used
% aside from lecture_advml/attic/xx-automl/slides.Rnw
% \usepackage{pifont}

% Quotes (inline and display), provdes \enquote
% https://ctan.org/pkg/csquotes
\usepackage{csquotes}

% Adds arg to enumerate env, technically superseded by enumitem according
% to https://ctan.org/pkg/enumerate
% Replace with https://ctan.org/pkg/enumitem ?
\usepackage{enumerate}

% Line spacing - provides \singlespacing \doublespacing \onehalfspacing
% https://ctan.org/pkg/setspace
% TODO: Check if really unused?
%\usepackage{setspace}

% mathtools – Mathematical tools to use with amsmath
% https://ctan.org/pkg/mathtools?lang=en
% latex-math dependency according to latex-math repo
\usepackage{mathtools}

%--------------------------------------------------------%
%       Displaying code and algorithms
%--------------------------------------------------------%
\usepackage{verbatim}
\usepackage{algorithm}
\usepackage{algpseudocode}

%--------------------------------------------------------%
%       Tables
%--------------------------------------------------------%

% multi-row table cells: https://www.namsu.de/Extra/pakete/Multirow.html
\usepackage{multirow}

% long/multi-page tables: https://texdoc.org/serve/longtable.pdf/0
% TODO: Check if really unused?

\usepackage{longtable}

% pretty table env: https://ctan.org/pkg/booktabs?lang=en
% TODO: Check if really unused?
\usepackage{booktabs}

%--------------------------------------------------------%
%       Figures: Creating, placing, verbing
%--------------------------------------------------------%

% wrapfig - Wrapping text around figures https://de.overleaf.com/learn/latex/Wrapping_text_around_figures
\usepackage{wrapfig}

% Sub figures in figures and tables
% https://ctan.org/pkg/subfig -- supersedes subfigure package
% TODO: Check if really unused?
\usepackage{subfig}

% Actually it's pronounced PGF https://en.wikibooks.org/wiki/LaTeX/PGF/TikZ
\usepackage{tikz}

\usetikzlibrary{shapes,arrows,automata,positioning,calc,chains,trees, shadows}
\tikzset{
  %Define standard arrow tip
  >=stealth',
  %Define style for boxes
  punkt/.style={
    rectangle,
    rounded corners,
    draw=black, very thick,
    text width=6.5em,
    minimum height=2em,
    text centered},
  % Define arrow style
  pil/.style={
    ->,
    thick,
    shorten <=2pt,
    shorten >=2pt,}
}


% Unsorted
% textpos – Place boxes at arbitrary positions on the LATEX page
% https://ctan.org/pkg/textpos?lang=en
% Provides \begin{textblock}
 % TODO: Check if really unused?
\usepackage[absolute,overlay]{textpos}

% psfrag – Replace strings in encapsulated PostScript figures
% https://www.overleaf.com/latex/examples/psfrag-example/tggxhgzwrzhn
% https://ftp.mpi-inf.mpg.de/pub/tex/mirror/ftp.dante.de/pub/tex/macros/latex/contrib/psfrag/pfgguide.pdf
% Can't tell if this is needed
% TODO: Check if really unused?
\usepackage{psfrag}

% Maybe not great to use this https://tex.stackexchange.com/a/197/19093
% Use align instead -- TODO: Global search & replace to check
\usepackage{eqnarray}

\usepackage{colortbl}

% arydshln – Draw dash-lines in array/tabular
% https://www.ctan.org/pkg/arydshln
% !! "arydshln has to be loaded after array, longtable, colortab and/or colortbl"
% Provides \hdashline and \cdashline
% TODO: Check if really unused?
% \usepackage{arydshln}

% tabularx – Tabulars with adjustable-width columns
% https://ctan.org/pkg/tabularx
% Provides \begin{tabularx}
% TODO: Check if really unused?
% \usepackage{tabularx}

% placeins – Control float placement
% https://ctan.org/pkg/placeins
% Defines a \FloatBarrier command
% TODO: Check if really unused?
% \usepackage{placeins}


% framed – Framed or shaded regions that can break across pages
% https://ctan.org/pkg/framed
% Provides \begin{framed} which uses \colorbox{shadecolor} relying on \definecolor{shadecolor}.
% TODO: Check if really unused?
% \usepackage{framed}

% Used often in conjunction with \definecolor{shadecolor}{rgb}{0.969, 0.969, 0.969}
% Might be able to be removed or at least redefined to only have shadecolor (if needed)
\definecolor{fgcolor}{rgb}{0.345, 0.345, 0.345}
\definecolor{shadecolor}{rgb}{0.969, 0.969, 0.969}
\newenvironment{knitrout}{}{} % an empty environment to be redefined in TeX


% Defines macros and environments
\usepackage{../../style/lmu-lecture}

\let\code=\texttt % Used regularly
\let\proglang=\textsf % Unused?

% Not sure what/why this does
\setkeys{Gin}{width=0.9\textwidth}

\setbeamertemplate{frametitle}{\expandafter\uppercase\expandafter\insertframetitle}

% Can't find a reason why common.tex is not just part of this file?

% basic latex stuff
\newcommand{\pkg}[1]{{\fontseries{b}\selectfont #1}} %fontstyle for R packages
\newcommand{\lz}{\vspace{0.5cm}} %vertical space
\newcommand{\dlz}{\vspace{1cm}} %double vertical space
\newcommand{\oneliner}[1] % Oneliner for important statements
{\begin{block}{}\begin{center}\begin{Large}#1\end{Large}\end{center}\end{block}}


%new environments
\newenvironment{vbframe}  %frame with breaks and verbatim
{
 \begin{frame}[containsverbatim,allowframebreaks]
}
{
\end{frame}
}

\newenvironment{vframe}  %frame with verbatim without breaks (to avoid numbering one slided frames)
{
 \begin{frame}[containsverbatim]
}
{
\end{frame}
}

\newenvironment{blocki}[1]   % itemize block
{
 \begin{block}{#1}\begin{itemize}
}
{
\end{itemize}\end{block}
}

\newenvironment{fragileframe}[2]{  %fragile frame with framebreaks
\begin{frame}[allowframebreaks, fragile, environment = fragileframe]
\frametitle{#1}
#2}
{\end{frame}}


\newcommand{\myframe}[2]{  %short for frame with framebreaks
\begin{frame}[allowframebreaks]
\frametitle{#1}
#2
\end{frame}}

\newcommand{\remark}[1]{
  \textbf{Remark:} #1
}


\newenvironment{deleteframe}
{
\begingroup
\usebackgroundtemplate{\includegraphics[width=\paperwidth,height=\paperheight]{../style/color/red.png}}
 \begin{frame}
}
{
\end{frame}
\endgroup
}
\newenvironment{simplifyframe}
{
\begingroup
\usebackgroundtemplate{\includegraphics[width=\paperwidth,height=\paperheight]{../style/color/yellow.png}}
 \begin{frame}
}
{
\end{frame}
\endgroup
}\newenvironment{draftframe}
{
\begingroup
\usebackgroundtemplate{\includegraphics[width=\paperwidth,height=\paperheight]{../style/color/green.jpg}}
 \begin{frame}
}
{
\end{frame}
\endgroup
}
% https://tex.stackexchange.com/a/261480: textcolor that works in mathmode
\makeatletter
\renewcommand*{\@textcolor}[3]{%
  \protect\leavevmode
  \begingroup
    \color#1{#2}#3%
  \endgroup
}
\makeatother


%-------------------------------------------------------------------------------------------------------%
%  Unused stuff that needs to go but is kept here currently juuuust in case it was important after all  %
%-------------------------------------------------------------------------------------------------------%

% \newcommand{\hlnum}[1]{\textcolor[rgb]{0.686,0.059,0.569}{#1}}%
% \newcommand{\hlstr}[1]{\textcolor[rgb]{0.192,0.494,0.8}{#1}}%
% \newcommand{\hlcom}[1]{\textcolor[rgb]{0.678,0.584,0.686}{\textit{#1}}}%
% \newcommand{\hlopt}[1]{\textcolor[rgb]{0,0,0}{#1}}%
% \newcommand{\hlstd}[1]{\textcolor[rgb]{0.345,0.345,0.345}{#1}}%
% \newcommand{\hlkwa}[1]{\textcolor[rgb]{0.161,0.373,0.58}{\textbf{#1}}}%
% \newcommand{\hlkwb}[1]{\textcolor[rgb]{0.69,0.353,0.396}{#1}}%
% \newcommand{\hlkwc}[1]{\textcolor[rgb]{0.333,0.667,0.333}{#1}}%
% \newcommand{\hlkwd}[1]{\textcolor[rgb]{0.737,0.353,0.396}{\textbf{#1}}}%
% \let\hlipl\hlkwb

% \makeatletter
% \newenvironment{kframe}{%
%  \def\at@end@of@kframe{}%
%  \ifinner\ifhmode%
%   \def\at@end@of@kframe{\end{minipage}}%
%   \begin{minipage}{\columnwidth}%
%  \fi\fi%
%  \def\FrameCommand##1{\hskip\@totalleftmargin \hskip-\fboxsep
%  \colorbox{shadecolor}{##1}\hskip-\fboxsep
%      % There is no \\@totalrightmargin, so:
%      \hskip-\linewidth \hskip-\@totalleftmargin \hskip\columnwidth}%
%  \MakeFramed {\advance\hsize-\width
%    \@totalleftmargin\z@ \linewidth\hsize
%    \@setminipage}}%
%  {\par\unskip\endMakeFramed%
%  \at@end@of@kframe}
% \makeatother

% \definecolor{shadecolor}{rgb}{.97, .97, .97}
% \definecolor{messagecolor}{rgb}{0, 0, 0}
% \definecolor{warningcolor}{rgb}{1, 0, 1}
% \definecolor{errorcolor}{rgb}{1, 0, 0}
% \newenvironment{knitrout}{}{} % an empty environment to be redefined in TeX

% \usepackage{alltt}
% \newcommand{\SweaveOpts}[1]{}  % do not interfere with LaTeX
% \newcommand{\SweaveInput}[1]{} % because they are not real TeX commands
% \newcommand{\Sexpr}[1]{}       % will only be parsed by R
% \newcommand{\xmark}{\ding{55}}%

% math spaces
\newcommand{\N}{\mathds{N}}                                                 % N, naturals
\newcommand{\Z}{\mathds{Z}}                                                 % Z, integers
\newcommand{\Q}{\mathds{Q}}                                                 % Q, rationals
\newcommand{\R}{\mathds{R}}                                                 % R, reals
\newcommand{\C}{\mathds{C}}                                                 % C, complex
\newcommand{\HS}{\mathcal{H}}                                               % H, hilbertspace
\newcommand{\continuous}{\mathcal{C}}                                       % C, space of continuous functions
\newcommand{\M}{\mathcal{M}} 												% machine numbers
\newcommand{\epsm}{\epsilon_m} 												% maximum error


% basic math stuff
\newcommand{\xt}{\tilde x}													% x tilde
\def\argmax{\mathop{\sf arg\,max}}                                          % argmax
\def\argmin{\mathop{\sf arg\,min}}                                          % argmin
\newcommand{\sign}{\operatorname{sign}}                                     % sign, signum
\newcommand{\I}{\mathbb{I}}                                                 % I, indicator
\newcommand{\order}{\mathcal{O}}                                            % O, order
\newcommand{\fp}[2]{\frac{\partial #1}{\partial #2}}                        % partial derivative
\newcommand{\pd}[2]{\frac{\partial{#1}}{\partial #2}}						% partial derivative

% sums and products
\newcommand{\sumin}{\sum_{i=1}^n}											% summation from i=1 to n
\newcommand{\sumkg}{\sum_{k=1}^g}											% summation from k=1 to g
\newcommand{\prodin}{\prod_{i=1}^n}											% product from i=1 to n
\newcommand{\prodkg}{\prod_{k=1}^g}											% product from k=1 to g

% linear algebra
\newcommand{\one}{\boldsymbol{1}}                                           % 1, unitvector
\newcommand{\id}{\mathrm{I}}                                                % I, identity
\newcommand{\diag}{\operatorname{diag}}                                     % diag, diagonal
\newcommand{\trace}{\operatorname{tr}}                                      % tr, trace
\newcommand{\spn}{\operatorname{span}}                                      % span
\newcommand{\scp}[2]{\left\langle #1, #2 \right\rangle}                     % <.,.>, scalarproduct
\newcommand{\mat}[1]{ 														% short pmatrix command
	\begin{pmatrix}
		#1
	\end{pmatrix}
}
\newcommand{\Amat}{\bm{A}}													% matrix A
\newcommand{\xv}{\bm{x}}													% vector x (bold)
\newcommand{\yv}{\bm{y}}														% vector y (bold)
\newcommand{\Deltab}{\bm{\Delta}}											% error term for vectors
															

% basic probability + stats
\renewcommand{\P}{\mathds{P}}                                               % P, probability
\newcommand{\E}{\mathds{E}}                                                 % E, expectation
\newcommand{\var}{\mathsf{Var}}                                             % Var, variance
\newcommand{\cov}{\mathsf{Cov}}                                             % Cov, covariance
\newcommand{\corr}{\mathsf{Corr}}                                           % Corr, correlation
\newcommand{\normal}{\mathcal{N}}                                           % N of the normal distribution
\newcommand{\iid}{\overset{i.i.d}{\sim}}                                    % dist with i.i.d superscript
\newcommand{\distas}[1]{\overset{#1}{\sim}}                                 % ... is distributed as ... 
% machine learning

%%%%%% ml - data
\newcommand{\Xspace}{\mathcal{X}}                                           % X, input space
\newcommand{\Yspace}{\mathcal{Y}}                                           % Y, output space
\newcommand{\nset}{\{1, \ldots, n\}}                                        % set from 1 to n
\newcommand{\pset}{\{1, \ldots, p\}}                                        % set from 1 to p
\newcommand{\gset}{\{1, \ldots, g\}}                                        % set from 1 to g
\newcommand{\Pxy}{\P_{xy}}                                                  % P_xy
\newcommand{\xy}{(x, y)}                                                    % observation (x, y)
\newcommand{\xvec}{(x_1, \ldots, x_p)^T}                                    % (x1, ..., xp) 
\newcommand{\D}{\mathcal{D}}                                                % D, data 
\newcommand{\Dset}{\{ (x^{(1)}, y^{(1)}), \ldots, (x^{(n)},  y^{(n)})\}}    % {(x1,y1)), ..., (xn,yn)}, data
\newcommand{\xdat}{\{ x^{(1)}, \ldots, x^{(n)}\}}   						 % {x1, ..., xn}, input data
\newcommand{\ydat}{\mathbf{y}}                                              % y (bold), vector of outcomes
\newcommand{\yvec}{(y^{(1)}, \hdots, y^{(n)})^T}                            % (y1, ..., yn), vector of outcomes
\renewcommand{\xi}[1][i]{x^{(#1)}}                                          % x^i, i-th observed value of x
\newcommand{\yi}[1][i]{y^{(#1)}}                                            % y^i, i-th observed value of y 
\newcommand{\xyi}{(\xi, \yi)}                                               % (x^i, y^i), i-th observation
\newcommand{\xivec}{(x^{(i)}_1, \ldots, x^{(i)}_p)^T}                       % (x1^i, ..., xp^i), i-th observation vector
\newcommand{\xj}{x_j}                                                       % x_j, j-th feature
\newcommand{\xjb}{\mathbf{x}_j}                                             % x_j (bold), j-th feature vecor
\newcommand{\xjvec}{(x^{(1)}_j, \ldots, x^{(n)}_j)^T}                       % (x^1_j, ..., x^n_j), j-th feature vector
\newcommand{\Dtrain}{\mathcal{D}_{\text{train}}}                            % D_train, training set
\newcommand{\Dtest}{\mathcal{D}_{\text{test}}}                              % D_test, test set

%%%%%% ml - models general

% continuous prediction function f
\newcommand{\fx}{f(x)}                                                      % f(x), continuous prediction function
\newcommand{\Hspace}{H}														% hypothesis space where f is from
\newcommand{\fh}{\hat{f}}                                                   % f hat, estimated prediction function
\newcommand{\fxh}{\fh(x)}                                                   % fhat(x)
\newcommand{\fxt}{f(x | \theta)}                                            % f(x | theta)
\newcommand{\fxi}{f(\xi)}                                                   % f(x^(i))
\newcommand{\fxih}{\hat{f}(\xi)}                                            % f(x^(i))
\newcommand{\fxit}{f(x^{(i)} | \theta)}                                     % f(x^(i) | theta)
\newcommand{\fhD}{\fh_{\D}}                                                 % fhat_D, estimate of f based on D
\newcommand{\fhDtrain}{\fh_{\Dtrain}}                                       % fhat_Dtrain, estimate of f based on D

% discrete prediction function h
\newcommand{\hx}{h(x)}                                                      % h(x), discrete prediction function
\newcommand{\hh}{\hat{h}}                                                   % h hat
\newcommand{\hxh}{\hat{h}(x)}                                               % hhat(x)
\newcommand{\hxt}{h(x | \theta)}                                            % h(x | theta)
\newcommand{\hxi}{h(\xi)}                                                   % h(x^(i))
\newcommand{\hxit}{h(x^{(i)} | \theta)}                                     % h(x^(i) | theta)

% yhat
\newcommand{\yh}{\hat{y}}                                                   % y hat for prediction of target
\newcommand{\yih}{\hat{y}}                                                  % y hat for prediction of target

% theta
\newcommand{\thetah}{\hat{\theta}}                                          % theta hat

% densities + probabilities
% pdf of x 
\newcommand{\pdf}{p}                                                        % p
\newcommand{\pdfx}{p(x)}                                                    % p(x)
\newcommand{\pixt}{\pi(x | \theta)}                                         % pi(x|theta), pdf of x given theta

% pdf of (x, y)
\newcommand{\pdfxy}{p(x,y)}                                                 % p(x, y)
\newcommand{\pdfxyt}{p(x, y | \theta)}                                      % p(x, y | theta)
\newcommand{\pdfxyit}{p(\xi, \yi | \theta)}                                 % p(x^(i), y^(i) | theta)

% pdf of x given y
\newcommand{\pdfxyk}{p(x | y=k)}                                            % p(x | y = k)
\newcommand{\lpdfxyk}{\log \pdfxyk}                                         % log p(x | y = k)
\newcommand{\pdfxiyk}{p(\xi | y=k)}                                         % p(x^i | y = k)

% prior probabilities
\newcommand{\pik}{\pi_k}                                                    % pi_k, prior
\newcommand{\lpik}{\log \pik}                                               % log pi_k, log of the prior

% posterior probabilities
\newcommand{\post}{\P(y = 1 | x)}                                           % P(y = 1 | x), post. prob for y=1
\newcommand{\pix}{\pi(x)}                                                   % pi(x), P(y = 1 | x)
\newcommand{\postk}{\P(y = k | x)}                                          % P(y = k | y), post. prob for y=k
\newcommand{\pikx}{\pi_k(x)}                                                % pi_k(x), P(y = k | x)
\newcommand{\pikxt}{\pi_k(x | \theta)}                                      % pi_k(x | theta), P(y = k | x, theta)
\newcommand{\pijx}{\pi_j(x)}                                                % pi_j(x), P(y = j | x)
\newcommand{\pdfygxt}{p(y |x, \theta)}                                      % p(y | x, theta)
\newcommand{\pdfyigxit}{p(\yi |\xi, \theta)}                                % p(y^i |x^i, theta)
\newcommand{\lpdfygxt}{\log \pdfygxt }                                      % log p(y | x, theta)
\newcommand{\lpdfyigxit}{\log \pdfyigxit}                                   % log p(y^i |x^i, theta)
\newcommand{\pixh}{\hat \pi(x)}                                             % pi(x) hat, P(y = 1 | x) hat
\newcommand{\pikxh}{\hat \pi_k(x)}                                          % pi_k(x) hat, P(y = k | x) hat

% residual and margin
\newcommand{\eps}{\epsilon}                                                 % residual, stochastic
\newcommand{\epsi}{\epsilon^{(i)}}                                          % epsilon^i, residual, stochastic
\newcommand{\epsh}{\hat{\epsilon}}                                          % residual, estimated
\newcommand{\yf}{y \fx}                                                     % y f(x), margin
\newcommand{\yfi}{\yi \fxi}                                                 % y^i f(x^i), margin
\newcommand{\Sigmah}{\hat \Sigma}											% estimated covariance matrix
\newcommand{\Sigmahj}{\hat \Sigma_j}										% estimated covariance matrix for the j-th class

% ml - loss, risk, likelihood
\newcommand{\Lxy}{L(y, f(x))}                                               % L(y, f(x)), loss function
\newcommand{\Lxyi}{L(\yi, \fxi)}                                            % L(y^i, f(x^i))
\newcommand{\Lxyt}{L(y, \fxt)}                                              % L(y, f(x | theta))
\newcommand{\Lxyit}{L(\yi, \fxit)}                                          % L(y^i, f(x^i | theta)
\newcommand{\risk}{\mathcal{R}}                                             % R, risk
\newcommand{\riskf}{\risk(f)}                                               % R(f), risk
\newcommand{\riske}{\mathcal{R}_{\text{emp}}}                               % R_emp, empirical risk
\newcommand{\riskef}{\riske(f)}                                             % R_emp(f)
\newcommand{\risket}{\mathcal{R}_{\text{emp}}(\theta)}                      % R_emp(theta)
\newcommand{\riskr}{\mathcal{R}_{\text{reg}}}                               % R_reg, regularized risk
\newcommand{\riskrt}{\mathcal{R}_{\text{reg}}(\theta)}                      % R_reg(theta)
\newcommand{\riskrf}{\riskr(f)}                                             % R_reg(f)
\newcommand{\LL}{\mathcal{L}}                                               % L, likelihood
\newcommand{\LLt}{\mathcal{L}(\theta)}                                      % L(theta), likelihood
\renewcommand{\ll}{\ell}                                                    % l, log-likelihood
\newcommand{\llt}{\ell(\theta)}                                             % l(theta), log-likelihood
\newcommand{\LS}{\mathfrak{L}}                                              % ????????????
\newcommand{\TS}{\mathfrak{T}}                                              % ??????????????
\newcommand{\errtrain}{\text{err}_{\text{train}}}                           % training error
\newcommand{\errtest}{\text{err}_{\text{test}}}                             % training error
\newcommand{\errexp}{\overline{\text{err}_{\text{test}}}}                   % training error

% resampling
\newcommand{\GE}[1]{GE(\fh_{#1})}                                           % Generalization error GE
\newcommand{\GEh}[1]{\widehat{GE}_{#1}}                                     % Estimated train error
\newcommand{\GED}{\GE{\D}}                                                  % Generalization error GE
\newcommand{\EGEn}{EGE_n}                                                   % Generalization error GE
\newcommand{\EDn}{\E_{|D| = n}}                                             % Generalization error GE


% ml - irace
\newcommand{\costs}{\mathcal{C}} % costs
\newcommand{\Celite}{\theta^*} % elite configurations
\newcommand{\instances}{\mathcal{I}} % sequence of instances
\newcommand{\budget}{\mathcal{B}} % computational budget

\newcommand{\titlefigure}{figure_man/automl2.png}
\newcommand{\learninggoals}{
  \item ...
  \item ...
}

\title{Introduction to Machine Learning}
\date{}

\begin{document}

\lecturechapter{Feature and Target Transformations}
\lecture{Introduction to Machine Learning}


\begin{vbframe}{Target Transformation}
  
  Sometimes using the raw target or raw features is not enough to build an adequate model.
  For example, the linear model requires a normally distributed target variable.
  But the house prices do not seems to be normally distributed.
  A linear model trained on that target overestimates the target variable:
  
  
  \begin{center}
    \includegraphics[width=0.9\textwidth]{figure_man/target01.png}
  \end{center}
  
  
  \framebreak
  
  A common trick for skewed distributions is to model the log-transformation:
  
  
  \begin{center}
    \includegraphics[width=0.9\textwidth]{figure_man/target02.png}
  \end{center}
  
  \framebreak
  
  Benchmarking the logarithmic transformation against the raw data yields a significant improvement of the mean absolute error:
  
  \begin{center}
    \includegraphics[width=0.9\textwidth]{figure_man/target03.png}
  \end{center}
  
  \framebreak
  
  Nevertheless, there are also methods that are able to deal with skewed data:
  
  
  \begin{center}
    \includegraphics[width=0.9\textwidth]{figure_man/target04.png}
  \end{center}
  
\end{vbframe}

\begin{vbframe}{Feature Transformations}
  \begin{itemize}
    \item \textbf{Normalization}: The feature is transformed to have a mean of 0 and standard deviation of 1
          $$
            z_j^{(i)} = \frac{x_j^{(i)} - \operatorname{mean}(x_j)}{\operatorname{sd}(x_j)}
          $$
          
    \item \textbf{Box-Cox Transformation}: Stabilizes variance, makes the data more normal distribution-like
          $$
            z_j^{(i)} = \left\{\begin{array}{cc}
              \frac{\left(x_j^{(i)}\right)^\lambda - 1}{\lambda} & \ \ \text{if} \ \ \lambda \neq 0 \\
              \log(x_j^{(i)})                                    & \ \ \text{if} \ \ \lambda = 0    \\
            \end{array}\right.
          $$
  \end{itemize}
  \framebreak
  
  To illustrate the effect of transforming the features we evaluate a k-NN learner without scaling, with normalization, and with a Box-Cox transformation:
  
  \begin{center}
    \includegraphics[width=0.9\textwidth]{figure_man/feature01.png}
  \end{center}
  
\end{vbframe}

\begin{vbframe}{Other Common Transformations}
  \vspace{+1cm}
  \begin{itemize}
    \item Polynomials: $x_j \longrightarrow x_j, x_j^2, x_j^3, ...$
          
    \item Interactions: $x_j, x_k \longrightarrow x_j, x_k, x_j \times x_k$
          
    \item Basis expansions: BSplines, TPB, ...
  \end{itemize}
  \vspace{+.5cm}
  
  These transformations are used to improve simple models, e.g. linear regression, and most likely will \textbf{not} improve complex machine learning models.
\end{vbframe}

\begin{vbframe}{Feature Extraction vs. Selection}
  \begin{center}
    \includegraphics[width=\textwidth, trim=1cm 3cm 2cm 2cm]{figure_man/feat_extr_vs_selection.pdf}
  \end{center}
  
  Feature extraction / dimensionality reduction:
  \begin{itemize}
    \item PCA, ICA, autoencoder, ...
  \end{itemize}
  
  Feature selection:
  
  \begin{itemize}
    \item Filter, stepwise selection, model-based selection, ...
  \end{itemize}
\end{vbframe}


\section{Categorical Features}
\begin{vbframe}{Categorical Features}
  
  A categorical feature is a feature with a finite number of discrete (unordered) \textbf{levels} $c_1, \dots, c_K$%, e.g.,
  % \textbf{House.Style=2Story}$\stackrel{?}{>}$\textbf{SFoyer}.
  
  \begin{itemize}
    \item Categorical features are very common in practical applications.
          
    \item Except for few machine learning algorithms like tree-based methods, categorical features have to be encoded in a preprocessing step.
  \end{itemize}
  \medskip
  
  \textbf{Encoding} is the creation of a fully numeric representation of a categorical feature.
  \begin{itemize}
    \item Choosing the optimal encoding can be a challenge, especially when the number of levels $k$ becomes very large.
  \end{itemize}
\end{vbframe}

\begin{vbframe}{One-Hot Encoding}
  \begin{itemize}
    \item Convert each categorical feature to $K$ binary ($1/0$) features, where $K$ is the number of unique levels.
    \item One-hot encoding does not lose any information contained in the feature;  many models can correctly handle binary features.
    \item Given a categorical feature $x_j$ with levels $c_1,\dots, c_K$, the new features are
          
          $$
            \tilde x_{j,k} = \mathbf{1}_{[x_j = c_k]} \quad \text{for } k \in \{1, 2, ..., K\},
          $$
          $$
            \text{with}\quad \mathbf{1}_{[x_j = c_k]} = \begin{cases} 1 & \text{ if } x_j = c_k \\
              0 & \text{ otherwise}\end{cases}\text{.}
          $$
  \end{itemize}
  
  \lz 
  
  One-hot encoding is often the \textbf{go-to} choice for encoding of categorial features! 
  
\end{vbframe}

\begin{vbframe}{One-Hot Encoding: Example}
  Original slice of the dataset:
  \vspace{+.4cm}
  \footnotesize
  
  \begin{center}
    \begin{tabular}{c|c|c}
      \hline
      SalePrice & Central.Air & Bldg.Type \\
      \hline
      189900    & Y           & 1Fam      \\
      \hline
      195500    & Y           & 1Fam      \\
      \hline
      213500    & Y           & TwnhsE    \\
      \hline
      191500    & Y           & TwnhsE    \\
      \hline
      236500    & Y           & TwnhsE    \\
      \hline
    \end{tabular}
  \end{center}
  
  
  \normalsize{One-hot encoded:}
  \vspace{+.4cm}
  
  \scriptsize
  \begin{center}
    \begin{tabular}{c|c|c}
      \hline
      SalePrice & Central.Air & Bldg.Type \\
      \hline
      189900    & Y           & 1Fam      \\
      \hline
      195500    & Y           & 1Fam      \\
      \hline
      213500    & Y           & TwnhsE    \\
      \hline
      191500    & Y           & TwnhsE    \\
      \hline
      236500    & Y           & TwnhsE    \\
      \hline
    \end{tabular}
  \end{center}
  
\end{vbframe}

\begin{vbframe}{Dummy Encoding}
  \begin{itemize}
    \item Dummy encoding is very similar to one-hot encoding with the difference that only $K-1$ binary features are created.
    \item A \textbf{reference} category is chosen that has as all binary features set to $0$, i.e.,
          
          $$
            \tilde x_{j,1} = 0, \dots, \tilde x_{j,K-1} = 0.
          $$
          
    \item Each feature $\tilde x_{j,1}$ represents the \textbf{deviation} from the reference category.
    \item While using a reference category is required for stability and interpretability in statistical models like (generalized) linear models, it is not necessary, rarely done in ML and can even have negative influence on performance.
  \end{itemize}
\end{vbframe}

\begin{vbframe}{Ames Housing - Encoding}
  
  \begin{center}
    \includegraphics[width = 0.6\textwidth]{figure_man/ames-encoding.png}
  \end{center}
  
  \begin{footnotesize}
    Result of linear model depends on actual implementation, e.g., R's `lm()` produces a \textbf{rank-deficient fit} warning and recovers by dropping the intercept.
  \end{footnotesize}
\end{vbframe}

\begin{vbframe}{One-Hot Encoding: Limitations}
  \begin{itemize}
    \item One-hot encoding can become extremely inefficient when the number of levels becomes too large, because one additional feature is introduced for every level.
    \item Assume a categorical feature with $K = 4000$ levels. When using dummy encoding, 4000 new features are added to the dataset.
    \item These additional features are very sparse.
    \item Handling such \textbf{high-cardinality categorical features} is a challenge. Possible solutions are
          \begin{itemize}

            \item specialized methods such as \textbf{factorization machines},
            \item \textbf{target/impact encoding},
            \item clustering feature levels or
            \item feature hashing.
          \end{itemize}
  \end{itemize}
\end{vbframe}


\begin{vbframe}{Target Encoding}
  \begin{itemize}
    \item Developed to solve limitations of dummy encoding for high cardinality categorical features.
    \item \textbf{Goal}: Each categorical feature $\xv$ should be encoded in a single numeric feature $\tilde \xv$.
          \medskip
    \item Basic definition for regression by Micci-Barreca (2001):
          
          $$
            \tilde \xv = \frac{\sum_{i:\xv=k} \yi}{n_k}, \quad k = 1,\dots,K,
          $$
          
          where $n_k$ is the number of observations of the $k$'th level of feature $\xv$.
          
  \end{itemize}
  
\end{vbframe}

\begin{vbframe}{Target Encoding - Example}
  \vspace{+.4cm}
  
  
  \footnotesize
  \begin{center}
    \begin{tabular}{c|c|c|c|c|c|c}
      \hline
      Foundation & BrkTil & CBlock & PConc & Slab & Stone & Wood \\
      \hline
      nk         & 311    & 1244   & 1310  & 49   & 11    & 5    \\
      \hline
    \end{tabular}
  \end{center}
  
  
  \normalsize
  \begin{itemize}
    \item Encoding for wooden foundation:
  \end{itemize}
  \vspace{+.4cm}
  
  \footnotesize
  \begin{center}
    \begin{tabular}{c|c|c|c|c|c}
      \hline
      house.id   & 17     & 893    & 986    & 2898   & 2899   \\
      \hline
      SalePrice  & 164000 & 145500 & 143000 & 250000 & 202000 \\
      \hline
      Foundation & Wood   & Wood   & Wood   & Wood   & Wood   \\
      \hline
    \end{tabular}
  \end{center}
  
  \vspace{+.4cm}
  $$
    \frac{164000 + 145500 + 143000 + 250000 + 202000}{5} = 180900
  $$
  \framebreak
  
  
  \normalsize
  \begin{itemize}
    \item For all foundation types:
  \end{itemize}
  \vspace{+.4cm}
  
  \footnotesize
  \begin{tabular}{c|c|c|c|c|c|c}
    \hline
    Foundation      & BrkTil & CBlock & PConc  & Slab   & Stone  & Wood   \\
    \hline
    Foundation(enc) & 128107 & 148284 & 227069 & 110458 & 149787 & 180900 \\
    \hline
  \end{tabular}
  
  
  \vspace{+.4cm}
  \normalsize
  This mapping is calculated on training data and later applied to test data.
\end{vbframe}

\begin{vbframe}{Target Encoding for Classification}
  \begin{itemize}
    \item Extending encoding to binary classification is straightforward, instead of the average target value the relative frequency of the positive class is used.
    \item Multi-class classification extends this by creating one feature for each target class in the same way as binary classification.
  \end{itemize}
\end{vbframe}

\begin{vbframe}{Target Encoding - Issues}
  \textbf{Problem:} Target encoding can assign extreme values to rarely occurring levels.
  
  \vspace*{0.2cm}
  
  \textbf{Solution:} Encoding as weighted sum between global average target value and encoding value of level.
  
  $$
    \tilde \xv = \lambda_k\frac{\sum_{i:\xv=k}\yi}{n_k} + (1-\lambda_k)\frac{\sum_{i=1}^n  \yi} {n}, \quad k=1,\dots,K.
  $$
  
  \begin{itemize}
    \item $\lambda_k$ can be parameterized and tuned, but tuning should optimally be done for each feature and level separately (most likely infeasible!).
    \item Simple solution: Set $\lambda_k=\frac{n_k}{n_k+\epsilon}$ with regularization parameter $\epsilon$.
    \item This shrinks small levels stronger to the global mean target value than large classes.
  \end{itemize}
  \framebreak
  
  
  \textbf{Problem:} Label leakage! Information of $\yi$ is used to calculate $\tilde \xv$. This can cause overfitting issues, especially for rarely occurring classes.
  
  \vspace*{0.2cm}
  
  \textbf{Solution:} Use internal cross-validation to calculate $\tilde \xv$.
  \vspace{+.4cm}
  
  \begin{itemize}
    \item It is unclear how serious this problem is in practice.
    \item But: calculation of $\tilde \xv$ is very cheap, so it doesn't hurt.
    \item An alternative is to add some noise $\tilde x_j^{(n)} + N(0,\sigma_\epsilon)$ to the encoded samples.
  \end{itemize}
\end{vbframe}

\endlecture
\end{document}
