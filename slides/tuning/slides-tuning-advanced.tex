\usepackage[]{graphicx}
\usepackage[]{color}
% maxwidth is the original width if it is less than linewidth
% otherwise use linewidth (to make sure the graphics do not exceed the margin)
\makeatletter
\def\maxwidth{ %
  \ifdim\Gin@nat@width>\linewidth
    \linewidth
  \else
    \Gin@nat@width
  \fi
}
\makeatother

% ---------------------------------%
% latex-math dependencies, do not remove:
% - \usepackage{mathtools}
% - \usepackage{bm}
% - \usepackage{siunitx}
% - \usepackage{dsfont}
% - \usepackage{xspace}
% ---------------------------------%

%--------------------------------------------------------%
%       Language, encoding, typography
%--------------------------------------------------------%

\usepackage[english]{babel}
\usepackage[utf8]{inputenc} % Enables inputting UTF-8 symbols
% Standard AMS suite
\usepackage{amsmath,amsfonts,amssymb}

% Font four double-stroke / blackboard letters for sets of numbers (N, R, ...)
% Distribution name is "doublestroke"
% According to https://mirror.physik.tu-berlin.de/pub/CTAN/fonts/doublestroke/dsdoc.pdf
% the "bbm" package does a similar thing and may be superfluous.
% Required for latex-math
\usepackage{dsfont}

% bbm – "Blackboard-style" cm fonts (https://www.ctan.org/pkg/bbm)
% Used to be in common.tex, loaded directly after this file
% Maybe superfluous given dsfont is loaded
% TODO: Check if really unused?
% \usepackage{bbm}

% bm – Access bold symbols in maths mode - https://ctan.org/pkg/bm
% Required for latex-math
% https://tex.stackexchange.com/questions/3238/bm-package-versus-boldsymbol
\usepackage{bm}

% pifont – Access to PostScript standard Symbol and Dingbats fonts
% Used for \newcommand{\xmark}{\ding{55}, which is never used
% aside from lecture_advml/attic/xx-automl/slides.Rnw
% \usepackage{pifont}

% Quotes (inline and display), provdes \enquote
% https://ctan.org/pkg/csquotes
\usepackage{csquotes}

% Adds arg to enumerate env, technically superseded by enumitem according
% to https://ctan.org/pkg/enumerate
% Replace with https://ctan.org/pkg/enumitem ?
\usepackage{enumerate}

% Line spacing - provides \singlespacing \doublespacing \onehalfspacing
% https://ctan.org/pkg/setspace
% TODO: Check if really unused?
%\usepackage{setspace}

% mathtools – Mathematical tools to use with amsmath
% https://ctan.org/pkg/mathtools?lang=en
% latex-math dependency according to latex-math repo
\usepackage{mathtools}

%--------------------------------------------------------%
%       Displaying code and algorithms
%--------------------------------------------------------%
\usepackage{verbatim}
\usepackage{algorithm}
\usepackage{algpseudocode}

%--------------------------------------------------------%
%       Tables
%--------------------------------------------------------%

% multi-row table cells: https://www.namsu.de/Extra/pakete/Multirow.html
\usepackage{multirow}

% long/multi-page tables: https://texdoc.org/serve/longtable.pdf/0
% TODO: Check if really unused?

\usepackage{longtable}

% pretty table env: https://ctan.org/pkg/booktabs?lang=en
% TODO: Check if really unused?
\usepackage{booktabs}

%--------------------------------------------------------%
%       Figures: Creating, placing, verbing
%--------------------------------------------------------%

% wrapfig - Wrapping text around figures https://de.overleaf.com/learn/latex/Wrapping_text_around_figures
\usepackage{wrapfig}

% Sub figures in figures and tables
% https://ctan.org/pkg/subfig -- supersedes subfigure package
% TODO: Check if really unused?
\usepackage{subfig}

% Actually it's pronounced PGF https://en.wikibooks.org/wiki/LaTeX/PGF/TikZ
\usepackage{tikz}

\usetikzlibrary{shapes,arrows,automata,positioning,calc,chains,trees, shadows}
\tikzset{
  %Define standard arrow tip
  >=stealth',
  %Define style for boxes
  punkt/.style={
    rectangle,
    rounded corners,
    draw=black, very thick,
    text width=6.5em,
    minimum height=2em,
    text centered},
  % Define arrow style
  pil/.style={
    ->,
    thick,
    shorten <=2pt,
    shorten >=2pt,}
}


% Unsorted
% textpos – Place boxes at arbitrary positions on the LATEX page
% https://ctan.org/pkg/textpos?lang=en
% Provides \begin{textblock}
 % TODO: Check if really unused?
\usepackage[absolute,overlay]{textpos}

% psfrag – Replace strings in encapsulated PostScript figures
% https://www.overleaf.com/latex/examples/psfrag-example/tggxhgzwrzhn
% https://ftp.mpi-inf.mpg.de/pub/tex/mirror/ftp.dante.de/pub/tex/macros/latex/contrib/psfrag/pfgguide.pdf
% Can't tell if this is needed
% TODO: Check if really unused?
\usepackage{psfrag}

% Maybe not great to use this https://tex.stackexchange.com/a/197/19093
% Use align instead -- TODO: Global search & replace to check
\usepackage{eqnarray}

\usepackage{colortbl}

% arydshln – Draw dash-lines in array/tabular
% https://www.ctan.org/pkg/arydshln
% !! "arydshln has to be loaded after array, longtable, colortab and/or colortbl"
% Provides \hdashline and \cdashline
% TODO: Check if really unused?
% \usepackage{arydshln}

% tabularx – Tabulars with adjustable-width columns
% https://ctan.org/pkg/tabularx
% Provides \begin{tabularx}
% TODO: Check if really unused?
% \usepackage{tabularx}

% placeins – Control float placement
% https://ctan.org/pkg/placeins
% Defines a \FloatBarrier command
% TODO: Check if really unused?
% \usepackage{placeins}


% framed – Framed or shaded regions that can break across pages
% https://ctan.org/pkg/framed
% Provides \begin{framed} which uses \colorbox{shadecolor} relying on \definecolor{shadecolor}.
% TODO: Check if really unused?
% \usepackage{framed}

% Used often in conjunction with \definecolor{shadecolor}{rgb}{0.969, 0.969, 0.969}
% Might be able to be removed or at least redefined to only have shadecolor (if needed)
\definecolor{fgcolor}{rgb}{0.345, 0.345, 0.345}
\definecolor{shadecolor}{rgb}{0.969, 0.969, 0.969}
\newenvironment{knitrout}{}{} % an empty environment to be redefined in TeX


% Defines macros and environments
\usepackage{../../style/lmu-lecture}

\let\code=\texttt % Used regularly
\let\proglang=\textsf % Unused?

% Not sure what/why this does
\setkeys{Gin}{width=0.9\textwidth}

\setbeamertemplate{frametitle}{\expandafter\uppercase\expandafter\insertframetitle}

% Can't find a reason why common.tex is not just part of this file?

% basic latex stuff
\newcommand{\pkg}[1]{{\fontseries{b}\selectfont #1}} %fontstyle for R packages
\newcommand{\lz}{\vspace{0.5cm}} %vertical space
\newcommand{\dlz}{\vspace{1cm}} %double vertical space
\newcommand{\oneliner}[1] % Oneliner for important statements
{\begin{block}{}\begin{center}\begin{Large}#1\end{Large}\end{center}\end{block}}


%new environments
\newenvironment{vbframe}  %frame with breaks and verbatim
{
 \begin{frame}[containsverbatim,allowframebreaks]
}
{
\end{frame}
}

\newenvironment{vframe}  %frame with verbatim without breaks (to avoid numbering one slided frames)
{
 \begin{frame}[containsverbatim]
}
{
\end{frame}
}

\newenvironment{blocki}[1]   % itemize block
{
 \begin{block}{#1}\begin{itemize}
}
{
\end{itemize}\end{block}
}

\newenvironment{fragileframe}[2]{  %fragile frame with framebreaks
\begin{frame}[allowframebreaks, fragile, environment = fragileframe]
\frametitle{#1}
#2}
{\end{frame}}


\newcommand{\myframe}[2]{  %short for frame with framebreaks
\begin{frame}[allowframebreaks]
\frametitle{#1}
#2
\end{frame}}

\newcommand{\remark}[1]{
  \textbf{Remark:} #1
}


\newenvironment{deleteframe}
{
\begingroup
\usebackgroundtemplate{\includegraphics[width=\paperwidth,height=\paperheight]{../style/color/red.png}}
 \begin{frame}
}
{
\end{frame}
\endgroup
}
\newenvironment{simplifyframe}
{
\begingroup
\usebackgroundtemplate{\includegraphics[width=\paperwidth,height=\paperheight]{../style/color/yellow.png}}
 \begin{frame}
}
{
\end{frame}
\endgroup
}\newenvironment{draftframe}
{
\begingroup
\usebackgroundtemplate{\includegraphics[width=\paperwidth,height=\paperheight]{../style/color/green.jpg}}
 \begin{frame}
}
{
\end{frame}
\endgroup
}
% https://tex.stackexchange.com/a/261480: textcolor that works in mathmode
\makeatletter
\renewcommand*{\@textcolor}[3]{%
  \protect\leavevmode
  \begingroup
    \color#1{#2}#3%
  \endgroup
}
\makeatother


%-------------------------------------------------------------------------------------------------------%
%  Unused stuff that needs to go but is kept here currently juuuust in case it was important after all  %
%-------------------------------------------------------------------------------------------------------%

% \newcommand{\hlnum}[1]{\textcolor[rgb]{0.686,0.059,0.569}{#1}}%
% \newcommand{\hlstr}[1]{\textcolor[rgb]{0.192,0.494,0.8}{#1}}%
% \newcommand{\hlcom}[1]{\textcolor[rgb]{0.678,0.584,0.686}{\textit{#1}}}%
% \newcommand{\hlopt}[1]{\textcolor[rgb]{0,0,0}{#1}}%
% \newcommand{\hlstd}[1]{\textcolor[rgb]{0.345,0.345,0.345}{#1}}%
% \newcommand{\hlkwa}[1]{\textcolor[rgb]{0.161,0.373,0.58}{\textbf{#1}}}%
% \newcommand{\hlkwb}[1]{\textcolor[rgb]{0.69,0.353,0.396}{#1}}%
% \newcommand{\hlkwc}[1]{\textcolor[rgb]{0.333,0.667,0.333}{#1}}%
% \newcommand{\hlkwd}[1]{\textcolor[rgb]{0.737,0.353,0.396}{\textbf{#1}}}%
% \let\hlipl\hlkwb

% \makeatletter
% \newenvironment{kframe}{%
%  \def\at@end@of@kframe{}%
%  \ifinner\ifhmode%
%   \def\at@end@of@kframe{\end{minipage}}%
%   \begin{minipage}{\columnwidth}%
%  \fi\fi%
%  \def\FrameCommand##1{\hskip\@totalleftmargin \hskip-\fboxsep
%  \colorbox{shadecolor}{##1}\hskip-\fboxsep
%      % There is no \\@totalrightmargin, so:
%      \hskip-\linewidth \hskip-\@totalleftmargin \hskip\columnwidth}%
%  \MakeFramed {\advance\hsize-\width
%    \@totalleftmargin\z@ \linewidth\hsize
%    \@setminipage}}%
%  {\par\unskip\endMakeFramed%
%  \at@end@of@kframe}
% \makeatother

% \definecolor{shadecolor}{rgb}{.97, .97, .97}
% \definecolor{messagecolor}{rgb}{0, 0, 0}
% \definecolor{warningcolor}{rgb}{1, 0, 1}
% \definecolor{errorcolor}{rgb}{1, 0, 0}
% \newenvironment{knitrout}{}{} % an empty environment to be redefined in TeX

% \usepackage{alltt}
% \newcommand{\SweaveOpts}[1]{}  % do not interfere with LaTeX
% \newcommand{\SweaveInput}[1]{} % because they are not real TeX commands
% \newcommand{\Sexpr}[1]{}       % will only be parsed by R
% \newcommand{\xmark}{\ding{55}}%

\newcommand{\titlefigure}{figure/cart_tuning_ad_4.pdf}
\newcommand{\learninggoals}{
\item Understand the idea of model based optimization
\item Be able to explein the terms 'surrogate model' and 'expected improvement'
\item Understand the idea of hyperband}
\usepackage{../../style/lmu-lecture}
% machine learning

%%%%%% ml - data
\newcommand{\Xspace}{\mathcal{X}}                                           % X, input space
\newcommand{\Yspace}{\mathcal{Y}}                                           % Y, output space
\newcommand{\nset}{\{1, \ldots, n\}}                                        % set from 1 to n
\newcommand{\pset}{\{1, \ldots, p\}}                                        % set from 1 to p
\newcommand{\gset}{\{1, \ldots, g\}}                                        % set from 1 to g
\newcommand{\Pxy}{\P_{xy}}                                                  % P_xy
\newcommand{\xy}{(x, y)}                                                    % observation (x, y)
\newcommand{\xvec}{(x_1, \ldots, x_p)^T}                                    % (x1, ..., xp) 
\newcommand{\D}{\mathcal{D}}                                                % D, data 
\newcommand{\Dset}{\{ (x^{(1)}, y^{(1)}), \ldots, (x^{(n)},  y^{(n)})\}}    % {(x1,y1)), ..., (xn,yn)}, data
\newcommand{\xdat}{\{ x^{(1)}, \ldots, x^{(n)}\}}   						 % {x1, ..., xn}, input data
\newcommand{\ydat}{\mathbf{y}}                                              % y (bold), vector of outcomes
\newcommand{\yvec}{(y^{(1)}, \hdots, y^{(n)})^T}                            % (y1, ..., yn), vector of outcomes
\renewcommand{\xi}[1][i]{x^{(#1)}}                                          % x^i, i-th observed value of x
\newcommand{\yi}[1][i]{y^{(#1)}}                                            % y^i, i-th observed value of y 
\newcommand{\xyi}{(\xi, \yi)}                                               % (x^i, y^i), i-th observation
\newcommand{\xivec}{(x^{(i)}_1, \ldots, x^{(i)}_p)^T}                       % (x1^i, ..., xp^i), i-th observation vector
\newcommand{\xj}{x_j}                                                       % x_j, j-th feature
\newcommand{\xjb}{\mathbf{x}_j}                                             % x_j (bold), j-th feature vecor
\newcommand{\xjvec}{(x^{(1)}_j, \ldots, x^{(n)}_j)^T}                       % (x^1_j, ..., x^n_j), j-th feature vector
\newcommand{\Dtrain}{\mathcal{D}_{\text{train}}}                            % D_train, training set
\newcommand{\Dtest}{\mathcal{D}_{\text{test}}}                              % D_test, test set

%%%%%% ml - models general

% continuous prediction function f
\newcommand{\fx}{f(x)}                                                      % f(x), continuous prediction function
\newcommand{\Hspace}{H}														% hypothesis space where f is from
\newcommand{\fh}{\hat{f}}                                                   % f hat, estimated prediction function
\newcommand{\fxh}{\fh(x)}                                                   % fhat(x)
\newcommand{\fxt}{f(x | \theta)}                                            % f(x | theta)
\newcommand{\fxi}{f(\xi)}                                                   % f(x^(i))
\newcommand{\fxih}{\hat{f}(\xi)}                                            % f(x^(i))
\newcommand{\fxit}{f(x^{(i)} | \theta)}                                     % f(x^(i) | theta)
\newcommand{\fhD}{\fh_{\D}}                                                 % fhat_D, estimate of f based on D
\newcommand{\fhDtrain}{\fh_{\Dtrain}}                                       % fhat_Dtrain, estimate of f based on D

% discrete prediction function h
\newcommand{\hx}{h(x)}                                                      % h(x), discrete prediction function
\newcommand{\hh}{\hat{h}}                                                   % h hat
\newcommand{\hxh}{\hat{h}(x)}                                               % hhat(x)
\newcommand{\hxt}{h(x | \theta)}                                            % h(x | theta)
\newcommand{\hxi}{h(\xi)}                                                   % h(x^(i))
\newcommand{\hxit}{h(x^{(i)} | \theta)}                                     % h(x^(i) | theta)

% yhat
\newcommand{\yh}{\hat{y}}                                                   % y hat for prediction of target
\newcommand{\yih}{\hat{y}}                                                  % y hat for prediction of target

% theta
\newcommand{\thetah}{\hat{\theta}}                                          % theta hat

% densities + probabilities
% pdf of x 
\newcommand{\pdf}{p}                                                        % p
\newcommand{\pdfx}{p(x)}                                                    % p(x)
\newcommand{\pixt}{\pi(x | \theta)}                                         % pi(x|theta), pdf of x given theta

% pdf of (x, y)
\newcommand{\pdfxy}{p(x,y)}                                                 % p(x, y)
\newcommand{\pdfxyt}{p(x, y | \theta)}                                      % p(x, y | theta)
\newcommand{\pdfxyit}{p(\xi, \yi | \theta)}                                 % p(x^(i), y^(i) | theta)

% pdf of x given y
\newcommand{\pdfxyk}{p(x | y=k)}                                            % p(x | y = k)
\newcommand{\lpdfxyk}{\log \pdfxyk}                                         % log p(x | y = k)
\newcommand{\pdfxiyk}{p(\xi | y=k)}                                         % p(x^i | y = k)

% prior probabilities
\newcommand{\pik}{\pi_k}                                                    % pi_k, prior
\newcommand{\lpik}{\log \pik}                                               % log pi_k, log of the prior

% posterior probabilities
\newcommand{\post}{\P(y = 1 | x)}                                           % P(y = 1 | x), post. prob for y=1
\newcommand{\pix}{\pi(x)}                                                   % pi(x), P(y = 1 | x)
\newcommand{\postk}{\P(y = k | x)}                                          % P(y = k | y), post. prob for y=k
\newcommand{\pikx}{\pi_k(x)}                                                % pi_k(x), P(y = k | x)
\newcommand{\pikxt}{\pi_k(x | \theta)}                                      % pi_k(x | theta), P(y = k | x, theta)
\newcommand{\pijx}{\pi_j(x)}                                                % pi_j(x), P(y = j | x)
\newcommand{\pdfygxt}{p(y |x, \theta)}                                      % p(y | x, theta)
\newcommand{\pdfyigxit}{p(\yi |\xi, \theta)}                                % p(y^i |x^i, theta)
\newcommand{\lpdfygxt}{\log \pdfygxt }                                      % log p(y | x, theta)
\newcommand{\lpdfyigxit}{\log \pdfyigxit}                                   % log p(y^i |x^i, theta)
\newcommand{\pixh}{\hat \pi(x)}                                             % pi(x) hat, P(y = 1 | x) hat
\newcommand{\pikxh}{\hat \pi_k(x)}                                          % pi_k(x) hat, P(y = k | x) hat

% residual and margin
\newcommand{\eps}{\epsilon}                                                 % residual, stochastic
\newcommand{\epsi}{\epsilon^{(i)}}                                          % epsilon^i, residual, stochastic
\newcommand{\epsh}{\hat{\epsilon}}                                          % residual, estimated
\newcommand{\yf}{y \fx}                                                     % y f(x), margin
\newcommand{\yfi}{\yi \fxi}                                                 % y^i f(x^i), margin
\newcommand{\Sigmah}{\hat \Sigma}											% estimated covariance matrix
\newcommand{\Sigmahj}{\hat \Sigma_j}										% estimated covariance matrix for the j-th class

% ml - loss, risk, likelihood
\newcommand{\Lxy}{L(y, f(x))}                                               % L(y, f(x)), loss function
\newcommand{\Lxyi}{L(\yi, \fxi)}                                            % L(y^i, f(x^i))
\newcommand{\Lxyt}{L(y, \fxt)}                                              % L(y, f(x | theta))
\newcommand{\Lxyit}{L(\yi, \fxit)}                                          % L(y^i, f(x^i | theta)
\newcommand{\risk}{\mathcal{R}}                                             % R, risk
\newcommand{\riskf}{\risk(f)}                                               % R(f), risk
\newcommand{\riske}{\mathcal{R}_{\text{emp}}}                               % R_emp, empirical risk
\newcommand{\riskef}{\riske(f)}                                             % R_emp(f)
\newcommand{\risket}{\mathcal{R}_{\text{emp}}(\theta)}                      % R_emp(theta)
\newcommand{\riskr}{\mathcal{R}_{\text{reg}}}                               % R_reg, regularized risk
\newcommand{\riskrt}{\mathcal{R}_{\text{reg}}(\theta)}                      % R_reg(theta)
\newcommand{\riskrf}{\riskr(f)}                                             % R_reg(f)
\newcommand{\LL}{\mathcal{L}}                                               % L, likelihood
\newcommand{\LLt}{\mathcal{L}(\theta)}                                      % L(theta), likelihood
\renewcommand{\ll}{\ell}                                                    % l, log-likelihood
\newcommand{\llt}{\ell(\theta)}                                             % l(theta), log-likelihood
\newcommand{\LS}{\mathfrak{L}}                                              % ????????????
\newcommand{\TS}{\mathfrak{T}}                                              % ??????????????
\newcommand{\errtrain}{\text{err}_{\text{train}}}                           % training error
\newcommand{\errtest}{\text{err}_{\text{test}}}                             % training error
\newcommand{\errexp}{\overline{\text{err}_{\text{test}}}}                   % training error

% resampling
\newcommand{\GE}[1]{GE(\fh_{#1})}                                           % Generalization error GE
\newcommand{\GEh}[1]{\widehat{GE}_{#1}}                                     % Estimated train error
\newcommand{\GED}{\GE{\D}}                                                  % Generalization error GE
\newcommand{\EGEn}{EGE_n}                                                   % Generalization error GE
\newcommand{\EDn}{\E_{|D| = n}}                                             % Generalization error GE


% ml - irace
\newcommand{\costs}{\mathcal{C}} % costs
\newcommand{\Celite}{\theta^*} % elite configurations
\newcommand{\instances}{\mathcal{I}} % sequence of instances
\newcommand{\budget}{\mathcal{B}} % computational budget
% math spaces
\newcommand{\N}{\mathds{N}}                                                 % N, naturals
\newcommand{\Z}{\mathds{Z}}                                                 % Z, integers
\newcommand{\Q}{\mathds{Q}}                                                 % Q, rationals
\newcommand{\R}{\mathds{R}}                                                 % R, reals
\newcommand{\C}{\mathds{C}}                                                 % C, complex
\newcommand{\HS}{\mathcal{H}}                                               % H, hilbertspace
\newcommand{\continuous}{\mathcal{C}}                                       % C, space of continuous functions
\newcommand{\M}{\mathcal{M}} 												% machine numbers
\newcommand{\epsm}{\epsilon_m} 												% maximum error


% basic math stuff
\newcommand{\xt}{\tilde x}													% x tilde
\def\argmax{\mathop{\sf arg\,max}}                                          % argmax
\def\argmin{\mathop{\sf arg\,min}}                                          % argmin
\newcommand{\sign}{\operatorname{sign}}                                     % sign, signum
\newcommand{\I}{\mathbb{I}}                                                 % I, indicator
\newcommand{\order}{\mathcal{O}}                                            % O, order
\newcommand{\fp}[2]{\frac{\partial #1}{\partial #2}}                        % partial derivative
\newcommand{\pd}[2]{\frac{\partial{#1}}{\partial #2}}						% partial derivative

% sums and products
\newcommand{\sumin}{\sum_{i=1}^n}											% summation from i=1 to n
\newcommand{\sumkg}{\sum_{k=1}^g}											% summation from k=1 to g
\newcommand{\prodin}{\prod_{i=1}^n}											% product from i=1 to n
\newcommand{\prodkg}{\prod_{k=1}^g}											% product from k=1 to g

% linear algebra
\newcommand{\one}{\boldsymbol{1}}                                           % 1, unitvector
\newcommand{\id}{\mathrm{I}}                                                % I, identity
\newcommand{\diag}{\operatorname{diag}}                                     % diag, diagonal
\newcommand{\trace}{\operatorname{tr}}                                      % tr, trace
\newcommand{\spn}{\operatorname{span}}                                      % span
\newcommand{\scp}[2]{\left\langle #1, #2 \right\rangle}                     % <.,.>, scalarproduct
\newcommand{\mat}[1]{ 														% short pmatrix command
	\begin{pmatrix}
		#1
	\end{pmatrix}
}
\newcommand{\Amat}{\bm{A}}													% matrix A
\newcommand{\xv}{\bm{x}}													% vector x (bold)
\newcommand{\yv}{\bm{y}}														% vector y (bold)
\newcommand{\Deltab}{\bm{\Delta}}											% error term for vectors
															

% basic probability + stats
\renewcommand{\P}{\mathds{P}}                                               % P, probability
\newcommand{\E}{\mathds{E}}                                                 % E, expectation
\newcommand{\var}{\mathsf{Var}}                                             % Var, variance
\newcommand{\cov}{\mathsf{Cov}}                                             % Cov, covariance
\newcommand{\corr}{\mathsf{Corr}}                                           % Corr, correlation
\newcommand{\normal}{\mathcal{N}}                                           % N of the normal distribution
\newcommand{\iid}{\overset{i.i.d}{\sim}}                                    % dist with i.i.d superscript
\newcommand{\distas}[1]{\overset{#1}{\sim}}                                 % ... is distributed as ... 
\input{../../latex-math/ml-hpo.tex}
\input{../../latex-math/ml-eval.tex}

\title{Introduction to Machine Learning}
% \author{Bernd Bischl, Christoph Molnar, Daniel Schalk, Fabian Scheipl}
\institute{\href{https://compstat-lmu.github.io/lecture_i2ml/}{compstat-lmu.github.io/lecture\_i2ml}}
\date{}


\begin{document}

\lecturechapter{Hyperparameter Tuning - Advanced Tuning Techniques: MBO \& Hyperband}
\lecture{Introduction to Machine Learning}
\sloppy


\begin{vbframe}{HPO -- Many approaches}
\begin{itemize}
    \item Evolutionary algorithms
    \item Bayesian / model-based optimization
    \item Multi-fidelity optimization, e.g.\ Hyperband
\end{itemize}

\includegraphics[width = 0.37\textwidth]{figure/ea.eps}
\includegraphics[width = 0.37\textwidth]{figure/mbo.png}
\includegraphics[trim=0 0 285 0, clip, width = 0.235\textwidth]{figure/hyperband.png}

HPO methods can be characterized by: 
\begin{itemize}
    \item how the exploration vs. exploitation trade-off is handled
    \item how the inference vs. search trade-off is handled% (how much time is spent is spent to learn a surrogate model on the currently observed data)
\end{itemize}

Further aspects: Parallelizability, local vs. global behavior, handling of noisy observations, multifidelity and search space complexity. 

\end{vbframe}

\begin{vbframe}{Evolutionary strategies}

\begin{figure}[htb]
    \centering
    \includegraphics[width=0.65\textwidth]{figure/ea.eps}
%    \caption{%
%        Schematic representation of a single iteration of an ES as a four dimensional discrete problem.
%        Parameter values are symbolized by geometric shapes.
%    }
    \label{fig:ea}
\end{figure}
\begin{footnotesize}
\begin{itemize}
\item Are  a  class  of  stochastic  population-based  optimization  methods  inspired  by  the  concepts  of  biological  evolution
\item Are applicable to HPO since they  do  not  require  gradients
\item Mutation is  the  (randomized)  change  of  one  or  a  few  HP  values  in  a  configuration.
\item Crossover creates a new HPC by (randomly) mixing the values of two other configurations. 
\end{itemize}
\end{footnotesize}

\end{vbframe}

\begin{vbframe}{Bayesian Optimization}
BO sequentially iterates:
\begin{columns}
\column{0.45\textwidth}
\begin{enumerate}
\item \textbf{Approximate} $\bm{\lambda} \mapsto c(\bm{\lambda})$ by  (nonlin) regression model $\hat c(\bm{\lambda})$, from evaluated configurations (archive)
\item \textbf{Propose candidates} via optimizing an acquisition function that is based on the surrogate $\hat c(\bm{\lambda})$
\item \textbf{Evaluate} candidate(s) proposed in 2, then go to 1 
\end{enumerate}
\column{0.5\textwidth}
\vspace*{0.5cm}
\includegraphics[width = 1\textwidth]{figure/mbo.png}
\end{columns}
Important trade-off: \textbf{Exploration} (evaluate candidates in under-explored areas) vs. \textbf{exploitation} (search near promising areas)

\end{vbframe}

\begin{frame}{Bayesian Optimization}

\begin{columns}
\column{0.45\textwidth}
\textbf{Surrogate Model}: 
\begin{itemize}
    \item Probabilistic modeling of $C(\lamv) \sim (\hat c(\bm{\lambda}), \hat \sigma(\bm{\lambda}))$ with posterior mean $\hat c(\bm{\lambda})$ and uncertainty  $\hat \sigma(\bm{\lambda})$.  
    \item Typical choices for numeric spaces are Gaussian Processes; random forests for mixed spaces
\end{itemize}
\column{0.5\textwidth}
\vspace*{0.5cm}
\includegraphics[width = 1\textwidth]{figure/mbo.png}
\end{columns}

\textbf{Acquisition Function}: 
\begin{itemize}
    \item Balance exploration (high $\hat \sigma$) vs. exploitation (low $\hat c$). 
    \item Lower confidence bound (LCB): $\quad a(\bm{\lamv}) = \hat c(\bm{\lambda}) - \kappa \cdot \hat \sigma(\lamv)$ 
    \item Expected improvement (EI): $\quad a(\lamv) = \E\left[\max\left\{c_{\min} - C(\lamv), 0\right\}\right]$ where ($c_{\min}$ is best cost value from archive)
    %($c_{\min}$ is best observed value)
    \item Optimizing $a(\lamv)$ is still difficult, but cheap(er)
\end{itemize}

\end{frame}

\begin{vbframe}{Bayesian Optimization}

% Model-based optimization (MBO) is a sequential optimization procedure. We start with an initial design, i.e., a set of configurations $\lambda_i$ where we have evaluated the corresponding (resampling) performance. 
% 
% Repeat:
%   \begin{enumerate}
% \item From the available performance measurements, we build a \textbf{surrogate model} that models the relationship between model hyperparameters and estimated generalization error. It serves as a cheap approximation of the expensive objective. 
% \item Based on information provided by the surrogate model, a new configuration $\lambda^{(\text{new})}$ is proposed: we pick a value for which the surrogate model predicts a large potential improvement over the already evaluated configurations.
% \item The resampling performance of the learner with hyperparameter setting $\lambda^{(\text{new})}$ is evaluated and added to the set of design points.  
% \end{enumerate}

%\framebreak 

\begin{knitrout}\scriptsize
\definecolor{shadecolor}{rgb}{0.969, 0.969, 0.969}\color{fgcolor}

{\centering \includegraphics[width=0.95\textwidth]{figure/cart_tuning_ad_1.pdf} 

}
%FIGURE SOURCE: fig-cart_tuning_ad_1.R


\end{knitrout}
  
  \begin{footnotesize}
Upper plot: The surrogate model (black, dashed) models the \emph{unknown} relationship between input and output (black, solid) based on the initial design (red points).\\
Lower plot: Mean and variance of the surrogate model are used to derive the expected improvement (EI) criterion. The point that maximizes the EI is proposed (blue point). 
\end{footnotesize}


\end{vbframe}
\begin{vbframe}{Model-based Optimization}
\addtocounter{framenumber}{-1}

\begin{knitrout}\scriptsize
\definecolor{shadecolor}{rgb}{0.969, 0.969, 0.969}\color{fgcolor}

{\centering \includegraphics[width=0.95\textwidth]{figure/cart_tuning_ad_2.pdf} 
}
%FIGURE SOURCE: fig-cart_tuning_ad_1.R


\end{knitrout}
  \begin{footnotesize}
Upper plot: The surrogate model (black, dashed) models the \emph{unknown} relationship between input and output (black, solid) based on the initial design (red points).\\
Lower plot: Mean and variance of the surrogate model are used to derive the expected improvement (EI) criterion. The point that maximizes the EI is proposed (blue point). 
\end{footnotesize}

\end{vbframe}



\begin{vbframe}{Model-based Optimization}
\addtocounter{framenumber}{-1}

\begin{knitrout}\scriptsize
\definecolor{shadecolor}{rgb}{0.969, 0.969, 0.969}\color{fgcolor}

{\centering \includegraphics[width=0.95\textwidth]{figure/cart_tuning_ad_3.pdf} 
}
%FIGURE SOURCE: fig-cart_tuning_ad_1.R


\end{knitrout}
  \begin{footnotesize}
Upper plot: The surrogate model (black, dashed) models the \emph{unknown} relationship between input and output (black, solid) based on the initial design (red points).\\
Lower plot: Mean and variance of the surrogate model are used to derive the expected improvement (EI) criterion. The point that maximizes the EI is proposed (blue point). 
\end{footnotesize}

\end{vbframe}
\begin{vbframe}{Model-based Optimization}
\addtocounter{framenumber}{-1}

\begin{knitrout}\scriptsize
\definecolor{shadecolor}{rgb}{0.969, 0.969, 0.969}\color{fgcolor}

{\centering \includegraphics[width=0.95\textwidth]{figure/cart_tuning_ad_4.pdf} 
}
%FIGURE SOURCE: fig-cart_tuning_ad_1.R


\end{knitrout}
  \begin{footnotesize}
Upper plot: The surrogate model (black, dashed) models the \emph{unknown} relationship between input and output (black, solid) based on the initial design (red points).\\
Lower plot: Mean and variance of the surrogate model are used to derive the expected improvement (EI) criterion. The point that maximizes the EI is proposed (blue point). 
\end{footnotesize}
\framebreak

Since we use the sequentially updated surrogate model predictions of performance to propose new configurations,
we are guided to \enquote{interesting} regions of $\bm{\Lambda}$ and avoid irrelevant evaluations:
  
  \begin{center}
\begin{figure}
% see rsrc/mbo-example.R
\includegraphics[width=0.65\textwidth]{figure/mbo_random_tune.pdf}
\caption{\footnotesize{Tuning complexity and minimal node size for splits for CART on the \texttt{titanic} data (10-fold CV maximizing accuracy). \\
  Left panel: MBO, 50 configurations; right panel: random search, 50 iterations.}}
\end{figure}
\end{center}


\end{vbframe}



\begin{vbframe}{Multifidelity optimization}
\begin{itemize}
\item Prerequiste: Fidelity  HP $\lamfid$, i.e.,  a  component  of $\lamv$,  which  influences  the computational  cost  of  the  fitting  procedure  in  a  monotonically  increasing  manner
\item Methods of multifidelity optimization in HPO are all tuning  approaches  that  can  efficiently handle  a $\inducer$ with  a  HP $\lambda_{\text{fid}}$
\item The  lower  we  set $\lamfid$,  the more points we can explore in our search space, albeit with much less reliable information w.r.t. their true performance.
\item We  assume  to  know  box-constraints  of $\lamfid$,  so $\lamfid \in [\lamfidl, \lamfidu]$,  where  the  upper  limit  implies  the  highest  fidelity  returning  values closest to the true objective value at the highest computational cost. 
\end{itemize}
\end{vbframe}

\begin{vbframe}{Successive Halving}

\begin{columns}
\column{0.59\textwidth}
\begin{itemize}
\item Races down set of HPCs to the best
\item Idea: Discard bad configurations early
\item Train HPCs with fraction of full budget (SGD epochs, training set size); the control param for this is called \textbf{multi-fidelity HP}
\item Continue with better half of HPCs (w.r.t $\GEh{}$); 
with doubled budget
\item Repeat until budget depleted or single HPC remains
\end{itemize}
\column{0.41\textwidth}
\begin{center}
\includegraphics[trim=0 0 285 0, clip, width = 1.0\textwidth]{figure/hyperband.eps}
\end{center}
\end{columns}
\end{vbframe}

\begin{vbframe}{Multifidelity optimization -- Hyperband}

\begin{columns}
\column{0.70\textwidth}
\textbf{Problem with SH}
\begin{itemize}
    \item Good HPCs could be killed off too early, 
    depends on evaluation schedule
\end{itemize}
\textbf{Solution: Hyperband}
\begin{itemize}
\item Repeat SH with different start budgets $\lambda_{\text{budget}}^{[0]}$ and initial number of HPCs $p^{[0]}$ 
\item Each SH run is called bracket
\item Each bracket consumes ca. the same budget
%\item It is a multi-fidelity optimization method since it uses information of multiple fidelity levels to identify promising candidates
\end{itemize}
\column{0.30\textwidth}
\begin{center}
\includegraphics[trim=620 0 0 0, clip, width = 0.70\textwidth]{figure/hyperband.eps}
\end{center}
\end{columns}
\end{vbframe}

% \begin{vbframe}{Hyperband}
% 
% \begin{itemize}
% \item It is extremely expensive to train complex models on large data sets
% \item For many configurations, it might be clear early on that further training is not likely to significantly improve the performance
% \item More importantly, the relative ordering of configurations (for a given data set) can also become evident early on. 
% \item \textbf{Idea:} \enquote{weed out} poor configurations early during training
% \item One approach is \textbf{successive halving}: Given an initial set of configurations, all trained for a small initial budget, repeat:
%   \begin{itemize}
% \item Remove the half that performed worst, double the budget
% \item Continue until the new budget is exhausted
% \end{itemize}  
% \item Successful halving is performed several times with different trade-offs between the number of configurations considered and the budget that is spent on them. 
% \end{itemize}
% 
% \framebreak 
% 
% Only the most promising configuration(s) are trained to completion: 
%   
%   \begin{center}
% \begin{figure}
% \includegraphics[width=0.8\textwidth]{figure_man/hyperband3.png}
% %src: https://www.automl.org/wp-content/uploads/2019/05/AutoML_Book_Chapter1.pdf
% \end{figure}
% \end{center}
% \tiny taken from Hutter, Kotthoff, Vanschoren. \textit{Automated Machine Learning -- Methods, Systems, Challenges}. Springer, 2019. (Fig. 1.3)
% \end{vbframe}

\begin{frame}{More tuning algorithms:}

Other advanced techniques besides model-based optimization and the hyperband algorithm are: 
  
  \begin{itemize}
\item Stochastic local search, e.g., simulated annealing
\item Genetic algorithms / CMAES
\item Iterated F-Racing
\item Many more $\ldots$
  \end{itemize}

\medskip
For more information see \textit{Hyperparameter Optimization: Foundations, Algorithms,
Best Practices and Open Challenges}, Bischl (2021)

\end{frame}
\endlecture
\end{document}
