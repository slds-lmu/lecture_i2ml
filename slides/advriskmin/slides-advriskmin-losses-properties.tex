%<<setup-child, include = FALSE>>=
%library(knitr)
%library(qrmix)
%library(mlr)
%library(quantreg)
%library(reshape2)
%set_parent("../style/preamble.Rnw")
%@


\usepackage[]{graphicx}
\usepackage[]{color}
% maxwidth is the original width if it is less than linewidth
% otherwise use linewidth (to make sure the graphics do not exceed the margin)
\makeatletter
\def\maxwidth{ %
  \ifdim\Gin@nat@width>\linewidth
    \linewidth
  \else
    \Gin@nat@width
  \fi
}
\makeatother

% ---------------------------------%
% latex-math dependencies, do not remove:
% - \usepackage{mathtools}
% - \usepackage{bm}
% - \usepackage{siunitx}
% - \usepackage{dsfont}
% - \usepackage{xspace}
% ---------------------------------%

%--------------------------------------------------------%
%       Language, encoding, typography
%--------------------------------------------------------%

\usepackage[english]{babel}
\usepackage[utf8]{inputenc} % Enables inputting UTF-8 symbols
% Standard AMS suite
\usepackage{amsmath,amsfonts,amssymb}

% Font four double-stroke / blackboard letters for sets of numbers (N, R, ...)
% Distribution name is "doublestroke"
% According to https://mirror.physik.tu-berlin.de/pub/CTAN/fonts/doublestroke/dsdoc.pdf
% the "bbm" package does a similar thing and may be superfluous.
% Required for latex-math
\usepackage{dsfont}

% bbm – "Blackboard-style" cm fonts (https://www.ctan.org/pkg/bbm)
% Used to be in common.tex, loaded directly after this file
% Maybe superfluous given dsfont is loaded
% TODO: Check if really unused?
% \usepackage{bbm}

% bm – Access bold symbols in maths mode - https://ctan.org/pkg/bm
% Required for latex-math
% https://tex.stackexchange.com/questions/3238/bm-package-versus-boldsymbol
\usepackage{bm}

% pifont – Access to PostScript standard Symbol and Dingbats fonts
% Used for \newcommand{\xmark}{\ding{55}, which is never used
% aside from lecture_advml/attic/xx-automl/slides.Rnw
% \usepackage{pifont}

% Quotes (inline and display), provdes \enquote
% https://ctan.org/pkg/csquotes
\usepackage{csquotes}

% Adds arg to enumerate env, technically superseded by enumitem according
% to https://ctan.org/pkg/enumerate
% Replace with https://ctan.org/pkg/enumitem ?
\usepackage{enumerate}

% Line spacing - provides \singlespacing \doublespacing \onehalfspacing
% https://ctan.org/pkg/setspace
% TODO: Check if really unused?
%\usepackage{setspace}

% mathtools – Mathematical tools to use with amsmath
% https://ctan.org/pkg/mathtools?lang=en
% latex-math dependency according to latex-math repo
\usepackage{mathtools}

%--------------------------------------------------------%
%       Displaying code and algorithms
%--------------------------------------------------------%
\usepackage{verbatim}
\usepackage{algorithm}
\usepackage{algpseudocode}

%--------------------------------------------------------%
%       Tables
%--------------------------------------------------------%

% multi-row table cells: https://www.namsu.de/Extra/pakete/Multirow.html
\usepackage{multirow}

% long/multi-page tables: https://texdoc.org/serve/longtable.pdf/0
% TODO: Check if really unused?

\usepackage{longtable}

% pretty table env: https://ctan.org/pkg/booktabs?lang=en
% TODO: Check if really unused?
\usepackage{booktabs}

%--------------------------------------------------------%
%       Figures: Creating, placing, verbing
%--------------------------------------------------------%

% wrapfig - Wrapping text around figures https://de.overleaf.com/learn/latex/Wrapping_text_around_figures
\usepackage{wrapfig}

% Sub figures in figures and tables
% https://ctan.org/pkg/subfig -- supersedes subfigure package
% TODO: Check if really unused?
\usepackage{subfig}

% Actually it's pronounced PGF https://en.wikibooks.org/wiki/LaTeX/PGF/TikZ
\usepackage{tikz}

\usetikzlibrary{shapes,arrows,automata,positioning,calc,chains,trees, shadows}
\tikzset{
  %Define standard arrow tip
  >=stealth',
  %Define style for boxes
  punkt/.style={
    rectangle,
    rounded corners,
    draw=black, very thick,
    text width=6.5em,
    minimum height=2em,
    text centered},
  % Define arrow style
  pil/.style={
    ->,
    thick,
    shorten <=2pt,
    shorten >=2pt,}
}


% Unsorted
% textpos – Place boxes at arbitrary positions on the LATEX page
% https://ctan.org/pkg/textpos?lang=en
% Provides \begin{textblock}
 % TODO: Check if really unused?
\usepackage[absolute,overlay]{textpos}

% psfrag – Replace strings in encapsulated PostScript figures
% https://www.overleaf.com/latex/examples/psfrag-example/tggxhgzwrzhn
% https://ftp.mpi-inf.mpg.de/pub/tex/mirror/ftp.dante.de/pub/tex/macros/latex/contrib/psfrag/pfgguide.pdf
% Can't tell if this is needed
% TODO: Check if really unused?
\usepackage{psfrag}

% Maybe not great to use this https://tex.stackexchange.com/a/197/19093
% Use align instead -- TODO: Global search & replace to check
\usepackage{eqnarray}

\usepackage{colortbl}

% arydshln – Draw dash-lines in array/tabular
% https://www.ctan.org/pkg/arydshln
% !! "arydshln has to be loaded after array, longtable, colortab and/or colortbl"
% Provides \hdashline and \cdashline
% TODO: Check if really unused?
% \usepackage{arydshln}

% tabularx – Tabulars with adjustable-width columns
% https://ctan.org/pkg/tabularx
% Provides \begin{tabularx}
% TODO: Check if really unused?
% \usepackage{tabularx}

% placeins – Control float placement
% https://ctan.org/pkg/placeins
% Defines a \FloatBarrier command
% TODO: Check if really unused?
% \usepackage{placeins}


% framed – Framed or shaded regions that can break across pages
% https://ctan.org/pkg/framed
% Provides \begin{framed} which uses \colorbox{shadecolor} relying on \definecolor{shadecolor}.
% TODO: Check if really unused?
% \usepackage{framed}

% Used often in conjunction with \definecolor{shadecolor}{rgb}{0.969, 0.969, 0.969}
% Might be able to be removed or at least redefined to only have shadecolor (if needed)
\definecolor{fgcolor}{rgb}{0.345, 0.345, 0.345}
\definecolor{shadecolor}{rgb}{0.969, 0.969, 0.969}
\newenvironment{knitrout}{}{} % an empty environment to be redefined in TeX


% Defines macros and environments
\usepackage{../../style/lmu-lecture}

\let\code=\texttt % Used regularly
\let\proglang=\textsf % Unused?

% Not sure what/why this does
\setkeys{Gin}{width=0.9\textwidth}

\setbeamertemplate{frametitle}{\expandafter\uppercase\expandafter\insertframetitle}

% Can't find a reason why common.tex is not just part of this file?

% basic latex stuff
\newcommand{\pkg}[1]{{\fontseries{b}\selectfont #1}} %fontstyle for R packages
\newcommand{\lz}{\vspace{0.5cm}} %vertical space
\newcommand{\dlz}{\vspace{1cm}} %double vertical space
\newcommand{\oneliner}[1] % Oneliner for important statements
{\begin{block}{}\begin{center}\begin{Large}#1\end{Large}\end{center}\end{block}}


%new environments
\newenvironment{vbframe}  %frame with breaks and verbatim
{
 \begin{frame}[containsverbatim,allowframebreaks]
}
{
\end{frame}
}

\newenvironment{vframe}  %frame with verbatim without breaks (to avoid numbering one slided frames)
{
 \begin{frame}[containsverbatim]
}
{
\end{frame}
}

\newenvironment{blocki}[1]   % itemize block
{
 \begin{block}{#1}\begin{itemize}
}
{
\end{itemize}\end{block}
}

\newenvironment{fragileframe}[2]{  %fragile frame with framebreaks
\begin{frame}[allowframebreaks, fragile, environment = fragileframe]
\frametitle{#1}
#2}
{\end{frame}}


\newcommand{\myframe}[2]{  %short for frame with framebreaks
\begin{frame}[allowframebreaks]
\frametitle{#1}
#2
\end{frame}}

\newcommand{\remark}[1]{
  \textbf{Remark:} #1
}


\newenvironment{deleteframe}
{
\begingroup
\usebackgroundtemplate{\includegraphics[width=\paperwidth,height=\paperheight]{../style/color/red.png}}
 \begin{frame}
}
{
\end{frame}
\endgroup
}
\newenvironment{simplifyframe}
{
\begingroup
\usebackgroundtemplate{\includegraphics[width=\paperwidth,height=\paperheight]{../style/color/yellow.png}}
 \begin{frame}
}
{
\end{frame}
\endgroup
}\newenvironment{draftframe}
{
\begingroup
\usebackgroundtemplate{\includegraphics[width=\paperwidth,height=\paperheight]{../style/color/green.jpg}}
 \begin{frame}
}
{
\end{frame}
\endgroup
}
% https://tex.stackexchange.com/a/261480: textcolor that works in mathmode
\makeatletter
\renewcommand*{\@textcolor}[3]{%
  \protect\leavevmode
  \begingroup
    \color#1{#2}#3%
  \endgroup
}
\makeatother


%-------------------------------------------------------------------------------------------------------%
%  Unused stuff that needs to go but is kept here currently juuuust in case it was important after all  %
%-------------------------------------------------------------------------------------------------------%

% \newcommand{\hlnum}[1]{\textcolor[rgb]{0.686,0.059,0.569}{#1}}%
% \newcommand{\hlstr}[1]{\textcolor[rgb]{0.192,0.494,0.8}{#1}}%
% \newcommand{\hlcom}[1]{\textcolor[rgb]{0.678,0.584,0.686}{\textit{#1}}}%
% \newcommand{\hlopt}[1]{\textcolor[rgb]{0,0,0}{#1}}%
% \newcommand{\hlstd}[1]{\textcolor[rgb]{0.345,0.345,0.345}{#1}}%
% \newcommand{\hlkwa}[1]{\textcolor[rgb]{0.161,0.373,0.58}{\textbf{#1}}}%
% \newcommand{\hlkwb}[1]{\textcolor[rgb]{0.69,0.353,0.396}{#1}}%
% \newcommand{\hlkwc}[1]{\textcolor[rgb]{0.333,0.667,0.333}{#1}}%
% \newcommand{\hlkwd}[1]{\textcolor[rgb]{0.737,0.353,0.396}{\textbf{#1}}}%
% \let\hlipl\hlkwb

% \makeatletter
% \newenvironment{kframe}{%
%  \def\at@end@of@kframe{}%
%  \ifinner\ifhmode%
%   \def\at@end@of@kframe{\end{minipage}}%
%   \begin{minipage}{\columnwidth}%
%  \fi\fi%
%  \def\FrameCommand##1{\hskip\@totalleftmargin \hskip-\fboxsep
%  \colorbox{shadecolor}{##1}\hskip-\fboxsep
%      % There is no \\@totalrightmargin, so:
%      \hskip-\linewidth \hskip-\@totalleftmargin \hskip\columnwidth}%
%  \MakeFramed {\advance\hsize-\width
%    \@totalleftmargin\z@ \linewidth\hsize
%    \@setminipage}}%
%  {\par\unskip\endMakeFramed%
%  \at@end@of@kframe}
% \makeatother

% \definecolor{shadecolor}{rgb}{.97, .97, .97}
% \definecolor{messagecolor}{rgb}{0, 0, 0}
% \definecolor{warningcolor}{rgb}{1, 0, 1}
% \definecolor{errorcolor}{rgb}{1, 0, 0}
% \newenvironment{knitrout}{}{} % an empty environment to be redefined in TeX

% \usepackage{alltt}
% \newcommand{\SweaveOpts}[1]{}  % do not interfere with LaTeX
% \newcommand{\SweaveInput}[1]{} % because they are not real TeX commands
% \newcommand{\Sexpr}[1]{}       % will only be parsed by R
% \newcommand{\xmark}{\ding{55}}%

% math spaces
\newcommand{\N}{\mathds{N}}                                                 % N, naturals
\newcommand{\Z}{\mathds{Z}}                                                 % Z, integers
\newcommand{\Q}{\mathds{Q}}                                                 % Q, rationals
\newcommand{\R}{\mathds{R}}                                                 % R, reals
\newcommand{\C}{\mathds{C}}                                                 % C, complex
\newcommand{\HS}{\mathcal{H}}                                               % H, hilbertspace
\newcommand{\continuous}{\mathcal{C}}                                       % C, space of continuous functions
\newcommand{\M}{\mathcal{M}} 												% machine numbers
\newcommand{\epsm}{\epsilon_m} 												% maximum error


% basic math stuff
\newcommand{\xt}{\tilde x}													% x tilde
\def\argmax{\mathop{\sf arg\,max}}                                          % argmax
\def\argmin{\mathop{\sf arg\,min}}                                          % argmin
\newcommand{\sign}{\operatorname{sign}}                                     % sign, signum
\newcommand{\I}{\mathbb{I}}                                                 % I, indicator
\newcommand{\order}{\mathcal{O}}                                            % O, order
\newcommand{\fp}[2]{\frac{\partial #1}{\partial #2}}                        % partial derivative
\newcommand{\pd}[2]{\frac{\partial{#1}}{\partial #2}}						% partial derivative

% sums and products
\newcommand{\sumin}{\sum_{i=1}^n}											% summation from i=1 to n
\newcommand{\sumkg}{\sum_{k=1}^g}											% summation from k=1 to g
\newcommand{\prodin}{\prod_{i=1}^n}											% product from i=1 to n
\newcommand{\prodkg}{\prod_{k=1}^g}											% product from k=1 to g

% linear algebra
\newcommand{\one}{\boldsymbol{1}}                                           % 1, unitvector
\newcommand{\id}{\mathrm{I}}                                                % I, identity
\newcommand{\diag}{\operatorname{diag}}                                     % diag, diagonal
\newcommand{\trace}{\operatorname{tr}}                                      % tr, trace
\newcommand{\spn}{\operatorname{span}}                                      % span
\newcommand{\scp}[2]{\left\langle #1, #2 \right\rangle}                     % <.,.>, scalarproduct
\newcommand{\mat}[1]{ 														% short pmatrix command
	\begin{pmatrix}
		#1
	\end{pmatrix}
}
\newcommand{\Amat}{\bm{A}}													% matrix A
\newcommand{\xv}{\bm{x}}													% vector x (bold)
\newcommand{\yv}{\bm{y}}														% vector y (bold)
\newcommand{\Deltab}{\bm{\Delta}}											% error term for vectors
															

% basic probability + stats
\renewcommand{\P}{\mathds{P}}                                               % P, probability
\newcommand{\E}{\mathds{E}}                                                 % E, expectation
\newcommand{\var}{\mathsf{Var}}                                             % Var, variance
\newcommand{\cov}{\mathsf{Cov}}                                             % Cov, covariance
\newcommand{\corr}{\mathsf{Corr}}                                           % Corr, correlation
\newcommand{\normal}{\mathcal{N}}                                           % N of the normal distribution
\newcommand{\iid}{\overset{i.i.d}{\sim}}                                    % dist with i.i.d superscript
\newcommand{\distas}[1]{\overset{#1}{\sim}}                                 % ... is distributed as ... 
% machine learning

%%%%%% ml - data
\newcommand{\Xspace}{\mathcal{X}}                                           % X, input space
\newcommand{\Yspace}{\mathcal{Y}}                                           % Y, output space
\newcommand{\nset}{\{1, \ldots, n\}}                                        % set from 1 to n
\newcommand{\pset}{\{1, \ldots, p\}}                                        % set from 1 to p
\newcommand{\gset}{\{1, \ldots, g\}}                                        % set from 1 to g
\newcommand{\Pxy}{\P_{xy}}                                                  % P_xy
\newcommand{\xy}{(x, y)}                                                    % observation (x, y)
\newcommand{\xvec}{(x_1, \ldots, x_p)^T}                                    % (x1, ..., xp) 
\newcommand{\D}{\mathcal{D}}                                                % D, data 
\newcommand{\Dset}{\{ (x^{(1)}, y^{(1)}), \ldots, (x^{(n)},  y^{(n)})\}}    % {(x1,y1)), ..., (xn,yn)}, data
\newcommand{\xdat}{\{ x^{(1)}, \ldots, x^{(n)}\}}   						 % {x1, ..., xn}, input data
\newcommand{\ydat}{\mathbf{y}}                                              % y (bold), vector of outcomes
\newcommand{\yvec}{(y^{(1)}, \hdots, y^{(n)})^T}                            % (y1, ..., yn), vector of outcomes
\renewcommand{\xi}[1][i]{x^{(#1)}}                                          % x^i, i-th observed value of x
\newcommand{\yi}[1][i]{y^{(#1)}}                                            % y^i, i-th observed value of y 
\newcommand{\xyi}{(\xi, \yi)}                                               % (x^i, y^i), i-th observation
\newcommand{\xivec}{(x^{(i)}_1, \ldots, x^{(i)}_p)^T}                       % (x1^i, ..., xp^i), i-th observation vector
\newcommand{\xj}{x_j}                                                       % x_j, j-th feature
\newcommand{\xjb}{\mathbf{x}_j}                                             % x_j (bold), j-th feature vecor
\newcommand{\xjvec}{(x^{(1)}_j, \ldots, x^{(n)}_j)^T}                       % (x^1_j, ..., x^n_j), j-th feature vector
\newcommand{\Dtrain}{\mathcal{D}_{\text{train}}}                            % D_train, training set
\newcommand{\Dtest}{\mathcal{D}_{\text{test}}}                              % D_test, test set

%%%%%% ml - models general

% continuous prediction function f
\newcommand{\fx}{f(x)}                                                      % f(x), continuous prediction function
\newcommand{\Hspace}{H}														% hypothesis space where f is from
\newcommand{\fh}{\hat{f}}                                                   % f hat, estimated prediction function
\newcommand{\fxh}{\fh(x)}                                                   % fhat(x)
\newcommand{\fxt}{f(x | \theta)}                                            % f(x | theta)
\newcommand{\fxi}{f(\xi)}                                                   % f(x^(i))
\newcommand{\fxih}{\hat{f}(\xi)}                                            % f(x^(i))
\newcommand{\fxit}{f(x^{(i)} | \theta)}                                     % f(x^(i) | theta)
\newcommand{\fhD}{\fh_{\D}}                                                 % fhat_D, estimate of f based on D
\newcommand{\fhDtrain}{\fh_{\Dtrain}}                                       % fhat_Dtrain, estimate of f based on D

% discrete prediction function h
\newcommand{\hx}{h(x)}                                                      % h(x), discrete prediction function
\newcommand{\hh}{\hat{h}}                                                   % h hat
\newcommand{\hxh}{\hat{h}(x)}                                               % hhat(x)
\newcommand{\hxt}{h(x | \theta)}                                            % h(x | theta)
\newcommand{\hxi}{h(\xi)}                                                   % h(x^(i))
\newcommand{\hxit}{h(x^{(i)} | \theta)}                                     % h(x^(i) | theta)

% yhat
\newcommand{\yh}{\hat{y}}                                                   % y hat for prediction of target
\newcommand{\yih}{\hat{y}}                                                  % y hat for prediction of target

% theta
\newcommand{\thetah}{\hat{\theta}}                                          % theta hat

% densities + probabilities
% pdf of x 
\newcommand{\pdf}{p}                                                        % p
\newcommand{\pdfx}{p(x)}                                                    % p(x)
\newcommand{\pixt}{\pi(x | \theta)}                                         % pi(x|theta), pdf of x given theta

% pdf of (x, y)
\newcommand{\pdfxy}{p(x,y)}                                                 % p(x, y)
\newcommand{\pdfxyt}{p(x, y | \theta)}                                      % p(x, y | theta)
\newcommand{\pdfxyit}{p(\xi, \yi | \theta)}                                 % p(x^(i), y^(i) | theta)

% pdf of x given y
\newcommand{\pdfxyk}{p(x | y=k)}                                            % p(x | y = k)
\newcommand{\lpdfxyk}{\log \pdfxyk}                                         % log p(x | y = k)
\newcommand{\pdfxiyk}{p(\xi | y=k)}                                         % p(x^i | y = k)

% prior probabilities
\newcommand{\pik}{\pi_k}                                                    % pi_k, prior
\newcommand{\lpik}{\log \pik}                                               % log pi_k, log of the prior

% posterior probabilities
\newcommand{\post}{\P(y = 1 | x)}                                           % P(y = 1 | x), post. prob for y=1
\newcommand{\pix}{\pi(x)}                                                   % pi(x), P(y = 1 | x)
\newcommand{\postk}{\P(y = k | x)}                                          % P(y = k | y), post. prob for y=k
\newcommand{\pikx}{\pi_k(x)}                                                % pi_k(x), P(y = k | x)
\newcommand{\pikxt}{\pi_k(x | \theta)}                                      % pi_k(x | theta), P(y = k | x, theta)
\newcommand{\pijx}{\pi_j(x)}                                                % pi_j(x), P(y = j | x)
\newcommand{\pdfygxt}{p(y |x, \theta)}                                      % p(y | x, theta)
\newcommand{\pdfyigxit}{p(\yi |\xi, \theta)}                                % p(y^i |x^i, theta)
\newcommand{\lpdfygxt}{\log \pdfygxt }                                      % log p(y | x, theta)
\newcommand{\lpdfyigxit}{\log \pdfyigxit}                                   % log p(y^i |x^i, theta)
\newcommand{\pixh}{\hat \pi(x)}                                             % pi(x) hat, P(y = 1 | x) hat
\newcommand{\pikxh}{\hat \pi_k(x)}                                          % pi_k(x) hat, P(y = k | x) hat

% residual and margin
\newcommand{\eps}{\epsilon}                                                 % residual, stochastic
\newcommand{\epsi}{\epsilon^{(i)}}                                          % epsilon^i, residual, stochastic
\newcommand{\epsh}{\hat{\epsilon}}                                          % residual, estimated
\newcommand{\yf}{y \fx}                                                     % y f(x), margin
\newcommand{\yfi}{\yi \fxi}                                                 % y^i f(x^i), margin
\newcommand{\Sigmah}{\hat \Sigma}											% estimated covariance matrix
\newcommand{\Sigmahj}{\hat \Sigma_j}										% estimated covariance matrix for the j-th class

% ml - loss, risk, likelihood
\newcommand{\Lxy}{L(y, f(x))}                                               % L(y, f(x)), loss function
\newcommand{\Lxyi}{L(\yi, \fxi)}                                            % L(y^i, f(x^i))
\newcommand{\Lxyt}{L(y, \fxt)}                                              % L(y, f(x | theta))
\newcommand{\Lxyit}{L(\yi, \fxit)}                                          % L(y^i, f(x^i | theta)
\newcommand{\risk}{\mathcal{R}}                                             % R, risk
\newcommand{\riskf}{\risk(f)}                                               % R(f), risk
\newcommand{\riske}{\mathcal{R}_{\text{emp}}}                               % R_emp, empirical risk
\newcommand{\riskef}{\riske(f)}                                             % R_emp(f)
\newcommand{\risket}{\mathcal{R}_{\text{emp}}(\theta)}                      % R_emp(theta)
\newcommand{\riskr}{\mathcal{R}_{\text{reg}}}                               % R_reg, regularized risk
\newcommand{\riskrt}{\mathcal{R}_{\text{reg}}(\theta)}                      % R_reg(theta)
\newcommand{\riskrf}{\riskr(f)}                                             % R_reg(f)
\newcommand{\LL}{\mathcal{L}}                                               % L, likelihood
\newcommand{\LLt}{\mathcal{L}(\theta)}                                      % L(theta), likelihood
\renewcommand{\ll}{\ell}                                                    % l, log-likelihood
\newcommand{\llt}{\ell(\theta)}                                             % l(theta), log-likelihood
\newcommand{\LS}{\mathfrak{L}}                                              % ????????????
\newcommand{\TS}{\mathfrak{T}}                                              % ??????????????
\newcommand{\errtrain}{\text{err}_{\text{train}}}                           % training error
\newcommand{\errtest}{\text{err}_{\text{test}}}                             % training error
\newcommand{\errexp}{\overline{\text{err}_{\text{test}}}}                   % training error

% resampling
\newcommand{\GE}[1]{GE(\fh_{#1})}                                           % Generalization error GE
\newcommand{\GEh}[1]{\widehat{GE}_{#1}}                                     % Estimated train error
\newcommand{\GED}{\GE{\D}}                                                  % Generalization error GE
\newcommand{\EGEn}{EGE_n}                                                   % Generalization error GE
\newcommand{\EDn}{\E_{|D| = n}}                                             % Generalization error GE


% ml - irace
\newcommand{\costs}{\mathcal{C}} % costs
\newcommand{\Celite}{\theta^*} % elite configurations
\newcommand{\instances}{\mathcal{I}} % sequence of instances
\newcommand{\budget}{\mathcal{B}} % computational budget

\usepackage{booktabs}

\newcommand{\titlefigure}{figure_man/vgg_example.png}
\newcommand{\learninggoals}{
  % \item Understand why you should care about properties of loss functions
  \item Know the concept of robustness 
  \item Learn about analytical and computational properties of loss functions 
  \item Understand that the loss function may influence convergence of the optimizer
}

\title{Introduction to Machine Learning}
% \author{Bernd Bi{}schl, Christoph Molnar, Daniel Schalk, Fabian Scheipl}
\institute{\href{https://compstat-lmu.github.io/lecture_i2ml/}{compstat-lmu.github.io/lecture\_i2ml}}
\date{}


\begin{document}

% ------------------------------------------------------------------------------

\lecturechapter{Properties of Loss Functions}
\lecture{Introduction to Machine Learning}

\begin{vbframe}{The role of Loss Functions}

Why should we care about how to choose the loss function $\Lxy$?

\begin{itemize}
% \item For regression, the loss usually only depends on residual $\Lxy = L\left(y - \fx\right) = L(\eps)$, this is a \emph{translation invariant} loss
\item \textbf{Statistical} properties: choice of loss implies statistical assumptions on the distribution of $y ~|~ \xv = \xv$ (see \emph{maximum likelihood estimation vs.
empirical risk minimization}). 
\item \textbf{Robustness} properties: some loss functions are more robust towards outliers than others. 
\item \textbf{Analytical} properties: the computational / optimization complexity of the problem 
$$
\argmin_{\thetab \in \Theta} \risket
$$
is influenced by the choice of the loss function. 
\end{itemize}

\end{vbframe}

% ------------------------------------------------------------------------------

\begin{vbframe}{Basic Types of Regression Losses}

% https://davidrosenberg.github.io/mlcourse/Archive/2017/Lectures/3b.loss-functions.pdf

\begin{itemize}
  \small
  \item Regression losses usually only depend on the \textbf{residuals}
  $r := y - \fx.$
  \item Classification losses are usually expressed in terms of the 
  \textbf{margin} $\nu := y \cdot \fx.$
  \item A loss is called \textbf{distance-based} if
  \begin{itemize}
    \small
    \item it can be written in terms of the residual, i.e., 
    $\Lxy = \psi (r)$ \\for some $\psi: \R \to \R, ~ \text{and}$
    \item $\psi(r) = 0 \Leftrightarrow r = 0$.
  \end{itemize}
  \item A loss is \textbf{translation-invariant} if $L(y + a, \fx + a) = \Lxy$, 
  $a \in \R$.
  \item Losses are called \textbf{symmetric} if $\Lxy = L\left(\fx, y\right)$. 
\end{itemize}

\vfill

\begin{minipage}[b]{0.33\textwidth}
  \includegraphics[width=\textwidth]{figure/loss_dist_based}
  \scriptsize \centering
  Distance-based: $L1$ loss
\end{minipage}%
\begin{minipage}[b]{0.33\textwidth}
  \includegraphics[width=\textwidth]{figure/loss_transl_inv.png}
  \scriptsize \centering
  Translation-invariant: $L2$ loss
\end{minipage}%
\begin{minipage}[b]{0.33\textwidth}
  \includegraphics[width=\textwidth]{figure/loss_symmetric}
  \scriptsize \centering
  Symmetric: Brier score
\end{minipage}

\end{vbframe}

% ------------------------------------------------------------------------------

\begin{vbframe}{Robustness}

\small

Outliers (in $y$) have large residuals $r = y - \fx$. Some losses are more
strongly affected by large residuals than others. 

\vspace{0.5cm}

\begin{minipage}[c]{0.55\textwidth}
  \footnotesize
  \begin{table}[]
  \begin{tabular}{r|r|r|r}
  \toprule
  $y - \fxh$ & $L1$ & $L2$ & Huber ($\eps = 5$) \\ \hline
  1 & 1 & 1 & 0.5 \\
  5 & 5 & 25 & 12.5 \\
  10 & 10 & 100 & 37.5 \\
  50 & 50 & 2500  & 237.5
\end{tabular}
\end{table}
\end{minipage}%
\begin{minipage}[c]{0.05\textwidth}
  \phantom{foo}
\end{minipage}%
\begin{minipage}[c]{0.4\textwidth}
  \small
  As a consequence, a model is less influenced by outliers than by inliers if 
  the loss is \textbf{robust}.
\end{minipage}%

\vfill

\begin{minipage}[c]{0.55\textwidth}
  \includegraphics[width=\textwidth]{figure/robustness}
  \footnotesize \centering
\end{minipage}%
\begin{minipage}[c]{0.05\textwidth}
  \phantom{foo}
\end{minipage}%
\begin{minipage}[c]{0.4\textwidth}
  \small \raggedright
  $L2$ is an example for a loss function that is not very robust towards
  outliers. It penalizes large residuals more than $L1$ or Huber loss, which are
  considered robust.
\end{minipage}%

% \framebreak 
% 
% The L2 loss is an example for a loss function that is not very robust towards outliers. It penalizes large residuals more than the L1 or the Huber loss. The L1 and the Huber loss are thus regarded robust. 
% 
% \begin{center}
% \includegraphics[width=0.8\textwidth]{figure/robustness.png}
% \end{center}

\end{vbframe}

% ------------------------------------------------------------------------------

\begin{vbframe}{Analytical Properties: Smoothness}

% \textcolor{blue}{LW: points 1-3 together w/ plot on subsequent slide; scnd slide w/ concrete examples (e.g., gd failing w/ lasso}

\begin{itemize}
  \small
  \item \textbf{Smoothness} of a function is a property measured by 
  the number of continuous derivatives. 
  \item A function is said to be $\mathcal{C}^k$ if it is $k$ times 
  continuously differentiable. A function is $\mathcal{C}^\infty$ if it is 
  continuously differently for all orders $k$. 
  \item Derivative-based methods require a certain level of smoothness of the 
  risk function $\risket$. 
  \begin{itemize}
    \small
    \item If the loss function is not smooth, the risk minimization problem will
    generally not be smooth either. 
    \item This may require the use of derivative-free optimization (which 
    might not be desirable).
  \end{itemize}
\end{itemize}

\vfill

\begin{minipage}[c]{0.4\textwidth}
  \includegraphics[width=0.9\textwidth]{figure/plot_loss_overview_classif}
\end{minipage}%
\begin{minipage}[c]{0.05\textwidth}
  \phantom{foo}
\end{minipage}%
\begin{minipage}[c]{0.55\textwidth}
  \footnotesize \raggedright
  Squared loss, exponential loss and squared hinge loss are continuously 
  differentiable. Hinge loss is continuous but not differentiable. 
  0-1 loss is not even continuous.
\end{minipage}%

\framebreak

\small
\textbf{Example: Lasso regression}

\begin{itemize}
  \small
  % \item Most optimization methods require differentiability of loss 
  % function, e.g.,
  \item Problem: Lasso has non-differentiable 
  objective function $$\riskrt = \| \yv - \Xmat \thetab \|^2_2
  % \sumin \left( y - \fxit \right)^2 
  + \lambda \| 
  \thetab \|_1 ~ \in \mathcal{C}^0,$$
  but many optimization methods are derivative-based, e.g.,
  \begin{itemize}
    \small
    \item Gradient descent: requires existence of gradient $\nabla \risket$, 
    \item Newton-Raphson: requires existence of Hessian $\nabla^2 \risket$.
  \end{itemize}
  \item We must therefore resort to alternative optimization 
  techniques -- for instance, coordinate descent with subgradients.
\end{itemize}

\vfill

\begin{minipage}[c]{0.3\textwidth}
  \includegraphics[width=0.9\textwidth]{figure/lasso_unpenalized}
\end{minipage}%
\begin{minipage}[c]{0.05\textwidth}
  \phantom{foo}
\end{minipage}%
\begin{minipage}[c]{0.3\textwidth}
  \includegraphics[width=0.9\textwidth]{figure/lasso_penalty}
\end{minipage}%
\begin{minipage}[c]{0.05\textwidth}
  \phantom{foo}
\end{minipage}%
\begin{minipage}[c]{0.3\textwidth}
  \includegraphics[width=0.9\textwidth]{figure/lasso_penalized}
\end{minipage}%

\tiny Example: XYZ. \textit{Left:} unpenalized objective, \textit{middle:}
$L1$ penalty, \textit{right:} penalized objective. \textcolor{blue}{screenshot}
    % \item Example: Gradient descent requires differentiability of the $\risket$ (existence of $\nabla \risket$), Newton-Raphson requires $\risket$ to be twice differentiable (existence of Hessian $\nabla^2 \risket$). 

\end{vbframe}

% ------------------------------------------------------------------------------

\begin{vbframe}{Analytical Properties: Convexity}

\begin{itemize}
  \footnotesize
  \item A function $\risket$ is convex if
  $$
    \risk\left(t \cdot \thetab + (1 - t) \cdot \tilde \thetab\right) \le t \cdot
    \risk\left(\thetab\right) + (1 - t) \cdot \risk\left(\tilde \thetab \right)
  $$
  $\forall$ $t \in [0, 1], ~\thetab, \tilde \thetab \in \Theta$
  (strictly convex if the above holds with equality).
  \item In optimization, convex optimization problems are desirable because 
  they have a number of conventient properties. 
  \item In particular, it holds for convex problems that local optima are 
  global optima \\$\rightarrow$ a strictly convex function has at most 
  \textbf{one} global minimum (uniqueness). 
  \item Note, however, that convexity of $\risket$ depends both on convexity of 
  \begin{itemize} 
    \footnotesize
    \item $L(\cdot)$ -- given in most cases -- and 
    \item $\fxt$ -- often problematic.
  \end{itemize}  
\end{itemize} 

\vfill

\begin{minipage}[b]{0.5\textwidth}
  \footnotesize \raggedright
  \href{https://arxiv.org/pdf/1712.09913.pdf}{Li et al., 2018: 
  \textit{Visualizing the Loss Landscape of Neural Nets}}. 
  The problem on the bottom right is convex, the others are not (note that 
  very high-dimensional surfaces are coerced into 3D here).  
  \\
  \phantom{foo}
\end{minipage}%
\begin{minipage}[b]{0.05\textwidth}
  \phantom{foo}
\end{minipage}%
\begin{minipage}[b]{0.45\textwidth}
  \includegraphics[width=0.75\textwidth]{
  figure_man/convex-vs-nonconfex-landscape}
\end{minipage}%

\end{vbframe}

% ------------------------------------------------------------------------------

\begin{vbframe}{Analytical Properties: Convergence}

% \textcolor{blue}{LW: not that much detail (belongs to logreg instead, is addressed in demo); story: if loss never reaches 0, w/ linearly separable data, it becomes desirable to have an infinitely steep logistic fun --> theta to infty}
% 
% \textcolor{red}{@BB: Do we want to cover this case here in such a detailed manner? }

\small
The choice of the loss function may also impact the convergence behavior of the 
optimization problem. 

\begin{itemize} 
  \small
  \item Example: gradient descent in logistic regression will not converge for 
  linearly separable data (\textbf{complete separation}). 
  \item This is a direct consequence of the convergence behavior of Bernoulli 
  loss, which reaches 0 only in the infinite limit of the margin.
  \item In the case of complete separation, we have
  \begin{flalign*}
    \risket &= \sumin \log \left( 1 + \exp \left( - \yi \thetab^T \xi \right) 
    \right) \\ &=
    \sumin \log \left( 1 + \exp \left( - | \thetab^T \xi| \right) 
    \right),
  \end{flalign*}
  as every observation is correctly classified (i.e., $\thetab^T \xi < 0$ \\for
  $\yi = -1$ and $\thetab^T \xi > 0$ for $\yi = 1$).
\end{itemize}  

\framebreak

\phantom{foo}

% \vspace{0.3cm}

\begin{minipage}[b]{0.55\textwidth}
  \footnotesize
  $\risket$ thus monotonically decreases in $\thetab$: if a parameter 
  vector $\thetab^\prime$ is able to classify the samples perfectly, then 
  $2\thetab^\prime$ also classifies the samples perfectly, at lower risk.
\end{minipage}%
\begin{minipage}[b]{0.05\textwidth}
  \phantom{foo}
\end{minipage}%
\begin{minipage}[b]{0.4\textwidth}
  \includegraphics[width=\textwidth]{figure_man/snap_bernoulli_loss}
\end{minipage}%

\vfill

\begin{minipage}[b]{0.55\textwidth}
  \footnotesize
  Geometrically, this translates to an ever steeper slope of the 
  logistic/softmax function, leading to increasingly sharp discrimination and 
  infinitely running optimization.
\end{minipage}%
\begin{minipage}[b]{0.05\textwidth}
  \phantom{foo}
\end{minipage}%
\begin{minipage}[b]{0.2\textwidth}
  \includegraphics[width=\textwidth]{figure_man/snap_softmax_1}
  \textcolor{blue}{screenshot}
\end{minipage}%
\begin{minipage}[b]{0.2\textwidth}
  \includegraphics[width=\textwidth]{figure_man/snap_softmax_2}
\end{minipage}%

\vfill

In practice, data are seldom linearly separable and misclassified 
examples act as counterweights to increasing parameter values. Besides, we 
can apply \textbf{regularization} to encourage convergence to robust solutions.

\end{vbframe}

% ------------------------------------------------------------------------------

\begin{vbframe}{Analytical Properties: Convergence}

\textcolor{red}{@BB: Do we want to cover this case here in such a detailed manner? }

The choice of the loss function may also imply convergence behavior of the optimization problem. 

\vspace*{0.2cm}

\textbf{Example: } Gradient descent will not converge if we minimize the Bernoulli loss for linearly separable data. 

\vspace*{0.2cm}

First, we take a look at logistic regression for an almost linearly separable dataset consisting of the observations $\xv^{(1)}, \dots, \xv^{(8)}$.
\vfill

\begin{figure}
\includegraphics[width=0.9\textwidth]{figure_man/undet-problem01.png}\\
\end{figure}


Note: WLOG we estimate the
model without intercept, s.t. we can visualize the regression coefficient 
$\bm{\theta}$ in 2D. Also, the symmetry of the data does not influence the generality of our conclusions.

\vspace*{0.2cm}

Because of the symmetry of the data, the direction\footnote[frame]{$\bm{\theta}$ is perpendicular to the decision boundary and points to the "1"-space.} of $\bm{\theta}$ is $\tilde\thetab := (\frac{1}{\sqrt{2}}, -\frac{1}{\sqrt{2}})^\top$.

\medskip

To find $\overline{\theta} := ||\bm{\theta}||_2$, we consider the empirical risk $\riske$ along $\tilde\thetab$:
\begin{align*}
\riske &= \sum^8_{i=1} \log \left[1 + \exp \left(-y^{(i)}\thetab^\top \xv^{(i)}\right)\right] \\
&= \underbrace{\sum^6_{i=1} \log \left[1 + \exp \left(- \overline{\theta} \left|\tilde{\thetab}^\top \xv^{(i)}\right|\right)\right]}_{=: \; f_{\text{cor}}(\overline{\theta}) \;\text{(correctly classified)}} +
\underbrace{\sum^8_{i=7}\log \left[1 + \exp \left( \overline{\theta} \left| \tilde{\thetab}^\top \xv^{(i)}\right|\right)\right]}_{=:\; f_{\text{incor}}(\overline{\theta}) \;\text{(incorrectly classified)}}.
\end{align*}

\end{vbframe}


\begin{vbframe}{Analytical Properties: Convergence}



Clearly, $f_{\text{cor}}$ / $f_{\text{incor}}$ are monotonically decreasing/increasing with rising length of $\thetab$:

\begin{figure}
\includegraphics[width=0.8\textwidth]{figure_man/undet-problem02.png}\\
\end{figure}

\begin{itemize}
\item By removing obs. 7 and 8, we get a linearly separable dataset. \\
\item This also means that we lose our "counterweight", i.e., if a parameter vector $\thetab$ is able to classify the samples perfectly, the vector $2\thetab$ also classifies the samples perfectly, with decreased risk.
\item Therefore, an iterative optimizer such as gradient descent will continually increase $\thetab$ and never halt (in theory).
\item In such cases, regularization can guarantee convergence (see chapter on regularization). 
\end{itemize}

\end{vbframe}

\endlecture

\end{document}