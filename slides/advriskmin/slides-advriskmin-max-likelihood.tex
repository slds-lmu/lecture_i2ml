%<<setup-child, include = FALSE>>=
%library(knitr)
%library(qrmix)
%library(mlr)
%library(quantreg)
%library(reshape2)
%set_parent("../style/preamble.Rnw")
%@

\usepackage[]{graphicx}
\usepackage[]{color}
% maxwidth is the original width if it is less than linewidth
% otherwise use linewidth (to make sure the graphics do not exceed the margin)
\makeatletter
\def\maxwidth{ %
  \ifdim\Gin@nat@width>\linewidth
    \linewidth
  \else
    \Gin@nat@width
  \fi
}
\makeatother

% ---------------------------------%
% latex-math dependencies, do not remove:
% - \usepackage{mathtools}
% - \usepackage{bm}
% - \usepackage{siunitx}
% - \usepackage{dsfont}
% - \usepackage{xspace}
% ---------------------------------%

%--------------------------------------------------------%
%       Language, encoding, typography
%--------------------------------------------------------%

\usepackage[english]{babel}
\usepackage[utf8]{inputenc} % Enables inputting UTF-8 symbols
% Standard AMS suite
\usepackage{amsmath,amsfonts,amssymb}

% Font four double-stroke / blackboard letters for sets of numbers (N, R, ...)
% Distribution name is "doublestroke"
% According to https://mirror.physik.tu-berlin.de/pub/CTAN/fonts/doublestroke/dsdoc.pdf
% the "bbm" package does a similar thing and may be superfluous.
% Required for latex-math
\usepackage{dsfont}

% bbm – "Blackboard-style" cm fonts (https://www.ctan.org/pkg/bbm)
% Used to be in common.tex, loaded directly after this file
% Maybe superfluous given dsfont is loaded
% TODO: Check if really unused?
% \usepackage{bbm}

% bm – Access bold symbols in maths mode - https://ctan.org/pkg/bm
% Required for latex-math
% https://tex.stackexchange.com/questions/3238/bm-package-versus-boldsymbol
\usepackage{bm}

% pifont – Access to PostScript standard Symbol and Dingbats fonts
% Used for \newcommand{\xmark}{\ding{55}, which is never used
% aside from lecture_advml/attic/xx-automl/slides.Rnw
% \usepackage{pifont}

% Quotes (inline and display), provdes \enquote
% https://ctan.org/pkg/csquotes
\usepackage{csquotes}

% Adds arg to enumerate env, technically superseded by enumitem according
% to https://ctan.org/pkg/enumerate
% Replace with https://ctan.org/pkg/enumitem ?
\usepackage{enumerate}

% Line spacing - provides \singlespacing \doublespacing \onehalfspacing
% https://ctan.org/pkg/setspace
% TODO: Check if really unused?
%\usepackage{setspace}

% mathtools – Mathematical tools to use with amsmath
% https://ctan.org/pkg/mathtools?lang=en
% latex-math dependency according to latex-math repo
\usepackage{mathtools}

%--------------------------------------------------------%
%       Displaying code and algorithms
%--------------------------------------------------------%
\usepackage{verbatim}
\usepackage{algorithm}
\usepackage{algpseudocode}

%--------------------------------------------------------%
%       Tables
%--------------------------------------------------------%

% multi-row table cells: https://www.namsu.de/Extra/pakete/Multirow.html
\usepackage{multirow}

% long/multi-page tables: https://texdoc.org/serve/longtable.pdf/0
% TODO: Check if really unused?

\usepackage{longtable}

% pretty table env: https://ctan.org/pkg/booktabs?lang=en
% TODO: Check if really unused?
\usepackage{booktabs}

%--------------------------------------------------------%
%       Figures: Creating, placing, verbing
%--------------------------------------------------------%

% wrapfig - Wrapping text around figures https://de.overleaf.com/learn/latex/Wrapping_text_around_figures
\usepackage{wrapfig}

% Sub figures in figures and tables
% https://ctan.org/pkg/subfig -- supersedes subfigure package
% TODO: Check if really unused?
\usepackage{subfig}

% Actually it's pronounced PGF https://en.wikibooks.org/wiki/LaTeX/PGF/TikZ
\usepackage{tikz}

\usetikzlibrary{shapes,arrows,automata,positioning,calc,chains,trees, shadows}
\tikzset{
  %Define standard arrow tip
  >=stealth',
  %Define style for boxes
  punkt/.style={
    rectangle,
    rounded corners,
    draw=black, very thick,
    text width=6.5em,
    minimum height=2em,
    text centered},
  % Define arrow style
  pil/.style={
    ->,
    thick,
    shorten <=2pt,
    shorten >=2pt,}
}


% Unsorted
% textpos – Place boxes at arbitrary positions on the LATEX page
% https://ctan.org/pkg/textpos?lang=en
% Provides \begin{textblock}
 % TODO: Check if really unused?
\usepackage[absolute,overlay]{textpos}

% psfrag – Replace strings in encapsulated PostScript figures
% https://www.overleaf.com/latex/examples/psfrag-example/tggxhgzwrzhn
% https://ftp.mpi-inf.mpg.de/pub/tex/mirror/ftp.dante.de/pub/tex/macros/latex/contrib/psfrag/pfgguide.pdf
% Can't tell if this is needed
% TODO: Check if really unused?
\usepackage{psfrag}

% Maybe not great to use this https://tex.stackexchange.com/a/197/19093
% Use align instead -- TODO: Global search & replace to check
\usepackage{eqnarray}

\usepackage{colortbl}

% arydshln – Draw dash-lines in array/tabular
% https://www.ctan.org/pkg/arydshln
% !! "arydshln has to be loaded after array, longtable, colortab and/or colortbl"
% Provides \hdashline and \cdashline
% TODO: Check if really unused?
% \usepackage{arydshln}

% tabularx – Tabulars with adjustable-width columns
% https://ctan.org/pkg/tabularx
% Provides \begin{tabularx}
% TODO: Check if really unused?
% \usepackage{tabularx}

% placeins – Control float placement
% https://ctan.org/pkg/placeins
% Defines a \FloatBarrier command
% TODO: Check if really unused?
% \usepackage{placeins}


% framed – Framed or shaded regions that can break across pages
% https://ctan.org/pkg/framed
% Provides \begin{framed} which uses \colorbox{shadecolor} relying on \definecolor{shadecolor}.
% TODO: Check if really unused?
% \usepackage{framed}

% Used often in conjunction with \definecolor{shadecolor}{rgb}{0.969, 0.969, 0.969}
% Might be able to be removed or at least redefined to only have shadecolor (if needed)
\definecolor{fgcolor}{rgb}{0.345, 0.345, 0.345}
\definecolor{shadecolor}{rgb}{0.969, 0.969, 0.969}
\newenvironment{knitrout}{}{} % an empty environment to be redefined in TeX


% Defines macros and environments
\usepackage{../../style/lmu-lecture}

\let\code=\texttt % Used regularly
\let\proglang=\textsf % Unused?

% Not sure what/why this does
\setkeys{Gin}{width=0.9\textwidth}

\setbeamertemplate{frametitle}{\expandafter\uppercase\expandafter\insertframetitle}

% Can't find a reason why common.tex is not just part of this file?

% basic latex stuff
\newcommand{\pkg}[1]{{\fontseries{b}\selectfont #1}} %fontstyle for R packages
\newcommand{\lz}{\vspace{0.5cm}} %vertical space
\newcommand{\dlz}{\vspace{1cm}} %double vertical space
\newcommand{\oneliner}[1] % Oneliner for important statements
{\begin{block}{}\begin{center}\begin{Large}#1\end{Large}\end{center}\end{block}}


%new environments
\newenvironment{vbframe}  %frame with breaks and verbatim
{
 \begin{frame}[containsverbatim,allowframebreaks]
}
{
\end{frame}
}

\newenvironment{vframe}  %frame with verbatim without breaks (to avoid numbering one slided frames)
{
 \begin{frame}[containsverbatim]
}
{
\end{frame}
}

\newenvironment{blocki}[1]   % itemize block
{
 \begin{block}{#1}\begin{itemize}
}
{
\end{itemize}\end{block}
}

\newenvironment{fragileframe}[2]{  %fragile frame with framebreaks
\begin{frame}[allowframebreaks, fragile, environment = fragileframe]
\frametitle{#1}
#2}
{\end{frame}}


\newcommand{\myframe}[2]{  %short for frame with framebreaks
\begin{frame}[allowframebreaks]
\frametitle{#1}
#2
\end{frame}}

\newcommand{\remark}[1]{
  \textbf{Remark:} #1
}


\newenvironment{deleteframe}
{
\begingroup
\usebackgroundtemplate{\includegraphics[width=\paperwidth,height=\paperheight]{../style/color/red.png}}
 \begin{frame}
}
{
\end{frame}
\endgroup
}
\newenvironment{simplifyframe}
{
\begingroup
\usebackgroundtemplate{\includegraphics[width=\paperwidth,height=\paperheight]{../style/color/yellow.png}}
 \begin{frame}
}
{
\end{frame}
\endgroup
}\newenvironment{draftframe}
{
\begingroup
\usebackgroundtemplate{\includegraphics[width=\paperwidth,height=\paperheight]{../style/color/green.jpg}}
 \begin{frame}
}
{
\end{frame}
\endgroup
}
% https://tex.stackexchange.com/a/261480: textcolor that works in mathmode
\makeatletter
\renewcommand*{\@textcolor}[3]{%
  \protect\leavevmode
  \begingroup
    \color#1{#2}#3%
  \endgroup
}
\makeatother


%-------------------------------------------------------------------------------------------------------%
%  Unused stuff that needs to go but is kept here currently juuuust in case it was important after all  %
%-------------------------------------------------------------------------------------------------------%

% \newcommand{\hlnum}[1]{\textcolor[rgb]{0.686,0.059,0.569}{#1}}%
% \newcommand{\hlstr}[1]{\textcolor[rgb]{0.192,0.494,0.8}{#1}}%
% \newcommand{\hlcom}[1]{\textcolor[rgb]{0.678,0.584,0.686}{\textit{#1}}}%
% \newcommand{\hlopt}[1]{\textcolor[rgb]{0,0,0}{#1}}%
% \newcommand{\hlstd}[1]{\textcolor[rgb]{0.345,0.345,0.345}{#1}}%
% \newcommand{\hlkwa}[1]{\textcolor[rgb]{0.161,0.373,0.58}{\textbf{#1}}}%
% \newcommand{\hlkwb}[1]{\textcolor[rgb]{0.69,0.353,0.396}{#1}}%
% \newcommand{\hlkwc}[1]{\textcolor[rgb]{0.333,0.667,0.333}{#1}}%
% \newcommand{\hlkwd}[1]{\textcolor[rgb]{0.737,0.353,0.396}{\textbf{#1}}}%
% \let\hlipl\hlkwb

% \makeatletter
% \newenvironment{kframe}{%
%  \def\at@end@of@kframe{}%
%  \ifinner\ifhmode%
%   \def\at@end@of@kframe{\end{minipage}}%
%   \begin{minipage}{\columnwidth}%
%  \fi\fi%
%  \def\FrameCommand##1{\hskip\@totalleftmargin \hskip-\fboxsep
%  \colorbox{shadecolor}{##1}\hskip-\fboxsep
%      % There is no \\@totalrightmargin, so:
%      \hskip-\linewidth \hskip-\@totalleftmargin \hskip\columnwidth}%
%  \MakeFramed {\advance\hsize-\width
%    \@totalleftmargin\z@ \linewidth\hsize
%    \@setminipage}}%
%  {\par\unskip\endMakeFramed%
%  \at@end@of@kframe}
% \makeatother

% \definecolor{shadecolor}{rgb}{.97, .97, .97}
% \definecolor{messagecolor}{rgb}{0, 0, 0}
% \definecolor{warningcolor}{rgb}{1, 0, 1}
% \definecolor{errorcolor}{rgb}{1, 0, 0}
% \newenvironment{knitrout}{}{} % an empty environment to be redefined in TeX

% \usepackage{alltt}
% \newcommand{\SweaveOpts}[1]{}  % do not interfere with LaTeX
% \newcommand{\SweaveInput}[1]{} % because they are not real TeX commands
% \newcommand{\Sexpr}[1]{}       % will only be parsed by R
% \newcommand{\xmark}{\ding{55}}%

% math spaces
\newcommand{\N}{\mathds{N}}                                                 % N, naturals
\newcommand{\Z}{\mathds{Z}}                                                 % Z, integers
\newcommand{\Q}{\mathds{Q}}                                                 % Q, rationals
\newcommand{\R}{\mathds{R}}                                                 % R, reals
\newcommand{\C}{\mathds{C}}                                                 % C, complex
\newcommand{\HS}{\mathcal{H}}                                               % H, hilbertspace
\newcommand{\continuous}{\mathcal{C}}                                       % C, space of continuous functions
\newcommand{\M}{\mathcal{M}} 												% machine numbers
\newcommand{\epsm}{\epsilon_m} 												% maximum error


% basic math stuff
\newcommand{\xt}{\tilde x}													% x tilde
\def\argmax{\mathop{\sf arg\,max}}                                          % argmax
\def\argmin{\mathop{\sf arg\,min}}                                          % argmin
\newcommand{\sign}{\operatorname{sign}}                                     % sign, signum
\newcommand{\I}{\mathbb{I}}                                                 % I, indicator
\newcommand{\order}{\mathcal{O}}                                            % O, order
\newcommand{\fp}[2]{\frac{\partial #1}{\partial #2}}                        % partial derivative
\newcommand{\pd}[2]{\frac{\partial{#1}}{\partial #2}}						% partial derivative

% sums and products
\newcommand{\sumin}{\sum_{i=1}^n}											% summation from i=1 to n
\newcommand{\sumkg}{\sum_{k=1}^g}											% summation from k=1 to g
\newcommand{\prodin}{\prod_{i=1}^n}											% product from i=1 to n
\newcommand{\prodkg}{\prod_{k=1}^g}											% product from k=1 to g

% linear algebra
\newcommand{\one}{\boldsymbol{1}}                                           % 1, unitvector
\newcommand{\id}{\mathrm{I}}                                                % I, identity
\newcommand{\diag}{\operatorname{diag}}                                     % diag, diagonal
\newcommand{\trace}{\operatorname{tr}}                                      % tr, trace
\newcommand{\spn}{\operatorname{span}}                                      % span
\newcommand{\scp}[2]{\left\langle #1, #2 \right\rangle}                     % <.,.>, scalarproduct
\newcommand{\mat}[1]{ 														% short pmatrix command
	\begin{pmatrix}
		#1
	\end{pmatrix}
}
\newcommand{\Amat}{\bm{A}}													% matrix A
\newcommand{\xv}{\bm{x}}													% vector x (bold)
\newcommand{\yv}{\bm{y}}														% vector y (bold)
\newcommand{\Deltab}{\bm{\Delta}}											% error term for vectors
															

% basic probability + stats
\renewcommand{\P}{\mathds{P}}                                               % P, probability
\newcommand{\E}{\mathds{E}}                                                 % E, expectation
\newcommand{\var}{\mathsf{Var}}                                             % Var, variance
\newcommand{\cov}{\mathsf{Cov}}                                             % Cov, covariance
\newcommand{\corr}{\mathsf{Corr}}                                           % Corr, correlation
\newcommand{\normal}{\mathcal{N}}                                           % N of the normal distribution
\newcommand{\iid}{\overset{i.i.d}{\sim}}                                    % dist with i.i.d superscript
\newcommand{\distas}[1]{\overset{#1}{\sim}}                                 % ... is distributed as ... 
% machine learning

%%%%%% ml - data
\newcommand{\Xspace}{\mathcal{X}}                                           % X, input space
\newcommand{\Yspace}{\mathcal{Y}}                                           % Y, output space
\newcommand{\nset}{\{1, \ldots, n\}}                                        % set from 1 to n
\newcommand{\pset}{\{1, \ldots, p\}}                                        % set from 1 to p
\newcommand{\gset}{\{1, \ldots, g\}}                                        % set from 1 to g
\newcommand{\Pxy}{\P_{xy}}                                                  % P_xy
\newcommand{\xy}{(x, y)}                                                    % observation (x, y)
\newcommand{\xvec}{(x_1, \ldots, x_p)^T}                                    % (x1, ..., xp) 
\newcommand{\D}{\mathcal{D}}                                                % D, data 
\newcommand{\Dset}{\{ (x^{(1)}, y^{(1)}), \ldots, (x^{(n)},  y^{(n)})\}}    % {(x1,y1)), ..., (xn,yn)}, data
\newcommand{\xdat}{\{ x^{(1)}, \ldots, x^{(n)}\}}   						 % {x1, ..., xn}, input data
\newcommand{\ydat}{\mathbf{y}}                                              % y (bold), vector of outcomes
\newcommand{\yvec}{(y^{(1)}, \hdots, y^{(n)})^T}                            % (y1, ..., yn), vector of outcomes
\renewcommand{\xi}[1][i]{x^{(#1)}}                                          % x^i, i-th observed value of x
\newcommand{\yi}[1][i]{y^{(#1)}}                                            % y^i, i-th observed value of y 
\newcommand{\xyi}{(\xi, \yi)}                                               % (x^i, y^i), i-th observation
\newcommand{\xivec}{(x^{(i)}_1, \ldots, x^{(i)}_p)^T}                       % (x1^i, ..., xp^i), i-th observation vector
\newcommand{\xj}{x_j}                                                       % x_j, j-th feature
\newcommand{\xjb}{\mathbf{x}_j}                                             % x_j (bold), j-th feature vecor
\newcommand{\xjvec}{(x^{(1)}_j, \ldots, x^{(n)}_j)^T}                       % (x^1_j, ..., x^n_j), j-th feature vector
\newcommand{\Dtrain}{\mathcal{D}_{\text{train}}}                            % D_train, training set
\newcommand{\Dtest}{\mathcal{D}_{\text{test}}}                              % D_test, test set

%%%%%% ml - models general

% continuous prediction function f
\newcommand{\fx}{f(x)}                                                      % f(x), continuous prediction function
\newcommand{\Hspace}{H}														% hypothesis space where f is from
\newcommand{\fh}{\hat{f}}                                                   % f hat, estimated prediction function
\newcommand{\fxh}{\fh(x)}                                                   % fhat(x)
\newcommand{\fxt}{f(x | \theta)}                                            % f(x | theta)
\newcommand{\fxi}{f(\xi)}                                                   % f(x^(i))
\newcommand{\fxih}{\hat{f}(\xi)}                                            % f(x^(i))
\newcommand{\fxit}{f(x^{(i)} | \theta)}                                     % f(x^(i) | theta)
\newcommand{\fhD}{\fh_{\D}}                                                 % fhat_D, estimate of f based on D
\newcommand{\fhDtrain}{\fh_{\Dtrain}}                                       % fhat_Dtrain, estimate of f based on D

% discrete prediction function h
\newcommand{\hx}{h(x)}                                                      % h(x), discrete prediction function
\newcommand{\hh}{\hat{h}}                                                   % h hat
\newcommand{\hxh}{\hat{h}(x)}                                               % hhat(x)
\newcommand{\hxt}{h(x | \theta)}                                            % h(x | theta)
\newcommand{\hxi}{h(\xi)}                                                   % h(x^(i))
\newcommand{\hxit}{h(x^{(i)} | \theta)}                                     % h(x^(i) | theta)

% yhat
\newcommand{\yh}{\hat{y}}                                                   % y hat for prediction of target
\newcommand{\yih}{\hat{y}}                                                  % y hat for prediction of target

% theta
\newcommand{\thetah}{\hat{\theta}}                                          % theta hat

% densities + probabilities
% pdf of x 
\newcommand{\pdf}{p}                                                        % p
\newcommand{\pdfx}{p(x)}                                                    % p(x)
\newcommand{\pixt}{\pi(x | \theta)}                                         % pi(x|theta), pdf of x given theta

% pdf of (x, y)
\newcommand{\pdfxy}{p(x,y)}                                                 % p(x, y)
\newcommand{\pdfxyt}{p(x, y | \theta)}                                      % p(x, y | theta)
\newcommand{\pdfxyit}{p(\xi, \yi | \theta)}                                 % p(x^(i), y^(i) | theta)

% pdf of x given y
\newcommand{\pdfxyk}{p(x | y=k)}                                            % p(x | y = k)
\newcommand{\lpdfxyk}{\log \pdfxyk}                                         % log p(x | y = k)
\newcommand{\pdfxiyk}{p(\xi | y=k)}                                         % p(x^i | y = k)

% prior probabilities
\newcommand{\pik}{\pi_k}                                                    % pi_k, prior
\newcommand{\lpik}{\log \pik}                                               % log pi_k, log of the prior

% posterior probabilities
\newcommand{\post}{\P(y = 1 | x)}                                           % P(y = 1 | x), post. prob for y=1
\newcommand{\pix}{\pi(x)}                                                   % pi(x), P(y = 1 | x)
\newcommand{\postk}{\P(y = k | x)}                                          % P(y = k | y), post. prob for y=k
\newcommand{\pikx}{\pi_k(x)}                                                % pi_k(x), P(y = k | x)
\newcommand{\pikxt}{\pi_k(x | \theta)}                                      % pi_k(x | theta), P(y = k | x, theta)
\newcommand{\pijx}{\pi_j(x)}                                                % pi_j(x), P(y = j | x)
\newcommand{\pdfygxt}{p(y |x, \theta)}                                      % p(y | x, theta)
\newcommand{\pdfyigxit}{p(\yi |\xi, \theta)}                                % p(y^i |x^i, theta)
\newcommand{\lpdfygxt}{\log \pdfygxt }                                      % log p(y | x, theta)
\newcommand{\lpdfyigxit}{\log \pdfyigxit}                                   % log p(y^i |x^i, theta)
\newcommand{\pixh}{\hat \pi(x)}                                             % pi(x) hat, P(y = 1 | x) hat
\newcommand{\pikxh}{\hat \pi_k(x)}                                          % pi_k(x) hat, P(y = k | x) hat

% residual and margin
\newcommand{\eps}{\epsilon}                                                 % residual, stochastic
\newcommand{\epsi}{\epsilon^{(i)}}                                          % epsilon^i, residual, stochastic
\newcommand{\epsh}{\hat{\epsilon}}                                          % residual, estimated
\newcommand{\yf}{y \fx}                                                     % y f(x), margin
\newcommand{\yfi}{\yi \fxi}                                                 % y^i f(x^i), margin
\newcommand{\Sigmah}{\hat \Sigma}											% estimated covariance matrix
\newcommand{\Sigmahj}{\hat \Sigma_j}										% estimated covariance matrix for the j-th class

% ml - loss, risk, likelihood
\newcommand{\Lxy}{L(y, f(x))}                                               % L(y, f(x)), loss function
\newcommand{\Lxyi}{L(\yi, \fxi)}                                            % L(y^i, f(x^i))
\newcommand{\Lxyt}{L(y, \fxt)}                                              % L(y, f(x | theta))
\newcommand{\Lxyit}{L(\yi, \fxit)}                                          % L(y^i, f(x^i | theta)
\newcommand{\risk}{\mathcal{R}}                                             % R, risk
\newcommand{\riskf}{\risk(f)}                                               % R(f), risk
\newcommand{\riske}{\mathcal{R}_{\text{emp}}}                               % R_emp, empirical risk
\newcommand{\riskef}{\riske(f)}                                             % R_emp(f)
\newcommand{\risket}{\mathcal{R}_{\text{emp}}(\theta)}                      % R_emp(theta)
\newcommand{\riskr}{\mathcal{R}_{\text{reg}}}                               % R_reg, regularized risk
\newcommand{\riskrt}{\mathcal{R}_{\text{reg}}(\theta)}                      % R_reg(theta)
\newcommand{\riskrf}{\riskr(f)}                                             % R_reg(f)
\newcommand{\LL}{\mathcal{L}}                                               % L, likelihood
\newcommand{\LLt}{\mathcal{L}(\theta)}                                      % L(theta), likelihood
\renewcommand{\ll}{\ell}                                                    % l, log-likelihood
\newcommand{\llt}{\ell(\theta)}                                             % l(theta), log-likelihood
\newcommand{\LS}{\mathfrak{L}}                                              % ????????????
\newcommand{\TS}{\mathfrak{T}}                                              % ??????????????
\newcommand{\errtrain}{\text{err}_{\text{train}}}                           % training error
\newcommand{\errtest}{\text{err}_{\text{test}}}                             % training error
\newcommand{\errexp}{\overline{\text{err}_{\text{test}}}}                   % training error

% resampling
\newcommand{\GE}[1]{GE(\fh_{#1})}                                           % Generalization error GE
\newcommand{\GEh}[1]{\widehat{GE}_{#1}}                                     % Estimated train error
\newcommand{\GED}{\GE{\D}}                                                  % Generalization error GE
\newcommand{\EGEn}{EGE_n}                                                   % Generalization error GE
\newcommand{\EDn}{\E_{|D| = n}}                                             % Generalization error GE


% ml - irace
\newcommand{\costs}{\mathcal{C}} % costs
\newcommand{\Celite}{\theta^*} % elite configurations
\newcommand{\instances}{\mathcal{I}} % sequence of instances
\newcommand{\budget}{\mathcal{B}} % computational budget

\newcommand{\titlefigure}{figure_man/optimization_steps.jpeg}
\newcommand{\learninggoals}{
\item Understand the connection between Maximum Likelihood and Risk Minimization
\item Learn the correspondence of loss functions and distributions
}

\title{Introduction to Machine Learning}
% \author{Bernd Bischl, Christoph Molnar, Daniel Schalk, Fabian Scheipl}
\institute{\href{https://compstat-lmu.github.io/lecture_i2ml/}{compstat-lmu.github.io/lecture\_i2ml}}
\date{}

\begin{document}

\lecturechapter{Maximum Likelihood Estimation vs.
Empirical Risk Minimization}
\lecture{Introduction to Machine Learning}

\section{Regression}

\begin{vbframe}{Maximum Likelihood}

Let us approach regression from a maximum likelihood perspective. 

\lz 

We assume that 

$$
y = \ftrue(\xv) + \eps,
$$
where $\ftrue$ is a function that is parameterized by $\thetab$ with $\eps$ being a random variable that follows some distribution $\P_\eps$, with $\E[\eps] = 0$. Further, we assume $\epsilon$ to be independent of $\xv$. 

\lz 

It follows that 
\begin{itemize}
\item $y~|~\xv$ follows a distribution with mean $\ftrue(\xv)$ and variance $\var(\eps)$. 
\item We denote the corresponding density function by $p(y~|~\xv, \thetab)$. 
\end{itemize}


\framebreak 

\begin{itemize}
\item Given data 
$$
\D = \Dset
$$ 

the maximum-likelihood principle is to maximize the \textbf{likelihood}

$$ \LLt = \prod_{i=1}^n \pdfyigxit $$
or to minimize the \textbf{negative log-likelihood}:
$$ -\llt = -\sumin \lpdfyigxit $$

\framebreak 

\item Let us now simply define the negative log-likelihood as \textbf{loss function} 
$$ \Lxyt := - \lpdfygxt $$
\item Maximum-likelihood optimization can be formulated as an empirical risk minimization problem
$$\risket = \sumin \Lxyit$$

% \item Then the maximum-likelihood estimator $\thetah$, which we obtain by optimizing $\LLt$ is identical
% to the loss-minimal $\thetah$ we obtain by minimizing $\risket$.
\item We can even disregard multiplicative or additive constants in the loss as they do not change the minimizer.

\framebreak 

\item For every error distribution $\P_\eps$ we can derive an equivalent loss function, which leads to the same point estimator for the parameter vector $\thetab$ as maximum-likelihood.
\lz
\item \textbf{NB}: The other way around does not always work: We cannot derive a probability density function or error distribution corresponding to every loss function -- the Hinge loss is a prominent example.
\end{itemize}

\end{vbframe}


\begin{vbframe}{Gaussian Errors - L2-Loss} 

Let us assume that errors are Gaussian, i.e. $\epsi \sim \mathcal{N}(0, \sigma^2)$. Then


$$
y = \ftrue(\xv) + \eps \sim N\left(\ftrue(\xv), \sigma^2\right). 
$$

The likelihood is then 

\begin{eqnarray*}
\LL(\thetab) &=& \prod_{i=1}^n \pdf\left(\yi ~\bigg|~ \fxit, \sigma^2\right) \\ &\propto& \exp\left(-\frac{1}{2\sigma^2}\sumin \left(\yi - \fxit\right)^2\right)\,.
\end{eqnarray*}

\framebreak 

It is easy to see that minimizing the negative log-likelihood is equivalent to the $L2$-loss minimization approach since

\begin{eqnarray*}
- \llt &=& - \log\left(\LL(\thetab)\right) \\
&=& - \log\left(\exp\left(-\frac{1}{2\sigma^2}\sumin \left(\yi - \fxit\right)^2\right)\right) \\
&\propto& \sumin \left(\yi - \fxit\right)^2.
\end{eqnarray*}


\begin{footnotesize}
\textbf{Note:} We use $\propto$ as \enquote{proportional to ... up to multiplicative and additive constants}. 
\end{footnotesize}

\framebreak 

\begin{itemize}
\item We can plot the \enquote{empirical} error distribution, i.e. the distribution of the residuals after fitting a model w.r.t. $L2$-loss.
\item With the help of a Q-Q-plot we can compare the empirical residuals vs. the theoretical quantiles of a Gaussian distribution.  
\end{itemize}

\includegraphics{figure_man/residuals_plot_L2.pdf}

\end{vbframe}


\begin{vbframe}{Laplace Errors - L1-Loss}

Let us assume that errors are Laplacian, i.e. $\eps$ follows a Laplace distribution which has the density 

$$
 \frac{1}{2\sigma} \exp\left(-\frac{|x|}{\sigma}\right)\,, \sigma > 0.
$$ 

Then

$$
y = \ftrue(\xv) + \eps 
$$

follows a Laplace distribution with mean $\fxit$ and scale parameter $\sigma$. 

\framebreak 

The likelihood is then 

\begin{eqnarray*}
\LL(\thetab) &=& \prod_{i=1}^n \pdf\left(\yi ~\bigg|~ \fxit, \sigma\right) \\ &\propto& \exp\left(-\frac{1}{\sigma}\sumin \left|\yi - \fxit\right|\right)\,.
\end{eqnarray*}


The negative log-likelihood is

$$
- \llt \propto - \sumin \left|\yi - \fxit\right|.
$$

Minimizing the negative log-likelihood for Laplacian error terms corresponds to empirical risk minimization with L1-loss. 


\framebreak 

\begin{itemize}
\item Distribution of empirical residuals and their comparison to the theoretical quantiles of a Laplace-distribution. 
\end{itemize}

\includegraphics{figure_man/residuals_plot_L1.pdf}

\end{vbframe}

\begin{vbframe}{Other Error Distributions}

\begin{itemize}
\item There are losses that do not correspond to \enquote{real} error densities, like the Huber loss. (In the QQ-plot below we show residuals against quantiles of a normal. )
\end{itemize}

\begin{center}
\includegraphics{figure_man/residuals_plot_Huber.pdf}
\end{center}

\framebreak 

However, intuitively, we see that a certain type of loss function corresponds to a certain error distribution. 

\begin{table}[]
\begin{tabular}{ll}
Loss function & Error Distribution \\
\hline
$L2$-Loss & Gaussian Errors \\
$L1$-Loss & Laplace Errors \\
Huber Loss & \enquote{Huber Errors}
\end{tabular}
\end{table}

\end{vbframe}


\section{Classification}


\begin{vbframe}{Maximum Likelihood in Classification}

Let us assume the outputs $\yi$ to be Bernoulli-distributed, i.e.  

$$
  \yi \sim \text{Ber}(\pix) 
$$

with probability $\pix$ that depends on $\xv$. 

\lz 

The maximization of the negative log-likelihood is based on

\begin{eqnarray*}
- \llt &=& -\sumin \lpdfyigxit \\ &=& \sumin -\yi \log[\pi\left(\xi\right)] - \left(1-\yi\right) \log [1 - \pi\left(\xi\right)]. \\
\end{eqnarray*}


\framebreak 

This gives rise to the following loss function 

$$
  L_{0, 1}(y, \pix) = -y\ln(\pix)-(1-y)\ln(1-\pix)
$$

which we introduced as \textbf{Bernoulli} loss. 

\vspace{0.2cm}

\begin{center}
\includegraphics[width = 11cm ]{figure_man/bernoulli-loss.png} \\
\end{center}



% \framebreak 

% The Bernoulli loss is used in \textbf{logistic regression} in combination with the hypothesis space of linear functions

% \vspace*{-0.3cm}

% \begin{eqnarray*}
%   \Hspace = \left\{f: \Xspace \to \R ~|~\fx = \thetab^\top \xv\right\}
% \end{eqnarray*}

% Scores are afterwards transformed into probabilities by the logistic function $\pix = \left(1 + \exp(- \fx)\right)^{-1}$  

% \begin{center}
%   \includegraphics[width = 0.4\textwidth]{figure_man/logreg-2vars-data.png}~~\includegraphics[width = 0.4\textwidth]{figure_man/logreg-2vars-score-vs-prob.png}
% \end{center}

% \framebreak

% \begin{center}
%   \includegraphics[width=0.8\textwidth]{figure_man/logreg-2vars-surface.png}
% \end{center}


\end{vbframe}




% \begin{vbframe}{Classification Losses: Bernoulli Loss}
% \begin{itemize}
%   \item $L(y, \pix) = -y\ln(\pix)-(1-y)\ln(1-\pix) $
%   % \item Also called \textbf{Bernoulli Loss}
%   \item Convex, differentiable (gradient methods can be used), not robust
%   \item Also called logarithmic loss or cross-entropy loss (which will be motivated later)
% \end{itemize}


% \framebreak 

% \lz 

% The constant model $\pix =< \theta$ that is optimal w.r.t. the empirical risk is the fraction of class $1$ observations

% $$
% \hat \pi (\xv) = \frac{1}{n}\sumin \yi.
% $$

% \textbf{Proof:} Exercise.


% \end{vbframe}



% \begin{vbframe}{Classification Losses: Cross-Entropy Loss}

% \begin{itemize}
%   \item The cross entropy loss 
%   $$
%   \Lxy = -y\ln(\pix)-(1-y)\ln(1-\pix)$$
%   % is equivalent to logistic loss when 
%   \item  The cross entropy loss is closely related to the Kullback-Leibler divergence, which will be introduced later in the chapter.
%   \item Very often used in neural networks with binary output nodes for classification.
% \end{itemize}

% \end{vbframe}

% \section{Outlook}

% \begin{vbframe}{Outlook}

% When introducing different learning algorithms, we will come back to the loss functions introduced in this chapter or even introduce new ones. For example:  

% \begin{itemize}
%   \item Ordinary Linear Regression: L2-loss
%   \item Logistic Regression: Logistic loss
%   \item Support Vector Machine Classification: Hinge-Loss (to be introduced) (see \textbf{SVM} chapter)
%   \item Support Vector Machine Regression: $\epsilon$-insensitive loss (see \textbf{SVM} chapter)
%   \item AdaBoost: Exponential loss (see \textbf{Boosting} chapter)
% \end{itemize}

% Once knowing the theory of risk minimization and properties of loss functions, we can combine model classes and loss functions as needed or even tailor loss functions to our needs. 

% \end{vbframe}

% \framebreak

% we get the risk function

% \begin{eqnarray*}
% \risk(f) &=& \mathbb{E}_x [\max\{0, 1 - \fx\} \pix + \max\{0, 1 + y\fx\} (1-\pix)].
% \end{eqnarray*}

% The minimizer of $\risk(f)$ for the hinge loss function is

% \begin{eqnarray*}
  % $fh(x) =  \footnotesize \begin{cases} 1 \quad \text{ if } \pix > 1/2 \\ -1 \quad \pix < 1/2  \end{cases}$
% \end{eqnarray*}



% \section{Selected methods for regression and classification}

% \begin{vbframe}{Normal linear and additive regression}

% For $i \in \nset$ (simple case and with basis functions):
% \begin{eqnarray*}
% \yi & = & \fxi + \epsi = \theta_0 + \theta^T \xi + \epsi\\
% \yi & = & \fxi + \epsi = \theta_0 + \theta^T \phi(\xi) + \epsi
% \end{eqnarray*}

% \begin{itemize}
% \item basis functions $\phi(x)=(\phi_1(x), \ldots, \phi_m(x))^T$
% \item assumption: $\epsi \iid N(0, \sigma^2)$
% \end{itemize}

% \lz

% Given observed data $\D$  we want to address the questions
% \begin{itemize}
% \item  Given basis functions, how to find $\theta$? ({\bf Parameter estimation})
% \item  How to select basis functions for my problem? ({\bf Model selection} )
% \end{itemize}

% \framebreak

% We'll address the model selection problem later, for now leave it at:
% Doing this \enquote{manually} is rather frowned upon...

% \lz

% A typical ML way to estimate the parameters is to not require the assumption
% $\epsi \iid N(0, \sigma^2)$, but instead assume the that prediction error is measured
% by \emph{squared error} as our \emph{loss function} in \emph{risk minimization}:

% $$
% \riske(\thetab) = SSE(\thetab) = \sumin \Lxyit = \sumin \left(\yi - \theta^T \xi\right)^2
% $$

% NB: We assume here and from now on that $\theta_0$ is included in $\theta$.

% Using matrix notation the empirical risk can be written as
% $$
% SSE(\thetab) = (\ydat - \Xmat\thetab)^T(\ydat - \Xmat\thetab).
% $$


% Differentiating w.r.t $\theta$ yields the so-called \emph{normal equations}:
% $$
% \Xmat^T(\ydat - \Xmat\thetab) = 0
% $$
% The optimal $\theta$ is
% $$
% \thetah = (\Xmat^T \Xmat)^{-1} \Xmat^T\ydat
% $$

% In statistics, we would start from a maximum-likelihood perspective
% $$
% \yi = \fxi + \epsi \sim N\left(\fxi, \sigma^2\right)
% $$
% $$
% \LLt = \prod_{i=1}^n \pdf\left(\yi | \fxit, \sigma^2\right) \propto \exp\left(-\frac{\sumin (\yi - \fxit)^2}{2\sigma^2}\right)
% $$
% It's easy to see that minimizing the neg. log-likelihood is equivalent to the
% loss minimization approach since
% $$
% \llt \propto - \sumin \left(\yi - \fxit\right)^2.
% $$


% \framebreak

% <<lm-mtcars-plot>>=
% data(mtcars)
% regr.task = makeRegrTask(data = mtcars, target = "mpg")
% plotLearnerPrediction("regr.lm", regr.task, features = "disp")
% @
% \end{vbframe}

% \begin{vbframe}{Example: Linear Regr. with L1 vs L2 loss}

% <<l1-vs-l2-loss-prep>>=
% set.seed(123)

% # prediction with f, based on vec x and param vec beta
% f = function(x, beta) {
%   crossprod(x, beta)
% }

% # L1 and L2 loss, based on design mat X, vec, param vec beta, computed with f
% loss1 = function(X, y, beta) {
%   yhat = apply(X, 1, f, beta = beta)
%   sum((y - yhat)^2)
% }
% loss2 = function(X, y, beta) {
%   yhat = apply(X, 1, f, beta = beta)
%   sum(abs(y - yhat))
% }

% # optimize loss (1 or 2) with optim
% # yes, neldermead not really the best, who cares it is 1d
% optLoss = function(X, y, loss) {
%   start = rep(0, ncol(X))
%   res = optim(start, loss, method = "Nelder-Mead", X = X, y = y)
%   res$par
% }

% # plot data and a couple of linear models
% plotIt = function(X, y, models = list()) {
%   gd = data.frame(x = X[, 2],  y = y, outlier = c(TRUE, rep(FALSE, length(y) - 1)))
%   pl = ggplot(data = gd, aes(x = x, y = y, shape = outlier))
%   pl = pl + geom_point(alpha = .8) + guides(shape = FALSE)
%   for (i in seq_along(models)) {
%     m = models[[i]]
%     pl = pl + geom_abline(intercept = m$beta[1], slope = m$beta[2], 
%       col = m$col, lty = m$lty, alpha = .8)
%   }
%   return(pl)
% }


% # generate some data, sample from line with gaussian errors
% # make the leftmost obs an outlier
% n = 10
% x = sort(runif(n = n, min = 0, max = 10))
% y = 3 * x + 1 + rnorm(n, sd = 5)
% X = cbind(x0 = 1, x1 = x)
% y[1] = 50
% @

% L1 loss is less sensitive to outliers than L2 loss:
% <<l1-vs-l2-loss-plot, fig.height=4.5>>=
% # fit l1/2 models on data without then with outlier data
% b1 = optLoss(X[-1,], y[-1], loss = loss1)
% b2 = optLoss(X[-1,], y[-1], loss = loss2)
% b3 = optLoss(X, y, loss = loss1)
% b4 = optLoss(X, y, loss = loss2)

% # plot all 4 models
% pl = plotIt(X, y, models = list(
%   list(beta = b1, col = pal_2[1], lty = "solid"),
%   list(beta = b2, col = pal_2[2], lty = "solid"),
%   list(beta = b3, col = pal_2[1], lty = "dashed"),
%   list(beta = b4, col = pal_2[2], lty = "dashed")
% ))
% print(pl)
% @
% Violet = L2, green = L1 loss.\\
% Solid = fit without, dashed = fit with outlier ($\blacktriangle$ at (\Sexpr{c(x[1], y[1])})).
% \end{vbframe}

\endlecture
\end{document}
