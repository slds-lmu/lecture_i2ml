\usepackage[]{graphicx}
\usepackage[]{color}
% maxwidth is the original width if it is less than linewidth
% otherwise use linewidth (to make sure the graphics do not exceed the margin)
\makeatletter
\def\maxwidth{ %
  \ifdim\Gin@nat@width>\linewidth
    \linewidth
  \else
    \Gin@nat@width
  \fi
}
\makeatother

% ---------------------------------%
% latex-math dependencies, do not remove:
% - \usepackage{mathtools}
% - \usepackage{bm}
% - \usepackage{siunitx}
% - \usepackage{dsfont}
% - \usepackage{xspace}
% ---------------------------------%

%--------------------------------------------------------%
%       Language, encoding, typography
%--------------------------------------------------------%

\usepackage[english]{babel}
\usepackage[utf8]{inputenc} % Enables inputting UTF-8 symbols
% Standard AMS suite
\usepackage{amsmath,amsfonts,amssymb}

% Font four double-stroke / blackboard letters for sets of numbers (N, R, ...)
% Distribution name is "doublestroke"
% According to https://mirror.physik.tu-berlin.de/pub/CTAN/fonts/doublestroke/dsdoc.pdf
% the "bbm" package does a similar thing and may be superfluous.
% Required for latex-math
\usepackage{dsfont}

% bbm – "Blackboard-style" cm fonts (https://www.ctan.org/pkg/bbm)
% Used to be in common.tex, loaded directly after this file
% Maybe superfluous given dsfont is loaded
% TODO: Check if really unused?
% \usepackage{bbm}

% bm – Access bold symbols in maths mode - https://ctan.org/pkg/bm
% Required for latex-math
% https://tex.stackexchange.com/questions/3238/bm-package-versus-boldsymbol
\usepackage{bm}

% pifont – Access to PostScript standard Symbol and Dingbats fonts
% Used for \newcommand{\xmark}{\ding{55}, which is never used
% aside from lecture_advml/attic/xx-automl/slides.Rnw
% \usepackage{pifont}

% Quotes (inline and display), provdes \enquote
% https://ctan.org/pkg/csquotes
\usepackage{csquotes}

% Adds arg to enumerate env, technically superseded by enumitem according
% to https://ctan.org/pkg/enumerate
% Replace with https://ctan.org/pkg/enumitem ?
\usepackage{enumerate}

% Line spacing - provides \singlespacing \doublespacing \onehalfspacing
% https://ctan.org/pkg/setspace
% TODO: Check if really unused?
%\usepackage{setspace}

% mathtools – Mathematical tools to use with amsmath
% https://ctan.org/pkg/mathtools?lang=en
% latex-math dependency according to latex-math repo
\usepackage{mathtools}

%--------------------------------------------------------%
%       Displaying code and algorithms
%--------------------------------------------------------%
\usepackage{verbatim}
\usepackage{algorithm}
\usepackage{algpseudocode}

%--------------------------------------------------------%
%       Tables
%--------------------------------------------------------%

% multi-row table cells: https://www.namsu.de/Extra/pakete/Multirow.html
\usepackage{multirow}

% long/multi-page tables: https://texdoc.org/serve/longtable.pdf/0
% TODO: Check if really unused?

\usepackage{longtable}

% pretty table env: https://ctan.org/pkg/booktabs?lang=en
% TODO: Check if really unused?
\usepackage{booktabs}

%--------------------------------------------------------%
%       Figures: Creating, placing, verbing
%--------------------------------------------------------%

% wrapfig - Wrapping text around figures https://de.overleaf.com/learn/latex/Wrapping_text_around_figures
\usepackage{wrapfig}

% Sub figures in figures and tables
% https://ctan.org/pkg/subfig -- supersedes subfigure package
% TODO: Check if really unused?
\usepackage{subfig}

% Actually it's pronounced PGF https://en.wikibooks.org/wiki/LaTeX/PGF/TikZ
\usepackage{tikz}

\usetikzlibrary{shapes,arrows,automata,positioning,calc,chains,trees, shadows}
\tikzset{
  %Define standard arrow tip
  >=stealth',
  %Define style for boxes
  punkt/.style={
    rectangle,
    rounded corners,
    draw=black, very thick,
    text width=6.5em,
    minimum height=2em,
    text centered},
  % Define arrow style
  pil/.style={
    ->,
    thick,
    shorten <=2pt,
    shorten >=2pt,}
}


% Unsorted
% textpos – Place boxes at arbitrary positions on the LATEX page
% https://ctan.org/pkg/textpos?lang=en
% Provides \begin{textblock}
 % TODO: Check if really unused?
\usepackage[absolute,overlay]{textpos}

% psfrag – Replace strings in encapsulated PostScript figures
% https://www.overleaf.com/latex/examples/psfrag-example/tggxhgzwrzhn
% https://ftp.mpi-inf.mpg.de/pub/tex/mirror/ftp.dante.de/pub/tex/macros/latex/contrib/psfrag/pfgguide.pdf
% Can't tell if this is needed
% TODO: Check if really unused?
\usepackage{psfrag}

% Maybe not great to use this https://tex.stackexchange.com/a/197/19093
% Use align instead -- TODO: Global search & replace to check
\usepackage{eqnarray}

\usepackage{colortbl}

% arydshln – Draw dash-lines in array/tabular
% https://www.ctan.org/pkg/arydshln
% !! "arydshln has to be loaded after array, longtable, colortab and/or colortbl"
% Provides \hdashline and \cdashline
% TODO: Check if really unused?
% \usepackage{arydshln}

% tabularx – Tabulars with adjustable-width columns
% https://ctan.org/pkg/tabularx
% Provides \begin{tabularx}
% TODO: Check if really unused?
% \usepackage{tabularx}

% placeins – Control float placement
% https://ctan.org/pkg/placeins
% Defines a \FloatBarrier command
% TODO: Check if really unused?
% \usepackage{placeins}


% framed – Framed or shaded regions that can break across pages
% https://ctan.org/pkg/framed
% Provides \begin{framed} which uses \colorbox{shadecolor} relying on \definecolor{shadecolor}.
% TODO: Check if really unused?
% \usepackage{framed}

% Used often in conjunction with \definecolor{shadecolor}{rgb}{0.969, 0.969, 0.969}
% Might be able to be removed or at least redefined to only have shadecolor (if needed)
\definecolor{fgcolor}{rgb}{0.345, 0.345, 0.345}
\definecolor{shadecolor}{rgb}{0.969, 0.969, 0.969}
\newenvironment{knitrout}{}{} % an empty environment to be redefined in TeX


% Defines macros and environments
\usepackage{../../style/lmu-lecture}

\let\code=\texttt % Used regularly
\let\proglang=\textsf % Unused?

% Not sure what/why this does
\setkeys{Gin}{width=0.9\textwidth}

\setbeamertemplate{frametitle}{\expandafter\uppercase\expandafter\insertframetitle}

% Can't find a reason why common.tex is not just part of this file?

% basic latex stuff
\newcommand{\pkg}[1]{{\fontseries{b}\selectfont #1}} %fontstyle for R packages
\newcommand{\lz}{\vspace{0.5cm}} %vertical space
\newcommand{\dlz}{\vspace{1cm}} %double vertical space
\newcommand{\oneliner}[1] % Oneliner for important statements
{\begin{block}{}\begin{center}\begin{Large}#1\end{Large}\end{center}\end{block}}


%new environments
\newenvironment{vbframe}  %frame with breaks and verbatim
{
 \begin{frame}[containsverbatim,allowframebreaks]
}
{
\end{frame}
}

\newenvironment{vframe}  %frame with verbatim without breaks (to avoid numbering one slided frames)
{
 \begin{frame}[containsverbatim]
}
{
\end{frame}
}

\newenvironment{blocki}[1]   % itemize block
{
 \begin{block}{#1}\begin{itemize}
}
{
\end{itemize}\end{block}
}

\newenvironment{fragileframe}[2]{  %fragile frame with framebreaks
\begin{frame}[allowframebreaks, fragile, environment = fragileframe]
\frametitle{#1}
#2}
{\end{frame}}


\newcommand{\myframe}[2]{  %short for frame with framebreaks
\begin{frame}[allowframebreaks]
\frametitle{#1}
#2
\end{frame}}

\newcommand{\remark}[1]{
  \textbf{Remark:} #1
}


\newenvironment{deleteframe}
{
\begingroup
\usebackgroundtemplate{\includegraphics[width=\paperwidth,height=\paperheight]{../style/color/red.png}}
 \begin{frame}
}
{
\end{frame}
\endgroup
}
\newenvironment{simplifyframe}
{
\begingroup
\usebackgroundtemplate{\includegraphics[width=\paperwidth,height=\paperheight]{../style/color/yellow.png}}
 \begin{frame}
}
{
\end{frame}
\endgroup
}\newenvironment{draftframe}
{
\begingroup
\usebackgroundtemplate{\includegraphics[width=\paperwidth,height=\paperheight]{../style/color/green.jpg}}
 \begin{frame}
}
{
\end{frame}
\endgroup
}
% https://tex.stackexchange.com/a/261480: textcolor that works in mathmode
\makeatletter
\renewcommand*{\@textcolor}[3]{%
  \protect\leavevmode
  \begingroup
    \color#1{#2}#3%
  \endgroup
}
\makeatother


%-------------------------------------------------------------------------------------------------------%
%  Unused stuff that needs to go but is kept here currently juuuust in case it was important after all  %
%-------------------------------------------------------------------------------------------------------%

% \newcommand{\hlnum}[1]{\textcolor[rgb]{0.686,0.059,0.569}{#1}}%
% \newcommand{\hlstr}[1]{\textcolor[rgb]{0.192,0.494,0.8}{#1}}%
% \newcommand{\hlcom}[1]{\textcolor[rgb]{0.678,0.584,0.686}{\textit{#1}}}%
% \newcommand{\hlopt}[1]{\textcolor[rgb]{0,0,0}{#1}}%
% \newcommand{\hlstd}[1]{\textcolor[rgb]{0.345,0.345,0.345}{#1}}%
% \newcommand{\hlkwa}[1]{\textcolor[rgb]{0.161,0.373,0.58}{\textbf{#1}}}%
% \newcommand{\hlkwb}[1]{\textcolor[rgb]{0.69,0.353,0.396}{#1}}%
% \newcommand{\hlkwc}[1]{\textcolor[rgb]{0.333,0.667,0.333}{#1}}%
% \newcommand{\hlkwd}[1]{\textcolor[rgb]{0.737,0.353,0.396}{\textbf{#1}}}%
% \let\hlipl\hlkwb

% \makeatletter
% \newenvironment{kframe}{%
%  \def\at@end@of@kframe{}%
%  \ifinner\ifhmode%
%   \def\at@end@of@kframe{\end{minipage}}%
%   \begin{minipage}{\columnwidth}%
%  \fi\fi%
%  \def\FrameCommand##1{\hskip\@totalleftmargin \hskip-\fboxsep
%  \colorbox{shadecolor}{##1}\hskip-\fboxsep
%      % There is no \\@totalrightmargin, so:
%      \hskip-\linewidth \hskip-\@totalleftmargin \hskip\columnwidth}%
%  \MakeFramed {\advance\hsize-\width
%    \@totalleftmargin\z@ \linewidth\hsize
%    \@setminipage}}%
%  {\par\unskip\endMakeFramed%
%  \at@end@of@kframe}
% \makeatother

% \definecolor{shadecolor}{rgb}{.97, .97, .97}
% \definecolor{messagecolor}{rgb}{0, 0, 0}
% \definecolor{warningcolor}{rgb}{1, 0, 1}
% \definecolor{errorcolor}{rgb}{1, 0, 0}
% \newenvironment{knitrout}{}{} % an empty environment to be redefined in TeX

% \usepackage{alltt}
% \newcommand{\SweaveOpts}[1]{}  % do not interfere with LaTeX
% \newcommand{\SweaveInput}[1]{} % because they are not real TeX commands
% \newcommand{\Sexpr}[1]{}       % will only be parsed by R
% \newcommand{\xmark}{\ding{55}}%

% math spaces
\newcommand{\N}{\mathds{N}}                                                 % N, naturals
\newcommand{\Z}{\mathds{Z}}                                                 % Z, integers
\newcommand{\Q}{\mathds{Q}}                                                 % Q, rationals
\newcommand{\R}{\mathds{R}}                                                 % R, reals
\newcommand{\C}{\mathds{C}}                                                 % C, complex
\newcommand{\HS}{\mathcal{H}}                                               % H, hilbertspace
\newcommand{\continuous}{\mathcal{C}}                                       % C, space of continuous functions
\newcommand{\M}{\mathcal{M}} 												% machine numbers
\newcommand{\epsm}{\epsilon_m} 												% maximum error


% basic math stuff
\newcommand{\xt}{\tilde x}													% x tilde
\def\argmax{\mathop{\sf arg\,max}}                                          % argmax
\def\argmin{\mathop{\sf arg\,min}}                                          % argmin
\newcommand{\sign}{\operatorname{sign}}                                     % sign, signum
\newcommand{\I}{\mathbb{I}}                                                 % I, indicator
\newcommand{\order}{\mathcal{O}}                                            % O, order
\newcommand{\fp}[2]{\frac{\partial #1}{\partial #2}}                        % partial derivative
\newcommand{\pd}[2]{\frac{\partial{#1}}{\partial #2}}						% partial derivative

% sums and products
\newcommand{\sumin}{\sum_{i=1}^n}											% summation from i=1 to n
\newcommand{\sumkg}{\sum_{k=1}^g}											% summation from k=1 to g
\newcommand{\prodin}{\prod_{i=1}^n}											% product from i=1 to n
\newcommand{\prodkg}{\prod_{k=1}^g}											% product from k=1 to g

% linear algebra
\newcommand{\one}{\boldsymbol{1}}                                           % 1, unitvector
\newcommand{\id}{\mathrm{I}}                                                % I, identity
\newcommand{\diag}{\operatorname{diag}}                                     % diag, diagonal
\newcommand{\trace}{\operatorname{tr}}                                      % tr, trace
\newcommand{\spn}{\operatorname{span}}                                      % span
\newcommand{\scp}[2]{\left\langle #1, #2 \right\rangle}                     % <.,.>, scalarproduct
\newcommand{\mat}[1]{ 														% short pmatrix command
	\begin{pmatrix}
		#1
	\end{pmatrix}
}
\newcommand{\Amat}{\bm{A}}													% matrix A
\newcommand{\xv}{\bm{x}}													% vector x (bold)
\newcommand{\yv}{\bm{y}}														% vector y (bold)
\newcommand{\Deltab}{\bm{\Delta}}											% error term for vectors
															

% basic probability + stats
\renewcommand{\P}{\mathds{P}}                                               % P, probability
\newcommand{\E}{\mathds{E}}                                                 % E, expectation
\newcommand{\var}{\mathsf{Var}}                                             % Var, variance
\newcommand{\cov}{\mathsf{Cov}}                                             % Cov, covariance
\newcommand{\corr}{\mathsf{Corr}}                                           % Corr, correlation
\newcommand{\normal}{\mathcal{N}}                                           % N of the normal distribution
\newcommand{\iid}{\overset{i.i.d}{\sim}}                                    % dist with i.i.d superscript
\newcommand{\distas}[1]{\overset{#1}{\sim}}                                 % ... is distributed as ... 
% machine learning

%%%%%% ml - data
\newcommand{\Xspace}{\mathcal{X}}                                           % X, input space
\newcommand{\Yspace}{\mathcal{Y}}                                           % Y, output space
\newcommand{\nset}{\{1, \ldots, n\}}                                        % set from 1 to n
\newcommand{\pset}{\{1, \ldots, p\}}                                        % set from 1 to p
\newcommand{\gset}{\{1, \ldots, g\}}                                        % set from 1 to g
\newcommand{\Pxy}{\P_{xy}}                                                  % P_xy
\newcommand{\xy}{(x, y)}                                                    % observation (x, y)
\newcommand{\xvec}{(x_1, \ldots, x_p)^T}                                    % (x1, ..., xp) 
\newcommand{\D}{\mathcal{D}}                                                % D, data 
\newcommand{\Dset}{\{ (x^{(1)}, y^{(1)}), \ldots, (x^{(n)},  y^{(n)})\}}    % {(x1,y1)), ..., (xn,yn)}, data
\newcommand{\xdat}{\{ x^{(1)}, \ldots, x^{(n)}\}}   						 % {x1, ..., xn}, input data
\newcommand{\ydat}{\mathbf{y}}                                              % y (bold), vector of outcomes
\newcommand{\yvec}{(y^{(1)}, \hdots, y^{(n)})^T}                            % (y1, ..., yn), vector of outcomes
\renewcommand{\xi}[1][i]{x^{(#1)}}                                          % x^i, i-th observed value of x
\newcommand{\yi}[1][i]{y^{(#1)}}                                            % y^i, i-th observed value of y 
\newcommand{\xyi}{(\xi, \yi)}                                               % (x^i, y^i), i-th observation
\newcommand{\xivec}{(x^{(i)}_1, \ldots, x^{(i)}_p)^T}                       % (x1^i, ..., xp^i), i-th observation vector
\newcommand{\xj}{x_j}                                                       % x_j, j-th feature
\newcommand{\xjb}{\mathbf{x}_j}                                             % x_j (bold), j-th feature vecor
\newcommand{\xjvec}{(x^{(1)}_j, \ldots, x^{(n)}_j)^T}                       % (x^1_j, ..., x^n_j), j-th feature vector
\newcommand{\Dtrain}{\mathcal{D}_{\text{train}}}                            % D_train, training set
\newcommand{\Dtest}{\mathcal{D}_{\text{test}}}                              % D_test, test set

%%%%%% ml - models general

% continuous prediction function f
\newcommand{\fx}{f(x)}                                                      % f(x), continuous prediction function
\newcommand{\Hspace}{H}														% hypothesis space where f is from
\newcommand{\fh}{\hat{f}}                                                   % f hat, estimated prediction function
\newcommand{\fxh}{\fh(x)}                                                   % fhat(x)
\newcommand{\fxt}{f(x | \theta)}                                            % f(x | theta)
\newcommand{\fxi}{f(\xi)}                                                   % f(x^(i))
\newcommand{\fxih}{\hat{f}(\xi)}                                            % f(x^(i))
\newcommand{\fxit}{f(x^{(i)} | \theta)}                                     % f(x^(i) | theta)
\newcommand{\fhD}{\fh_{\D}}                                                 % fhat_D, estimate of f based on D
\newcommand{\fhDtrain}{\fh_{\Dtrain}}                                       % fhat_Dtrain, estimate of f based on D

% discrete prediction function h
\newcommand{\hx}{h(x)}                                                      % h(x), discrete prediction function
\newcommand{\hh}{\hat{h}}                                                   % h hat
\newcommand{\hxh}{\hat{h}(x)}                                               % hhat(x)
\newcommand{\hxt}{h(x | \theta)}                                            % h(x | theta)
\newcommand{\hxi}{h(\xi)}                                                   % h(x^(i))
\newcommand{\hxit}{h(x^{(i)} | \theta)}                                     % h(x^(i) | theta)

% yhat
\newcommand{\yh}{\hat{y}}                                                   % y hat for prediction of target
\newcommand{\yih}{\hat{y}}                                                  % y hat for prediction of target

% theta
\newcommand{\thetah}{\hat{\theta}}                                          % theta hat

% densities + probabilities
% pdf of x 
\newcommand{\pdf}{p}                                                        % p
\newcommand{\pdfx}{p(x)}                                                    % p(x)
\newcommand{\pixt}{\pi(x | \theta)}                                         % pi(x|theta), pdf of x given theta

% pdf of (x, y)
\newcommand{\pdfxy}{p(x,y)}                                                 % p(x, y)
\newcommand{\pdfxyt}{p(x, y | \theta)}                                      % p(x, y | theta)
\newcommand{\pdfxyit}{p(\xi, \yi | \theta)}                                 % p(x^(i), y^(i) | theta)

% pdf of x given y
\newcommand{\pdfxyk}{p(x | y=k)}                                            % p(x | y = k)
\newcommand{\lpdfxyk}{\log \pdfxyk}                                         % log p(x | y = k)
\newcommand{\pdfxiyk}{p(\xi | y=k)}                                         % p(x^i | y = k)

% prior probabilities
\newcommand{\pik}{\pi_k}                                                    % pi_k, prior
\newcommand{\lpik}{\log \pik}                                               % log pi_k, log of the prior

% posterior probabilities
\newcommand{\post}{\P(y = 1 | x)}                                           % P(y = 1 | x), post. prob for y=1
\newcommand{\pix}{\pi(x)}                                                   % pi(x), P(y = 1 | x)
\newcommand{\postk}{\P(y = k | x)}                                          % P(y = k | y), post. prob for y=k
\newcommand{\pikx}{\pi_k(x)}                                                % pi_k(x), P(y = k | x)
\newcommand{\pikxt}{\pi_k(x | \theta)}                                      % pi_k(x | theta), P(y = k | x, theta)
\newcommand{\pijx}{\pi_j(x)}                                                % pi_j(x), P(y = j | x)
\newcommand{\pdfygxt}{p(y |x, \theta)}                                      % p(y | x, theta)
\newcommand{\pdfyigxit}{p(\yi |\xi, \theta)}                                % p(y^i |x^i, theta)
\newcommand{\lpdfygxt}{\log \pdfygxt }                                      % log p(y | x, theta)
\newcommand{\lpdfyigxit}{\log \pdfyigxit}                                   % log p(y^i |x^i, theta)
\newcommand{\pixh}{\hat \pi(x)}                                             % pi(x) hat, P(y = 1 | x) hat
\newcommand{\pikxh}{\hat \pi_k(x)}                                          % pi_k(x) hat, P(y = k | x) hat

% residual and margin
\newcommand{\eps}{\epsilon}                                                 % residual, stochastic
\newcommand{\epsi}{\epsilon^{(i)}}                                          % epsilon^i, residual, stochastic
\newcommand{\epsh}{\hat{\epsilon}}                                          % residual, estimated
\newcommand{\yf}{y \fx}                                                     % y f(x), margin
\newcommand{\yfi}{\yi \fxi}                                                 % y^i f(x^i), margin
\newcommand{\Sigmah}{\hat \Sigma}											% estimated covariance matrix
\newcommand{\Sigmahj}{\hat \Sigma_j}										% estimated covariance matrix for the j-th class

% ml - loss, risk, likelihood
\newcommand{\Lxy}{L(y, f(x))}                                               % L(y, f(x)), loss function
\newcommand{\Lxyi}{L(\yi, \fxi)}                                            % L(y^i, f(x^i))
\newcommand{\Lxyt}{L(y, \fxt)}                                              % L(y, f(x | theta))
\newcommand{\Lxyit}{L(\yi, \fxit)}                                          % L(y^i, f(x^i | theta)
\newcommand{\risk}{\mathcal{R}}                                             % R, risk
\newcommand{\riskf}{\risk(f)}                                               % R(f), risk
\newcommand{\riske}{\mathcal{R}_{\text{emp}}}                               % R_emp, empirical risk
\newcommand{\riskef}{\riske(f)}                                             % R_emp(f)
\newcommand{\risket}{\mathcal{R}_{\text{emp}}(\theta)}                      % R_emp(theta)
\newcommand{\riskr}{\mathcal{R}_{\text{reg}}}                               % R_reg, regularized risk
\newcommand{\riskrt}{\mathcal{R}_{\text{reg}}(\theta)}                      % R_reg(theta)
\newcommand{\riskrf}{\riskr(f)}                                             % R_reg(f)
\newcommand{\LL}{\mathcal{L}}                                               % L, likelihood
\newcommand{\LLt}{\mathcal{L}(\theta)}                                      % L(theta), likelihood
\renewcommand{\ll}{\ell}                                                    % l, log-likelihood
\newcommand{\llt}{\ell(\theta)}                                             % l(theta), log-likelihood
\newcommand{\LS}{\mathfrak{L}}                                              % ????????????
\newcommand{\TS}{\mathfrak{T}}                                              % ??????????????
\newcommand{\errtrain}{\text{err}_{\text{train}}}                           % training error
\newcommand{\errtest}{\text{err}_{\text{test}}}                             % training error
\newcommand{\errexp}{\overline{\text{err}_{\text{test}}}}                   % training error

% resampling
\newcommand{\GE}[1]{GE(\fh_{#1})}                                           % Generalization error GE
\newcommand{\GEh}[1]{\widehat{GE}_{#1}}                                     % Estimated train error
\newcommand{\GED}{\GE{\D}}                                                  % Generalization error GE
\newcommand{\EGEn}{EGE_n}                                                   % Generalization error GE
\newcommand{\EDn}{\E_{|D| = n}}                                             % Generalization error GE


% ml - irace
\newcommand{\costs}{\mathcal{C}} % costs
\newcommand{\Celite}{\theta^*} % elite configurations
\newcommand{\instances}{\mathcal{I}} % sequence of instances
\newcommand{\budget}{\mathcal{B}} % computational budget

\newcommand{\sens}{\mathbf{A}} % vector x (bold)
\newcommand{\ba}{\mathbf{a}}
\newcommand{\batilde}{\tilde{\mathbf{a}}}
\newcommand{\Px}{\mathbb{P}_{x}} % P_x
\newcommand{\Pxj}{\mathbb{P}_{x_j}} % P_{x_j}
\newcommand{\indep}{\perp \!\!\! \perp} % independence symbol
% ml - ROC
\newcommand{\np}{n_{+}} % no. of positive instances
\newcommand{\nn}{n_{-}} % no. of negative instances
\newcommand{\rn}{\pi_{-}} % proportion negative instances
\newcommand{\rp}{\pi_{+}} % proportion negative instances
% true/false pos/neg:
\newcommand{\tp}{\# \text{TP}} % true pos
\newcommand{\fap}{\# \text{FP}} % false pos (fp taken for partial derivs)
\newcommand{\tn}{\# \text{TN}} % true neg
\newcommand{\fan}{\# \text{FN}} % false neg

\newcommand{\Tspace}{\mathcal{T}}
\newcommand{\tv}{\mathbf{t}}
\newcommand{\tj}{\mathbf{t}_j}

\usepackage{multicol}
\usepackage{color,colortbl} 
\definecolor{putblue}{RGB}{0,0,124}
\definecolor{putred}{RGB}{204,33,69}

\newcommand{\titlefigure}{figure/fmeasure}
\newcommand{\learninggoals}{
  \item Get to know loss functions for multi-target prediction problems
  \item Know the Bayes predictor for Hamming loss and subset $0/1$ loss
  \item Understand the difference between macro-, micro-, and instance-wise-losses 
}

\title{Advanced Machine Learning}
\date{}

\begin{document}

\lecturechapter{Loss Functions for Multi-Target Prediction}
\lecture{Advanced Machine Learning}



\sloppy


\begin{frame}
	\frametitle{Multivariate Loss Functions}
	\begin{itemize}
%		
		\small
%		
		\item In multi-target prediction we want to the following: For a feature vector $\xv$, predict a vector of scores $\yv = (y_1, y_2, \ldots, y_m)^\top$ by means of a function (hypothesis) $f$:
		$$
		\xv = (x_1,x_2,\ldots,x_p)^\top \quad \xrightarrow{~~\fx~~} \quad \yh = ( \hat{y}_1, \hat{y}_2, \ldots, \hat{y}_m)^\top
		$$
%		
		\item If we want to follow the machine learning paradigm based on loss minimization, we need a \emph{multivariate loss functions} 
		$$
		\ell: \, \Yspace^m \times \Yspace^m \rightarrow \mathbb{R}.
		$$  
		Compared to single-target prediction,  a broad spectrum of such multivariate loss functions	is conceivable. 
%		
		\item In case we have an appropriate multivariate loss function $\ell$, we want to find a (Bayes) predictor $\fbayes$ that minimizes expected loss with regard to $\ell:$
%		
		\begin{eqnarray*}
			\fbayes &=& \argmin_{f: \Xspace \to \Yspace^m} \risk_\ell \left(f\right) = \argmin_{f: \Xspace \to \Yspace^m}\Exy\left[ \ell(y,\fx) \right]\\ &=&  \argmin_{f: \Xspace \to \Yspace^m }\int \ell(y,\fx) \text{d}\Pxy. 
		\end{eqnarray*}
%				
%		\item Can we achieve this goal through simple reduction, i.e., by training one model for each target independently? Or can we do better with more sophisticated methods?  
%		
%		\item There are \emph{two views:} the individual target and joint target view.
%
	\end{itemize}
	
\end{frame}




\begin{frame}
	\frametitle{Examples of MTP loss functions}
	\begin{itemize}
		
		\small 
		\item \emph{Squared error loss} (typically used in multivariate regression):
		$$
		\ell(\yv, \yh) = \sum_{j=1}^m (y_j - \hat{y}_j)^2 \, ,
		$$
		where $\yv , \yh \in \mathbb{R}^m$.
		
		
		\medskip
		
		
		%		
		%		\item \emph{F-measure}:
		%		$$
		%		F(\mathbf{Y}, \hat{\mathbf{Y}}) = \frac{2 \sum_{i=1}^K y_i \hat{y}_i}{\sum_{i=1}^K y_i + \sum_{i=1}^K \hat{y}_i} \, ,
		%		$$
		%		where $\mathbf{Y} = (y_1, \ldots , y_K) \in \{ 0,1 \}^K$ and $\hat{\mathbf{Y}} = (\hat{y}_1, \ldots , \hat{y}_K) \in \{ 0,1 \}^K$. Can be used in macro- and micro-averaging, but also instance-wise. 
		%		
		
		
		\item The \emph{Hamming loss} averages over mistakes on individual scores:    
		$$
		\displaystyle \ell_H(\yv, \hat{y}) = \frac{1}{m}  \, \sum_{j=1}^m \, \mathds{1}_{[y_j \neq  \hat y_j ]}
		$$
		
		\item The \emph{subset 0/1 loss} simply checks for entire correctness:  
		\begin{align*}
			\displaystyle \ell_{0/1}(\yv, \hat{y}) & = \mathds{1}_{[ \yv \ne \hat{y} ]}  =  \max_j \, \mathds{1}_{[y_j \neq  \hat y_j]}
		\end{align*}
		
		
	\end{itemize}
\end{frame}



\begin{frame}
	\frametitle{Hamming vs.\ subset 0/1 loss}
	\begin{itemize}
		\item The risk minimizer for the Hamming loss is the  \emph{marginal mode}:
		$$
		f_j^*(\xv) = \arg \max_{y_j \in \{0,1\}} \Pr( y_j  ~|~ \xv )\,, \quad j = 1,\ldots,m ,
		$$
		%\item 
		while for the subset 0/1 loss it is the \emph{joint mode}:
		$$
		\mathbf{f}^*(\xv) = \arg \max_{\yv \in \mathcal{Y}^m} \Pr(\yv ~|~ \xv) \,.
		$$
		\item Marginal mode vs. joint mode:\\[6pt]
		\begin{center}
			\begin{tabular}{@{}cc@{}}
				\toprule
				$\yv$ & $\Pr(\yv)$ \\
				\hline
				$0~0~0~0$ & $0.30$ \\
				$0~1~1~1$ & $0.17$ \\
				$1~0~1~1$ & $0.18$ \\
				$1~1~0~1$ & $0.17$ \\
				$1~1~1~0$ & $0.18$ \\
				\toprule
			\end{tabular}
			$\qquad$
			\footnotesize{
				\begin{tabular}{lr}
					Marginal mode: & $1~1~1~1$ \\
					Joint mode: & $0~0~0~0$ \\
				\end{tabular}
			}
		\end{center}
	\end{itemize}
\end{frame}



%\begin{frame}
%	\frametitle{The individual target view}
%	
%	\begin{itemize} 
%		\item How can we improve the predictive accuracy of a single label \yv exploiting information about other labels?
%		\item Goal: predict a value of $y_i$ using $\xv$ and any available information on other targets $y_j$.
%		\item The problem is usually defined through univariate losses $\ell_i(y_i, \hat{y}_i)$.
%		%\item The problem is usually decomposable over the targets.
%		\item  Domain of $y_i$ is either continuous or nominal.
%		\item  Independent models vs.\ regularized (shrunken) models.
%%		\item James-Stein paradox (to be discussed later).
%		
%	\end{itemize}
%	
%\end{frame}
%
%
%
%\begin{frame}
%	\frametitle{The joint target view}
%	
%	\begin{itemize}
%		\item The problem is defined through multivariate losses $\ell(\yv, \yh)$.
%		%\item What are examples of such loss functions ?
%		\item Is reduction to single-target prediction (decomposition over targets) still possible, and even if so, can we improve over such strategies \yv using more expressive models?
%		%\item What are the relations between the losses?
%		%\item Goal: predict a vector $\yv$ using $\xv$.
%		\item Important: \emph{Structure of loss} $\ell(\cdot, \cdot)$, possible \emph{dependencies between targets}, multivariate distribution of $\yv$.
%		
%		%\item The problem might not be easily decomposable over the targets.
%		%\item Domain of $\yv$ is usually finite, but contains a large number of elements.
%		%\item Independent models vs.\ more expressive models.
%		%\item Loss function perspective.
%		
%	\end{itemize}
%\end{frame}


\begin{frame}
	\frametitle{Multivariate loss functions}
	\small
	\begin{itemize}
%		
%		\item 
%
		\item A loss $L$ (on test data) is decomposable over examples if it can be written in the form
		$$
		L = \sum_{i=1}^n \ell(\yv^{(i)}, f(\xi)) 
		%\quad \text{vs.} \quad L \ne \sum_{i=1}^n \ell(\yv^{(i)}, f(\xi)) 
		\, ,
		$$
		i.e., as a sum of losses over all (test) examples. 
		
		
		\item A multivariate loss $\ell$ is decomposable over targets if it can be written as
		$$
		\ell(\yv, \fx) = \sum_{j=1}^m \ell_j(y_j, f_j(\xv)) 
		%\quad \text{vs.} \quad \ell(\yv, \fx) \ne \sum_{i=1}^m \ell(y_i, h_i(\xv))
		$$
		with suitable single-target losses $\ell_j$. 
		
%		

		\begin{minipage}{0.45\textwidth}
%			
			\item  In general, we distinguish between three categories of losses: macro-, micro-, and instance-wise-losses. 
%			
		\end{minipage}
%	
		\begin{minipage}{0.45\textwidth}
			\begin{center}
				\includegraphics[width=0.9\textwidth]{figure/fmeasure}
			\end{center}
		\end{minipage}
%		
	\end{itemize}
%
\end{frame}






\begin{frame}
	\frametitle{Macro- and micro-Losses}
	
	\begin{itemize}
		\item<1-> Macro-losses: The overall loss corresponds to aggregating the losses over the targets.
		
		\begin{center}
			\begin{tabular}{|c|c|c|c|}
				\multicolumn{4}{c}{True scores} \\
				\hline
				{\only<2>{\color{putred}}$y_{11}$} & {\only<3>{\color{putred}}$y_{12}$} & {\only<4>{\color{putred}}$y_{13}$} & {\only<5>{\color{putred}}$y_{14}$} \\
				{\only<2>{\color{putred}}$y_{21}$} & {\only<3>{\color{putred}}$y_{22}$} & {\only<4>{\color{putred}}$y_{23}$} & {\only<5>{\color{putred}}$y_{24}$} \\
				{\only<2>{\color{putred}}$y_{31}$} & {\only<3>{\color{putred}}$y_{32}$} & {\only<4>{\color{putred}}$y_{33}$} & {\only<5>{\color{putred}}$y_{34}$} \\
				{\only<2>{\color{putred}}$y_{41}$} & {\only<3>{\color{putred}}$y_{42}$} & {\only<4>{\color{putred}}$y_{43}$} & {\only<5>{\color{putred}}$y_{44}$} \\
				{\only<2>{\color{putred}}$y_{51}$} & {\only<3>{\color{putred}}$y_{52}$} & {\only<4>{\color{putred}}$y_{53}$} & {\only<5>{\color{putred}}$y_{54}$} \\
				{\only<2>{\color{putred}}$y_{61}$} & {\only<3>{\color{putred}}$y_{62}$} & {\only<4>{\color{putred}}$y_{63}$} & {\only<5>{\color{putred}}$y_{64}$} \\
				\hline
			\end{tabular}
			$\quad$
			\begin{tabular}{|c|c|c|c|}
				\multicolumn{4}{c}{Predicted scores} \\
				\hline
				{\only<2>{\color{putred}}$\hat{y}_{11}$} & {\only<3>{\color{putred}}$\hat{y}_{12}$} & {\only<4>{\color{putred}}$\hat{y}_{13}$} & {\only<5>{\color{putred}}$\hat{y}_{14}$} \\
				{\only<2>{\color{putred}}$\hat{y}_{21}$} & {\only<3>{\color{putred}}$\hat{y}_{22}$} & {\only<4>{\color{putred}}$\hat{y}_{23}$} & {\only<5>{\color{putred}}$\hat{y}_{24}$} \\
				{\only<2>{\color{putred}}$\hat{y}_{31}$} & {\only<3>{\color{putred}}$\hat{y}_{32}$} & {\only<4>{\color{putred}}$\hat{y}_{33}$} & {\only<5>{\color{putred}}$\hat{y}_{34}$} \\
				{\only<2>{\color{putred}}$\hat{y}_{41}$} & {\only<3>{\color{putred}}$\hat{y}_{42}$} & {\only<4>{\color{putred}}$\hat{y}_{43}$} & {\only<5>{\color{putred}}$\hat{y}_{44}$} \\
				{\only<2>{\color{putred}}$\hat{y}_{51}$} & {\only<3>{\color{putred}}$\hat{y}_{52}$} & {\only<4>{\color{putred}}$\hat{y}_{53}$} & {\only<5>{\color{putred}}$\hat{y}_{54}$} \\
				{\only<2>{\color{putred}}$\hat{y}_{61}$} & {\only<3>{\color{putred}}$\hat{y}_{62}$} & {\only<4>{\color{putred}}$\hat{y}_{63}$} & {\only<5>{\color{putred}}$\hat{y}_{64}$} \\
				\hline
			\end{tabular}
		\end{center}
	\lz
	\item Example: Averaging the target losses.
	$$
	L = \frac{1}{4} \left( {\only<2>{\color{putred}}L_1} + {\only<3>{\color{putred}}L_2} + {\only<4>{\color{putred}}L_3} + {\only<5>{\color{putred}}L_4} \right)
	$$
	\end{itemize}
	

	
\end{frame}




\begin{frame}
	\frametitle{Macro- and micro-Losses}
	
	\begin{itemize}
		\item<1-> Micro-losses: The overall loss corresponds to aggregating the pointwise losses over the targets and the instances.
		
		\begin{center}
			\begin{tabular}{|c|c|c|c|}
				\multicolumn{4}{c}{True scores} \\
				\hline
				\color{putred}$y_{11}$ & \color{putred}$y_{12}$ & \color{putred}$y_{13}$ & \color{putred}$y_{14}$ \\
				\color{putred}$y_{21}$ & \color{putred}$y_{22}$ & \color{putred}$y_{23}$ & \color{putred}$y_{24}$ \\
				\color{putred}$y_{31}$ & \color{putred}$y_{32}$ & \color{putred}$y_{33}$ & \color{putred}$y_{34}$ \\
				\color{putred}$y_{41}$ & \color{putred}$y_{42}$ & \color{putred}$y_{43}$ & \color{putred}$y_{44}$ \\
				\color{putred}$y_{51}$ & \color{putred}$y_{52}$ & \color{putred}$y_{53}$ & \color{putred}$y_{54}$ \\
				\color{putred}$y_{61}$ & \color{putred}$y_{62}$ & \color{putred}$y_{63}$ & \color{putred}$y_{64}$ \\
				\hline
			\end{tabular}
			$\quad$
			\begin{tabular}{|c|c|c|c|}
				\multicolumn{4}{c}{Predicted scores} \\
				\hline
				\color{putred}$\hat{y}_{11}$ & \color{putred}$\hat{y}_{12}$ & \color{putred}$\hat{y}_{13}$ & \color{putred}$\hat{y}_{14}$ \\
				\color{putred}$\hat{y}_{21}$ & \color{putred}$\hat{y}_{22}$ & \color{putred}$\hat{y}_{23}$ & \color{putred}$\hat{y}_{24}$ \\
				\color{putred}$\hat{y}_{31}$ & \color{putred}$\hat{y}_{32}$ & \color{putred}$\hat{y}_{33}$ & \color{putred}$\hat{y}_{34}$ \\
				\color{putred}$\hat{y}_{41}$ & \color{putred}$\hat{y}_{42}$ & \color{putred}$\hat{y}_{43}$ & \color{putred}$\hat{y}_{44}$ \\
				\color{putred}$\hat{y}_{51}$ & \color{putred}$\hat{y}_{52}$ & \color{putred}$\hat{y}_{53}$ & \color{putred}$\hat{y}_{54}$ \\
				\color{putred}$\hat{y}_{61}$ & \color{putred}$\hat{y}_{62}$ & \color{putred}$\hat{y}_{63}$ & \color{putred}$\hat{y}_{64}$ \\
				\hline
			\end{tabular}
		\end{center}
	\lz
	\item Thus, we have	
	$$
	L =  \sum_{i,j} \ell(y_{ij} , \hat{y}_{ij}),
	$$
	where $\ell: \Yspace \times \Yspace \to \R$ in this case.
%	
	\end{itemize}

\end{frame}


\begin{frame}
	\frametitle{Macro- and micro-Losses}
	
	\begin{itemize}
		\item<1-> Micro-losses: The overall loss corresponds to averaging the pointwise losses over the targets and the instances.
		
		\begin{center}
			\begin{tabular}{|c|c|c|c|}
				\multicolumn{4}{c}{True scores} \\
				\hline
				\color{putred}$y_{11}$ & \color{putred}$y_{12}$ &   & \color{putred}$y_{14}$ \\
				\color{putred}$y_{21}$ &   & \color{putred}$y_{23}$ & \color{putred}$y_{24}$ \\
				\color{putred}$y_{31}$ & \color{putred}$y_{32}$ & \color{putred}$y_{33}$ & \color{putred}$y_{34}$ \\
				\color{putred}$y_{41}$ &   & \color{putred}$y_{43}$ & \color{putred}$y_{44}$ \\
				\color{putred}$y_{51}$ & \color{putred}$y_{52}$ & \color{putred}$y_{53}$ & \color{putred}$y_{54}$ \\
				& \color{putred}$y_{62}$ & \color{putred}$y_{63}$ &   \\
				\hline
			\end{tabular}
			$\quad$
			\begin{tabular}{|c|c|c|c|}
				\multicolumn{4}{c}{Predicted scores} \\
				\hline
				\color{putred}$\hat{y}_{11}$ & \color{putred}$\hat{y}_{12}$ &   & \color{putred}$\hat{y}_{14}$ \\
				\color{putred}$\hat{y}_{21}$ &   & \color{putred}$\hat{y}_{23}$ & \color{putred}$\hat{y}_{24}$ \\
				\color{putred}$\hat{y}_{31}$ & \color{putred}$\hat{y}_{32}$ & \color{putred}$\hat{y}_{33}$ & \color{putred}$\hat{y}_{34}$ \\
				\color{putred}$\hat{y}_{41}$ &   & \color{putred}$\hat{y}_{43}$ & \color{putred}$\hat{y}_{44}$ \\
				\color{putred}$\hat{y}_{51}$ & \color{putred}$\hat{y}_{52}$ & \color{putred}$\hat{y}_{53}$ & \color{putred}$\hat{y}_{54}$ \\
				& \color{putred}$\hat{y}_{62}$ & \color{putred}$\hat{y}_{63}$ &   \\
				\hline
			\end{tabular}
		\end{center}	
	\lz
	\item Thus, we have	
		$$
	L =  \sum_{i,j} \ell(y_{ij} , \hat{y}_{ij}),
	$$
	where $\ell: \Yspace \times \Yspace \to \R$ in this case.
%	
	\item 	
		Can be used also for cases with missing entries.
	\end{itemize}
 

	
\end{frame}




\begin{frame}
	\frametitle{Instance-wise losses}
	
	\begin{itemize}
		\item<1-> Instance-wise losses: Aggregating the losses over the instances.
		
		\begin{center}
			\begin{tabular}{|c|c|c|c|}
				\multicolumn{4}{c}{True scores} \\
				\hline
				{\only<2>{\color{putred}}$y_{11}$} & {\only<2>{\color{putred}}$y_{12}$} & {\only<2>{\color{putred}}$y_{13}$} & {\only<2>{\color{putred}}$y_{14}$} \\
				{\only<3>{\color{putred}}$y_{21}$} & {\only<3>{\color{putred}}$y_{22}$} & {\only<3>{\color{putred}}$y_{23}$} & {\only<3>{\color{putred}}$y_{24}$} \\
				{\only<4>{\color{putred}}$y_{31}$} & {\only<4>{\color{putred}}$y_{32}$} & {\only<4>{\color{putred}}$y_{33}$} & {\only<4>{\color{putred}}$y_{34}$} \\
				{\only<5>{\color{putred}}$y_{41}$} & {\only<5>{\color{putred}}$y_{42}$} & {\only<5>{\color{putred}}$y_{43}$} & {\only<5>{\color{putred}}$y_{44}$} \\
				{\only<6>{\color{putred}}$y_{51}$} & {\only<6>{\color{putred}}$y_{52}$} & {\only<6>{\color{putred}}$y_{53}$} & {\only<6>{\color{putred}}$y_{54}$} \\
				{\only<7>{\color{putred}}$y_{61}$} & {\only<7>{\color{putred}}$y_{62}$} & {\only<7>{\color{putred}}$y_{63}$} & {\only<7>{\color{putred}}$y_{64}$} \\
				\hline
			\end{tabular}
			$\quad$
			\begin{tabular}{|c|c|c|c|}
				\multicolumn{4}{c}{Predicted scores} \\
				\hline
				{\only<2>{\color{putred}}$\hat{y}_{11}$} & {\only<2>{\color{putred}}$\hat{y}_{12}$} & {\only<2>{\color{putred}}$\hat{y}_{13}$} & {\only<2>{\color{putred}}$\hat{y}_{14}$} \\
				{\only<3>{\color{putred}}$\hat{y}_{21}$} & {\only<3>{\color{putred}}$\hat{y}_{22}$} & {\only<3>{\color{putred}}$\hat{y}_{23}$} & {\only<3>{\color{putred}}$\hat{y}_{24}$} \\
				{\only<4>{\color{putred}}$\hat{y}_{31}$} & {\only<4>{\color{putred}}$\hat{y}_{32}$} & {\only<4>{\color{putred}}$\hat{y}_{33}$} & {\only<4>{\color{putred}}$\hat{y}_{34}$} \\
				{\only<5>{\color{putred}}$\hat{y}_{41}$} & {\only<5>{\color{putred}}$\hat{y}_{42}$} & {\only<5>{\color{putred}}$\hat{y}_{43}$} & {\only<5>{\color{putred}}$\hat{y}_{44}$} \\
				{\only<6>{\color{putred}}$\hat{y}_{51}$} & {\only<6>{\color{putred}}$\hat{y}_{52}$} & {\only<6>{\color{putred}}$\hat{y}_{53}$} & {\only<6>{\color{putred}}$\hat{y}_{54}$} \\
				{\only<7>{\color{putred}}$\hat{y}_{61}$} & {\only<7>{\color{putred}}$\hat{y}_{62}$} & {\only<7>{\color{putred}}$\hat{y}_{63}$} & {\only<7>{\color{putred}}$\hat{y}_{64}$} \\
				\hline
			\end{tabular}
		\end{center}
		\lz
%		
		\item Example: Averaging over the instance-losses.
%		
		\begin{align*}
%			
			L = \frac{1}{6} \Big( {\only<2>{\color{putred}}\ell(\yv^{(1)},\hat{y}^{(1)})}  & + 
			{\only<3>{\color{putred}}\ell(\yv^{(2)},\hat{y}^{(2)})} +
			{\only<4>{\color{putred}}\ell(\yv^{(3)},\hat{y}^{(3)})} + \\
			& {\only<5>{\color{putred}}\ell(\yv^{(4)},\hat{y}^{(4)})} +
			{\only<6>{\color{putred}}\ell(\yv^{(5)},\hat{y}^{(5)})} +
			{\only<7>{\color{putred}}\ell(\yv^{(6)},\hat{y}^{(6)})} 
			\Big)
%			
		\end{align*}
%
	\end{itemize}
	

	
\end{frame}











%
\endlecture
\end{document}
