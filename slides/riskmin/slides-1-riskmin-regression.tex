%<<setup-child, include = FALSE>>=
%library(knitr)
%library(qrmix)
%library(mlr)
%library(quantreg)
%library(reshape2)
%set_parent("../style/preamble.Rnw")
%@

\usepackage[]{graphicx}
\usepackage[]{color}
% maxwidth is the original width if it is less than linewidth
% otherwise use linewidth (to make sure the graphics do not exceed the margin)
\makeatletter
\def\maxwidth{ %
  \ifdim\Gin@nat@width>\linewidth
    \linewidth
  \else
    \Gin@nat@width
  \fi
}
\makeatother

% ---------------------------------%
% latex-math dependencies, do not remove:
% - \usepackage{mathtools}
% - \usepackage{bm}
% - \usepackage{siunitx}
% - \usepackage{dsfont}
% - \usepackage{xspace}
% ---------------------------------%

%--------------------------------------------------------%
%       Language, encoding, typography
%--------------------------------------------------------%

\usepackage[english]{babel}
\usepackage[utf8]{inputenc} % Enables inputting UTF-8 symbols
% Standard AMS suite
\usepackage{amsmath,amsfonts,amssymb}

% Font four double-stroke / blackboard letters for sets of numbers (N, R, ...)
% Distribution name is "doublestroke"
% According to https://mirror.physik.tu-berlin.de/pub/CTAN/fonts/doublestroke/dsdoc.pdf
% the "bbm" package does a similar thing and may be superfluous.
% Required for latex-math
\usepackage{dsfont}

% bbm – "Blackboard-style" cm fonts (https://www.ctan.org/pkg/bbm)
% Used to be in common.tex, loaded directly after this file
% Maybe superfluous given dsfont is loaded
% TODO: Check if really unused?
% \usepackage{bbm}

% bm – Access bold symbols in maths mode - https://ctan.org/pkg/bm
% Required for latex-math
% https://tex.stackexchange.com/questions/3238/bm-package-versus-boldsymbol
\usepackage{bm}

% pifont – Access to PostScript standard Symbol and Dingbats fonts
% Used for \newcommand{\xmark}{\ding{55}, which is never used
% aside from lecture_advml/attic/xx-automl/slides.Rnw
% \usepackage{pifont}

% Quotes (inline and display), provdes \enquote
% https://ctan.org/pkg/csquotes
\usepackage{csquotes}

% Adds arg to enumerate env, technically superseded by enumitem according
% to https://ctan.org/pkg/enumerate
% Replace with https://ctan.org/pkg/enumitem ?
\usepackage{enumerate}

% Line spacing - provides \singlespacing \doublespacing \onehalfspacing
% https://ctan.org/pkg/setspace
% TODO: Check if really unused?
%\usepackage{setspace}

% mathtools – Mathematical tools to use with amsmath
% https://ctan.org/pkg/mathtools?lang=en
% latex-math dependency according to latex-math repo
\usepackage{mathtools}

%--------------------------------------------------------%
%       Displaying code and algorithms
%--------------------------------------------------------%
\usepackage{verbatim}
\usepackage{algorithm}
\usepackage{algpseudocode}

%--------------------------------------------------------%
%       Tables
%--------------------------------------------------------%

% multi-row table cells: https://www.namsu.de/Extra/pakete/Multirow.html
\usepackage{multirow}

% long/multi-page tables: https://texdoc.org/serve/longtable.pdf/0
% TODO: Check if really unused?

\usepackage{longtable}

% pretty table env: https://ctan.org/pkg/booktabs?lang=en
% TODO: Check if really unused?
\usepackage{booktabs}

%--------------------------------------------------------%
%       Figures: Creating, placing, verbing
%--------------------------------------------------------%

% wrapfig - Wrapping text around figures https://de.overleaf.com/learn/latex/Wrapping_text_around_figures
\usepackage{wrapfig}

% Sub figures in figures and tables
% https://ctan.org/pkg/subfig -- supersedes subfigure package
% TODO: Check if really unused?
\usepackage{subfig}

% Actually it's pronounced PGF https://en.wikibooks.org/wiki/LaTeX/PGF/TikZ
\usepackage{tikz}

\usetikzlibrary{shapes,arrows,automata,positioning,calc,chains,trees, shadows}
\tikzset{
  %Define standard arrow tip
  >=stealth',
  %Define style for boxes
  punkt/.style={
    rectangle,
    rounded corners,
    draw=black, very thick,
    text width=6.5em,
    minimum height=2em,
    text centered},
  % Define arrow style
  pil/.style={
    ->,
    thick,
    shorten <=2pt,
    shorten >=2pt,}
}


% Unsorted
% textpos – Place boxes at arbitrary positions on the LATEX page
% https://ctan.org/pkg/textpos?lang=en
% Provides \begin{textblock}
 % TODO: Check if really unused?
\usepackage[absolute,overlay]{textpos}

% psfrag – Replace strings in encapsulated PostScript figures
% https://www.overleaf.com/latex/examples/psfrag-example/tggxhgzwrzhn
% https://ftp.mpi-inf.mpg.de/pub/tex/mirror/ftp.dante.de/pub/tex/macros/latex/contrib/psfrag/pfgguide.pdf
% Can't tell if this is needed
% TODO: Check if really unused?
\usepackage{psfrag}

% Maybe not great to use this https://tex.stackexchange.com/a/197/19093
% Use align instead -- TODO: Global search & replace to check
\usepackage{eqnarray}

\usepackage{colortbl}

% arydshln – Draw dash-lines in array/tabular
% https://www.ctan.org/pkg/arydshln
% !! "arydshln has to be loaded after array, longtable, colortab and/or colortbl"
% Provides \hdashline and \cdashline
% TODO: Check if really unused?
% \usepackage{arydshln}

% tabularx – Tabulars with adjustable-width columns
% https://ctan.org/pkg/tabularx
% Provides \begin{tabularx}
% TODO: Check if really unused?
% \usepackage{tabularx}

% placeins – Control float placement
% https://ctan.org/pkg/placeins
% Defines a \FloatBarrier command
% TODO: Check if really unused?
% \usepackage{placeins}


% framed – Framed or shaded regions that can break across pages
% https://ctan.org/pkg/framed
% Provides \begin{framed} which uses \colorbox{shadecolor} relying on \definecolor{shadecolor}.
% TODO: Check if really unused?
% \usepackage{framed}

% Used often in conjunction with \definecolor{shadecolor}{rgb}{0.969, 0.969, 0.969}
% Might be able to be removed or at least redefined to only have shadecolor (if needed)
\definecolor{fgcolor}{rgb}{0.345, 0.345, 0.345}
\definecolor{shadecolor}{rgb}{0.969, 0.969, 0.969}
\newenvironment{knitrout}{}{} % an empty environment to be redefined in TeX


% Defines macros and environments
\usepackage{../../style/lmu-lecture}

\let\code=\texttt % Used regularly
\let\proglang=\textsf % Unused?

% Not sure what/why this does
\setkeys{Gin}{width=0.9\textwidth}

\setbeamertemplate{frametitle}{\expandafter\uppercase\expandafter\insertframetitle}

% Can't find a reason why common.tex is not just part of this file?

% basic latex stuff
\newcommand{\pkg}[1]{{\fontseries{b}\selectfont #1}} %fontstyle for R packages
\newcommand{\lz}{\vspace{0.5cm}} %vertical space
\newcommand{\dlz}{\vspace{1cm}} %double vertical space
\newcommand{\oneliner}[1] % Oneliner for important statements
{\begin{block}{}\begin{center}\begin{Large}#1\end{Large}\end{center}\end{block}}


%new environments
\newenvironment{vbframe}  %frame with breaks and verbatim
{
 \begin{frame}[containsverbatim,allowframebreaks]
}
{
\end{frame}
}

\newenvironment{vframe}  %frame with verbatim without breaks (to avoid numbering one slided frames)
{
 \begin{frame}[containsverbatim]
}
{
\end{frame}
}

\newenvironment{blocki}[1]   % itemize block
{
 \begin{block}{#1}\begin{itemize}
}
{
\end{itemize}\end{block}
}

\newenvironment{fragileframe}[2]{  %fragile frame with framebreaks
\begin{frame}[allowframebreaks, fragile, environment = fragileframe]
\frametitle{#1}
#2}
{\end{frame}}


\newcommand{\myframe}[2]{  %short for frame with framebreaks
\begin{frame}[allowframebreaks]
\frametitle{#1}
#2
\end{frame}}

\newcommand{\remark}[1]{
  \textbf{Remark:} #1
}


\newenvironment{deleteframe}
{
\begingroup
\usebackgroundtemplate{\includegraphics[width=\paperwidth,height=\paperheight]{../style/color/red.png}}
 \begin{frame}
}
{
\end{frame}
\endgroup
}
\newenvironment{simplifyframe}
{
\begingroup
\usebackgroundtemplate{\includegraphics[width=\paperwidth,height=\paperheight]{../style/color/yellow.png}}
 \begin{frame}
}
{
\end{frame}
\endgroup
}\newenvironment{draftframe}
{
\begingroup
\usebackgroundtemplate{\includegraphics[width=\paperwidth,height=\paperheight]{../style/color/green.jpg}}
 \begin{frame}
}
{
\end{frame}
\endgroup
}
% https://tex.stackexchange.com/a/261480: textcolor that works in mathmode
\makeatletter
\renewcommand*{\@textcolor}[3]{%
  \protect\leavevmode
  \begingroup
    \color#1{#2}#3%
  \endgroup
}
\makeatother


%-------------------------------------------------------------------------------------------------------%
%  Unused stuff that needs to go but is kept here currently juuuust in case it was important after all  %
%-------------------------------------------------------------------------------------------------------%

% \newcommand{\hlnum}[1]{\textcolor[rgb]{0.686,0.059,0.569}{#1}}%
% \newcommand{\hlstr}[1]{\textcolor[rgb]{0.192,0.494,0.8}{#1}}%
% \newcommand{\hlcom}[1]{\textcolor[rgb]{0.678,0.584,0.686}{\textit{#1}}}%
% \newcommand{\hlopt}[1]{\textcolor[rgb]{0,0,0}{#1}}%
% \newcommand{\hlstd}[1]{\textcolor[rgb]{0.345,0.345,0.345}{#1}}%
% \newcommand{\hlkwa}[1]{\textcolor[rgb]{0.161,0.373,0.58}{\textbf{#1}}}%
% \newcommand{\hlkwb}[1]{\textcolor[rgb]{0.69,0.353,0.396}{#1}}%
% \newcommand{\hlkwc}[1]{\textcolor[rgb]{0.333,0.667,0.333}{#1}}%
% \newcommand{\hlkwd}[1]{\textcolor[rgb]{0.737,0.353,0.396}{\textbf{#1}}}%
% \let\hlipl\hlkwb

% \makeatletter
% \newenvironment{kframe}{%
%  \def\at@end@of@kframe{}%
%  \ifinner\ifhmode%
%   \def\at@end@of@kframe{\end{minipage}}%
%   \begin{minipage}{\columnwidth}%
%  \fi\fi%
%  \def\FrameCommand##1{\hskip\@totalleftmargin \hskip-\fboxsep
%  \colorbox{shadecolor}{##1}\hskip-\fboxsep
%      % There is no \\@totalrightmargin, so:
%      \hskip-\linewidth \hskip-\@totalleftmargin \hskip\columnwidth}%
%  \MakeFramed {\advance\hsize-\width
%    \@totalleftmargin\z@ \linewidth\hsize
%    \@setminipage}}%
%  {\par\unskip\endMakeFramed%
%  \at@end@of@kframe}
% \makeatother

% \definecolor{shadecolor}{rgb}{.97, .97, .97}
% \definecolor{messagecolor}{rgb}{0, 0, 0}
% \definecolor{warningcolor}{rgb}{1, 0, 1}
% \definecolor{errorcolor}{rgb}{1, 0, 0}
% \newenvironment{knitrout}{}{} % an empty environment to be redefined in TeX

% \usepackage{alltt}
% \newcommand{\SweaveOpts}[1]{}  % do not interfere with LaTeX
% \newcommand{\SweaveInput}[1]{} % because they are not real TeX commands
% \newcommand{\Sexpr}[1]{}       % will only be parsed by R
% \newcommand{\xmark}{\ding{55}}%

% math spaces
\newcommand{\N}{\mathds{N}}                                                 % N, naturals
\newcommand{\Z}{\mathds{Z}}                                                 % Z, integers
\newcommand{\Q}{\mathds{Q}}                                                 % Q, rationals
\newcommand{\R}{\mathds{R}}                                                 % R, reals
\newcommand{\C}{\mathds{C}}                                                 % C, complex
\newcommand{\HS}{\mathcal{H}}                                               % H, hilbertspace
\newcommand{\continuous}{\mathcal{C}}                                       % C, space of continuous functions
\newcommand{\M}{\mathcal{M}} 												% machine numbers
\newcommand{\epsm}{\epsilon_m} 												% maximum error


% basic math stuff
\newcommand{\xt}{\tilde x}													% x tilde
\def\argmax{\mathop{\sf arg\,max}}                                          % argmax
\def\argmin{\mathop{\sf arg\,min}}                                          % argmin
\newcommand{\sign}{\operatorname{sign}}                                     % sign, signum
\newcommand{\I}{\mathbb{I}}                                                 % I, indicator
\newcommand{\order}{\mathcal{O}}                                            % O, order
\newcommand{\fp}[2]{\frac{\partial #1}{\partial #2}}                        % partial derivative
\newcommand{\pd}[2]{\frac{\partial{#1}}{\partial #2}}						% partial derivative

% sums and products
\newcommand{\sumin}{\sum_{i=1}^n}											% summation from i=1 to n
\newcommand{\sumkg}{\sum_{k=1}^g}											% summation from k=1 to g
\newcommand{\prodin}{\prod_{i=1}^n}											% product from i=1 to n
\newcommand{\prodkg}{\prod_{k=1}^g}											% product from k=1 to g

% linear algebra
\newcommand{\one}{\boldsymbol{1}}                                           % 1, unitvector
\newcommand{\id}{\mathrm{I}}                                                % I, identity
\newcommand{\diag}{\operatorname{diag}}                                     % diag, diagonal
\newcommand{\trace}{\operatorname{tr}}                                      % tr, trace
\newcommand{\spn}{\operatorname{span}}                                      % span
\newcommand{\scp}[2]{\left\langle #1, #2 \right\rangle}                     % <.,.>, scalarproduct
\newcommand{\mat}[1]{ 														% short pmatrix command
	\begin{pmatrix}
		#1
	\end{pmatrix}
}
\newcommand{\Amat}{\bm{A}}													% matrix A
\newcommand{\xv}{\bm{x}}													% vector x (bold)
\newcommand{\yv}{\bm{y}}														% vector y (bold)
\newcommand{\Deltab}{\bm{\Delta}}											% error term for vectors
															

% basic probability + stats
\renewcommand{\P}{\mathds{P}}                                               % P, probability
\newcommand{\E}{\mathds{E}}                                                 % E, expectation
\newcommand{\var}{\mathsf{Var}}                                             % Var, variance
\newcommand{\cov}{\mathsf{Cov}}                                             % Cov, covariance
\newcommand{\corr}{\mathsf{Corr}}                                           % Corr, correlation
\newcommand{\normal}{\mathcal{N}}                                           % N of the normal distribution
\newcommand{\iid}{\overset{i.i.d}{\sim}}                                    % dist with i.i.d superscript
\newcommand{\distas}[1]{\overset{#1}{\sim}}                                 % ... is distributed as ... 
% machine learning

%%%%%% ml - data
\newcommand{\Xspace}{\mathcal{X}}                                           % X, input space
\newcommand{\Yspace}{\mathcal{Y}}                                           % Y, output space
\newcommand{\nset}{\{1, \ldots, n\}}                                        % set from 1 to n
\newcommand{\pset}{\{1, \ldots, p\}}                                        % set from 1 to p
\newcommand{\gset}{\{1, \ldots, g\}}                                        % set from 1 to g
\newcommand{\Pxy}{\P_{xy}}                                                  % P_xy
\newcommand{\xy}{(x, y)}                                                    % observation (x, y)
\newcommand{\xvec}{(x_1, \ldots, x_p)^T}                                    % (x1, ..., xp) 
\newcommand{\D}{\mathcal{D}}                                                % D, data 
\newcommand{\Dset}{\{ (x^{(1)}, y^{(1)}), \ldots, (x^{(n)},  y^{(n)})\}}    % {(x1,y1)), ..., (xn,yn)}, data
\newcommand{\xdat}{\{ x^{(1)}, \ldots, x^{(n)}\}}   						 % {x1, ..., xn}, input data
\newcommand{\ydat}{\mathbf{y}}                                              % y (bold), vector of outcomes
\newcommand{\yvec}{(y^{(1)}, \hdots, y^{(n)})^T}                            % (y1, ..., yn), vector of outcomes
\renewcommand{\xi}[1][i]{x^{(#1)}}                                          % x^i, i-th observed value of x
\newcommand{\yi}[1][i]{y^{(#1)}}                                            % y^i, i-th observed value of y 
\newcommand{\xyi}{(\xi, \yi)}                                               % (x^i, y^i), i-th observation
\newcommand{\xivec}{(x^{(i)}_1, \ldots, x^{(i)}_p)^T}                       % (x1^i, ..., xp^i), i-th observation vector
\newcommand{\xj}{x_j}                                                       % x_j, j-th feature
\newcommand{\xjb}{\mathbf{x}_j}                                             % x_j (bold), j-th feature vecor
\newcommand{\xjvec}{(x^{(1)}_j, \ldots, x^{(n)}_j)^T}                       % (x^1_j, ..., x^n_j), j-th feature vector
\newcommand{\Dtrain}{\mathcal{D}_{\text{train}}}                            % D_train, training set
\newcommand{\Dtest}{\mathcal{D}_{\text{test}}}                              % D_test, test set

%%%%%% ml - models general

% continuous prediction function f
\newcommand{\fx}{f(x)}                                                      % f(x), continuous prediction function
\newcommand{\Hspace}{H}														% hypothesis space where f is from
\newcommand{\fh}{\hat{f}}                                                   % f hat, estimated prediction function
\newcommand{\fxh}{\fh(x)}                                                   % fhat(x)
\newcommand{\fxt}{f(x | \theta)}                                            % f(x | theta)
\newcommand{\fxi}{f(\xi)}                                                   % f(x^(i))
\newcommand{\fxih}{\hat{f}(\xi)}                                            % f(x^(i))
\newcommand{\fxit}{f(x^{(i)} | \theta)}                                     % f(x^(i) | theta)
\newcommand{\fhD}{\fh_{\D}}                                                 % fhat_D, estimate of f based on D
\newcommand{\fhDtrain}{\fh_{\Dtrain}}                                       % fhat_Dtrain, estimate of f based on D

% discrete prediction function h
\newcommand{\hx}{h(x)}                                                      % h(x), discrete prediction function
\newcommand{\hh}{\hat{h}}                                                   % h hat
\newcommand{\hxh}{\hat{h}(x)}                                               % hhat(x)
\newcommand{\hxt}{h(x | \theta)}                                            % h(x | theta)
\newcommand{\hxi}{h(\xi)}                                                   % h(x^(i))
\newcommand{\hxit}{h(x^{(i)} | \theta)}                                     % h(x^(i) | theta)

% yhat
\newcommand{\yh}{\hat{y}}                                                   % y hat for prediction of target
\newcommand{\yih}{\hat{y}}                                                  % y hat for prediction of target

% theta
\newcommand{\thetah}{\hat{\theta}}                                          % theta hat

% densities + probabilities
% pdf of x 
\newcommand{\pdf}{p}                                                        % p
\newcommand{\pdfx}{p(x)}                                                    % p(x)
\newcommand{\pixt}{\pi(x | \theta)}                                         % pi(x|theta), pdf of x given theta

% pdf of (x, y)
\newcommand{\pdfxy}{p(x,y)}                                                 % p(x, y)
\newcommand{\pdfxyt}{p(x, y | \theta)}                                      % p(x, y | theta)
\newcommand{\pdfxyit}{p(\xi, \yi | \theta)}                                 % p(x^(i), y^(i) | theta)

% pdf of x given y
\newcommand{\pdfxyk}{p(x | y=k)}                                            % p(x | y = k)
\newcommand{\lpdfxyk}{\log \pdfxyk}                                         % log p(x | y = k)
\newcommand{\pdfxiyk}{p(\xi | y=k)}                                         % p(x^i | y = k)

% prior probabilities
\newcommand{\pik}{\pi_k}                                                    % pi_k, prior
\newcommand{\lpik}{\log \pik}                                               % log pi_k, log of the prior

% posterior probabilities
\newcommand{\post}{\P(y = 1 | x)}                                           % P(y = 1 | x), post. prob for y=1
\newcommand{\pix}{\pi(x)}                                                   % pi(x), P(y = 1 | x)
\newcommand{\postk}{\P(y = k | x)}                                          % P(y = k | y), post. prob for y=k
\newcommand{\pikx}{\pi_k(x)}                                                % pi_k(x), P(y = k | x)
\newcommand{\pikxt}{\pi_k(x | \theta)}                                      % pi_k(x | theta), P(y = k | x, theta)
\newcommand{\pijx}{\pi_j(x)}                                                % pi_j(x), P(y = j | x)
\newcommand{\pdfygxt}{p(y |x, \theta)}                                      % p(y | x, theta)
\newcommand{\pdfyigxit}{p(\yi |\xi, \theta)}                                % p(y^i |x^i, theta)
\newcommand{\lpdfygxt}{\log \pdfygxt }                                      % log p(y | x, theta)
\newcommand{\lpdfyigxit}{\log \pdfyigxit}                                   % log p(y^i |x^i, theta)
\newcommand{\pixh}{\hat \pi(x)}                                             % pi(x) hat, P(y = 1 | x) hat
\newcommand{\pikxh}{\hat \pi_k(x)}                                          % pi_k(x) hat, P(y = k | x) hat

% residual and margin
\newcommand{\eps}{\epsilon}                                                 % residual, stochastic
\newcommand{\epsi}{\epsilon^{(i)}}                                          % epsilon^i, residual, stochastic
\newcommand{\epsh}{\hat{\epsilon}}                                          % residual, estimated
\newcommand{\yf}{y \fx}                                                     % y f(x), margin
\newcommand{\yfi}{\yi \fxi}                                                 % y^i f(x^i), margin
\newcommand{\Sigmah}{\hat \Sigma}											% estimated covariance matrix
\newcommand{\Sigmahj}{\hat \Sigma_j}										% estimated covariance matrix for the j-th class

% ml - loss, risk, likelihood
\newcommand{\Lxy}{L(y, f(x))}                                               % L(y, f(x)), loss function
\newcommand{\Lxyi}{L(\yi, \fxi)}                                            % L(y^i, f(x^i))
\newcommand{\Lxyt}{L(y, \fxt)}                                              % L(y, f(x | theta))
\newcommand{\Lxyit}{L(\yi, \fxit)}                                          % L(y^i, f(x^i | theta)
\newcommand{\risk}{\mathcal{R}}                                             % R, risk
\newcommand{\riskf}{\risk(f)}                                               % R(f), risk
\newcommand{\riske}{\mathcal{R}_{\text{emp}}}                               % R_emp, empirical risk
\newcommand{\riskef}{\riske(f)}                                             % R_emp(f)
\newcommand{\risket}{\mathcal{R}_{\text{emp}}(\theta)}                      % R_emp(theta)
\newcommand{\riskr}{\mathcal{R}_{\text{reg}}}                               % R_reg, regularized risk
\newcommand{\riskrt}{\mathcal{R}_{\text{reg}}(\theta)}                      % R_reg(theta)
\newcommand{\riskrf}{\riskr(f)}                                             % R_reg(f)
\newcommand{\LL}{\mathcal{L}}                                               % L, likelihood
\newcommand{\LLt}{\mathcal{L}(\theta)}                                      % L(theta), likelihood
\renewcommand{\ll}{\ell}                                                    % l, log-likelihood
\newcommand{\llt}{\ell(\theta)}                                             % l(theta), log-likelihood
\newcommand{\LS}{\mathfrak{L}}                                              % ????????????
\newcommand{\TS}{\mathfrak{T}}                                              % ??????????????
\newcommand{\errtrain}{\text{err}_{\text{train}}}                           % training error
\newcommand{\errtest}{\text{err}_{\text{test}}}                             % training error
\newcommand{\errexp}{\overline{\text{err}_{\text{test}}}}                   % training error

% resampling
\newcommand{\GE}[1]{GE(\fh_{#1})}                                           % Generalization error GE
\newcommand{\GEh}[1]{\widehat{GE}_{#1}}                                     % Estimated train error
\newcommand{\GED}{\GE{\D}}                                                  % Generalization error GE
\newcommand{\EGEn}{EGE_n}                                                   % Generalization error GE
\newcommand{\EDn}{\E_{|D| = n}}                                             % Generalization error GE


% ml - irace
\newcommand{\costs}{\mathcal{C}} % costs
\newcommand{\Celite}{\theta^*} % elite configurations
\newcommand{\instances}{\mathcal{I}} % sequence of instances
\newcommand{\budget}{\mathcal{B}} % computational budget

\newcommand{\titlefigure}{figure_man/optimization_steps.jpeg}
\newcommand{\learninggoals}{
\item Understand that an ML model is simply a parametrized curve
\item Understand that the hypothesis space lists all admissible models
    for a learner
\item Understand the relationship between the hypothesis space and the parameter space
}

\title{Introduction to Machine Learning}
% \author{Bernd Bischl, Christoph Molnar, Daniel Schalk, Fabian Scheipl}
\institute{\href{https://compstat-lmu.github.io/lecture_i2ml/}{compstat-lmu.github.io/lecture\_i2ml}}
\date{}

\begin{document}



% 1 -- Goes out, in I2ML "Components of a Learner"

\begin{vbframe}{What is Learning?}

\begin{center}
\begin{table}[]
\begin{tabular}{ccccccc}
 Learning & = & Hypothesis space & + & Risk & + & Optimization
 % &  &  \checkmark &  & \textbf{?} & &  \textbf{?} \\
\end{tabular}
\end{table}
\end{center}

\begin{itemize}
  \item The \textbf{hypothesis space} $\Hspace$ is the search space of the learning algorithm. It is a predefined set of functions (also called models) from which the learning algorithm picks one function/model. \\ \vspace{1mm}
  Example: Space of linear models. \vspace{3mm}
  \item The \textbf{risk} is a metric to evaluate and compare the different models in the hypothesis space. \\ \vspace{1mm}
  Example: Sum of squared errors. \vspace{3mm}
  \item The \textbf{optimizer} is the algorithm used to minimize the risk over the hypothesis space. \\ \vspace{1mm}
  Example optimizer: Gradient descent. \vspace{3mm}
\end{itemize}

% \begin{itemize}
%   \item Assume we decided on a certain hypothesis space $\Hspace$ (e. g. the space of linear models) and we are given some data $\D$ we can use for training.

%   % \item How do we get the \enquote{best} model $\hat f$?
%   \item What does \enquote{best} mean? How do we distinguish good models from bad ones? (Cost function)
%   \item And how do we get there? (Optimization)
% \end{itemize}

\end{vbframe}




\section{Loss Functions}


% 3, 4, 5 -- Pointwise losses are in I2ML

\begin{vbframe}{Losses: Measuring Errors Point-wise}

Given the hypothesis space of linear models, which model will be returned by a learning algorithm (under \enquote{perfect} optimization)?

\vspace*{0.2cm}

\begin{center}
\includegraphics[width = 11cm ]{figure_man/example_intro.png} \\
\end{center}


\textbf{Answer:} It depends on the metric we use to compare models.

\end{vbframe}


\begin{vbframe}{Losses: Measuring Errors Point-wise}

\begin{itemize}
  \item Let us assume that there is a probability distribution $\Pxy$ defined on $\Xspace \times \Yspace$ induced by the process that generates the observed data $\D$. 
  \item Further, let $(\xv, y)$ denote the random variables that follow this distribution. 
  \item We consider a model $f \in \Hspace, f: \Xspace \to \R^g$, and want to quantify the \enquote{goodness} of the function. 
  \item Intuitively, a \enquote{good} function outputs values $\fx$ which are close to the targets $y \in \Yspace$
  $$
    y \approx \fx 
  $$
  for $(\xv, y) \sim \Pxy$. 
  \item We quantify the \enquote{goodness} of a model $\fx$ \textbf{point-wise} via a \textbf{loss} function 
    $$
    L: \Yspace \times \R^g \to \R,
    $$

  which compares the prediction and the real target $L(y, \fx)$.
\end{itemize}

\textbf{Example:} $\Lxy = (y - \fx)^2$ (point-wise squared errors)

\vspace*{-4mm}

\begin{center}
  \includegraphics[width = 10cm]{figure_man/loss_quadratic_plot1.png} \\
\end{center}

\end{vbframe}


% 6, 7, 8 - Will go into Chapter Types of Losses, Pseudo-residuals

\begin{vbframe}{Losses, Residuals and Pseudo-Residuals}

\begin{itemize}
\item Regression losses usually only depend on the \textbf{ residuals}

\vspace*{-0.5cm}

\begin{eqnarray*}
  r &:=& y - \fx %\\
  %r^{(i)} &:=& \yi - \fxi .
\end{eqnarray*}


\item A loss is called \textbf{distance-based} if
\begin{itemize}
  \item it can be written in terms of the residual
  $$
    \Lxy = \psi (r) \text{ for some } \psi: \R \to \R
  $$
  \item $\psi(r) = 0 \Leftrightarrow r = 0$ .
\end{itemize}
\item A loss is \textbf{translation-invariant}, if $L(y + a, \fx + a) = \Lxy$.

  % We will see later that in case of the L2-loss, pseudo-residuals correspond to the residuals - hence the name.

\framebreak 


%% DR: I don't think the following is correct. E.g., take the translation-invariant Loss function
%% L(x,y) = 1 + (x-y)^2 = 1 + r^2 = psi(r) with r = x-y. 
%% It holds L(x+a,y+a) = 1+(x+a-y-a)^2 = L(x,y), BUT
%% psi(0) = 1 and thus L is not distance-based.
% \textbf{Note}: A loss is translation-invariant iff it is distance-based:
% 
% \begin{itemize}
%   \item[$\Rightarrow$] If a loss is translation-invariant, then it is also distance-based:
%   \begin{footnotesize}
%   \begin{eqnarray*}
%     \Lxy &=& L(y - y, \fx - y) = L(0, - r) =: \psi(r).
%   \end{eqnarray*}
%   \end{footnotesize}
%   \item [$\Leftarrow$] The residual $r = y - \fx$ is translation-invariant, thus any loss that depends on the residual only is also translation-invariant.
% \end{itemize}

% \begin{itemize}
%   \item[$\Rightarrow$] If a loss is distance-based, then it is also translation-invariant:
% The residual $r = y - \fx$ is translation-invariant, thus any loss that depends on the residual only is also translation-invariant.
%   \begin{footnotesize}
%   \begin{eqnarray*}
%     \Lxy &=& L(y, \fx) = \psi(r) = \psi(y - \fx) = \psi(y - \fx - a + a) = \psi((y-a) - (\fx - a)) = L(y - a, \fx - a)
%   \end{eqnarray*}
%   \end{footnotesize}
%   \item [$\Leftarrow$] The residual $r = y - \fx$ is translation-invariant, thus any loss that depends on the residual only is also translation-invariant.
% \end{itemize}
\framebreak 

\item We further define \textbf{pseudo-residuals} as the negative first derivatives of loss functions w.r.t. $\fx$

  \begin{eqnarray*}
    % \tilde r &:=& - \frac{\partial \Lxy}{\partial \fx} \qquad 
    % \tilde r^{(i)} := - \frac{\partial \Lxyi}{\partial \fxi} 
    \tilde r &:=& - \frac{\partial \Lxy}{\partial \fx}.  % \qquad 
    % \tilde r^{(i)} := - \frac{\partial \Lxyi}{\partial f} 
  \end{eqnarray*}
  % (Note that pseudo-residuals are functions of $y$ and $\fx$)
\item We will gain more intuition about the principle of pseudo-residuals in a later chapter. 
\end{itemize}

\end{vbframe}



\begin{vbframe}{Loss Plots}

We call the plot that shows the point-wise error, i.e. the loss $\Lxy$ vs. the \textbf{residuals} $r := y - \fx$ (for regression), \textbf{loss plot}. The pseudo-residual corresponds to the slope of the tangent in $\left(y - \fx, \Lxy \right)$. 

\vspace*{0.5cm}


\begin{figure}
\includegraphics[width = 1\linewidth]{figure_man/loss.png}
\end{figure}

%<<echo=FALSE, warning=FALSE, message=FALSE, fig.height = 4>>=
%   xx = seq(-2, 2, by = 0.01); 
%   yy = xx^2
%   plot(xx, yy, type = "l", xlab = "y - f(x)", ylab = "L(y, f(x))")
%   points(1, 1, col = "red")
%   lines(x = c(1, 1), y = c(0, 1), col = "red")
%   points(xx, 2 * xx - 1, type = "l", col = "red")
%  @

% We will define a similar plot for classification later on in this chapter.

\end{vbframe}



\section{Theoretical Risk Minimization}


% 10, 11-- Covered in I2ML
\begin{vbframe}{(Theoretical) Risk Minimization}

\begin{itemize}
  \item The (theoretical) \textbf{risk} associated with a certain hypothesis $\fx$ measured by a loss function $\Lxy$ is the \textbf{expected loss}
  $$ \riskf := \E_{xy} [\Lxy] = \int \Lxy \text{d}\Pxy. $$
  \item Our goal is to find a hypothesis $\fx \in \Hspace$ that \textbf{minimizes} the risk.
\end{itemize}

\end{vbframe}

\begin{vbframe}{(Theoretical) Risk Minimization: Limitation} 

% The goodness of the prediction $y=\fx$ is measured by a \emph{loss function} $\Lxy $.
% Its expectation is the so-called \emph{risk}:
%   $$ \riskf = \E [\Lxy] = \int \Lxy d\Pxy. $$

\textbf{Problem}: Minimizing $\riskf$ over $f$ is generally not feasible or practical:

\begin{itemize}
\item $\Pxy$ is unknown (if it were known, we could use it directly to construct optimal predictions).
\item We could estimate $\Pxy$ in non-parametric fashion from the data $\D$ (i.i.d. drawn from $\Pxy$), e.g. by kernel density estimation, but this really does not scale to higher dimensions (see \enquote{curse of dimensionality}).
\item We can efficiently estimate $\Pxy$, if we place rigorous assumptions on its distributional form, and methods like discriminant analysis work exactly this way. \textbf{Machine learning} usually studies more flexible models.
\end{itemize}

\end{vbframe}


\section{Empirical Risk Minimization}


% 13, 14-- Covered in I2ML

\begin{vbframe}{Empirical Risk Minimization}

Let 
$$
\D = \Dset,
$$
with observations $\xyi \overset{\text{i.i.d.}}{\sim}\Pxy$.

\vspace{0.2cm}

An alternative (without directly assuming anything about $\P_{xy}$) is to approximate $\riskf$ based on $\D$ by means of the \textbf{empirical risk}

\vspace*{-0.2cm}

\begin{eqnarray*}
\riske(f) &:=& \sumin \Lxyi
\end{eqnarray*}

Learning then amounts to \textbf{empirical risk minimization}

$$
\fh = \argmin_{f \in \Hspace} \riske(f).
$$

\framebreak 

\textbf{Notes: }

\begin{itemize}
  \item The risk is often denoted as empirical mean over $\Lxy$
  $$
    \riskeb(f) = \frac{1}{n}\sumin \Lxyi. 
  $$
  The factor $\frac{1}{n}$ does not make a difference in optimization, so we will consider $\riske(f)$ most of the time.
  \item If $f$ is parameterized by $\thetab \in \Theta$, this becomes:

  \begin{eqnarray*}
  \risket & = & \sumin \Lxyit \cr
  \hat{\thetab} & = & \argmin_{\thetab \in \Theta} \riske(\thetab)
  \end{eqnarray*}

\end{itemize}

\end{vbframe}



% 15 -- Goes into advanced Optimization? 

\begin{vbframe}{Machine Learning = Optimization?}

Learning (often) means solving the above \textbf{optimization problem}.
There is a very tight connection between ML and optimization, but still, there are substantial differences:

\begin{itemize}
  \item In machine learning, we want to find a model that is optimal w.r.t. the theoretical risk $\risk(f)$.
  \item In general, we cannot compute the theoretical risk, because the data generating process $\Pxy$ is not known.
  \item Instead, we use observed data $\D$ to formulate the empirical risk $\riskef$.
  \item However, $\riske(f)$ is a good approximation for $\risk(f)$ only if $\D$ is an unbiased, independent and large enough sample from $\Pxy$.
  \item So in machine learning, we optimize an approximated version of the problem we are actually interested in.
\end{itemize}

\end{vbframe}

% Slide 16 -- As intro in Advanced Regression Losses

\begin{vbframe}{The role of Loss Functions}

Why should we care about how to choose the loss function $\Lxy$?

\begin{itemize}
% \item For regression, the loss usually only depends on residual $\Lxy = L\left(y - \fx\right) = L(\eps)$, this is a \emph{translation invariant} loss
\item \textbf{Statistical} properties of $f$: Choice of loss implies statistical properties of $f$ like robustness and an implicit error distribution.
\item \textbf{Computational / Optimization} complexity of the optimization problem: The complexity of the optimization problem
$$
\argmin_{\thetab \in \Theta} \risket
$$
is influenced by the choice of the loss function, i.e.\

  \begin{itemize}
    \item Smoothness of the objective \\
    \begin{footnotesize} 
    Some optimization methods require smoothness (e.g. gradient methods).
    \end{footnotesize}
    \item Uni- or multimodality of the problem \\
    \begin{footnotesize} 
    If $\Lxy$ is convex in its second argument, and $\fxt$ is linear in $\thetab$, then $\risket$ is convex; every local minimum of $\risket$ is a global one. If $L$ is not convex, $\risket$ might have multiple local minima (bad!).
    \end{footnotesize}
  \end{itemize}
\end{itemize}


\end{vbframe}

\endlecture
\end{document}
