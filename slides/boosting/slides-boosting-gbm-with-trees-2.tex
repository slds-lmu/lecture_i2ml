\usepackage[]{graphicx}
\usepackage[]{color}
% maxwidth is the original width if it is less than linewidth
% otherwise use linewidth (to make sure the graphics do not exceed the margin)
\makeatletter
\def\maxwidth{ %
  \ifdim\Gin@nat@width>\linewidth
    \linewidth
  \else
    \Gin@nat@width
  \fi
}
\makeatother

% ---------------------------------%
% latex-math dependencies, do not remove:
% - \usepackage{mathtools}
% - \usepackage{bm}
% - \usepackage{siunitx}
% - \usepackage{dsfont}
% - \usepackage{xspace}
% ---------------------------------%

%--------------------------------------------------------%
%       Language, encoding, typography
%--------------------------------------------------------%

\usepackage[english]{babel}
\usepackage[utf8]{inputenc} % Enables inputting UTF-8 symbols
% Standard AMS suite
\usepackage{amsmath,amsfonts,amssymb}

% Font four double-stroke / blackboard letters for sets of numbers (N, R, ...)
% Distribution name is "doublestroke"
% According to https://mirror.physik.tu-berlin.de/pub/CTAN/fonts/doublestroke/dsdoc.pdf
% the "bbm" package does a similar thing and may be superfluous.
% Required for latex-math
\usepackage{dsfont}

% bbm – "Blackboard-style" cm fonts (https://www.ctan.org/pkg/bbm)
% Used to be in common.tex, loaded directly after this file
% Maybe superfluous given dsfont is loaded
% TODO: Check if really unused?
% \usepackage{bbm}

% bm – Access bold symbols in maths mode - https://ctan.org/pkg/bm
% Required for latex-math
% https://tex.stackexchange.com/questions/3238/bm-package-versus-boldsymbol
\usepackage{bm}

% pifont – Access to PostScript standard Symbol and Dingbats fonts
% Used for \newcommand{\xmark}{\ding{55}, which is never used
% aside from lecture_advml/attic/xx-automl/slides.Rnw
% \usepackage{pifont}

% Quotes (inline and display), provdes \enquote
% https://ctan.org/pkg/csquotes
\usepackage{csquotes}

% Adds arg to enumerate env, technically superseded by enumitem according
% to https://ctan.org/pkg/enumerate
% Replace with https://ctan.org/pkg/enumitem ?
\usepackage{enumerate}

% Line spacing - provides \singlespacing \doublespacing \onehalfspacing
% https://ctan.org/pkg/setspace
% TODO: Check if really unused?
%\usepackage{setspace}

% mathtools – Mathematical tools to use with amsmath
% https://ctan.org/pkg/mathtools?lang=en
% latex-math dependency according to latex-math repo
\usepackage{mathtools}

%--------------------------------------------------------%
%       Displaying code and algorithms
%--------------------------------------------------------%
\usepackage{verbatim}
\usepackage{algorithm}
\usepackage{algpseudocode}

%--------------------------------------------------------%
%       Tables
%--------------------------------------------------------%

% multi-row table cells: https://www.namsu.de/Extra/pakete/Multirow.html
\usepackage{multirow}

% long/multi-page tables: https://texdoc.org/serve/longtable.pdf/0
% TODO: Check if really unused?

\usepackage{longtable}

% pretty table env: https://ctan.org/pkg/booktabs?lang=en
% TODO: Check if really unused?
\usepackage{booktabs}

%--------------------------------------------------------%
%       Figures: Creating, placing, verbing
%--------------------------------------------------------%

% wrapfig - Wrapping text around figures https://de.overleaf.com/learn/latex/Wrapping_text_around_figures
\usepackage{wrapfig}

% Sub figures in figures and tables
% https://ctan.org/pkg/subfig -- supersedes subfigure package
% TODO: Check if really unused?
\usepackage{subfig}

% Actually it's pronounced PGF https://en.wikibooks.org/wiki/LaTeX/PGF/TikZ
\usepackage{tikz}

\usetikzlibrary{shapes,arrows,automata,positioning,calc,chains,trees, shadows}
\tikzset{
  %Define standard arrow tip
  >=stealth',
  %Define style for boxes
  punkt/.style={
    rectangle,
    rounded corners,
    draw=black, very thick,
    text width=6.5em,
    minimum height=2em,
    text centered},
  % Define arrow style
  pil/.style={
    ->,
    thick,
    shorten <=2pt,
    shorten >=2pt,}
}


% Unsorted
% textpos – Place boxes at arbitrary positions on the LATEX page
% https://ctan.org/pkg/textpos?lang=en
% Provides \begin{textblock}
 % TODO: Check if really unused?
\usepackage[absolute,overlay]{textpos}

% psfrag – Replace strings in encapsulated PostScript figures
% https://www.overleaf.com/latex/examples/psfrag-example/tggxhgzwrzhn
% https://ftp.mpi-inf.mpg.de/pub/tex/mirror/ftp.dante.de/pub/tex/macros/latex/contrib/psfrag/pfgguide.pdf
% Can't tell if this is needed
% TODO: Check if really unused?
\usepackage{psfrag}

% Maybe not great to use this https://tex.stackexchange.com/a/197/19093
% Use align instead -- TODO: Global search & replace to check
\usepackage{eqnarray}

\usepackage{colortbl}

% arydshln – Draw dash-lines in array/tabular
% https://www.ctan.org/pkg/arydshln
% !! "arydshln has to be loaded after array, longtable, colortab and/or colortbl"
% Provides \hdashline and \cdashline
% TODO: Check if really unused?
% \usepackage{arydshln}

% tabularx – Tabulars with adjustable-width columns
% https://ctan.org/pkg/tabularx
% Provides \begin{tabularx}
% TODO: Check if really unused?
% \usepackage{tabularx}

% placeins – Control float placement
% https://ctan.org/pkg/placeins
% Defines a \FloatBarrier command
% TODO: Check if really unused?
% \usepackage{placeins}


% framed – Framed or shaded regions that can break across pages
% https://ctan.org/pkg/framed
% Provides \begin{framed} which uses \colorbox{shadecolor} relying on \definecolor{shadecolor}.
% TODO: Check if really unused?
% \usepackage{framed}

% Used often in conjunction with \definecolor{shadecolor}{rgb}{0.969, 0.969, 0.969}
% Might be able to be removed or at least redefined to only have shadecolor (if needed)
\definecolor{fgcolor}{rgb}{0.345, 0.345, 0.345}
\definecolor{shadecolor}{rgb}{0.969, 0.969, 0.969}
\newenvironment{knitrout}{}{} % an empty environment to be redefined in TeX


% Defines macros and environments
\usepackage{../../style/lmu-lecture}

\let\code=\texttt % Used regularly
\let\proglang=\textsf % Unused?

% Not sure what/why this does
\setkeys{Gin}{width=0.9\textwidth}

\setbeamertemplate{frametitle}{\expandafter\uppercase\expandafter\insertframetitle}

% Can't find a reason why common.tex is not just part of this file?

% basic latex stuff
\newcommand{\pkg}[1]{{\fontseries{b}\selectfont #1}} %fontstyle for R packages
\newcommand{\lz}{\vspace{0.5cm}} %vertical space
\newcommand{\dlz}{\vspace{1cm}} %double vertical space
\newcommand{\oneliner}[1] % Oneliner for important statements
{\begin{block}{}\begin{center}\begin{Large}#1\end{Large}\end{center}\end{block}}


%new environments
\newenvironment{vbframe}  %frame with breaks and verbatim
{
 \begin{frame}[containsverbatim,allowframebreaks]
}
{
\end{frame}
}

\newenvironment{vframe}  %frame with verbatim without breaks (to avoid numbering one slided frames)
{
 \begin{frame}[containsverbatim]
}
{
\end{frame}
}

\newenvironment{blocki}[1]   % itemize block
{
 \begin{block}{#1}\begin{itemize}
}
{
\end{itemize}\end{block}
}

\newenvironment{fragileframe}[2]{  %fragile frame with framebreaks
\begin{frame}[allowframebreaks, fragile, environment = fragileframe]
\frametitle{#1}
#2}
{\end{frame}}


\newcommand{\myframe}[2]{  %short for frame with framebreaks
\begin{frame}[allowframebreaks]
\frametitle{#1}
#2
\end{frame}}

\newcommand{\remark}[1]{
  \textbf{Remark:} #1
}


\newenvironment{deleteframe}
{
\begingroup
\usebackgroundtemplate{\includegraphics[width=\paperwidth,height=\paperheight]{../style/color/red.png}}
 \begin{frame}
}
{
\end{frame}
\endgroup
}
\newenvironment{simplifyframe}
{
\begingroup
\usebackgroundtemplate{\includegraphics[width=\paperwidth,height=\paperheight]{../style/color/yellow.png}}
 \begin{frame}
}
{
\end{frame}
\endgroup
}\newenvironment{draftframe}
{
\begingroup
\usebackgroundtemplate{\includegraphics[width=\paperwidth,height=\paperheight]{../style/color/green.jpg}}
 \begin{frame}
}
{
\end{frame}
\endgroup
}
% https://tex.stackexchange.com/a/261480: textcolor that works in mathmode
\makeatletter
\renewcommand*{\@textcolor}[3]{%
  \protect\leavevmode
  \begingroup
    \color#1{#2}#3%
  \endgroup
}
\makeatother


%-------------------------------------------------------------------------------------------------------%
%  Unused stuff that needs to go but is kept here currently juuuust in case it was important after all  %
%-------------------------------------------------------------------------------------------------------%

% \newcommand{\hlnum}[1]{\textcolor[rgb]{0.686,0.059,0.569}{#1}}%
% \newcommand{\hlstr}[1]{\textcolor[rgb]{0.192,0.494,0.8}{#1}}%
% \newcommand{\hlcom}[1]{\textcolor[rgb]{0.678,0.584,0.686}{\textit{#1}}}%
% \newcommand{\hlopt}[1]{\textcolor[rgb]{0,0,0}{#1}}%
% \newcommand{\hlstd}[1]{\textcolor[rgb]{0.345,0.345,0.345}{#1}}%
% \newcommand{\hlkwa}[1]{\textcolor[rgb]{0.161,0.373,0.58}{\textbf{#1}}}%
% \newcommand{\hlkwb}[1]{\textcolor[rgb]{0.69,0.353,0.396}{#1}}%
% \newcommand{\hlkwc}[1]{\textcolor[rgb]{0.333,0.667,0.333}{#1}}%
% \newcommand{\hlkwd}[1]{\textcolor[rgb]{0.737,0.353,0.396}{\textbf{#1}}}%
% \let\hlipl\hlkwb

% \makeatletter
% \newenvironment{kframe}{%
%  \def\at@end@of@kframe{}%
%  \ifinner\ifhmode%
%   \def\at@end@of@kframe{\end{minipage}}%
%   \begin{minipage}{\columnwidth}%
%  \fi\fi%
%  \def\FrameCommand##1{\hskip\@totalleftmargin \hskip-\fboxsep
%  \colorbox{shadecolor}{##1}\hskip-\fboxsep
%      % There is no \\@totalrightmargin, so:
%      \hskip-\linewidth \hskip-\@totalleftmargin \hskip\columnwidth}%
%  \MakeFramed {\advance\hsize-\width
%    \@totalleftmargin\z@ \linewidth\hsize
%    \@setminipage}}%
%  {\par\unskip\endMakeFramed%
%  \at@end@of@kframe}
% \makeatother

% \definecolor{shadecolor}{rgb}{.97, .97, .97}
% \definecolor{messagecolor}{rgb}{0, 0, 0}
% \definecolor{warningcolor}{rgb}{1, 0, 1}
% \definecolor{errorcolor}{rgb}{1, 0, 0}
% \newenvironment{knitrout}{}{} % an empty environment to be redefined in TeX

% \usepackage{alltt}
% \newcommand{\SweaveOpts}[1]{}  % do not interfere with LaTeX
% \newcommand{\SweaveInput}[1]{} % because they are not real TeX commands
% \newcommand{\Sexpr}[1]{}       % will only be parsed by R
% \newcommand{\xmark}{\ding{55}}%
 
% math spaces
\newcommand{\N}{\mathds{N}}                                                 % N, naturals
\newcommand{\Z}{\mathds{Z}}                                                 % Z, integers
\newcommand{\Q}{\mathds{Q}}                                                 % Q, rationals
\newcommand{\R}{\mathds{R}}                                                 % R, reals
\newcommand{\C}{\mathds{C}}                                                 % C, complex
\newcommand{\HS}{\mathcal{H}}                                               % H, hilbertspace
\newcommand{\continuous}{\mathcal{C}}                                       % C, space of continuous functions
\newcommand{\M}{\mathcal{M}} 												% machine numbers
\newcommand{\epsm}{\epsilon_m} 												% maximum error


% basic math stuff
\newcommand{\xt}{\tilde x}													% x tilde
\def\argmax{\mathop{\sf arg\,max}}                                          % argmax
\def\argmin{\mathop{\sf arg\,min}}                                          % argmin
\newcommand{\sign}{\operatorname{sign}}                                     % sign, signum
\newcommand{\I}{\mathbb{I}}                                                 % I, indicator
\newcommand{\order}{\mathcal{O}}                                            % O, order
\newcommand{\fp}[2]{\frac{\partial #1}{\partial #2}}                        % partial derivative
\newcommand{\pd}[2]{\frac{\partial{#1}}{\partial #2}}						% partial derivative

% sums and products
\newcommand{\sumin}{\sum_{i=1}^n}											% summation from i=1 to n
\newcommand{\sumkg}{\sum_{k=1}^g}											% summation from k=1 to g
\newcommand{\prodin}{\prod_{i=1}^n}											% product from i=1 to n
\newcommand{\prodkg}{\prod_{k=1}^g}											% product from k=1 to g

% linear algebra
\newcommand{\one}{\boldsymbol{1}}                                           % 1, unitvector
\newcommand{\id}{\mathrm{I}}                                                % I, identity
\newcommand{\diag}{\operatorname{diag}}                                     % diag, diagonal
\newcommand{\trace}{\operatorname{tr}}                                      % tr, trace
\newcommand{\spn}{\operatorname{span}}                                      % span
\newcommand{\scp}[2]{\left\langle #1, #2 \right\rangle}                     % <.,.>, scalarproduct
\newcommand{\mat}[1]{ 														% short pmatrix command
	\begin{pmatrix}
		#1
	\end{pmatrix}
}
\newcommand{\Amat}{\bm{A}}													% matrix A
\newcommand{\xv}{\bm{x}}													% vector x (bold)
\newcommand{\yv}{\bm{y}}														% vector y (bold)
\newcommand{\Deltab}{\bm{\Delta}}											% error term for vectors
															

% basic probability + stats
\renewcommand{\P}{\mathds{P}}                                               % P, probability
\newcommand{\E}{\mathds{E}}                                                 % E, expectation
\newcommand{\var}{\mathsf{Var}}                                             % Var, variance
\newcommand{\cov}{\mathsf{Cov}}                                             % Cov, covariance
\newcommand{\corr}{\mathsf{Corr}}                                           % Corr, correlation
\newcommand{\normal}{\mathcal{N}}                                           % N of the normal distribution
\newcommand{\iid}{\overset{i.i.d}{\sim}}                                    % dist with i.i.d superscript
\newcommand{\distas}[1]{\overset{#1}{\sim}}                                 % ... is distributed as ... 
% machine learning

%%%%%% ml - data
\newcommand{\Xspace}{\mathcal{X}}                                           % X, input space
\newcommand{\Yspace}{\mathcal{Y}}                                           % Y, output space
\newcommand{\nset}{\{1, \ldots, n\}}                                        % set from 1 to n
\newcommand{\pset}{\{1, \ldots, p\}}                                        % set from 1 to p
\newcommand{\gset}{\{1, \ldots, g\}}                                        % set from 1 to g
\newcommand{\Pxy}{\P_{xy}}                                                  % P_xy
\newcommand{\xy}{(x, y)}                                                    % observation (x, y)
\newcommand{\xvec}{(x_1, \ldots, x_p)^T}                                    % (x1, ..., xp) 
\newcommand{\D}{\mathcal{D}}                                                % D, data 
\newcommand{\Dset}{\{ (x^{(1)}, y^{(1)}), \ldots, (x^{(n)},  y^{(n)})\}}    % {(x1,y1)), ..., (xn,yn)}, data
\newcommand{\xdat}{\{ x^{(1)}, \ldots, x^{(n)}\}}   						 % {x1, ..., xn}, input data
\newcommand{\ydat}{\mathbf{y}}                                              % y (bold), vector of outcomes
\newcommand{\yvec}{(y^{(1)}, \hdots, y^{(n)})^T}                            % (y1, ..., yn), vector of outcomes
\renewcommand{\xi}[1][i]{x^{(#1)}}                                          % x^i, i-th observed value of x
\newcommand{\yi}[1][i]{y^{(#1)}}                                            % y^i, i-th observed value of y 
\newcommand{\xyi}{(\xi, \yi)}                                               % (x^i, y^i), i-th observation
\newcommand{\xivec}{(x^{(i)}_1, \ldots, x^{(i)}_p)^T}                       % (x1^i, ..., xp^i), i-th observation vector
\newcommand{\xj}{x_j}                                                       % x_j, j-th feature
\newcommand{\xjb}{\mathbf{x}_j}                                             % x_j (bold), j-th feature vecor
\newcommand{\xjvec}{(x^{(1)}_j, \ldots, x^{(n)}_j)^T}                       % (x^1_j, ..., x^n_j), j-th feature vector
\newcommand{\Dtrain}{\mathcal{D}_{\text{train}}}                            % D_train, training set
\newcommand{\Dtest}{\mathcal{D}_{\text{test}}}                              % D_test, test set

%%%%%% ml - models general

% continuous prediction function f
\newcommand{\fx}{f(x)}                                                      % f(x), continuous prediction function
\newcommand{\Hspace}{H}														% hypothesis space where f is from
\newcommand{\fh}{\hat{f}}                                                   % f hat, estimated prediction function
\newcommand{\fxh}{\fh(x)}                                                   % fhat(x)
\newcommand{\fxt}{f(x | \theta)}                                            % f(x | theta)
\newcommand{\fxi}{f(\xi)}                                                   % f(x^(i))
\newcommand{\fxih}{\hat{f}(\xi)}                                            % f(x^(i))
\newcommand{\fxit}{f(x^{(i)} | \theta)}                                     % f(x^(i) | theta)
\newcommand{\fhD}{\fh_{\D}}                                                 % fhat_D, estimate of f based on D
\newcommand{\fhDtrain}{\fh_{\Dtrain}}                                       % fhat_Dtrain, estimate of f based on D

% discrete prediction function h
\newcommand{\hx}{h(x)}                                                      % h(x), discrete prediction function
\newcommand{\hh}{\hat{h}}                                                   % h hat
\newcommand{\hxh}{\hat{h}(x)}                                               % hhat(x)
\newcommand{\hxt}{h(x | \theta)}                                            % h(x | theta)
\newcommand{\hxi}{h(\xi)}                                                   % h(x^(i))
\newcommand{\hxit}{h(x^{(i)} | \theta)}                                     % h(x^(i) | theta)

% yhat
\newcommand{\yh}{\hat{y}}                                                   % y hat for prediction of target
\newcommand{\yih}{\hat{y}}                                                  % y hat for prediction of target

% theta
\newcommand{\thetah}{\hat{\theta}}                                          % theta hat

% densities + probabilities
% pdf of x 
\newcommand{\pdf}{p}                                                        % p
\newcommand{\pdfx}{p(x)}                                                    % p(x)
\newcommand{\pixt}{\pi(x | \theta)}                                         % pi(x|theta), pdf of x given theta

% pdf of (x, y)
\newcommand{\pdfxy}{p(x,y)}                                                 % p(x, y)
\newcommand{\pdfxyt}{p(x, y | \theta)}                                      % p(x, y | theta)
\newcommand{\pdfxyit}{p(\xi, \yi | \theta)}                                 % p(x^(i), y^(i) | theta)

% pdf of x given y
\newcommand{\pdfxyk}{p(x | y=k)}                                            % p(x | y = k)
\newcommand{\lpdfxyk}{\log \pdfxyk}                                         % log p(x | y = k)
\newcommand{\pdfxiyk}{p(\xi | y=k)}                                         % p(x^i | y = k)

% prior probabilities
\newcommand{\pik}{\pi_k}                                                    % pi_k, prior
\newcommand{\lpik}{\log \pik}                                               % log pi_k, log of the prior

% posterior probabilities
\newcommand{\post}{\P(y = 1 | x)}                                           % P(y = 1 | x), post. prob for y=1
\newcommand{\pix}{\pi(x)}                                                   % pi(x), P(y = 1 | x)
\newcommand{\postk}{\P(y = k | x)}                                          % P(y = k | y), post. prob for y=k
\newcommand{\pikx}{\pi_k(x)}                                                % pi_k(x), P(y = k | x)
\newcommand{\pikxt}{\pi_k(x | \theta)}                                      % pi_k(x | theta), P(y = k | x, theta)
\newcommand{\pijx}{\pi_j(x)}                                                % pi_j(x), P(y = j | x)
\newcommand{\pdfygxt}{p(y |x, \theta)}                                      % p(y | x, theta)
\newcommand{\pdfyigxit}{p(\yi |\xi, \theta)}                                % p(y^i |x^i, theta)
\newcommand{\lpdfygxt}{\log \pdfygxt }                                      % log p(y | x, theta)
\newcommand{\lpdfyigxit}{\log \pdfyigxit}                                   % log p(y^i |x^i, theta)
\newcommand{\pixh}{\hat \pi(x)}                                             % pi(x) hat, P(y = 1 | x) hat
\newcommand{\pikxh}{\hat \pi_k(x)}                                          % pi_k(x) hat, P(y = k | x) hat

% residual and margin
\newcommand{\eps}{\epsilon}                                                 % residual, stochastic
\newcommand{\epsi}{\epsilon^{(i)}}                                          % epsilon^i, residual, stochastic
\newcommand{\epsh}{\hat{\epsilon}}                                          % residual, estimated
\newcommand{\yf}{y \fx}                                                     % y f(x), margin
\newcommand{\yfi}{\yi \fxi}                                                 % y^i f(x^i), margin
\newcommand{\Sigmah}{\hat \Sigma}											% estimated covariance matrix
\newcommand{\Sigmahj}{\hat \Sigma_j}										% estimated covariance matrix for the j-th class

% ml - loss, risk, likelihood
\newcommand{\Lxy}{L(y, f(x))}                                               % L(y, f(x)), loss function
\newcommand{\Lxyi}{L(\yi, \fxi)}                                            % L(y^i, f(x^i))
\newcommand{\Lxyt}{L(y, \fxt)}                                              % L(y, f(x | theta))
\newcommand{\Lxyit}{L(\yi, \fxit)}                                          % L(y^i, f(x^i | theta)
\newcommand{\risk}{\mathcal{R}}                                             % R, risk
\newcommand{\riskf}{\risk(f)}                                               % R(f), risk
\newcommand{\riske}{\mathcal{R}_{\text{emp}}}                               % R_emp, empirical risk
\newcommand{\riskef}{\riske(f)}                                             % R_emp(f)
\newcommand{\risket}{\mathcal{R}_{\text{emp}}(\theta)}                      % R_emp(theta)
\newcommand{\riskr}{\mathcal{R}_{\text{reg}}}                               % R_reg, regularized risk
\newcommand{\riskrt}{\mathcal{R}_{\text{reg}}(\theta)}                      % R_reg(theta)
\newcommand{\riskrf}{\riskr(f)}                                             % R_reg(f)
\newcommand{\LL}{\mathcal{L}}                                               % L, likelihood
\newcommand{\LLt}{\mathcal{L}(\theta)}                                      % L(theta), likelihood
\renewcommand{\ll}{\ell}                                                    % l, log-likelihood
\newcommand{\llt}{\ell(\theta)}                                             % l(theta), log-likelihood
\newcommand{\LS}{\mathfrak{L}}                                              % ????????????
\newcommand{\TS}{\mathfrak{T}}                                              % ??????????????
\newcommand{\errtrain}{\text{err}_{\text{train}}}                           % training error
\newcommand{\errtest}{\text{err}_{\text{test}}}                             % training error
\newcommand{\errexp}{\overline{\text{err}_{\text{test}}}}                   % training error

% resampling
\newcommand{\GE}[1]{GE(\fh_{#1})}                                           % Generalization error GE
\newcommand{\GEh}[1]{\widehat{GE}_{#1}}                                     % Estimated train error
\newcommand{\GED}{\GE{\D}}                                                  % Generalization error GE
\newcommand{\EGEn}{EGE_n}                                                   % Generalization error GE
\newcommand{\EDn}{\E_{|D| = n}}                                             % Generalization error GE


% ml - irace
\newcommand{\costs}{\mathcal{C}} % costs
\newcommand{\Celite}{\theta^*} % elite configurations
\newcommand{\instances}{\mathcal{I}} % sequence of instances
\newcommand{\budget}{\mathcal{B}} % computational budget
% ml - bagging, random forest
\newcommand{\bl}[1][m]{b^{[#1]}} % baselearner, default m
\newcommand{\blh}[1][m]{\hat{b}^{[#1]}} % estimated base learner, default m 
\newcommand{\blx}[1][m]{b^{[#1]}(\xv)} % baselearner, default m

\newcommand{\blfmh}{\hat{f}^{[m]}} % estimated baselearner: scores
\newcommand{\blfmhx}{\blfmh(\xv)} % estimated baselearner: scores of x
\newcommand{\blhmh}{\hat{h}^{[m]}} % estimated baselearner: hard labels
\newcommand{\blhmhx}{\blhmh(\xv)} % estimated baselearner: hard labels of x
\newcommand{\blpmh}{\hat \pi^{[m]}} % estimated baselearner: probabilities
\newcommand{\blpmhxk}{\hat \pi_{k}^{[m]}(\xv)} % estimated baselearner: probabilities of x for class k

\newcommand{\fM}{f^{[M]}(\xv)} % ensembled predictor
\newcommand{\fMh}{\hat f^{[M]}(\xv)} % estimated ensembled predictor
\newcommand{\ambifM}{\Delta\left(\fM\right)} % ambiguity/instability of ensemble
\newcommand{\betam}[1][m]{\beta^{[#1]}} % weight of basemodel m
\newcommand{\betamh}[1][m]{\hat{\beta}^{[#1]}} % weight of basemodel m with hat
\newcommand{\betaM}{\beta^{[M]}} % last baselearner
\newcommand{\summM}{\sum_{m=1}^M} % sum over m=1 to M baselearners
\newcommand{\avgmM}{\frac{1}{M} \sum_{m=1}^M} % averaging over m=1 to M baselearners

\newcommand{\ib}{\mathrm{IB}} % In-Bag (IB)
\newcommand{\ibm}{\ib^{[m]}} % In-Bag (IB) for m-th bootstrap
\newcommand{\oob}{\mathrm{OOB}} % Out-of-Bag (OOB)
\newcommand{\oobm}{\oob^{[m]}} % Out-of-Bag (OOB) for m-th bootstrap


% ml - boosting
\newcommand{\fm}[1][m]{f^{[#1]}} % prediction in iteration m
\newcommand{\fmh}[1][m]{\hat{f}^{[#1]}} % prediction in iteration m
\newcommand{\fmd}[1][m]{f^{[#1-1]}} % prediction m-1
\newcommand{\fmdh}[1][m]{\hat{f}^{[#1-1]}} % prediction m-1
\newcommand{\errm}[1][m]{\text{err}^{[#1]}} % weighted in-sample misclassification rate
\newcommand{\wm}[1][m]{w^{[#1]}} % weight vector of basemodel m
\newcommand{\wmi}[1][m]{w^{[#1](i)}} % weight of obs i of basemodel m
\newcommand{\thetam}[1][m]{\thetab^{[#1]}} % parameters of basemodel m
\newcommand{\thetamh}[1][m]{\hat{\thetab}^{[#1]}} % parameters of basemodel m with hat
\newcommand{\blxt}[1][m]{b(\xv, \thetab^{[#1]})} % baselearner, default m
\newcommand{\ens}{\sum_{m=1}^M \betam \blxt} % ensemble
\newcommand{\rmm}[1][m]{\tilde{r}^{[#1]}} % pseudo residuals
\newcommand{\rmi}[1][m]{\tilde{r}^{[#1](i)}} % pseudo residuals
\newcommand{\Rtm}[1][m]{R_{t}^{[#1]}} % terminal-region
\newcommand{\Tm}[1][m]{T^{[#1]}} % terminal-region
\newcommand{\ctm}[1][m]{c_t^{[#1]}} % mean, terminal-regions
\newcommand{\ctmh}[1][m]{\hat{c}_t^{[#1]}} % mean, terminal-regions with hat
\newcommand{\ctmt}[1][m]{\tilde{c}_t^{[#1]}} % mean, terminal-regions
\newcommand{\Lp}{L^\prime}
\newcommand{\Ldp}{L^{\prime\prime}}
\newcommand{\Lpleft}{\Lp_{\text{left}}}

% ml - boosting iml lecture
\newcommand{\ts}{\thetab^{\star}} % theta*
\newcommand{\bljt}{\bl[j](\xv, \thetab)} % BL j with theta
\newcommand{\bljts}{\bl[j](\xv, \ts)} % BL j with theta*

% ml - trees, extra trees

\newcommand{\Np}{\mathcal{N}}												% Parent node N
\newcommand{\Nl}{\Np_1}														% Left node N_1
\newcommand{\Nr}{\Np_2}														% Right node N_2


\newcommand{\titlefigure}{figure/gbm_anim_51.png}
\newcommand{\learninggoals}{
  \item \textcolor{blue}{@BB pls add}
}

\title{Introduction to Machine Learning}
\date{}

\begin{document}

\lecturechapter{Gradient Boosting with Trees 2}
\lecture{Introduction to Machine Learning}

% ------------------------------------------------------------------------------


\begin{vbframe}{Theoretical Background}
 
One can write a tree as: $ b(\xv) = \sum_{t=1}^{T} c_t \mathds{1}_{\{\xv \in R_t\}} $,
where $R_t$ are the terminal regions and $c_t$ the corresponding constant parameters.

\vspace*{0.2cm}

For a fitted tree with regions $R_t$, the special additive structure can be exploited in boosting:  %Finished here

\begin{align*}
  \fm(\xv) &= \fmd(\xv) +  \alpha^{[m]} \bl(\xv) \\
         &= \fmd(\xv) +  \alpha^{[m]} \sum_{t=1}^{\Tm} \ctm \mathds{1}_{\{\xv \in \Rtm\}}\\
         &= \fmd(\xv) +  \sum_{t=1}^{\Tm} \ctmt \mathds{1}_{\{\xv \in \Rtm\}}.
\end{align*}

With $\ctmt = \alpha^{[m]} \cdot \ctm$ in the case that $\alpha^{[m]}$ is a constant learning rate
%Actually, we do not have to find $\ctm$ and $\betam$ in two separate steps
%(fitting against pseudo-residuals, then line search) but find optimal 
%\textcolor{blue}{$\ctmt$} (including $\betam$).
%Also note that the \textcolor{blue}{$\ctmt$} will not really be loss-optimal as 
%we used squared error loss
%to fit them against the pseudo-residuals.

% ------------------------------------------------------------------------------

\framebreak
\begin{small}
We do the same steps as before: (1) calculate the pseudo-residuals, (2) fit a tree against pseudo-residuals, \textbf{but now} we keep only the structure of the tree and optimize the $c$ parameter in a (further) post-hoc step.

$$
\fm(\xv) = \fmd(\xv) +  \sum_{t=1}^{\Tm} \ctmt \mathds{1}_{\{\xv \in \Rtm\}}. 
$$

%We want to find the constant value $c$ that drives down risk the most w.r.t the squared error loss,
%when added to the respective terminal region.
We can determine/change all $\ctmt$ individually and directly $L$-optimally:


%\vspace{-0.2cm}

$$ \ctmt = \argmin_{c} \sum_{\xi \in \Rtm} L(\yi, \fmd(\xi) + c). $$

\vspace{-0.5cm}

\begin{center}

\includegraphics[width=0.38\textwidth]{figure_man/gbm_leaf_adjustment.pdf}

\end{center}

\end{small}

\framebreak

An alternative approach ist to directly fit a loss-optimal tree.
The risk function is then defined by:
$$
\mathcal{R}(\mathcal{N}') = \sum_{i \in \mathcal{N}'} L(\yi, \fmd(\xi) + c)
$$

with $\mathcal{N}'$ being the index set of a specific (left or right) node after splitting and $c$ being a constant value added to the current model for this node.

Thus, instead of having a two-step approach of first fitting a tree to the pseudo-residuals of the current model and then finding the optimal value for $c$, we now directly build a tree that finds $c$ loss-optimally. Since $c$ is unknown, it needs to be determined, which can either be done by a line search or by taking the derivative:
$$
\frac{\partial{\mathcal{R}(\mathcal{N}')}}{\partial c} = \sum_{i \in \mathcal{N}'} \frac{\partial{L(\yi, \fmd(\xi) + c)}}{\partial f \rvert_{f = \fmd + c}} = 0
$$
% ------------------------------------------------------------------------------
\framebreak
\begin{algorithm}[H]
  \begin{footnotesize}
  \begin{center}
  \caption{Tree Algorithm for Gradient Boosting.}
    \begin{algorithmic}[1]
      \State \textbf{input: } All observations $\mathcal{N}$ and risk function $\mathcal{R}$
      \State \textbf{output: } $\mathcal{N}_l^{j^\ast, s^\ast}$ and $\mathcal{N}_r^{j^\ast, s^\ast}$
      \For{$j = \xv_1 \dots \xv_p$}
      \For{every split $s$ on feature $j$}
          %\State For all $i$: $\rmi = -\left[\fp{\Lxyi}{\fxi}\right]_{f=\fmdh}$
          %\State Fit regr. tree to the $\rmi$, giving terminal regions $\Rtm,
          %\ t = 1,\ldots,\Tm$
        %\For{$t = 1 \to \Tm$}
       % $$
        %\hat{\mathbf{c}}^{[m]} = \argmin_{(c_1,\dots,c_{T^{[m]}})}\sum_{i = 1}^n L(\yi, \fmdh(\xi) + \bmm(\xi, c_1,\dots,c_{T^{[m]}})).
        %$$
         \State $\mathcal{N}_l^{j,s} = \{ i \in \mathcal{N}\}_{j^{(i)} \leq s}$
            \State $\mathcal{N}_r^{j,s} = \{i \in \mathcal{N}\}_{j^{(i)} > s}$ 
              
            % objective I for lambda_C and split point t
            \State Find $c$ which minimizes $\mathcal{R}$ for each node
            \State $\mathcal{I}(j, s) = \mathcal{R}(\mathcal{N}_l^{j,s}) + \mathcal{R}(\mathcal{N}_r^{j,s})$
        \EndFor
        % find optimal split point t* for lambda_C
        
       
         % \State $\fmh_t = \argmin_{\fmdh(\xi) + c} \sum \limits_{\xi \in \Rtm} L(\yi, \fmdh(\xi) + c)$
        %\EndFor
        %\State $\bmmh(\xv) = \sum_{t=1}^{T} \ctmh \mathds{1}_{\{x \in R_t\}} $
        %\State Update $\fmh(\xv) = \fmdh(\xv) + \bmmh(\xi, c_1,\dots,c_{T^{[m]}})$
      \EndFor
      \State $(j^\ast, s^\ast) \in \argmin\nolimits_{j, s} \mathcal{I}(j, s)$
    \end{algorithmic}
    \end{center}
    \end{footnotesize}
\end{algorithm}


The tree algorithm based on the CART algorithm of Breiman shows one partitioning step based on the risk function we introduced before. 
\end{vbframe}

% ------------------------------------------------------------------------------


\begin{vbframe}{GB Multiclass with Trees}

\begin{itemize}
  \item From Friedman, J. H. - Greedy Function Approximation: A Gradient Boosting Machine (1999)
  \item Determining the tree structure for each $\hat{b}^{[m]}_k$ by $L2$ loss works just like before in the 2-class problem.
\item In the estimation of the $c$ values, i.e., the heights of the terminal regions, however, all models depend on each other because of the definition
of $L$. Optimizing this is more difficult, so we will skip some details and present the main idea and results.
\end{itemize}

\framebreak

\begin{itemize}
  \item The post-hoc, loss-optimal heights of the terminals $\hat{c}_{tk}^{[m]}$ are:
  $$ 
  \hat{c}_{tk}^{[m]} = - \argmin_{c_{tk}^{[m]} } \sum_{i=1}^n \sum_{k=1}^g \mathds{1}_{\{y = k\}} \ln \pi_k^{[m]}(\xv^{(i)}) \,.
  $$
\item Softmax trafo: $\pi_k^{[m]}(\xv) = \frac{\exp(f_k^{[m]}(\xv))}{\sum_j \exp(f_j^{[m]}(\xv))},$ with 
\item The $k$-th model:
  $
  \hat{f}_k^{[m]}(\xv^{(i)})) = \hat{f}_k^{[m-1]}(\xv^{(i)}) + \sum_{t=1}^{T_k^{[m]}} \hat{c}_{tk}^{[m]} \mathds{1}_{\{\xv^{(i)} \in R_{tk}^{[m]}\}}. 
  $
  % resulting from the multinomial loss function $L(y, f_1(\xv), \ldots f_g(\xv)) = - \sumkg \mathds{1}_{\{y = k\}} \ln \pikx$.
  %and $\pikx = \frac{\exp(f_k(\xv))}{\sum_j \exp(f_j(\xv))}$ as before.\medskip

\end{itemize}


  % \item In each iteration $m$ we calculate the pseudo-residuals
        % $$\rmi_k = \mathds{1}_{\{\yi = k\}} - \pi_k^{[m-1]}(\xi),$$
        % where $\pi_k^{[m-1]}(\xi)$ is derived from $f^{[m-1]}(\mathbf{x}).$

  % \item Thus, $g$ trees are induced at each iteration $m$ to predict the corresponding current pseudo-residuals for each class on the probability scale.

  % \item Each of these trees has $T$ terminal nodes with corresponding regions $R_{tk}^{[m]}$.


\framebreak


\begin{itemize}

  \item There is no closed-form solution for finding the optimal $\hat{c}_{tk}^{[m]}$ values. Additionally, the regions corresponding to the different class trees overlap, so that the solution does not reduce to a separate calculation within each region of each tree.

  \item Hence, we approximate the solution with a single Newton-Raphson step, using a diagonal approximation to the Hessian (we leave out the details here).

  \item This decomposes the problem into a separate calculation for each terminal node of each tree.

  \item The result is

  $$\hat{c}_{tk}^{[m]} =
      \frac{g-1}{g}\frac{\sum_{\xi \in R_{tk}^{[m]}} \rmi_k}{\sum_{\xi \in R_{tk}^{[m]}} \left|\rmi_k\right|\left(1 - \left|\rmi_k\right|\right)}.$$

  % \item The update is then done by
  % $$
  % \hat{f}_k^{[m]}(\xv) = \hat{f}_k^{[m-1]}(\xv) + \sum_t \hat{c}_{tk}^{[m]} \mathds{1}_{\{\xv \in R_{tk}^{[m]}\}}.
  % $$

\end{itemize}

\framebreak


\begin{algorithm}[H]
  \begin{footnotesize}
  \begin{center}
  \caption{Gradient Boosting for $g$-class Classification.}
    \begin{algorithmic}[1]
      \State Initialize $f_{k}^{[0]}(\xv) = 0,\ k = 1,\ldots,g$
      \For{$m = 1 \to M$}
          \State Set $\pikx = \frac{\exp(f_k^{[m]}(\xv))}{\sum_j \exp(f_j^{[m]}(\xv))}, k = 1,\ldots,g$
            \For{$k = 1 \to g$}
              \State For all $i$: Compute $\rmi_k = \mathds{1}_{\{\yi = k\}} - \pi_k(\xi)$
              \State Fit regr. tree to the $\rmi_k$ giving terminal regions $R_{tk}^{[m]}$
              \State Compute
              \State \hskip\algorithmicindent\relax $\hat{c}_{tk}^{[m]} =
                \frac{g-1}{g}\frac{\sum_{\xi \in R_{tk}^{[m]}} \rmi_k}{\sum_{\xi \in R_{tk}^{[m]}} \left|\rmi_k\right|\left(1 - \left|\rmi_k\right|\right)}$
              \State Update $\hat{f}_k^{[m]}(\xv) = \hat{f}_k^{[m-1]}(\xv) + \sum_t \hat{c}_{tk}^{[m]} \mathds{1}_{\{\xv \in R_{tk}^{[m]}\}}$
            \EndFor
      \EndFor
    \State Output $\hat{f}_1^{[M]}, \ldots, \hat{f}_g^{[M]}$
    \end{algorithmic}
    \end{center}
    \end{footnotesize}
\end{algorithm}



\end{vbframe}


% \begin{vbframe}{Additional information}
% 
% By choosing a suitable loss function it is also possible to model a large number of different problem domains:
% \begin{itemize}
%   \item Regression
%   \item (Multiclass) Classification
%   \item Count data
%   \item Survival data
%   \item Ordinal data
%   \item Quantile regression
%   \item Ranking problems
%   \item ...
% \end{itemize}
% 
% \lz
% 
% % Boosting is closely related to L1 regularization.
% 
% % \lz
% 
% Different base learners increase flexibility (see componentwise gradient boosting).
% If we model only individual variables, the resulting regularized variable selection
% is closely related to $L1$ regularization.
% 
% \framebreak
% 
% For example, using the pinball loss in boosting
% $$
% L(y, f(\xv)) = \left\{
% \begin{array}{lc}
% (1 - \alpha)(f(\xv) - y), & \text{if}\ y < f(\xv) \\
% \alpha(y - f(\xv)),       & \text{if}\ y \geq f(\xv)
% \end{array}
% \right.
% $$
% models the $\alpha$-quantiles:
% 
% \begin{center}
% \includegraphics[scale=0.5]{figure_man/quantile_boosting.png}
% \end{center}
% 
% \framebreak
% 
% The AdaBoost fit has the structure of an additive model with \enquote{basis functions} $\bl (x)$.
% 
% \lz
% 
% It can be shown (see Hastie et al. 2009, Chapter 10) that AdaBoost corresponds to minimizing the empirical risk in each iteration $m$ using the \textbf{exponential} loss function:
% \begin{align*}
%   L(y, \fmh(\mathbf{x}))    &= \exp\left(-y\fmh(\mathbf{x})\right) \\
%   \riske(\fmh)              &= \sumin L(\yi, \fmh(\xi)) \\
%                             &= \sumin L(\yi, \fmdh(\xi) + \beta b(\xi))\,,
% \end{align*}
% 
% 
% % \begin{align*}
% %   \sum_{i=1}^n \exp\left(-\yi \cdot \left(\beta b\left(\xi\right)
% %   + \fmdh\left(\xi\right)\right)\right),
% % \end{align*}
% with minimization over $\beta$ and $b$ and where $\fmdh$ is the boosting fit in iteration $m-1$.
% 
% % \framebreak
% 
% % AdaBoost is the empirical equivalent to the forward piecewise solution of the minimization problem
% 
% % \begin{align*}
% %   \text{arg} \min_{f} \E_{y|x}( \exp (- y \cdot \fx))\ .
% % \end{align*}
% 
% % \lz
% 
% % Therefore, the boosting fit is an estimate of function
% % \begin{align*}
% %   f^*(x) = 0.5 \cdot \log \left( \frac{\text{P} (y = 1 | x)}
% %   {\text{P} (y = -1 | x)}\right) \ ,
% % \end{align*}
% % which solves the former problem theoretically.
% 
% % \lz
% 
% % Obvious idea: generalization on other loss functions, use of alternative basis methods.
% 
% \end{vbframe}

% 
% \begin{vbframe}{Take home message}
% Gradient boosting is a statistical reinterpretation of the older AdaBoost algorithm.
% 
% \lz
% 
% Base learners are added in a \enquote{greedy} fashion, so that they point in the direction of the negative gradient of the empirical risk.
% 
% \lz
% 
% Regression base learners are fitted even for classification problems.
% 
% \lz
% 
% Often the base learners are (shallow) trees, but arbitrary base learners are possible.
% 
% \lz
% 
% The method can be adjusted flexibly by changing the loss function, as long as it's differentiable.
% 
% \lz
% 
% Methods to evaluate variable importance and to do variable selection exist.
% 
% \end{vbframe}


\endlecture
\end{document}
