\usepackage[]{graphicx}
\usepackage[]{color}
% maxwidth is the original width if it is less than linewidth
% otherwise use linewidth (to make sure the graphics do not exceed the margin)
\makeatletter
\def\maxwidth{ %
  \ifdim\Gin@nat@width>\linewidth
    \linewidth
  \else
    \Gin@nat@width
  \fi
}
\makeatother

% ---------------------------------%
% latex-math dependencies, do not remove:
% - \usepackage{mathtools}
% - \usepackage{bm}
% - \usepackage{siunitx}
% - \usepackage{dsfont}
% - \usepackage{xspace}
% ---------------------------------%

%--------------------------------------------------------%
%       Language, encoding, typography
%--------------------------------------------------------%

\usepackage[english]{babel}
\usepackage[utf8]{inputenc} % Enables inputting UTF-8 symbols
% Standard AMS suite
\usepackage{amsmath,amsfonts,amssymb}

% Font four double-stroke / blackboard letters for sets of numbers (N, R, ...)
% Distribution name is "doublestroke"
% According to https://mirror.physik.tu-berlin.de/pub/CTAN/fonts/doublestroke/dsdoc.pdf
% the "bbm" package does a similar thing and may be superfluous.
% Required for latex-math
\usepackage{dsfont}

% bbm – "Blackboard-style" cm fonts (https://www.ctan.org/pkg/bbm)
% Used to be in common.tex, loaded directly after this file
% Maybe superfluous given dsfont is loaded
% TODO: Check if really unused?
% \usepackage{bbm}

% bm – Access bold symbols in maths mode - https://ctan.org/pkg/bm
% Required for latex-math
% https://tex.stackexchange.com/questions/3238/bm-package-versus-boldsymbol
\usepackage{bm}

% pifont – Access to PostScript standard Symbol and Dingbats fonts
% Used for \newcommand{\xmark}{\ding{55}, which is never used
% aside from lecture_advml/attic/xx-automl/slides.Rnw
% \usepackage{pifont}

% Quotes (inline and display), provdes \enquote
% https://ctan.org/pkg/csquotes
\usepackage{csquotes}

% Adds arg to enumerate env, technically superseded by enumitem according
% to https://ctan.org/pkg/enumerate
% Replace with https://ctan.org/pkg/enumitem ?
\usepackage{enumerate}

% Line spacing - provides \singlespacing \doublespacing \onehalfspacing
% https://ctan.org/pkg/setspace
% TODO: Check if really unused?
%\usepackage{setspace}

% mathtools – Mathematical tools to use with amsmath
% https://ctan.org/pkg/mathtools?lang=en
% latex-math dependency according to latex-math repo
\usepackage{mathtools}

%--------------------------------------------------------%
%       Displaying code and algorithms
%--------------------------------------------------------%
\usepackage{verbatim}
\usepackage{algorithm}
\usepackage{algpseudocode}

%--------------------------------------------------------%
%       Tables
%--------------------------------------------------------%

% multi-row table cells: https://www.namsu.de/Extra/pakete/Multirow.html
\usepackage{multirow}

% long/multi-page tables: https://texdoc.org/serve/longtable.pdf/0
% TODO: Check if really unused?

\usepackage{longtable}

% pretty table env: https://ctan.org/pkg/booktabs?lang=en
% TODO: Check if really unused?
\usepackage{booktabs}

%--------------------------------------------------------%
%       Figures: Creating, placing, verbing
%--------------------------------------------------------%

% wrapfig - Wrapping text around figures https://de.overleaf.com/learn/latex/Wrapping_text_around_figures
\usepackage{wrapfig}

% Sub figures in figures and tables
% https://ctan.org/pkg/subfig -- supersedes subfigure package
% TODO: Check if really unused?
\usepackage{subfig}

% Actually it's pronounced PGF https://en.wikibooks.org/wiki/LaTeX/PGF/TikZ
\usepackage{tikz}

\usetikzlibrary{shapes,arrows,automata,positioning,calc,chains,trees, shadows}
\tikzset{
  %Define standard arrow tip
  >=stealth',
  %Define style for boxes
  punkt/.style={
    rectangle,
    rounded corners,
    draw=black, very thick,
    text width=6.5em,
    minimum height=2em,
    text centered},
  % Define arrow style
  pil/.style={
    ->,
    thick,
    shorten <=2pt,
    shorten >=2pt,}
}


% Unsorted
% textpos – Place boxes at arbitrary positions on the LATEX page
% https://ctan.org/pkg/textpos?lang=en
% Provides \begin{textblock}
 % TODO: Check if really unused?
\usepackage[absolute,overlay]{textpos}

% psfrag – Replace strings in encapsulated PostScript figures
% https://www.overleaf.com/latex/examples/psfrag-example/tggxhgzwrzhn
% https://ftp.mpi-inf.mpg.de/pub/tex/mirror/ftp.dante.de/pub/tex/macros/latex/contrib/psfrag/pfgguide.pdf
% Can't tell if this is needed
% TODO: Check if really unused?
\usepackage{psfrag}

% Maybe not great to use this https://tex.stackexchange.com/a/197/19093
% Use align instead -- TODO: Global search & replace to check
\usepackage{eqnarray}

\usepackage{colortbl}

% arydshln – Draw dash-lines in array/tabular
% https://www.ctan.org/pkg/arydshln
% !! "arydshln has to be loaded after array, longtable, colortab and/or colortbl"
% Provides \hdashline and \cdashline
% TODO: Check if really unused?
% \usepackage{arydshln}

% tabularx – Tabulars with adjustable-width columns
% https://ctan.org/pkg/tabularx
% Provides \begin{tabularx}
% TODO: Check if really unused?
% \usepackage{tabularx}

% placeins – Control float placement
% https://ctan.org/pkg/placeins
% Defines a \FloatBarrier command
% TODO: Check if really unused?
% \usepackage{placeins}


% framed – Framed or shaded regions that can break across pages
% https://ctan.org/pkg/framed
% Provides \begin{framed} which uses \colorbox{shadecolor} relying on \definecolor{shadecolor}.
% TODO: Check if really unused?
% \usepackage{framed}

% Used often in conjunction with \definecolor{shadecolor}{rgb}{0.969, 0.969, 0.969}
% Might be able to be removed or at least redefined to only have shadecolor (if needed)
\definecolor{fgcolor}{rgb}{0.345, 0.345, 0.345}
\definecolor{shadecolor}{rgb}{0.969, 0.969, 0.969}
\newenvironment{knitrout}{}{} % an empty environment to be redefined in TeX


% Defines macros and environments
\usepackage{../../style/lmu-lecture}

\let\code=\texttt % Used regularly
\let\proglang=\textsf % Unused?

% Not sure what/why this does
\setkeys{Gin}{width=0.9\textwidth}

\setbeamertemplate{frametitle}{\expandafter\uppercase\expandafter\insertframetitle}

% Can't find a reason why common.tex is not just part of this file?

% basic latex stuff
\newcommand{\pkg}[1]{{\fontseries{b}\selectfont #1}} %fontstyle for R packages
\newcommand{\lz}{\vspace{0.5cm}} %vertical space
\newcommand{\dlz}{\vspace{1cm}} %double vertical space
\newcommand{\oneliner}[1] % Oneliner for important statements
{\begin{block}{}\begin{center}\begin{Large}#1\end{Large}\end{center}\end{block}}


%new environments
\newenvironment{vbframe}  %frame with breaks and verbatim
{
 \begin{frame}[containsverbatim,allowframebreaks]
}
{
\end{frame}
}

\newenvironment{vframe}  %frame with verbatim without breaks (to avoid numbering one slided frames)
{
 \begin{frame}[containsverbatim]
}
{
\end{frame}
}

\newenvironment{blocki}[1]   % itemize block
{
 \begin{block}{#1}\begin{itemize}
}
{
\end{itemize}\end{block}
}

\newenvironment{fragileframe}[2]{  %fragile frame with framebreaks
\begin{frame}[allowframebreaks, fragile, environment = fragileframe]
\frametitle{#1}
#2}
{\end{frame}}


\newcommand{\myframe}[2]{  %short for frame with framebreaks
\begin{frame}[allowframebreaks]
\frametitle{#1}
#2
\end{frame}}

\newcommand{\remark}[1]{
  \textbf{Remark:} #1
}


\newenvironment{deleteframe}
{
\begingroup
\usebackgroundtemplate{\includegraphics[width=\paperwidth,height=\paperheight]{../style/color/red.png}}
 \begin{frame}
}
{
\end{frame}
\endgroup
}
\newenvironment{simplifyframe}
{
\begingroup
\usebackgroundtemplate{\includegraphics[width=\paperwidth,height=\paperheight]{../style/color/yellow.png}}
 \begin{frame}
}
{
\end{frame}
\endgroup
}\newenvironment{draftframe}
{
\begingroup
\usebackgroundtemplate{\includegraphics[width=\paperwidth,height=\paperheight]{../style/color/green.jpg}}
 \begin{frame}
}
{
\end{frame}
\endgroup
}
% https://tex.stackexchange.com/a/261480: textcolor that works in mathmode
\makeatletter
\renewcommand*{\@textcolor}[3]{%
  \protect\leavevmode
  \begingroup
    \color#1{#2}#3%
  \endgroup
}
\makeatother


%-------------------------------------------------------------------------------------------------------%
%  Unused stuff that needs to go but is kept here currently juuuust in case it was important after all  %
%-------------------------------------------------------------------------------------------------------%

% \newcommand{\hlnum}[1]{\textcolor[rgb]{0.686,0.059,0.569}{#1}}%
% \newcommand{\hlstr}[1]{\textcolor[rgb]{0.192,0.494,0.8}{#1}}%
% \newcommand{\hlcom}[1]{\textcolor[rgb]{0.678,0.584,0.686}{\textit{#1}}}%
% \newcommand{\hlopt}[1]{\textcolor[rgb]{0,0,0}{#1}}%
% \newcommand{\hlstd}[1]{\textcolor[rgb]{0.345,0.345,0.345}{#1}}%
% \newcommand{\hlkwa}[1]{\textcolor[rgb]{0.161,0.373,0.58}{\textbf{#1}}}%
% \newcommand{\hlkwb}[1]{\textcolor[rgb]{0.69,0.353,0.396}{#1}}%
% \newcommand{\hlkwc}[1]{\textcolor[rgb]{0.333,0.667,0.333}{#1}}%
% \newcommand{\hlkwd}[1]{\textcolor[rgb]{0.737,0.353,0.396}{\textbf{#1}}}%
% \let\hlipl\hlkwb

% \makeatletter
% \newenvironment{kframe}{%
%  \def\at@end@of@kframe{}%
%  \ifinner\ifhmode%
%   \def\at@end@of@kframe{\end{minipage}}%
%   \begin{minipage}{\columnwidth}%
%  \fi\fi%
%  \def\FrameCommand##1{\hskip\@totalleftmargin \hskip-\fboxsep
%  \colorbox{shadecolor}{##1}\hskip-\fboxsep
%      % There is no \\@totalrightmargin, so:
%      \hskip-\linewidth \hskip-\@totalleftmargin \hskip\columnwidth}%
%  \MakeFramed {\advance\hsize-\width
%    \@totalleftmargin\z@ \linewidth\hsize
%    \@setminipage}}%
%  {\par\unskip\endMakeFramed%
%  \at@end@of@kframe}
% \makeatother

% \definecolor{shadecolor}{rgb}{.97, .97, .97}
% \definecolor{messagecolor}{rgb}{0, 0, 0}
% \definecolor{warningcolor}{rgb}{1, 0, 1}
% \definecolor{errorcolor}{rgb}{1, 0, 0}
% \newenvironment{knitrout}{}{} % an empty environment to be redefined in TeX

% \usepackage{alltt}
% \newcommand{\SweaveOpts}[1]{}  % do not interfere with LaTeX
% \newcommand{\SweaveInput}[1]{} % because they are not real TeX commands
% \newcommand{\Sexpr}[1]{}       % will only be parsed by R
% \newcommand{\xmark}{\ding{55}}%
 
% math spaces
\newcommand{\N}{\mathds{N}}                                                 % N, naturals
\newcommand{\Z}{\mathds{Z}}                                                 % Z, integers
\newcommand{\Q}{\mathds{Q}}                                                 % Q, rationals
\newcommand{\R}{\mathds{R}}                                                 % R, reals
\newcommand{\C}{\mathds{C}}                                                 % C, complex
\newcommand{\HS}{\mathcal{H}}                                               % H, hilbertspace
\newcommand{\continuous}{\mathcal{C}}                                       % C, space of continuous functions
\newcommand{\M}{\mathcal{M}} 												% machine numbers
\newcommand{\epsm}{\epsilon_m} 												% maximum error


% basic math stuff
\newcommand{\xt}{\tilde x}													% x tilde
\def\argmax{\mathop{\sf arg\,max}}                                          % argmax
\def\argmin{\mathop{\sf arg\,min}}                                          % argmin
\newcommand{\sign}{\operatorname{sign}}                                     % sign, signum
\newcommand{\I}{\mathbb{I}}                                                 % I, indicator
\newcommand{\order}{\mathcal{O}}                                            % O, order
\newcommand{\fp}[2]{\frac{\partial #1}{\partial #2}}                        % partial derivative
\newcommand{\pd}[2]{\frac{\partial{#1}}{\partial #2}}						% partial derivative

% sums and products
\newcommand{\sumin}{\sum_{i=1}^n}											% summation from i=1 to n
\newcommand{\sumkg}{\sum_{k=1}^g}											% summation from k=1 to g
\newcommand{\prodin}{\prod_{i=1}^n}											% product from i=1 to n
\newcommand{\prodkg}{\prod_{k=1}^g}											% product from k=1 to g

% linear algebra
\newcommand{\one}{\boldsymbol{1}}                                           % 1, unitvector
\newcommand{\id}{\mathrm{I}}                                                % I, identity
\newcommand{\diag}{\operatorname{diag}}                                     % diag, diagonal
\newcommand{\trace}{\operatorname{tr}}                                      % tr, trace
\newcommand{\spn}{\operatorname{span}}                                      % span
\newcommand{\scp}[2]{\left\langle #1, #2 \right\rangle}                     % <.,.>, scalarproduct
\newcommand{\mat}[1]{ 														% short pmatrix command
	\begin{pmatrix}
		#1
	\end{pmatrix}
}
\newcommand{\Amat}{\bm{A}}													% matrix A
\newcommand{\xv}{\bm{x}}													% vector x (bold)
\newcommand{\yv}{\bm{y}}														% vector y (bold)
\newcommand{\Deltab}{\bm{\Delta}}											% error term for vectors
															

% basic probability + stats
\renewcommand{\P}{\mathds{P}}                                               % P, probability
\newcommand{\E}{\mathds{E}}                                                 % E, expectation
\newcommand{\var}{\mathsf{Var}}                                             % Var, variance
\newcommand{\cov}{\mathsf{Cov}}                                             % Cov, covariance
\newcommand{\corr}{\mathsf{Corr}}                                           % Corr, correlation
\newcommand{\normal}{\mathcal{N}}                                           % N of the normal distribution
\newcommand{\iid}{\overset{i.i.d}{\sim}}                                    % dist with i.i.d superscript
\newcommand{\distas}[1]{\overset{#1}{\sim}}                                 % ... is distributed as ... 
% machine learning

%%%%%% ml - data
\newcommand{\Xspace}{\mathcal{X}}                                           % X, input space
\newcommand{\Yspace}{\mathcal{Y}}                                           % Y, output space
\newcommand{\nset}{\{1, \ldots, n\}}                                        % set from 1 to n
\newcommand{\pset}{\{1, \ldots, p\}}                                        % set from 1 to p
\newcommand{\gset}{\{1, \ldots, g\}}                                        % set from 1 to g
\newcommand{\Pxy}{\P_{xy}}                                                  % P_xy
\newcommand{\xy}{(x, y)}                                                    % observation (x, y)
\newcommand{\xvec}{(x_1, \ldots, x_p)^T}                                    % (x1, ..., xp) 
\newcommand{\D}{\mathcal{D}}                                                % D, data 
\newcommand{\Dset}{\{ (x^{(1)}, y^{(1)}), \ldots, (x^{(n)},  y^{(n)})\}}    % {(x1,y1)), ..., (xn,yn)}, data
\newcommand{\xdat}{\{ x^{(1)}, \ldots, x^{(n)}\}}   						 % {x1, ..., xn}, input data
\newcommand{\ydat}{\mathbf{y}}                                              % y (bold), vector of outcomes
\newcommand{\yvec}{(y^{(1)}, \hdots, y^{(n)})^T}                            % (y1, ..., yn), vector of outcomes
\renewcommand{\xi}[1][i]{x^{(#1)}}                                          % x^i, i-th observed value of x
\newcommand{\yi}[1][i]{y^{(#1)}}                                            % y^i, i-th observed value of y 
\newcommand{\xyi}{(\xi, \yi)}                                               % (x^i, y^i), i-th observation
\newcommand{\xivec}{(x^{(i)}_1, \ldots, x^{(i)}_p)^T}                       % (x1^i, ..., xp^i), i-th observation vector
\newcommand{\xj}{x_j}                                                       % x_j, j-th feature
\newcommand{\xjb}{\mathbf{x}_j}                                             % x_j (bold), j-th feature vecor
\newcommand{\xjvec}{(x^{(1)}_j, \ldots, x^{(n)}_j)^T}                       % (x^1_j, ..., x^n_j), j-th feature vector
\newcommand{\Dtrain}{\mathcal{D}_{\text{train}}}                            % D_train, training set
\newcommand{\Dtest}{\mathcal{D}_{\text{test}}}                              % D_test, test set

%%%%%% ml - models general

% continuous prediction function f
\newcommand{\fx}{f(x)}                                                      % f(x), continuous prediction function
\newcommand{\Hspace}{H}														% hypothesis space where f is from
\newcommand{\fh}{\hat{f}}                                                   % f hat, estimated prediction function
\newcommand{\fxh}{\fh(x)}                                                   % fhat(x)
\newcommand{\fxt}{f(x | \theta)}                                            % f(x | theta)
\newcommand{\fxi}{f(\xi)}                                                   % f(x^(i))
\newcommand{\fxih}{\hat{f}(\xi)}                                            % f(x^(i))
\newcommand{\fxit}{f(x^{(i)} | \theta)}                                     % f(x^(i) | theta)
\newcommand{\fhD}{\fh_{\D}}                                                 % fhat_D, estimate of f based on D
\newcommand{\fhDtrain}{\fh_{\Dtrain}}                                       % fhat_Dtrain, estimate of f based on D

% discrete prediction function h
\newcommand{\hx}{h(x)}                                                      % h(x), discrete prediction function
\newcommand{\hh}{\hat{h}}                                                   % h hat
\newcommand{\hxh}{\hat{h}(x)}                                               % hhat(x)
\newcommand{\hxt}{h(x | \theta)}                                            % h(x | theta)
\newcommand{\hxi}{h(\xi)}                                                   % h(x^(i))
\newcommand{\hxit}{h(x^{(i)} | \theta)}                                     % h(x^(i) | theta)

% yhat
\newcommand{\yh}{\hat{y}}                                                   % y hat for prediction of target
\newcommand{\yih}{\hat{y}}                                                  % y hat for prediction of target

% theta
\newcommand{\thetah}{\hat{\theta}}                                          % theta hat

% densities + probabilities
% pdf of x 
\newcommand{\pdf}{p}                                                        % p
\newcommand{\pdfx}{p(x)}                                                    % p(x)
\newcommand{\pixt}{\pi(x | \theta)}                                         % pi(x|theta), pdf of x given theta

% pdf of (x, y)
\newcommand{\pdfxy}{p(x,y)}                                                 % p(x, y)
\newcommand{\pdfxyt}{p(x, y | \theta)}                                      % p(x, y | theta)
\newcommand{\pdfxyit}{p(\xi, \yi | \theta)}                                 % p(x^(i), y^(i) | theta)

% pdf of x given y
\newcommand{\pdfxyk}{p(x | y=k)}                                            % p(x | y = k)
\newcommand{\lpdfxyk}{\log \pdfxyk}                                         % log p(x | y = k)
\newcommand{\pdfxiyk}{p(\xi | y=k)}                                         % p(x^i | y = k)

% prior probabilities
\newcommand{\pik}{\pi_k}                                                    % pi_k, prior
\newcommand{\lpik}{\log \pik}                                               % log pi_k, log of the prior

% posterior probabilities
\newcommand{\post}{\P(y = 1 | x)}                                           % P(y = 1 | x), post. prob for y=1
\newcommand{\pix}{\pi(x)}                                                   % pi(x), P(y = 1 | x)
\newcommand{\postk}{\P(y = k | x)}                                          % P(y = k | y), post. prob for y=k
\newcommand{\pikx}{\pi_k(x)}                                                % pi_k(x), P(y = k | x)
\newcommand{\pikxt}{\pi_k(x | \theta)}                                      % pi_k(x | theta), P(y = k | x, theta)
\newcommand{\pijx}{\pi_j(x)}                                                % pi_j(x), P(y = j | x)
\newcommand{\pdfygxt}{p(y |x, \theta)}                                      % p(y | x, theta)
\newcommand{\pdfyigxit}{p(\yi |\xi, \theta)}                                % p(y^i |x^i, theta)
\newcommand{\lpdfygxt}{\log \pdfygxt }                                      % log p(y | x, theta)
\newcommand{\lpdfyigxit}{\log \pdfyigxit}                                   % log p(y^i |x^i, theta)
\newcommand{\pixh}{\hat \pi(x)}                                             % pi(x) hat, P(y = 1 | x) hat
\newcommand{\pikxh}{\hat \pi_k(x)}                                          % pi_k(x) hat, P(y = k | x) hat

% residual and margin
\newcommand{\eps}{\epsilon}                                                 % residual, stochastic
\newcommand{\epsi}{\epsilon^{(i)}}                                          % epsilon^i, residual, stochastic
\newcommand{\epsh}{\hat{\epsilon}}                                          % residual, estimated
\newcommand{\yf}{y \fx}                                                     % y f(x), margin
\newcommand{\yfi}{\yi \fxi}                                                 % y^i f(x^i), margin
\newcommand{\Sigmah}{\hat \Sigma}											% estimated covariance matrix
\newcommand{\Sigmahj}{\hat \Sigma_j}										% estimated covariance matrix for the j-th class

% ml - loss, risk, likelihood
\newcommand{\Lxy}{L(y, f(x))}                                               % L(y, f(x)), loss function
\newcommand{\Lxyi}{L(\yi, \fxi)}                                            % L(y^i, f(x^i))
\newcommand{\Lxyt}{L(y, \fxt)}                                              % L(y, f(x | theta))
\newcommand{\Lxyit}{L(\yi, \fxit)}                                          % L(y^i, f(x^i | theta)
\newcommand{\risk}{\mathcal{R}}                                             % R, risk
\newcommand{\riskf}{\risk(f)}                                               % R(f), risk
\newcommand{\riske}{\mathcal{R}_{\text{emp}}}                               % R_emp, empirical risk
\newcommand{\riskef}{\riske(f)}                                             % R_emp(f)
\newcommand{\risket}{\mathcal{R}_{\text{emp}}(\theta)}                      % R_emp(theta)
\newcommand{\riskr}{\mathcal{R}_{\text{reg}}}                               % R_reg, regularized risk
\newcommand{\riskrt}{\mathcal{R}_{\text{reg}}(\theta)}                      % R_reg(theta)
\newcommand{\riskrf}{\riskr(f)}                                             % R_reg(f)
\newcommand{\LL}{\mathcal{L}}                                               % L, likelihood
\newcommand{\LLt}{\mathcal{L}(\theta)}                                      % L(theta), likelihood
\renewcommand{\ll}{\ell}                                                    % l, log-likelihood
\newcommand{\llt}{\ell(\theta)}                                             % l(theta), log-likelihood
\newcommand{\LS}{\mathfrak{L}}                                              % ????????????
\newcommand{\TS}{\mathfrak{T}}                                              % ??????????????
\newcommand{\errtrain}{\text{err}_{\text{train}}}                           % training error
\newcommand{\errtest}{\text{err}_{\text{test}}}                             % training error
\newcommand{\errexp}{\overline{\text{err}_{\text{test}}}}                   % training error

% resampling
\newcommand{\GE}[1]{GE(\fh_{#1})}                                           % Generalization error GE
\newcommand{\GEh}[1]{\widehat{GE}_{#1}}                                     % Estimated train error
\newcommand{\GED}{\GE{\D}}                                                  % Generalization error GE
\newcommand{\EGEn}{EGE_n}                                                   % Generalization error GE
\newcommand{\EDn}{\E_{|D| = n}}                                             % Generalization error GE


% ml - irace
\newcommand{\costs}{\mathcal{C}} % costs
\newcommand{\Celite}{\theta^*} % elite configurations
\newcommand{\instances}{\mathcal{I}} % sequence of instances
\newcommand{\budget}{\mathcal{B}} % computational budget
% ml - bagging, random forest

\newcommand{\bl}[1]{b^{[#1]}(x)}											% baselearner with argument for m
\newcommand{\blm}{\bl{m}}												    % baselearner without argument for m
\newcommand{\blmh}{\hat{b}^{[m]}(x)}										% estimated base learner 

\input{../../latex-math/ml-boosting.tex}
% ml - trees, extra trees

\newcommand{\Np}{\mathcal{N}}												% Parent node N
\newcommand{\Nl}{\Np_1}														% Left node N_1
\newcommand{\Nr}{\Np_2}														% Right node N_2


\newcommand{\titlefigure}{figure_man/split-finding02.png}
\newcommand{\learninggoals}{
  \item \textcolor{blue}{XXX}
  \item \textcolor{blue}{XXX}
}

\title{Introduction to Machine Learning}
\date{}

\begin{document}

\lecturechapter{XGBoost}
\lecture{Introduction to Machine Learning}

% sources: https://homes.cs.washington.edu/~tqchen/pdf/BoostedTree.pdf
% sources: https://towardsdatascience.com/boosting-algorithm-xgboost-4d9ec0207d
% sources: https://devblogs.nvidia.com/parallelforall/gradient-boosting-decision-trees-xgboost-cuda/

\begin{vbframe}{Motivation}

\textbf{Chen and Guestrin (2016)}:

\pkg{XGBoost} (short for eXtreme Gradient Boosting) is an open-source software
library with R/Python/Julia/Spark interface.

\lz

A scalable regularized tree boosting system:
\begin{itemize}
  \item Efficient implementation in \emph{C++}.
  \item Parallel (approximate) split finding.
  \item Handling of \emph{sparse data}.
  \item Support of cluster-computing frameworks.
\end{itemize}

\lz

State-of-the-art machine learning method:
\begin{itemize}
  \item Many winning solutions at Kaggle use XGBoost.
  \item KDDCup 2015 Top 10 used XGBoost.
\end{itemize}

\end{vbframe}

\begin{vbframe}{Loss minimization}

\pkg{XGBoost} uses a risk function with 3 regularization terms:
\vskip -1em
\begin{multline*}
  \riskr^{[m]} = \sum_{i=1}^{n} L\left(\yi, \fmd(\xi) + \bmm(\xi)\right)\\
   + \gamma J_1(\bmm) + \lambda J_2(\bmm) + \alpha J_3(\bmm),
\end{multline*}


with $J_1(\bmm) = T^{[m]}$ to regularize tree depth, where $T^{[m]}$ is the number of leaves in the tree.

\lz

$J_2(\bmm) = \left\|\mathbf{c}^{[m]}\right\|^2_2$ and $J_3(\bmm) = \left\|\mathbf{c}^{[m]}\right\|_1$ are $L2$ and $L1$ regularizers of the terminal region scores $c_t^{[m]}, t=1,\dots,T^{[m]}$.

\lz

\textbf{Note:} In the following we define $J(\bmm) := \gamma J_1(\bmm) + \lambda J_2(\bmm) + \alpha J_3(\bmm)$ to simplify the notation.

\framebreak

Recall the way a tree base learner $\bmm(\xv)$ can be fitted loss-optimally in gradient boosting:

$$
\tilde{\mathbf{c}}^{[m]} = \argmin_{(c_1,\dots,c_{T^{[m]}})}\sum_{i = 1}^n L(\yi, \fmd(\xi) + \bmm(\xi, c_1,\dots,c_{T^{[m]}})).
$$

The direction of steepest descent for the update is then

$$
-\frac{\partial L(y, \fmd(\xv) + \bmm(\xv))}{\partial \bmm(\xv)}
$$

for each $\xi, i = 1,\dots, n$.

\lz

\textbf{Note:} $J(\bmm)$ is omitted for now but will be re-introduced later.

\framebreak

To approximate the loss in iteration $m$, use a second-order Taylor expansion around $\fmd(\xv)$:
\vskip  -1em
\begin{align*}
  &L(y, \fmd(\xv) + \bmm(\xv)) \approx \\
  &\qquad L(y, \fmd(\xv)) + g^{[m]}(\xv)\bmm(\xv) + \frac12 h^{[m]}(\xv)\bmm(\xv)^2,
\end{align*}

with gradient

$$
g^{[m]}(\xv) = \frac{\partial L(y, \fmd(\xv))}{\partial \fmd(\xv)}
$$

and Hessian

$$
h^{[m]}(\xv) = \frac{\partial^2 L(y, \fmd(\xv))}{\partial {\fmd(\xv)}^2}.
$$

\textbf{Note:} $g^{[m]}(\xv)$ are the negative pseudo-residuals $-\rmm$ we use in standard gradient boosting to determine the direction of the update.

\framebreak

Since $L(y, \fmd(\xv))$ is a constant, the optimization criterion simplifies to

\begin{align*}
\riskr^{[m]} \approx &\sum_{i = 1}^n g^{[m]}(\xi)\bmm(\xi) + \frac12 h^{[m]}(\xi)\bmm(\xi)^2 + J(\bmm) + const\\
\propto&\sum_{t=1}^{T^{[m]}}\sum_{\xi\in R_t^{[m]}} g^{[m]}(\xi)c^{[m]}_t + \frac12 h^{[m]}(\xi)(c^{[m]}_t)^2 + J(\bmm).\intertext{Defining $G^{[m]}_t$ and $H^{[m]}_t$ as the sum of the gradients and Hessians, respectively, in terminal node $t$ yields
}
=& \sum_{t=1}^{T^{[m]}}G^{[m]}_t c^{[m]}_t+\frac12 H^{[m]}_t (c^{[m]}_t)^2 + J(\bmm).
\end{align*}



\framebreak

Expanding $J(\bmm)$, we get
\begin{align*}
\riskr^{[m]} = &\sum_{t=1}^{T^{[m]}}\left(G^{[m]}_t c^{[m]}_t+\frac12 H^{[m]}_t (c^{[m]}_t)^2 + \frac12\lambda(c^{[m]}_t)^2 + \alpha |c^{[m]}_t|\right) + \gamma T^{[m]}\\
=&\sum_{t=1}^{T^{[m]}}\left(G^{[m]}_t c^{[m]}_t+\frac12 (H^{[m]}_t + \lambda) (c^{[m]}_t)^2 + \alpha |c_t^{[m]}|\right) + \gamma T^{[m]}.
\end{align*}

\lz

\textbf{Note:} The factor $\frac12$ is added to the $L2$ regularization to simplify the notation as shown in the second step.
This does not impact estimation since we can just define $\lambda = 2\tilde\lambda$.

\framebreak

Since

$$
\frac{\partial \riskr^{[m]}}{\partial c^{[m]}_t} =
\begin{cases}
  (G^{[m]}_t - \alpha) + (H^{[m]}_t + \lambda) c^{m}_t &\text{ for } c^{m}_t < 0 \\
  (G^{[m]}_t + \alpha) + (H^{[m]}_t + \lambda) c^{m}_t &\text{ for } c^{m}_t > 0, \\
\end{cases}
$$

the optimal weights $\tilde{c}^{[m]}_1,\dots, \tilde{c}^{[m]}_{T^{[m]}}$ can then be calculated as

\lz
$$
\tilde{c}^{[m]}_t = - \frac{t_\alpha\left(G^{[m]}_t\right)}{H^{[m]}_t + \lambda}, t=1,\dots T^{[m]},
$$
with $$t_\alpha(x) = \begin{cases}
  x - \alpha &\text{ for } x > \alpha \\
  x + \alpha &\text{ for } x < - \alpha \\
  0  &\text{ for } |x| \leq \alpha.
\end{cases}$$

\end{vbframe}

\begin{vbframe}{Loss minimization - split finding}

To evaluate the performance of a candidate split that divides the instances in region $R_t^{[m]}$ into a left and right node we use the \textbf{risk reduction} achieved by that split:
$$
\tilde S_{LR} =
 \frac12 \left[\frac{t_\alpha\left(G^{[m]}_{tL}\right)^2}{H^{[m]}_{tL} + \lambda} + \frac{t_\alpha\left(G^{[m]}_{tR}\right)^2}{H^{[m]}_{tR} + \lambda} - \frac{t_\alpha\left(G^{[m]}_{t}\right)^2}{H^{[m]}_{t} + \lambda}\right] - \gamma,
$$
where the subscripts $L$ and $R$ denote the left and right leaves after the split.

% derivation for this: write out change in loss function, case distinction acording to sign of t_a(G)/(H + lambda)

\lz

\framebreak

\begin{algorithm}[H]

\begin{footnotesize}
\begin{center}

  \begin{algorithmic}[1]
    \State \textbf{Input} $I$: \emph{instance set of current node}
    \State \textbf{Input} $p$: \emph{dimension of feature space}
    \State $gain \gets 0$
    \State $G \gets \sum_{i \in I} g(\xi), {H} \gets \sum_{i \in I} h(\xi)$
    \For{$j = 1 \to p$}
      \State $G_L \gets 0, {H}_L \gets 0$
      \For{$i$ in sorted($I$, by $x_{j}$)}
        \State ${G}_L \gets {G}_L + g(\xi), {H}_L \gets {H}_L + h(\xi)$
        \State ${G}_R \gets G - {G}_L, {H}_R \gets {H} - {H}_L$
        \State compute $\tilde S_{LR}$
      \EndFor
    \EndFor
    \State \textbf{Output} Split with maximal $\tilde S_{LR}$
  \end{algorithmic}
\end{center}
\end{footnotesize}
\caption{(Exact) Algorithm for split finding}
\end{algorithm}

\end{vbframe}

\begin{vbframe}{Approximate split-finding algorithms}

Three different algorithms to search for these splits are implemented in XGBoost.

\lz

\begin{itemize}
\item[] The \textbf{exact greedy algorithm} iterates over all possible splits of all features.
\lz
\item[] The \textbf{global approximate algorithm} iterates over percentiles of the empirical distribution of each feature.
\lz
\item[] The \textbf{local approximate algorithm} does the same as the global, but recalculates the percentiles after each split.
\end{itemize}

\framebreak

\begin{algorithm}[H]
\begin{footnotesize}
\begin{center}
  \begin{algorithmic}[1]
    \For{$j = 1 \to p$}
      \State Define possible split proposals $S_j = \{s_{j}^{(1)}, s_{j}^{(2)}, \hdots, s_{j}^{(l)}\}$ by percentiles on feature $j$.
      \State Proposal can be done once per tree (global), or in each node (local).
    \EndFor
    \For{$j = 1 \to p$}
      \State ${G}_{kv} \gets \sum_{i \in \{i|s_j^{(v)} \geq x_j^{(i)} > s_{k}^{(v - 1)}\}} g(\xi)$
      \State ${H}_{kv} \gets \sum_{i \in \{i|s_j^{(v)} \geq x_j^{(i)} > s_{k}^{(v - 1)}\}} h(\xi)$
    \EndFor
    \State Follow same steps as exact algorithm to find max score only among proposed splits.
  \end{algorithmic}
\end{center}
\end{footnotesize}
\caption{Approximate algorithm for split finding}
\end{algorithm}

\framebreak

\vspace{0.2cm}

\begin{center}
\includegraphics[width=0.9\textwidth]{figure_man/split-finding01.png}
\end{center}

Approximate \textbf{global} split finding.\\
Blue lines indicate percentiles, red line is the chosen split.
Proposals are not recomputed, allowing the right node only two possibilities.

\framebreak

\vspace{0.2cm}

\begin{center}
\includegraphics[width=0.9\textwidth]{figure_man/split-finding02.png}
\end{center}

Approximate \textbf{local} split finding.\\
Blue lines indicate percentiles, red line is the chosen split.
Percentiles are recalculated after each split.

\framebreak

Both approximate splitting algorithms are useful for large datasets as iteration over all splits becomes very expensive.

\lz

XGBoost also allows for sparsity-aware split finding, which means that for a large number of zero (or missing) values, a default direction in each node is introduced.

\lz

The calculation of the percentiles in the feature distributions can be parallelized, allowing parallel split finding for the approximate algorithms.

\end{vbframe}

\begin{vbframe}{Even more regularization}

Besides the regularization terms discussed above, XGBoost also incorporates a step length $\nu$ for the updates.

\lz

Furthermore, it uses \textbf{feature subsampling} similar to the random forest. Only a random subset of all features is considered for each split.

\lz

Stochastic gradient boosting, as introduced before, is utilized as well,
which means that in each iteration only a subset of the observations is used.

\lz

A final possibility for regularization is DART (Dropout Additive Regression Trees).
For each iteration only a fraction of previously fitted trees is considered.
After an update the joint contribution of the dropped tree and the tree new to the ensemble are scaled to avoid overshooting.

\end{vbframe}

\begin{vbframe}{Storage and out-of-memory calculation}

The data is stored in \textbf{blocks.}
Each of these blocks is stored in a compressed column format.

\lz

Depending on the split finding algorithm, these blocks contain the complete (sorted) data, or subsets of rows.

\lz

These blocks can be stored out-of-memory (on disk) to save memory and distributed across multiple machines for parallel computations.

\lz

For more details on this see \emph{Chen and Guestrin (2016)}.

\end{vbframe}

\begin{vbframe}{Summary}

XGBoost is an extremely powerful method, but also hard to configure correctly.
Overall, eight hyperparameters have to be set, which is difficult to do in practice and almost always requires tuning.

\lz

Different split finding algorithms can be selected, which allows XGBoost to be efficient even on very large datasets.

\lz

A large number of different regularization strategies is included to prevent overfitting.


\end{vbframe}

\begin{vbframe}{Comparison of major boosting systems}

\begin{tiny}
\begin{table}[]
\centering
\begin{tabular}{l|c|c|c|c|c|c}
System       & Exact algo. & Approx. algo. & Sparsity-aware & Variable importance & Parallel & Language   \\
\hline
ada          & yes         & no            & no             & no                  & no       & R          \\
GBM          & yes         & no            & partially      & yes                 & no       & R          \\
mboost       & yes         & no            & no             & no                  & no       & R          \\
compboost    & yes         & no            & yes            & yes                 & yes      & R          \\
H2O          & no          & yes           & partially      & yes                 & yes      & R (Java)   \\
XGBoost      & yes         & yes           & yes            & yes                 & yes      & R + Python \\
lightGBM     & no          & yes           & yes            & yes                 & yes      & R + Python \\
catboost     & no          & yes           & no             & yes                 & yes      & R + Python \\
scikit-learn & yes         & no            & no             & yes                 & no       & Python     \\
pGBRT        & no          & no            & no             & no                  & yes      & Python     \\
Spark MLLib  & no          & yes           & partially      & yes                 & yes      & R, Python, \\
             &             &               &                &                     &          & Java, Scala\\

\end{tabular}
\label{my-label}
\end{table}
\end{tiny}

\lz

\textbf{Note:} H2O is a commercial software written in Java with a solid R interface.
In the free version only two CPUs can be used.

%\framebreak
%
%We compare the performance in terms of accuray and runtime on five example data sets from OpenML.
%
%\lz
%
%All boosting algorithms use $100$ iterations, a learning rate of $0.1$ and a maximum tree depth of $4$ (except for mboost which uses linear models as base-learner).
%
%\lz
%
%We also compare to a random forest as a base-line.
%
%\framebreak
%<<echo=FALSE, fig.height=5>>=
%load("rsrc/benchmark.RData")
%plotBMRBoxplots(bmr, facet.wrap.ncol = 4)
%plotBMRBoxplots(bmr, measure = timetrain, facet.wrap.ncol = 4)
%@

%\framebreak

%Overall XGBoost performs well and is on par with commercial software like H2O.

%\lz

%The random forest is hard to beat in this benchmark. This is due to the fact that we did not do any tuning of the boosting hyperparameters.
%While this is quite important for boosting algorithms, a random forest is not as sensitive to its hyperparameters.

%\lz

%mboost with boosted linear models is overall worse than the other algorithms, but has the advantage of better interpretability.

\end{vbframe}

\endlecture
\end{document}

