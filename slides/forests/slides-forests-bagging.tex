\documentclass[11pt,compress,t,notes=noshow, xcolor=table]{beamer}
\usepackage[]{graphicx}\usepackage[]{color}
% maxwidth is the original width if it is less than linewidth
% otherwise use linewidth (to make sure the graphics do not exceed the margin)
\makeatletter
\def\maxwidth{ %
  \ifdim\Gin@nat@width>\linewidth
    \linewidth
  \else
    \Gin@nat@width
  \fi
}
\makeatother

\definecolor{fgcolor}{rgb}{0.345, 0.345, 0.345}
\newcommand{\hlnum}[1]{\textcolor[rgb]{0.686,0.059,0.569}{#1}}%
\newcommand{\hlstr}[1]{\textcolor[rgb]{0.192,0.494,0.8}{#1}}%
\newcommand{\hlcom}[1]{\textcolor[rgb]{0.678,0.584,0.686}{\textit{#1}}}%
\newcommand{\hlopt}[1]{\textcolor[rgb]{0,0,0}{#1}}%
\newcommand{\hlstd}[1]{\textcolor[rgb]{0.345,0.345,0.345}{#1}}%
\newcommand{\hlkwa}[1]{\textcolor[rgb]{0.161,0.373,0.58}{\textbf{#1}}}%
\newcommand{\hlkwb}[1]{\textcolor[rgb]{0.69,0.353,0.396}{#1}}%
\newcommand{\hlkwc}[1]{\textcolor[rgb]{0.333,0.667,0.333}{#1}}%
\newcommand{\hlkwd}[1]{\textcolor[rgb]{0.737,0.353,0.396}{\textbf{#1}}}%
\let\hlipl\hlkwb

\usepackage{framed}
\makeatletter
\newenvironment{kframe}{%
 \def\at@end@of@kframe{}%
 \ifinner\ifhmode%
  \def\at@end@of@kframe{\end{minipage}}%
  \begin{minipage}{\columnwidth}%
 \fi\fi%
 \def\FrameCommand##1{\hskip\@totalleftmargin \hskip-\fboxsep
 \colorbox{shadecolor}{##1}\hskip-\fboxsep
     % There is no \\@totalrightmargin, so:
     \hskip-\linewidth \hskip-\@totalleftmargin \hskip\columnwidth}%
 \MakeFramed {\advance\hsize-\width
   \@totalleftmargin\z@ \linewidth\hsize
   \@setminipage}}%
 {\par\unskip\endMakeFramed%
 \at@end@of@kframe}
\makeatother

\definecolor{shadecolor}{rgb}{.97, .97, .97}
\definecolor{messagecolor}{rgb}{0, 0, 0}
\definecolor{warningcolor}{rgb}{1, 0, 1}
\definecolor{errorcolor}{rgb}{1, 0, 0}
\newenvironment{knitrout}{}{} % an empty environment to be redefined in TeX

\usepackage{alltt}
\newcommand{\SweaveOpts}[1]{}  % do not interfere with LaTeX
\newcommand{\SweaveInput}[1]{} % because they are not real TeX commands
\newcommand{\Sexpr}[1]{}       % will only be parsed by R



\usepackage[english]{babel}
\usepackage[utf8]{inputenc}

\usepackage{dsfont}
\usepackage{verbatim}
\usepackage{amsmath}
\usepackage{amsfonts}
\usepackage{bm}
\usepackage{csquotes}
\usepackage{multirow}
\usepackage{longtable}
\usepackage{booktabs}
\usepackage{enumerate}
\usepackage[absolute,overlay]{textpos}
\usepackage{psfrag}
\usepackage{algorithm}
\usepackage{algpseudocode}
\usepackage{eqnarray}
\usepackage{arydshln}
\usepackage{tabularx}
\usepackage{placeins}
\usepackage{tikz}
\usepackage{setspace}
\usepackage{colortbl}
\usepackage{mathtools}
\usepackage{wrapfig}
\usepackage{bm}
\usetikzlibrary{shapes,arrows,automata,positioning,calc,chains,trees, shadows}
\tikzset{
  %Define standard arrow tip
  >=stealth',
  %Define style for boxes
  punkt/.style={
    rectangle,
    rounded corners,
    draw=black, very thick,
    text width=6.5em,
    minimum height=2em,
    text centered},
  % Define arrow style
  pil/.style={
    ->,
    thick,
    shorten <=2pt,
    shorten >=2pt,}
}
\usepackage{subfig}


% Defines macros and environments

% basic latex stuff
\newcommand{\pkg}[1]{{\fontseries{b}\selectfont #1}} %fontstyle for R packages
\newcommand{\lz}{\vspace{0.5cm}} %vertical space
\newcommand{\dlz}{\vspace{1cm}} %double vertical space
\newcommand{\oneliner}[1] % Oneliner for important statements
{\begin{block}{}\begin{center}\begin{Large}#1\end{Large}\end{center}\end{block}}


%new environments
\newenvironment{vbframe}  %frame with breaks and verbatim
{
 \begin{frame}[containsverbatim,allowframebreaks]
}
{
\end{frame}
}

\newenvironment{vframe}  %frame with verbatim without breaks (to avoid numbering one slided frames)
{
 \begin{frame}[containsverbatim]
}
{
\end{frame}
}

\newenvironment{blocki}[1]   % itemize block
{
 \begin{block}{#1}\begin{itemize}
}
{
\end{itemize}\end{block}
}

\newenvironment{fragileframe}[2]{  %fragile frame with framebreaks
\begin{frame}[allowframebreaks, fragile, environment = fragileframe]
\frametitle{#1}
#2}
{\end{frame}}


\newcommand{\myframe}[2]{  %short for frame with framebreaks
\begin{frame}[allowframebreaks]
\frametitle{#1}
#2
\end{frame}}

\newcommand{\remark}[1]{
  \textbf{Remark:} #1
}


\newenvironment{deleteframe}
{
\begingroup
\usebackgroundtemplate{\includegraphics[width=\paperwidth,height=\paperheight]{../style/color/red.png}}
 \begin{frame}
}
{
\end{frame}
\endgroup
}
\newenvironment{simplifyframe}
{
\begingroup
\usebackgroundtemplate{\includegraphics[width=\paperwidth,height=\paperheight]{../style/color/yellow.png}}
 \begin{frame}
}
{
\end{frame}
\endgroup
}\newenvironment{draftframe}
{
\begingroup
\usebackgroundtemplate{\includegraphics[width=\paperwidth,height=\paperheight]{../style/color/green.jpg}}
 \begin{frame}
}
{
\end{frame}
\endgroup
}
% https://tex.stackexchange.com/a/261480: textcolor that works in mathmode
\makeatletter
\renewcommand*{\@textcolor}[3]{%
  \protect\leavevmode
  \begingroup
    \color#1{#2}#3%
  \endgroup
}
\makeatother


%\usetheme{lmu-lecture}
\newcommand{\titlefigure}{figure_man/rf_majvot_averaging.png}
\newcommand{\learninggoals}{
\item Understand the basic idea of bagging
\item Be able to explain the connection of bagging and bootstrap
\item Understand how a prediction is computed for bagging
\item Understand why bagging improves the predictive power}
\usepackage{../../style/lmu-lecture}

\let\code=\texttt
\let\proglang=\textsf

\setkeys{Gin}{width=0.9\textwidth}

\title{Introduction to Machine Learning}
% \author{Bernd Bischl, Christoph Molnar, Daniel Schalk, Fabian Scheipl}
\institute{\href{https://compstat-lmu.github.io/lecture_i2ml/}{compstat-lmu.github.io/lecture\_i2ml}}
\date{}

\setbeamertemplate{frametitle}{\expandafter\uppercase\expandafter\insertframetitle}



\begin{document}
% Set style/preamble.Rnw as parent.


% Load all R packages and set up knitr

% This file loads R packages, configures knitr options and sets preamble.Rnw as parent file
% IF YOU MODIFY THIS, PLZ ALSO MODIFY setup.Rmd ACCORDINGLY...








% Defines macros and environments
% math spaces
\newcommand{\N}{\mathds{N}}                                                 % N, naturals
\newcommand{\Z}{\mathds{Z}}                                                 % Z, integers
\newcommand{\Q}{\mathds{Q}}                                                 % Q, rationals
\newcommand{\R}{\mathds{R}}                                                 % R, reals
\newcommand{\C}{\mathds{C}}                                                 % C, complex
\newcommand{\HS}{\mathcal{H}}                                               % H, hilbertspace
\newcommand{\continuous}{\mathcal{C}}                                       % C, space of continuous functions
\newcommand{\M}{\mathcal{M}} 												% machine numbers
\newcommand{\epsm}{\epsilon_m} 												% maximum error


% basic math stuff
\newcommand{\xt}{\tilde x}													% x tilde
\def\argmax{\mathop{\sf arg\,max}}                                          % argmax
\def\argmin{\mathop{\sf arg\,min}}                                          % argmin
\newcommand{\sign}{\operatorname{sign}}                                     % sign, signum
\newcommand{\I}{\mathbb{I}}                                                 % I, indicator
\newcommand{\order}{\mathcal{O}}                                            % O, order
\newcommand{\fp}[2]{\frac{\partial #1}{\partial #2}}                        % partial derivative
\newcommand{\pd}[2]{\frac{\partial{#1}}{\partial #2}}						% partial derivative

% sums and products
\newcommand{\sumin}{\sum_{i=1}^n}											% summation from i=1 to n
\newcommand{\sumkg}{\sum_{k=1}^g}											% summation from k=1 to g
\newcommand{\prodin}{\prod_{i=1}^n}											% product from i=1 to n
\newcommand{\prodkg}{\prod_{k=1}^g}											% product from k=1 to g

% linear algebra
\newcommand{\one}{\boldsymbol{1}}                                           % 1, unitvector
\newcommand{\id}{\mathrm{I}}                                                % I, identity
\newcommand{\diag}{\operatorname{diag}}                                     % diag, diagonal
\newcommand{\trace}{\operatorname{tr}}                                      % tr, trace
\newcommand{\spn}{\operatorname{span}}                                      % span
\newcommand{\scp}[2]{\left\langle #1, #2 \right\rangle}                     % <.,.>, scalarproduct
\newcommand{\mat}[1]{ 														% short pmatrix command
	\begin{pmatrix}
		#1
	\end{pmatrix}
}
\newcommand{\Amat}{\bm{A}}													% matrix A
\newcommand{\xv}{\bm{x}}													% vector x (bold)
\newcommand{\yv}{\bm{y}}														% vector y (bold)
\newcommand{\Deltab}{\bm{\Delta}}											% error term for vectors
															

% basic probability + stats
\renewcommand{\P}{\mathds{P}}                                               % P, probability
\newcommand{\E}{\mathds{E}}                                                 % E, expectation
\newcommand{\var}{\mathsf{Var}}                                             % Var, variance
\newcommand{\cov}{\mathsf{Cov}}                                             % Cov, covariance
\newcommand{\corr}{\mathsf{Corr}}                                           % Corr, correlation
\newcommand{\normal}{\mathcal{N}}                                           % N of the normal distribution
\newcommand{\iid}{\overset{i.i.d}{\sim}}                                    % dist with i.i.d superscript
\newcommand{\distas}[1]{\overset{#1}{\sim}}                                 % ... is distributed as ... 
% machine learning

%%%%%% ml - data
\newcommand{\Xspace}{\mathcal{X}}                                           % X, input space
\newcommand{\Yspace}{\mathcal{Y}}                                           % Y, output space
\newcommand{\nset}{\{1, \ldots, n\}}                                        % set from 1 to n
\newcommand{\pset}{\{1, \ldots, p\}}                                        % set from 1 to p
\newcommand{\gset}{\{1, \ldots, g\}}                                        % set from 1 to g
\newcommand{\Pxy}{\P_{xy}}                                                  % P_xy
\newcommand{\xy}{(x, y)}                                                    % observation (x, y)
\newcommand{\xvec}{(x_1, \ldots, x_p)^T}                                    % (x1, ..., xp) 
\newcommand{\D}{\mathcal{D}}                                                % D, data 
\newcommand{\Dset}{\{ (x^{(1)}, y^{(1)}), \ldots, (x^{(n)},  y^{(n)})\}}    % {(x1,y1)), ..., (xn,yn)}, data
\newcommand{\xdat}{\{ x^{(1)}, \ldots, x^{(n)}\}}   						 % {x1, ..., xn}, input data
\newcommand{\ydat}{\mathbf{y}}                                              % y (bold), vector of outcomes
\newcommand{\yvec}{(y^{(1)}, \hdots, y^{(n)})^T}                            % (y1, ..., yn), vector of outcomes
\renewcommand{\xi}[1][i]{x^{(#1)}}                                          % x^i, i-th observed value of x
\newcommand{\yi}[1][i]{y^{(#1)}}                                            % y^i, i-th observed value of y 
\newcommand{\xyi}{(\xi, \yi)}                                               % (x^i, y^i), i-th observation
\newcommand{\xivec}{(x^{(i)}_1, \ldots, x^{(i)}_p)^T}                       % (x1^i, ..., xp^i), i-th observation vector
\newcommand{\xj}{x_j}                                                       % x_j, j-th feature
\newcommand{\xjb}{\mathbf{x}_j}                                             % x_j (bold), j-th feature vecor
\newcommand{\xjvec}{(x^{(1)}_j, \ldots, x^{(n)}_j)^T}                       % (x^1_j, ..., x^n_j), j-th feature vector
\newcommand{\Dtrain}{\mathcal{D}_{\text{train}}}                            % D_train, training set
\newcommand{\Dtest}{\mathcal{D}_{\text{test}}}                              % D_test, test set

%%%%%% ml - models general

% continuous prediction function f
\newcommand{\fx}{f(x)}                                                      % f(x), continuous prediction function
\newcommand{\Hspace}{H}														% hypothesis space where f is from
\newcommand{\fh}{\hat{f}}                                                   % f hat, estimated prediction function
\newcommand{\fxh}{\fh(x)}                                                   % fhat(x)
\newcommand{\fxt}{f(x | \theta)}                                            % f(x | theta)
\newcommand{\fxi}{f(\xi)}                                                   % f(x^(i))
\newcommand{\fxih}{\hat{f}(\xi)}                                            % f(x^(i))
\newcommand{\fxit}{f(x^{(i)} | \theta)}                                     % f(x^(i) | theta)
\newcommand{\fhD}{\fh_{\D}}                                                 % fhat_D, estimate of f based on D
\newcommand{\fhDtrain}{\fh_{\Dtrain}}                                       % fhat_Dtrain, estimate of f based on D

% discrete prediction function h
\newcommand{\hx}{h(x)}                                                      % h(x), discrete prediction function
\newcommand{\hh}{\hat{h}}                                                   % h hat
\newcommand{\hxh}{\hat{h}(x)}                                               % hhat(x)
\newcommand{\hxt}{h(x | \theta)}                                            % h(x | theta)
\newcommand{\hxi}{h(\xi)}                                                   % h(x^(i))
\newcommand{\hxit}{h(x^{(i)} | \theta)}                                     % h(x^(i) | theta)

% yhat
\newcommand{\yh}{\hat{y}}                                                   % y hat for prediction of target
\newcommand{\yih}{\hat{y}}                                                  % y hat for prediction of target

% theta
\newcommand{\thetah}{\hat{\theta}}                                          % theta hat

% densities + probabilities
% pdf of x 
\newcommand{\pdf}{p}                                                        % p
\newcommand{\pdfx}{p(x)}                                                    % p(x)
\newcommand{\pixt}{\pi(x | \theta)}                                         % pi(x|theta), pdf of x given theta

% pdf of (x, y)
\newcommand{\pdfxy}{p(x,y)}                                                 % p(x, y)
\newcommand{\pdfxyt}{p(x, y | \theta)}                                      % p(x, y | theta)
\newcommand{\pdfxyit}{p(\xi, \yi | \theta)}                                 % p(x^(i), y^(i) | theta)

% pdf of x given y
\newcommand{\pdfxyk}{p(x | y=k)}                                            % p(x | y = k)
\newcommand{\lpdfxyk}{\log \pdfxyk}                                         % log p(x | y = k)
\newcommand{\pdfxiyk}{p(\xi | y=k)}                                         % p(x^i | y = k)

% prior probabilities
\newcommand{\pik}{\pi_k}                                                    % pi_k, prior
\newcommand{\lpik}{\log \pik}                                               % log pi_k, log of the prior

% posterior probabilities
\newcommand{\post}{\P(y = 1 | x)}                                           % P(y = 1 | x), post. prob for y=1
\newcommand{\pix}{\pi(x)}                                                   % pi(x), P(y = 1 | x)
\newcommand{\postk}{\P(y = k | x)}                                          % P(y = k | y), post. prob for y=k
\newcommand{\pikx}{\pi_k(x)}                                                % pi_k(x), P(y = k | x)
\newcommand{\pikxt}{\pi_k(x | \theta)}                                      % pi_k(x | theta), P(y = k | x, theta)
\newcommand{\pijx}{\pi_j(x)}                                                % pi_j(x), P(y = j | x)
\newcommand{\pdfygxt}{p(y |x, \theta)}                                      % p(y | x, theta)
\newcommand{\pdfyigxit}{p(\yi |\xi, \theta)}                                % p(y^i |x^i, theta)
\newcommand{\lpdfygxt}{\log \pdfygxt }                                      % log p(y | x, theta)
\newcommand{\lpdfyigxit}{\log \pdfyigxit}                                   % log p(y^i |x^i, theta)
\newcommand{\pixh}{\hat \pi(x)}                                             % pi(x) hat, P(y = 1 | x) hat
\newcommand{\pikxh}{\hat \pi_k(x)}                                          % pi_k(x) hat, P(y = k | x) hat

% residual and margin
\newcommand{\eps}{\epsilon}                                                 % residual, stochastic
\newcommand{\epsi}{\epsilon^{(i)}}                                          % epsilon^i, residual, stochastic
\newcommand{\epsh}{\hat{\epsilon}}                                          % residual, estimated
\newcommand{\yf}{y \fx}                                                     % y f(x), margin
\newcommand{\yfi}{\yi \fxi}                                                 % y^i f(x^i), margin
\newcommand{\Sigmah}{\hat \Sigma}											% estimated covariance matrix
\newcommand{\Sigmahj}{\hat \Sigma_j}										% estimated covariance matrix for the j-th class

% ml - loss, risk, likelihood
\newcommand{\Lxy}{L(y, f(x))}                                               % L(y, f(x)), loss function
\newcommand{\Lxyi}{L(\yi, \fxi)}                                            % L(y^i, f(x^i))
\newcommand{\Lxyt}{L(y, \fxt)}                                              % L(y, f(x | theta))
\newcommand{\Lxyit}{L(\yi, \fxit)}                                          % L(y^i, f(x^i | theta)
\newcommand{\risk}{\mathcal{R}}                                             % R, risk
\newcommand{\riskf}{\risk(f)}                                               % R(f), risk
\newcommand{\riske}{\mathcal{R}_{\text{emp}}}                               % R_emp, empirical risk
\newcommand{\riskef}{\riske(f)}                                             % R_emp(f)
\newcommand{\risket}{\mathcal{R}_{\text{emp}}(\theta)}                      % R_emp(theta)
\newcommand{\riskr}{\mathcal{R}_{\text{reg}}}                               % R_reg, regularized risk
\newcommand{\riskrt}{\mathcal{R}_{\text{reg}}(\theta)}                      % R_reg(theta)
\newcommand{\riskrf}{\riskr(f)}                                             % R_reg(f)
\newcommand{\LL}{\mathcal{L}}                                               % L, likelihood
\newcommand{\LLt}{\mathcal{L}(\theta)}                                      % L(theta), likelihood
\renewcommand{\ll}{\ell}                                                    % l, log-likelihood
\newcommand{\llt}{\ell(\theta)}                                             % l(theta), log-likelihood
\newcommand{\LS}{\mathfrak{L}}                                              % ????????????
\newcommand{\TS}{\mathfrak{T}}                                              % ??????????????
\newcommand{\errtrain}{\text{err}_{\text{train}}}                           % training error
\newcommand{\errtest}{\text{err}_{\text{test}}}                             % training error
\newcommand{\errexp}{\overline{\text{err}_{\text{test}}}}                   % training error

% resampling
\newcommand{\GE}[1]{GE(\fh_{#1})}                                           % Generalization error GE
\newcommand{\GEh}[1]{\widehat{GE}_{#1}}                                     % Estimated train error
\newcommand{\GED}{\GE{\D}}                                                  % Generalization error GE
\newcommand{\EGEn}{EGE_n}                                                   % Generalization error GE
\newcommand{\EDn}{\E_{|D| = n}}                                             % Generalization error GE


% ml - irace
\newcommand{\costs}{\mathcal{C}} % costs
\newcommand{\Celite}{\theta^*} % elite configurations
\newcommand{\instances}{\mathcal{I}} % sequence of instances
\newcommand{\budget}{\mathcal{B}} % computational budget
% ml - bagging, random forest

\newcommand{\bl}[1]{b^{[#1]}(x)}											% baselearner with argument for m
\newcommand{\blm}{\bl{m}}												    % baselearner without argument for m
\newcommand{\blmh}{\hat{b}^{[m]}(x)}										% estimated base learner 



%! includes: basics-riskminimization, basics-supervised,  evaluation-resampling

\lecturechapter{Random Forests: Bagging Ensembles}
\lecture{Introduction to Machine Learning}
\sloppy



\begin{vbframe}{Bagging}

\begin{itemize}
  \item Bagging is short for \textbf{B}ootstrap \textbf{Agg}regation.
  \item It's an \textbf{ensemble method}, i.e., it combines many models into one 
        big \enquote{meta-model}
  \item Such model ensembles often work much better than their members alone would.
  \item The constituent models of an ensemble are called \textbf{base learners} 
      % that improves instable / high variance learners by variance smoothing
\end{itemize}

\framebreak 
In a \textbf{bagging} ensemble, all base learners are of the same type. The only difference between the models is the data they are trained on.\\
Specifically, we train base learners $\blm, m = 1, \dots, M$ on $M$ \textbf{bootstrap} samples of training data $\D$:
\begin{itemize}
  \item Draw $n$ observations from $\D$ with replacement
  \item Fit the base learner on each of the $B$ bootstrap samples to get models $\fh(x) = \blmh, m = 1, \dots, M$
\end{itemize}

\begin{center}
% FIGURE SOURCE: https://docs.google.com/presentation/d/1xodP6ayu1Gay6mMKgzVWYEFmSoeG5kNuqsaTkFFmd78/edit
\includegraphics[width=0.55\textwidth]{figure_man/bagging.pdf}
\end{center}

\framebreak

\textbf{Aggregate} the predictions of the $M$ fitted base learners to get the
\textbf{ensemble model} $\fMh$:
  \begin{itemize}
    \item Aggregate via averaging (regression) or majority voting (classification)
    \item Posterior class probabilities $\pikxh$ can be estimated by calculating predicted class frequencies over the ensemble
  \end{itemize}

\begin{center}
% FIGURE SOURCE: No source
\includegraphics[width=0.6\textwidth]{figure_man/rf_majvot_averaging.png}
\end{center}
\end{vbframe}

% \begin{algorithm}[H]
%   \small
%   \setstretch{1.15}
%   \caption*{Bagging algorithm}
%   \begin{algorithmic}[1]
%     \State {\bf Input: } Dataset $\D$, base learner, number of bootstraps $M$
%     \For {$m = 1 \to M$}
%       \State Draw a bootstrap sample $\D^{[m]}$ from $\D$.
%       \State Train base learner on $\D^{[m]}$ to obtain model $\blm$
%     \EndFor
%     \State Aggregate the predictions of the $M$ estimators (via averaging or majority voting), to determine the bagging estimator:
%     \begin{align*}
%     \fM &= \frac{1}{M} \sum_{m=1}^M \blm \\
%     \text{or}\quad \fM &= \argmax_{k \in \Yspace} \sum_{m=1}^M \I\left(\blm = k\right)
%     \end{align*}
%   \end{algorithmic}
% \end{algorithm}

\begin{vbframe}{Why/when does Bagging help?}

In one sentence:\\
\lz

Because the variability of the average of the predictions of many base learner models is smaller than the variability of the predictions from one such base learner model.\\

If the error of a base learner model is mostly due to (random) variability and not due to structural reasons, combining many such base learners by bagging helps reducing this variability.


\framebreak
\begin{scriptsize}
 Assume we use quadratic loss and measure instability of the ensemble with $\ambifM = \tfrac{1}{M}\sum^M_{m} \left(\blm- \fM \right)^2$:
 \vskip -2em
 \begin{align*}
 \ambifM &= \tfrac{1}{M}\sum^M_{m} \left(\blm- \fM\right)^2 \\
         &= \tfrac{1}{M}\sum^M_{m} \left(\left(\blm - y\right)  + \left(y - \fM\right)\right)^2\\
         &= \tfrac{1}{M}\sum^M_{m} L(y, \blm) + L(y, \fM) \underbrace{- 2 \left(y - \tfrac{1}{M}\sum^M_{m=1}\blm\right)\left(y - \fM\right)}_{- 2 L\left(y, \fM\right)} \\[-.5em] \intertext{So, if we take the expected value over the data's distribution:}
         \Exy\left[L\left(y, \fM\right)\right] &= \tfrac{1}{M}\sum^M_{m} \Exy\left[L\left(y, \blm \right)\right] - \Exy\left[\ambifM\right] \end{align*}
\end{scriptsize}
\small
$\Rightarrow$ The expected loss of the ensemble is lower than the average loss of the single base learner by the amount of instability in the ensemble's base learners.\\ The more accurate and diverse the base learners, the better.
\normalsize
\framebreak
\end{vbframe}

\begin{vbframe}{Improving Bagging}
\begin{scriptsize}
\begin{align*}
\shortintertext{How to make $\Exy\left[\ambifM\right]$ as large as possible?}
\Exy\left[L\left(y, \fM \right)\right] &= \tfrac{1}{M}\sum^M_{m} \Exy\left[L\left(y, \blm \right)\right] - \Exy\left[\ambifM\right] \\
\shortintertext{Assume $\Exy\left[\blm\right] = 0$ for simplicity, $\var_{xy}\left[\blm\right] = \Exy\left[(\blm)^2\right] = \sigma^2$, $\corr_{xy}\left[\blm, \bl{m'}\right] = \rho$ for all $m, m'$.}
\implies 
\var_{xy}\left[\fM\right] &= \tfrac{1}{M} \sigma^2 +  \tfrac{M-1}{M} \rho \sigma^2 \qquad\left(... = \Exy\left[(\fM)^2\right]\right)\\
 \Exy\left[\ambifM\right] &= \tfrac{1}{M}\sum^M_{m} \Exy\left[\left(\blm- \fM\right)^2\right]\\
 & = \tfrac{1}{M}\left(M \Exy\left[(\blm)^2\right] + M \Exy\left[(\fM)^2\right] - 
     2 M \Exy\left[\blm\fM\right]\right) \\
  & = \sigma^2  + \Exy\left[(\fM)^2\right] - 2 \tfrac{1}{M}\sum^M_{m'} \underbrace{\Exy\left[\blm \bl{m'} \right]}_{\mathclap{\qquad\qquad\qquad\qquad= \cov_{xy}\left[\blm, \bl{m'} \right] + \Exy\left[\blm\right]\Exy\left[\bl{m'}\right]}} \\
  &=  \sigma^2  + \left(\tfrac{1}{M} \sigma^2 +   \tfrac{M-1}{M} \rho \sigma^2\right) - 2\left(\tfrac{M-1}{M} \rho\sigma^2 + \tfrac{1}{M}\sigma^2 + 0 \cdot 0 \right)\\
  &= \tfrac{M-1}{M} \sigma^2 (1-\rho)
\end{align*}
\end{scriptsize}

\begin{small}
\begin{align*}
\Exy\left[L\left(y, \fM\right)\right] &= \textcolor{blue}{\tfrac{1}{M}\sum^M_{m} \Exy\left[L\left(y, \blm \right)\right]} - \Exy\left[\ambifM\right]\\
\Exy\left[\ambifM\right] &\cong 
\textcolor{purple}{\frac{M-1}{M}} \textcolor{cyan}{\var_{xy}\left[\blm\right]} \textcolor{violet}{\left(1 - \corr_{xy}\left[\blm, \bl{m'}\right]\right)}
\end{align*}
\end{small}
\begin{itemize}
\item[$\Rightarrow$] \textcolor{blue}{\textbf{better base learners}} are better {\small (... duh)}
\item[$\Rightarrow$] \textcolor{purple}{\textbf{more base learners}} are better {\small (theoretically, at least...)}\\
\item[$\Rightarrow$] \textcolor{cyan}{\textbf{more variable base learners}} are better {\small(as long as their risk stays the same, of course!)}
\item[$\Rightarrow$] \textcolor{violet}{\textbf{less correlation between base learners}} is better:\\ bagging helps more if base learners are wrong in different ways so that their errors \enquote{cancel} each other out.\\
\end{itemize}


\end{vbframe}

\begin{vbframe}{Bagging: Synopsis}

  \begin{itemize}
    \item Basic idea: fit the same model repeatedly on many \textbf{bootstrap} replications of the training data set and \textbf{aggregate} the results
    \item Gains performance by reducing the variance of predictions, but (slightly) increases the bias: it reuses training data many times, so small mistakes can get amplified. 
    \item Works best for unstable/high-variance base learners, where small changes in the training set can cause large changes in predictions:\\
    e.g., CART, neural networks, step-wise/forward/backward variable selection for regression\\
     \item Works best if base learners' predictions are only weakly correlated: they don't all make the same mistakes.
         \item Can degrade performance for stable methods like $k$-NN, LDA, Naive Bayes, linear regression
  \end{itemize}
\end{vbframe}

\endlecture
\end{document}
