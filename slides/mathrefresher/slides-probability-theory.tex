%<<setup-child, include = FALSE>>=
%library(knitr)
%library(qrmix)
%library(mlr)
%library(quantreg)
%library(reshape2)
%set_parent("../style/preamble.Rnw")
%@

\usepackage[]{graphicx}
\usepackage[]{color}
% maxwidth is the original width if it is less than linewidth
% otherwise use linewidth (to make sure the graphics do not exceed the margin)
\makeatletter
\def\maxwidth{ %
  \ifdim\Gin@nat@width>\linewidth
    \linewidth
  \else
    \Gin@nat@width
  \fi
}
\makeatother

% ---------------------------------%
% latex-math dependencies, do not remove:
% - \usepackage{mathtools}
% - \usepackage{bm}
% - \usepackage{siunitx}
% - \usepackage{dsfont}
% - \usepackage{xspace}
% ---------------------------------%

%--------------------------------------------------------%
%       Language, encoding, typography
%--------------------------------------------------------%

\usepackage[english]{babel}
\usepackage[utf8]{inputenc} % Enables inputting UTF-8 symbols
% Standard AMS suite
\usepackage{amsmath,amsfonts,amssymb}

% Font four double-stroke / blackboard letters for sets of numbers (N, R, ...)
% Distribution name is "doublestroke"
% According to https://mirror.physik.tu-berlin.de/pub/CTAN/fonts/doublestroke/dsdoc.pdf
% the "bbm" package does a similar thing and may be superfluous.
% Required for latex-math
\usepackage{dsfont}

% bbm – "Blackboard-style" cm fonts (https://www.ctan.org/pkg/bbm)
% Used to be in common.tex, loaded directly after this file
% Maybe superfluous given dsfont is loaded
% TODO: Check if really unused?
% \usepackage{bbm}

% bm – Access bold symbols in maths mode - https://ctan.org/pkg/bm
% Required for latex-math
% https://tex.stackexchange.com/questions/3238/bm-package-versus-boldsymbol
\usepackage{bm}

% pifont – Access to PostScript standard Symbol and Dingbats fonts
% Used for \newcommand{\xmark}{\ding{55}, which is never used
% aside from lecture_advml/attic/xx-automl/slides.Rnw
% \usepackage{pifont}

% Quotes (inline and display), provdes \enquote
% https://ctan.org/pkg/csquotes
\usepackage{csquotes}

% Adds arg to enumerate env, technically superseded by enumitem according
% to https://ctan.org/pkg/enumerate
% Replace with https://ctan.org/pkg/enumitem ?
\usepackage{enumerate}

% Line spacing - provides \singlespacing \doublespacing \onehalfspacing
% https://ctan.org/pkg/setspace
% TODO: Check if really unused?
%\usepackage{setspace}

% mathtools – Mathematical tools to use with amsmath
% https://ctan.org/pkg/mathtools?lang=en
% latex-math dependency according to latex-math repo
\usepackage{mathtools}

%--------------------------------------------------------%
%       Displaying code and algorithms
%--------------------------------------------------------%
\usepackage{verbatim}
\usepackage{algorithm}
\usepackage{algpseudocode}

%--------------------------------------------------------%
%       Tables
%--------------------------------------------------------%

% multi-row table cells: https://www.namsu.de/Extra/pakete/Multirow.html
\usepackage{multirow}

% long/multi-page tables: https://texdoc.org/serve/longtable.pdf/0
% TODO: Check if really unused?

\usepackage{longtable}

% pretty table env: https://ctan.org/pkg/booktabs?lang=en
% TODO: Check if really unused?
\usepackage{booktabs}

%--------------------------------------------------------%
%       Figures: Creating, placing, verbing
%--------------------------------------------------------%

% wrapfig - Wrapping text around figures https://de.overleaf.com/learn/latex/Wrapping_text_around_figures
\usepackage{wrapfig}

% Sub figures in figures and tables
% https://ctan.org/pkg/subfig -- supersedes subfigure package
% TODO: Check if really unused?
\usepackage{subfig}

% Actually it's pronounced PGF https://en.wikibooks.org/wiki/LaTeX/PGF/TikZ
\usepackage{tikz}

\usetikzlibrary{shapes,arrows,automata,positioning,calc,chains,trees, shadows}
\tikzset{
  %Define standard arrow tip
  >=stealth',
  %Define style for boxes
  punkt/.style={
    rectangle,
    rounded corners,
    draw=black, very thick,
    text width=6.5em,
    minimum height=2em,
    text centered},
  % Define arrow style
  pil/.style={
    ->,
    thick,
    shorten <=2pt,
    shorten >=2pt,}
}


% Unsorted
% textpos – Place boxes at arbitrary positions on the LATEX page
% https://ctan.org/pkg/textpos?lang=en
% Provides \begin{textblock}
 % TODO: Check if really unused?
\usepackage[absolute,overlay]{textpos}

% psfrag – Replace strings in encapsulated PostScript figures
% https://www.overleaf.com/latex/examples/psfrag-example/tggxhgzwrzhn
% https://ftp.mpi-inf.mpg.de/pub/tex/mirror/ftp.dante.de/pub/tex/macros/latex/contrib/psfrag/pfgguide.pdf
% Can't tell if this is needed
% TODO: Check if really unused?
\usepackage{psfrag}

% Maybe not great to use this https://tex.stackexchange.com/a/197/19093
% Use align instead -- TODO: Global search & replace to check
\usepackage{eqnarray}

\usepackage{colortbl}

% arydshln – Draw dash-lines in array/tabular
% https://www.ctan.org/pkg/arydshln
% !! "arydshln has to be loaded after array, longtable, colortab and/or colortbl"
% Provides \hdashline and \cdashline
% TODO: Check if really unused?
% \usepackage{arydshln}

% tabularx – Tabulars with adjustable-width columns
% https://ctan.org/pkg/tabularx
% Provides \begin{tabularx}
% TODO: Check if really unused?
% \usepackage{tabularx}

% placeins – Control float placement
% https://ctan.org/pkg/placeins
% Defines a \FloatBarrier command
% TODO: Check if really unused?
% \usepackage{placeins}


% framed – Framed or shaded regions that can break across pages
% https://ctan.org/pkg/framed
% Provides \begin{framed} which uses \colorbox{shadecolor} relying on \definecolor{shadecolor}.
% TODO: Check if really unused?
% \usepackage{framed}

% Used often in conjunction with \definecolor{shadecolor}{rgb}{0.969, 0.969, 0.969}
% Might be able to be removed or at least redefined to only have shadecolor (if needed)
\definecolor{fgcolor}{rgb}{0.345, 0.345, 0.345}
\definecolor{shadecolor}{rgb}{0.969, 0.969, 0.969}
\newenvironment{knitrout}{}{} % an empty environment to be redefined in TeX


% Defines macros and environments
\usepackage{../../style/lmu-lecture}

\let\code=\texttt % Used regularly
\let\proglang=\textsf % Unused?

% Not sure what/why this does
\setkeys{Gin}{width=0.9\textwidth}

\setbeamertemplate{frametitle}{\expandafter\uppercase\expandafter\insertframetitle}

% Can't find a reason why common.tex is not just part of this file?

% basic latex stuff
\newcommand{\pkg}[1]{{\fontseries{b}\selectfont #1}} %fontstyle for R packages
\newcommand{\lz}{\vspace{0.5cm}} %vertical space
\newcommand{\dlz}{\vspace{1cm}} %double vertical space
\newcommand{\oneliner}[1] % Oneliner for important statements
{\begin{block}{}\begin{center}\begin{Large}#1\end{Large}\end{center}\end{block}}


%new environments
\newenvironment{vbframe}  %frame with breaks and verbatim
{
 \begin{frame}[containsverbatim,allowframebreaks]
}
{
\end{frame}
}

\newenvironment{vframe}  %frame with verbatim without breaks (to avoid numbering one slided frames)
{
 \begin{frame}[containsverbatim]
}
{
\end{frame}
}

\newenvironment{blocki}[1]   % itemize block
{
 \begin{block}{#1}\begin{itemize}
}
{
\end{itemize}\end{block}
}

\newenvironment{fragileframe}[2]{  %fragile frame with framebreaks
\begin{frame}[allowframebreaks, fragile, environment = fragileframe]
\frametitle{#1}
#2}
{\end{frame}}


\newcommand{\myframe}[2]{  %short for frame with framebreaks
\begin{frame}[allowframebreaks]
\frametitle{#1}
#2
\end{frame}}

\newcommand{\remark}[1]{
  \textbf{Remark:} #1
}


\newenvironment{deleteframe}
{
\begingroup
\usebackgroundtemplate{\includegraphics[width=\paperwidth,height=\paperheight]{../style/color/red.png}}
 \begin{frame}
}
{
\end{frame}
\endgroup
}
\newenvironment{simplifyframe}
{
\begingroup
\usebackgroundtemplate{\includegraphics[width=\paperwidth,height=\paperheight]{../style/color/yellow.png}}
 \begin{frame}
}
{
\end{frame}
\endgroup
}\newenvironment{draftframe}
{
\begingroup
\usebackgroundtemplate{\includegraphics[width=\paperwidth,height=\paperheight]{../style/color/green.jpg}}
 \begin{frame}
}
{
\end{frame}
\endgroup
}
% https://tex.stackexchange.com/a/261480: textcolor that works in mathmode
\makeatletter
\renewcommand*{\@textcolor}[3]{%
  \protect\leavevmode
  \begingroup
    \color#1{#2}#3%
  \endgroup
}
\makeatother


%-------------------------------------------------------------------------------------------------------%
%  Unused stuff that needs to go but is kept here currently juuuust in case it was important after all  %
%-------------------------------------------------------------------------------------------------------%

% \newcommand{\hlnum}[1]{\textcolor[rgb]{0.686,0.059,0.569}{#1}}%
% \newcommand{\hlstr}[1]{\textcolor[rgb]{0.192,0.494,0.8}{#1}}%
% \newcommand{\hlcom}[1]{\textcolor[rgb]{0.678,0.584,0.686}{\textit{#1}}}%
% \newcommand{\hlopt}[1]{\textcolor[rgb]{0,0,0}{#1}}%
% \newcommand{\hlstd}[1]{\textcolor[rgb]{0.345,0.345,0.345}{#1}}%
% \newcommand{\hlkwa}[1]{\textcolor[rgb]{0.161,0.373,0.58}{\textbf{#1}}}%
% \newcommand{\hlkwb}[1]{\textcolor[rgb]{0.69,0.353,0.396}{#1}}%
% \newcommand{\hlkwc}[1]{\textcolor[rgb]{0.333,0.667,0.333}{#1}}%
% \newcommand{\hlkwd}[1]{\textcolor[rgb]{0.737,0.353,0.396}{\textbf{#1}}}%
% \let\hlipl\hlkwb

% \makeatletter
% \newenvironment{kframe}{%
%  \def\at@end@of@kframe{}%
%  \ifinner\ifhmode%
%   \def\at@end@of@kframe{\end{minipage}}%
%   \begin{minipage}{\columnwidth}%
%  \fi\fi%
%  \def\FrameCommand##1{\hskip\@totalleftmargin \hskip-\fboxsep
%  \colorbox{shadecolor}{##1}\hskip-\fboxsep
%      % There is no \\@totalrightmargin, so:
%      \hskip-\linewidth \hskip-\@totalleftmargin \hskip\columnwidth}%
%  \MakeFramed {\advance\hsize-\width
%    \@totalleftmargin\z@ \linewidth\hsize
%    \@setminipage}}%
%  {\par\unskip\endMakeFramed%
%  \at@end@of@kframe}
% \makeatother

% \definecolor{shadecolor}{rgb}{.97, .97, .97}
% \definecolor{messagecolor}{rgb}{0, 0, 0}
% \definecolor{warningcolor}{rgb}{1, 0, 1}
% \definecolor{errorcolor}{rgb}{1, 0, 0}
% \newenvironment{knitrout}{}{} % an empty environment to be redefined in TeX

% \usepackage{alltt}
% \newcommand{\SweaveOpts}[1]{}  % do not interfere with LaTeX
% \newcommand{\SweaveInput}[1]{} % because they are not real TeX commands
% \newcommand{\Sexpr}[1]{}       % will only be parsed by R
% \newcommand{\xmark}{\ding{55}}%

% math spaces
\newcommand{\N}{\mathds{N}}                                                 % N, naturals
\newcommand{\Z}{\mathds{Z}}                                                 % Z, integers
\newcommand{\Q}{\mathds{Q}}                                                 % Q, rationals
\newcommand{\R}{\mathds{R}}                                                 % R, reals
\newcommand{\C}{\mathds{C}}                                                 % C, complex
\newcommand{\HS}{\mathcal{H}}                                               % H, hilbertspace
\newcommand{\continuous}{\mathcal{C}}                                       % C, space of continuous functions
\newcommand{\M}{\mathcal{M}} 												% machine numbers
\newcommand{\epsm}{\epsilon_m} 												% maximum error


% basic math stuff
\newcommand{\xt}{\tilde x}													% x tilde
\def\argmax{\mathop{\sf arg\,max}}                                          % argmax
\def\argmin{\mathop{\sf arg\,min}}                                          % argmin
\newcommand{\sign}{\operatorname{sign}}                                     % sign, signum
\newcommand{\I}{\mathbb{I}}                                                 % I, indicator
\newcommand{\order}{\mathcal{O}}                                            % O, order
\newcommand{\fp}[2]{\frac{\partial #1}{\partial #2}}                        % partial derivative
\newcommand{\pd}[2]{\frac{\partial{#1}}{\partial #2}}						% partial derivative

% sums and products
\newcommand{\sumin}{\sum_{i=1}^n}											% summation from i=1 to n
\newcommand{\sumkg}{\sum_{k=1}^g}											% summation from k=1 to g
\newcommand{\prodin}{\prod_{i=1}^n}											% product from i=1 to n
\newcommand{\prodkg}{\prod_{k=1}^g}											% product from k=1 to g

% linear algebra
\newcommand{\one}{\boldsymbol{1}}                                           % 1, unitvector
\newcommand{\id}{\mathrm{I}}                                                % I, identity
\newcommand{\diag}{\operatorname{diag}}                                     % diag, diagonal
\newcommand{\trace}{\operatorname{tr}}                                      % tr, trace
\newcommand{\spn}{\operatorname{span}}                                      % span
\newcommand{\scp}[2]{\left\langle #1, #2 \right\rangle}                     % <.,.>, scalarproduct
\newcommand{\mat}[1]{ 														% short pmatrix command
	\begin{pmatrix}
		#1
	\end{pmatrix}
}
\newcommand{\Amat}{\bm{A}}													% matrix A
\newcommand{\xv}{\bm{x}}													% vector x (bold)
\newcommand{\yv}{\bm{y}}														% vector y (bold)
\newcommand{\Deltab}{\bm{\Delta}}											% error term for vectors
															

% basic probability + stats
\renewcommand{\P}{\mathds{P}}                                               % P, probability
\newcommand{\E}{\mathds{E}}                                                 % E, expectation
\newcommand{\var}{\mathsf{Var}}                                             % Var, variance
\newcommand{\cov}{\mathsf{Cov}}                                             % Cov, covariance
\newcommand{\corr}{\mathsf{Corr}}                                           % Corr, correlation
\newcommand{\normal}{\mathcal{N}}                                           % N of the normal distribution
\newcommand{\iid}{\overset{i.i.d}{\sim}}                                    % dist with i.i.d superscript
\newcommand{\distas}[1]{\overset{#1}{\sim}}                                 % ... is distributed as ... 
% machine learning

%%%%%% ml - data
\newcommand{\Xspace}{\mathcal{X}}                                           % X, input space
\newcommand{\Yspace}{\mathcal{Y}}                                           % Y, output space
\newcommand{\nset}{\{1, \ldots, n\}}                                        % set from 1 to n
\newcommand{\pset}{\{1, \ldots, p\}}                                        % set from 1 to p
\newcommand{\gset}{\{1, \ldots, g\}}                                        % set from 1 to g
\newcommand{\Pxy}{\P_{xy}}                                                  % P_xy
\newcommand{\xy}{(x, y)}                                                    % observation (x, y)
\newcommand{\xvec}{(x_1, \ldots, x_p)^T}                                    % (x1, ..., xp) 
\newcommand{\D}{\mathcal{D}}                                                % D, data 
\newcommand{\Dset}{\{ (x^{(1)}, y^{(1)}), \ldots, (x^{(n)},  y^{(n)})\}}    % {(x1,y1)), ..., (xn,yn)}, data
\newcommand{\xdat}{\{ x^{(1)}, \ldots, x^{(n)}\}}   						 % {x1, ..., xn}, input data
\newcommand{\ydat}{\mathbf{y}}                                              % y (bold), vector of outcomes
\newcommand{\yvec}{(y^{(1)}, \hdots, y^{(n)})^T}                            % (y1, ..., yn), vector of outcomes
\renewcommand{\xi}[1][i]{x^{(#1)}}                                          % x^i, i-th observed value of x
\newcommand{\yi}[1][i]{y^{(#1)}}                                            % y^i, i-th observed value of y 
\newcommand{\xyi}{(\xi, \yi)}                                               % (x^i, y^i), i-th observation
\newcommand{\xivec}{(x^{(i)}_1, \ldots, x^{(i)}_p)^T}                       % (x1^i, ..., xp^i), i-th observation vector
\newcommand{\xj}{x_j}                                                       % x_j, j-th feature
\newcommand{\xjb}{\mathbf{x}_j}                                             % x_j (bold), j-th feature vecor
\newcommand{\xjvec}{(x^{(1)}_j, \ldots, x^{(n)}_j)^T}                       % (x^1_j, ..., x^n_j), j-th feature vector
\newcommand{\Dtrain}{\mathcal{D}_{\text{train}}}                            % D_train, training set
\newcommand{\Dtest}{\mathcal{D}_{\text{test}}}                              % D_test, test set

%%%%%% ml - models general

% continuous prediction function f
\newcommand{\fx}{f(x)}                                                      % f(x), continuous prediction function
\newcommand{\Hspace}{H}														% hypothesis space where f is from
\newcommand{\fh}{\hat{f}}                                                   % f hat, estimated prediction function
\newcommand{\fxh}{\fh(x)}                                                   % fhat(x)
\newcommand{\fxt}{f(x | \theta)}                                            % f(x | theta)
\newcommand{\fxi}{f(\xi)}                                                   % f(x^(i))
\newcommand{\fxih}{\hat{f}(\xi)}                                            % f(x^(i))
\newcommand{\fxit}{f(x^{(i)} | \theta)}                                     % f(x^(i) | theta)
\newcommand{\fhD}{\fh_{\D}}                                                 % fhat_D, estimate of f based on D
\newcommand{\fhDtrain}{\fh_{\Dtrain}}                                       % fhat_Dtrain, estimate of f based on D

% discrete prediction function h
\newcommand{\hx}{h(x)}                                                      % h(x), discrete prediction function
\newcommand{\hh}{\hat{h}}                                                   % h hat
\newcommand{\hxh}{\hat{h}(x)}                                               % hhat(x)
\newcommand{\hxt}{h(x | \theta)}                                            % h(x | theta)
\newcommand{\hxi}{h(\xi)}                                                   % h(x^(i))
\newcommand{\hxit}{h(x^{(i)} | \theta)}                                     % h(x^(i) | theta)

% yhat
\newcommand{\yh}{\hat{y}}                                                   % y hat for prediction of target
\newcommand{\yih}{\hat{y}}                                                  % y hat for prediction of target

% theta
\newcommand{\thetah}{\hat{\theta}}                                          % theta hat

% densities + probabilities
% pdf of x 
\newcommand{\pdf}{p}                                                        % p
\newcommand{\pdfx}{p(x)}                                                    % p(x)
\newcommand{\pixt}{\pi(x | \theta)}                                         % pi(x|theta), pdf of x given theta

% pdf of (x, y)
\newcommand{\pdfxy}{p(x,y)}                                                 % p(x, y)
\newcommand{\pdfxyt}{p(x, y | \theta)}                                      % p(x, y | theta)
\newcommand{\pdfxyit}{p(\xi, \yi | \theta)}                                 % p(x^(i), y^(i) | theta)

% pdf of x given y
\newcommand{\pdfxyk}{p(x | y=k)}                                            % p(x | y = k)
\newcommand{\lpdfxyk}{\log \pdfxyk}                                         % log p(x | y = k)
\newcommand{\pdfxiyk}{p(\xi | y=k)}                                         % p(x^i | y = k)

% prior probabilities
\newcommand{\pik}{\pi_k}                                                    % pi_k, prior
\newcommand{\lpik}{\log \pik}                                               % log pi_k, log of the prior

% posterior probabilities
\newcommand{\post}{\P(y = 1 | x)}                                           % P(y = 1 | x), post. prob for y=1
\newcommand{\pix}{\pi(x)}                                                   % pi(x), P(y = 1 | x)
\newcommand{\postk}{\P(y = k | x)}                                          % P(y = k | y), post. prob for y=k
\newcommand{\pikx}{\pi_k(x)}                                                % pi_k(x), P(y = k | x)
\newcommand{\pikxt}{\pi_k(x | \theta)}                                      % pi_k(x | theta), P(y = k | x, theta)
\newcommand{\pijx}{\pi_j(x)}                                                % pi_j(x), P(y = j | x)
\newcommand{\pdfygxt}{p(y |x, \theta)}                                      % p(y | x, theta)
\newcommand{\pdfyigxit}{p(\yi |\xi, \theta)}                                % p(y^i |x^i, theta)
\newcommand{\lpdfygxt}{\log \pdfygxt }                                      % log p(y | x, theta)
\newcommand{\lpdfyigxit}{\log \pdfyigxit}                                   % log p(y^i |x^i, theta)
\newcommand{\pixh}{\hat \pi(x)}                                             % pi(x) hat, P(y = 1 | x) hat
\newcommand{\pikxh}{\hat \pi_k(x)}                                          % pi_k(x) hat, P(y = k | x) hat

% residual and margin
\newcommand{\eps}{\epsilon}                                                 % residual, stochastic
\newcommand{\epsi}{\epsilon^{(i)}}                                          % epsilon^i, residual, stochastic
\newcommand{\epsh}{\hat{\epsilon}}                                          % residual, estimated
\newcommand{\yf}{y \fx}                                                     % y f(x), margin
\newcommand{\yfi}{\yi \fxi}                                                 % y^i f(x^i), margin
\newcommand{\Sigmah}{\hat \Sigma}											% estimated covariance matrix
\newcommand{\Sigmahj}{\hat \Sigma_j}										% estimated covariance matrix for the j-th class

% ml - loss, risk, likelihood
\newcommand{\Lxy}{L(y, f(x))}                                               % L(y, f(x)), loss function
\newcommand{\Lxyi}{L(\yi, \fxi)}                                            % L(y^i, f(x^i))
\newcommand{\Lxyt}{L(y, \fxt)}                                              % L(y, f(x | theta))
\newcommand{\Lxyit}{L(\yi, \fxit)}                                          % L(y^i, f(x^i | theta)
\newcommand{\risk}{\mathcal{R}}                                             % R, risk
\newcommand{\riskf}{\risk(f)}                                               % R(f), risk
\newcommand{\riske}{\mathcal{R}_{\text{emp}}}                               % R_emp, empirical risk
\newcommand{\riskef}{\riske(f)}                                             % R_emp(f)
\newcommand{\risket}{\mathcal{R}_{\text{emp}}(\theta)}                      % R_emp(theta)
\newcommand{\riskr}{\mathcal{R}_{\text{reg}}}                               % R_reg, regularized risk
\newcommand{\riskrt}{\mathcal{R}_{\text{reg}}(\theta)}                      % R_reg(theta)
\newcommand{\riskrf}{\riskr(f)}                                             % R_reg(f)
\newcommand{\LL}{\mathcal{L}}                                               % L, likelihood
\newcommand{\LLt}{\mathcal{L}(\theta)}                                      % L(theta), likelihood
\renewcommand{\ll}{\ell}                                                    % l, log-likelihood
\newcommand{\llt}{\ell(\theta)}                                             % l(theta), log-likelihood
\newcommand{\LS}{\mathfrak{L}}                                              % ????????????
\newcommand{\TS}{\mathfrak{T}}                                              % ??????????????
\newcommand{\errtrain}{\text{err}_{\text{train}}}                           % training error
\newcommand{\errtest}{\text{err}_{\text{test}}}                             % training error
\newcommand{\errexp}{\overline{\text{err}_{\text{test}}}}                   % training error

% resampling
\newcommand{\GE}[1]{GE(\fh_{#1})}                                           % Generalization error GE
\newcommand{\GEh}[1]{\widehat{GE}_{#1}}                                     % Estimated train error
\newcommand{\GED}{\GE{\D}}                                                  % Generalization error GE
\newcommand{\EGEn}{EGE_n}                                                   % Generalization error GE
\newcommand{\EDn}{\E_{|D| = n}}                                             % Generalization error GE


% ml - irace
\newcommand{\costs}{\mathcal{C}} % costs
\newcommand{\Celite}{\theta^*} % elite configurations
\newcommand{\instances}{\mathcal{I}} % sequence of instances
\newcommand{\budget}{\mathcal{B}} % computational budget
\input{../../latex-math/ml-hpo} 


\newcommand{\titlefigure}{figure_man/math_robot.jpg}
\newcommand{\learninggoals}{
	\item Refresher on the basics of probability theory
%	\item Bayes risk
%    \item Consistent learners
%    \item Bayes regret, estimation and approximation error    
%	\item Optimal constant model
%	\item Proper scoring rules
}

\title{Supervised Learning}
% \author{Bernd Bischl, Christoph Molnar, Daniel Schalk, Fabian Scheipl}
\institute{\href{https://compstat-lmu.github.io/lecture_i2ml/}{compstat-lmu.github.io/lecture\_i2ml}}
\date{}


\begin{document}

\lecturechapter{Refreshing Mathematical Tools}
\lecture{Supervised Learning}

\newcommand{\F}{\mathcal{F}}

\begin{vbframe}{Probability Space}
	
	\textbf{Probability space.} A probability space is a triple $(\Omega,\F,\P)$ which is modeling a specific random experiment/process.
		%		
		The components are
		%		
		\begin{itemize}
			%			
			\item a sample space $\Omega,$ which is the set of all possible outcomes of the random process modeled by the probability space.
			%			
			\item a $\sigma$-algebra $\F,$ which is a family of sets representing the allowable events of the random process modeled by the probability space.
			In particular, each set in $\F$ is a subset of the sample space $\Omega.$
			%
			\item a probability function $\P,$ which assigns each allowable event  a probability value, i.e., $\P:\F\to[0,1].$ It satisfies the following axioms of probability:
			%			
			\begin{itemize}
				%				
				\item  \emph{Completeness ---} $\P(\Omega)=1,$  
				%				
				\item  \emph{$\sigma$-additivity ---} For any finite or countably infinite sequence of mutually disjoint events $E_1,E_2,\ldots,$ it holds that
				%				
				$$\P(\cup_{i\geq1}  E_i) = \sum\nolimits_{i\geq 1} \P(E_i).$$
				%
			\end{itemize}
			%
		\end{itemize}
		%		
		\framebreak
		Examples:
		%		
		\begin{itemize}
			%			
			\item \emph{Coin Tossing ---} Possible samples are $\Omega=\{\texttt{H},\texttt{T} \}$ with \texttt{H} resp.\ \texttt{T} representing ''heads'' resp.\ ''tails''. The allowable events are contained in $\F=\{\varnothing,  \{\texttt{H}\} , \{\texttt{T}\}, \{\texttt{H},\texttt{T}\}  \}.$ If the coin is fair, then $\P(\texttt{H})=\P(\texttt{T})=1/2.$
			%			
			\item \emph{Dice Rolling---} Possible samples are $\Omega=\{1,2,3,4,5,6\}.$  The allowable events are contained in $2^\Omega,$ i.e., the power set of $\Omega.$
			If the dice is fair, then $\P(i)=1/6$ for $i=1,\ldots,6.$
			%			
		\end{itemize}
		%
		\textbf{Further properties.} For any probability space $(\Omega,\F,\P)$ the following properties hold
		%		
		\begin{itemize}
			%			
			\item \emph{Monotonicity ---} $A,B\in \F,$ and $A\subset B,$ then $\P(A)\leq \P(B).$
			%			
			\item \emph{Union bound ---}  For any finite or countably infinite sequence events $E_1,E_2,\ldots,$ 
			%			
			$$\P(\cup_{i\geq1}  E_i) \leq \sum\nolimits_{i\geq 1} \P(E_i).$$
			%			
		\end{itemize}

\end{vbframe}


\begin{vbframe}{Running Exercise}  
An urn contains five blue, three green, and one red ball. Two balls are randomly selected (without replacement).

\lz



What is the sample space of this experiment?


\lz

\lz
\lz
\lz

What is the probability of each point in the sample space?



\end{vbframe}

\begin{vbframe}{Independence}  


\textbf{Independence of events:} 
%		
\begin{itemize}
	%		
	\item 		Two events $A,B\in \F$ are \emph{independent} iff $\P(A \cap B)=\P(A)\P(B).$
	%			
	\item Events $E_1,\ldots,E_n \in \F$ are called \emph{pairwise independent} iff any pair $E_i,E_j$ with $i \neq j$ is independent.
	%	
	\item Events $E_1,\ldots,E_n \in \F$ are called \emph{mutually independent} iff for any  subset $I \subset \{1,\ldots,n\}$ it holds that  $\P(\bigcap_{i \in I} E_i )= \prod_{i \in I}\P(E_i).$
	%		
\end{itemize}
%		
Note that pairwise independence does not imply mutual independence! 

\lz

Example: One urn with four balls with the labels $110,101,011,000.$ We select one ball randomly.
\begin{itemize}
%	
	\item For $i=1,2,3$ let $E_i:=\{ \mbox{Selected ball has zero at the $i$-th position}  \}.$ 
%	
	\item $\P(E_i) = 1/2,$ $i=1,2,3$ and $\P(E_1 \cap E_2) = \P(E_1 \cap E_3) = \P(E_2 \cap E_3) =1/4.$
%	
	\item But $\P(E_1 \cap E_2 \cap E_3) = 1/4 \neq 1/8 = \P(E_1)\P(E_2)\P(E_3).$
%	
\end{itemize}
 


\framebreak

%
\textbf{Conditional probability.} The \emph{conditional probability} that event $A\in \F$ occurs given that event $B\in\F$ occurs is $\P(A|B)= \frac{\P(A\cap B)}{\P(B)}.$ The conditional probability is well-defined only if $\P(B) > 0.$


%	
\textbf{Bayes rule.} For two events $A,B\in \F$  it holds that $$\P(A|B)=\frac{\P(B|A) \P(A)}{\P(B)}.$$
%
\lz 

%
\textbf{The law of total probability.} Let $E_1,\ldots,E_n \in \F$  be mutually disjoint events, such that $\cup_{i=1}^n E_i = \Omega,$ then $ \forall A\in \F,$ {  $$\P(A) = \sum_{i=1}^n \P(A \cap E_i) = \sum_{i=1}^n \P(A | E_i) \P(E_i).$$}
%	

\end{vbframe}


\begin{vbframe}{Running Exercise}  
	An urn contains five blue, three green, and one red ball. Two balls are randomly selected (without replacement).
	
	\lz
	
	
	
	Consider the event $A=\{\mbox{First ball is red}\}$ and the event $B=\{\mbox{Second ball is red}\}.$ 
	Are these events independent?
	
	
	\lz
	
	\lz
	\lz
	\lz
	
	Are the events independent if we put a ball back into the urn after each selection?
	
	
	
\end{vbframe}


\begin{vbframe}{Random variables}  
 
 \textbf{Random variables.} A random variable $X$ on a sample space $\Omega$ is a real-valued function on $\Omega,$ that is $X:\Omega \to \R.$ The following observations can be made:
 %
 \begin{itemize}
 	%		
 	\item 	A random variable defines a probability space $(\Omega_X,\F_X,\P_X)$ with $\Omega_X= Im(X)$ and $\P_X(A)=\P(\{\omega\in\Omega \, | \, X(\omega) \in A \})$ for any $A\subset \Omega_X.$ Usually, one writes just $\P(X\in A)$ to denote the latter term, which is the \emph{probability distribution of $X$.}\\
 	
 	{\tiny (Technical remark: $\F_X$ is usually the Borel-$\sigma$-algebra on $\R$.)}
 	%	
 	\item In practical applications oftentimes the original probability space $(\Omega,\F,\P)$ is not the interesting object, but rather the induced probability space by $X.$ 
 	%		
 	One is rather interested in the probability of the outcome of the random variable:
 	%		
 	\begin{itemize}
 		%			
 		\item \emph{Coin tossing ---} Let $X$ be the random variable counting the number of tails after 100 flips.
 		%			
 		%			The 
 		%			
 		\item \emph{Financial market ---} Let $X_t$ be the  price of some asset in a future time $t.$
 		%			
 		%			
 	\end{itemize}
 	%	
 	\framebreak
 	%
 	\item Functions of random variables are again random variables, i.e., if $f:\R \to \R$ is some (measurable) function, then $Z=f(X)$ is also a random variable.\\
 	%	
% 	{\tiny (Technical remark: to be rigorously one has to require that the function is measurable.)}
 	%	
 	\item Identical distributed --- Two random variables $X$ and $Y$ are identical distributed if their probability distributions coincide, i.e., $\P_X = \P_Y.$ 
 	%		
 	%	\item The \emph{cumulative distribution function} of a random variable  determines the probability that the random variable will take a value smaller than or equal a specific value.
 	%%	\\
 	%	More specifically, $F_X:\R \to [0,1]$ with $F_X(x) := \P(X \leq x)$ for any $x\in \R$ is the cumulative distribution function of the random variable $X.$
 	%
 \end{itemize}
 %	
 One distinguishes between two types of random variables:
 %	
 \begin{itemize}
 	%		
 	
 	\item A \emph{discrete random variable} is a random variable that can take only a finite or countably
 	infinite number of values. Its probability distribution is determined by the \emph{probability mass function}  which assigns a probability to each value in the image of $X.$
 	%	
 	\item A \emph{continuous random variable} is a random variable which can take uncountably
 	infinite number of values. Usually its probability distribution is determined by a \emph{density function,} which assigns probabilities to intervals of the image of $X.$
 	%		
 \end{itemize}


\end{vbframe}



\begin{vbframe}{Discrete random variables} 

If the image $\Omega_X$ of $X$ is discrete (e.g., finite or countably infinite), then $X$ is called a \emph{discrete RV}.
 
 \lz
 
 For a discrete RV $X$, the function 
 $$p:\, \Omega_X \to [0,1], \,  x \mapsto  \P \left(X \in \{x\} \right) = \P \left(X = x \right)
 $$ 
 is called a probability function or probability mass function of $X$. 
 
 
 \lz 
 
 Obviously, $p(x) \geq 0$ and $\sum_{x \in \Omega_X} p(x) = 1$.
% 
 
	Examples:
	\begin{itemize}
		\item \emph{Bernoulli distribution}: For a binary RV with $\Omega_X = \{0,1\}$, $X \sim \mathrm{Ber}(\theta)$ if $p(1) = \theta$ and $p(0)= 1 - \theta$.
		\item \emph{Binomial distribution}: $X \sim \mathrm{Bin}(n,\theta)$ if $$
p(k) = \left\{ \begin{array}{cl}
	{n \choose k} \theta^k (1-\theta)^{n-k} & \text{ if } k \in \{0, \ldots , n\}\\
	0 & \text{ otherwise}
\end{array} \right. \, .
$$
	\end{itemize}
 


\end{vbframe}


\begin{vbframe}{Continuous random variables} 
	
	$X$ is a \emph{continuous RV} if $\Omega_X$ is non-discrete and there exists a function $p: \R \to \R$ such that 
%	
	\begin{itemize}
%		
		\item 
		$p(x) \geq 0$ for all $x \in \mathbb{R}$
%		
		\item $\int_{-\infty}^\infty p(x) \, dx = 1$,
%		
		\item for all $a \leq b$ it holds that	$
		\P(a \leq X \leq b) = \int_a^b p(x) \, dx .
		$
%
	\end{itemize} 
% 
	The function $p$ is called the probability density function (PDF) of $X$.
	
	\lz
	
	Examples.
	
	\begin{itemize}
			\item \emph{Uniform distribution}: $X \sim U(a,b)$ if 
		$$
		p(x) = \left\{ \begin{array}{cl}
			1/(b-a) & \text{ if } a \leq x \leq b  \\
			0 & \text{ otherwise}
		\end{array} \right. .
		$$
		\item \emph{Normal/Gaussian distribution}: $X \sim \normal(\mu, \sigma)$ if
		$$
		p(x) = \frac{1}{\sigma \sqrt{2\pi}} \exp \left( -\frac{(x- \mu)^2 }{2 \sigma^2} \right)\, .
		$$ 
	\end{itemize}
\end{vbframe}

\begin{vbframe}{Cumulative distribution function}
 
 The cumulative distribution function (CDF) of a random variable $X$ is the function
		$$
		F_X:\, \mathbb{R} \to [0,1], \, x \mapsto \P(X \leq x) \, .
		$$
		
A CDF fully characterizes a RV: If $F_X(x) = F_Y(x)$ for all $x \in \mathbb{R}$, then $X$ and $Y$ are identical distributed.
		
		\lz
		
%		A function $F:\, \mathbb{R} \to [0,1]$ is a CDF for some probability measure $\P$ if and only if the following properties are satisfied: 
%\begin{itemize}
%	\item $F$ is non-decreasing, i.e., $x_1 < x_2$ implies $F(x_1) \leq F(x_2)$;
%	\item $F$ is normalized: $\lim_{x \rightarrow - \infty} F(x) = 0$ and $\lim_{x \rightarrow \infty} F(x) = 1$;
%	\item $F$ is right-continuous, i.e., $\lim_{y \downarrow x} F(y) = F(x)$ for all $x \in \mathbb{R}$.
%\end{itemize}

If $X$ is 
\begin{itemize}
%	
	\item discrete with probability mass function $p$, then for all $x \in \mathbb{R}$, 
	$
	F_X(x) = \sum_{y \in \Omega_X \cap (-\infty, x]} p(y) \, .
	$
%	
	\item continuous with probability density function $p$, then 
	$
	F_X(x)  = \int_{-\infty}^x p(t) \, dt 
	$
	for all $x \in \mathbb{R}$, and $p(x) = F_X'(x)$ whenever $F_X$ is differentiable at $x$. 
%	
\end{itemize}

\end{vbframe}


\begin{vbframe}{Running Exercise}  
	An urn contains five blue, three green, and one red ball. Two balls are randomly selected (without replacement).
	
	\lz
	
	
	
	Let $X$ be the number of green balls selected. What are the possible values of $X?$
	
	
	\lz
	
	\lz
	\lz
	\lz
	
	What is the cumulative distribution function of $X$?
	
	
	
\end{vbframe}

\begin{frame}[t]{Expected value/Expectation} 
%	
	Expectation is the most basic characteristic of a random variable. Let $X$ be a random variable, then the expectation of $X$, denoted by $\E(X),$ is 
	%	
	$$	\E(X) = \int x \, dF(x) = \left\{ \begin{array}{ll}
		\sum_{x \in \Omega_X} x \, p(x) & \text{ if } X \text{ is discrete} \\
		\int_{\Omega_X} x \, p(x) \, dx& \text{ if } X \text{ is continuous} 
	\end{array} \right.  	$$
	%	
 provided the sum resp.\ the integral is well-defined and exists.
	
	%	
	Some important properties of the expected value are:
	%
	\begin{itemize}
		%		
		\item Linearity --- For any constants $c_1,c_2 \in \R$ and any pair of random variables $X$ and $Y$ it holds that $ \E(c_1 X + c_2 Y) = c_1 \E(X) + c_2 \E(Y).$
		%		 
%		\item Independence --- If $X$ and $Y$ are independent random variables, then $\E(XY)=\E(X)\E(Y).$
		%		
		\item Transformations --- If $f:\R \to \R$  is a (measurable) function, then the expectation of $f(X)$ is 
		%		
		$$\E(f(X)) = \int f(x) \, dF(x) = \left\{ \begin{array}{ll}
			\sum_{x \in \Omega_X} f(x) \, p(x) & \text{ if } X \text{ is discrete} \\
			\int_{\Omega_X} f(x) \, p(x) \, dx& \text{ if } X \text{ is continuous} 
		\end{array} \right.  	$$   
		%		 
		{\tiny (provided the sum resp.\ integral exists.)}
		% 
		%		
	\end{itemize}
\end{frame}


\begin{vbframe}{Running Exercise}  
	An urn contains five blue, three green, and one red ball. Two balls are randomly selected (without replacement).
	
	\lz
	
	
	
	Let $X$ be the number of green balls selected. What is the expected value of $X?$
	
	 
	
	
	
\end{vbframe}


\begin{vbframe}{Variance and covariance}
%	 
		The variance of a RV $X$ is defined as follows:
		$$
		\var(X) = \E \left[ (X - \E(X))^2 \right] = \int_{\Omega_X} (x-\E(X))^2  \, dF(x)  \, ,
		$$
		provided the integral on the right-hand side exists. 
		
		\lz
		 
		The \emph{standard deviation} is defined by $\sqrt{\var(X)}$.
		
		
		\lz
		 
		The \emph{covariance} between RVs $X$ and $Y$ is  
		$$
		\cov(X,Y) = \E \left[  (X- \E[X]) (Y - \E[Y]) \right]
		$$
%		 
\end{vbframe}



\begin{frame}[t]{Multivariate random variables}
	
	
	RVs $X_1  , \ldots  , X_n$ over the same probability space can be combined into a \emph{random vector} $\bm{X}=(X_1, \ldots , X_n)$. 
	
	\lz
	
	Their joint distribution is specified by the joint mass/density function $p_{\bm{X}}$, such that for any measurable set $A$ it holds that
	$$ \P(\bm{X}  \in A) =  \left\{ \begin{array}{ll}
		\sum_{  (x_1 , \ldots , x_n) \in A  } p_{\bm{X}}(x_1 , \ldots , x_n) & \text{ if } X_1,\ldots,X_n \text{ are discrete} \\
		\int_{  A}   p_{\bm{X}}(x_1 , \ldots , x_n) \, dx_1\ldots dx_n& \text{ if }  X_1,\ldots,X_n  \text{ are continuous} 
	\end{array} \right.      $$ 
%
	The marginal distribution $p_1$ of $X_1$ is given by
	\[
	p_1(x_1) =  \left\{ \begin{array}{ll}
		\sum_{  (x_2 , \ldots , x_n) \in \Omega_{X_2} \times \ldots \times \Omega_{X_n}  } p_{\bm{X}}(x_1 , \ldots , x_n) & \text{ if } \text{  discrete} \\
		\int_{  \Omega_{X_2} \times \ldots \times \Omega_{X_n} }   p_{\bm{X}}(x_1 , \ldots , x_n) \, dx_2\ldots dx_n& \text{ if }   \text{  continuous} 
	\end{array} \right.     
	\]
%	
	In the same way, the marginal distributions of $X_2, \ldots , X_n$ are defined. 
%	
 
	
%	\item 
	The same type of projection (summation/integration over all remaining variables) is used to define marginal distributions on subsets of variables $(X_{i_1} , \ldots  , X_{i_k})$ with 
	$\{ i_1 ,  \ldots  , i_k \} \subseteq \{1 , \ldots ,  n\}$.
	
	
\end{frame}



\begin{vbframe}{Running Exercise}  
	An urn contains five blue, three green, and one red ball. Two balls are randomly selected (without replacement).
	
	\lz
	
	
	
	Let $X_1$ be the number of green balls selected and $X_2$ the number of blue balls selected. What is the joint mass function of $(X_1,X_2)?$
	
	
	
	
	
\end{vbframe}


\begin{vbframe}{Independence of random variables}  
 
\begin{itemize}
	%		
	\item 	Two discrete random variables X and Y are independent iff for any $x\in \Omega_X,y\in \Omega_Y$ it holds that $	p_{XY}(x,y) =	p_X(x) p_Y(y).$
	%	
	\item Random variables $X_1, \ldots, X_n$ are pairwise independent iff for any pair $i,j$ and any $x_i\in \Omega_{X_i},x_j\in \Omega_{X_j}$ it holds that $	p_{X_iX_j}(x_i,x_j) =	p_{X_i}(x_i) p_{X_j}(x_j).$
%	$	\P(X_i=x_i, X_j=x_j) =	\P(X_i=x_i)\P(X_j=x_j).$
	%		
	\item Random variables  $X_1, \ldots, X_n$ are mutually independent iff for any subset $I \subset \{1,\ldots,n\}$ it holds that the joint probability mass/density function of $(X_i)_{i\in I}$ is given by
	$  \prod_{i \in I}	p_{X_i}(x_i)$  for any $x_i \in \Omega_{X_i},$ $i\in I.$
%	$	\P( \bigcap_{i \in I} X_i=x_i ) =	\prod_{i \in I}\P(X_i=x_i)$ for $x_i \in Im(X_i).$
	%		
\end{itemize}
%
Similar to independence of events, pairwise independence of random variables does not imply their
mutual independence. If we say that random variables are independent without further specifications we
are referring to mutual independence.
%
\framebreak

%
 Some important properties and concepts with respect to independence are:
%	
\begin{itemize}
	%		
	\item The iid assumption --- Random variables  $X_1, \ldots, X_n$ are called  \emph{independent and identically distributed} (iid) iff they are mutually independent and each random variable has the same probability distribution as the others.
	%		
	\item Independence under transformations --- Let $X$ and $Y$ be independent random variables and $f,g:\R \to \R$ are (measurable) functions. Then, $f(X)$ and $g(Y)$ are independent as well. 
	%		
\end{itemize}
%
\end{vbframe}



\begin{frame}[t]{Conditional distributions}
	 
		 If $(X,Y)$ have a joint distribution with mass function $p_{X,Y}$, then the \emph{conditional probability mass function} for $X$ given $Y$ is defined by
		$$
		p_{X|Y}(x ~|~ y) = \frac{p(X=x, Y=y)}{p(Y = y)}  = \frac{p_{X,Y}(x,y)}{p_Y(y)} 
		$$
		provided $p(Y = y) > 0$.
		\lz
		
		Likewise, in the continuous case, the \emph{conditional probability density function} is given by
		$$
		p_{X|Y}(x ~|~ y) = \frac{p_{X,Y}(x,y)}{p_Y(y)} 
		$$
		provided $p_Y(y) > 0$. Then,
		$$
		p(X \in A ~|~ Y= y) = \int_A  p_{X|Y}(x ~|~ y) \, dx \, .
		$$
		The soundness of this definition is less obvious than in the discrete case, due to conditioning on an event of probability 0 here. 
		
\end{frame}


\begin{vbframe}{Conditional expected value/expectation} 
%	
	Let $X$ and $Y$ be random variables, then the conditional expectation of $X$ given $Y=y$, denoted by $\E(X|Y=y),$ is given by
	%	
	$$	\E(X|Y=y) = \left\{ \begin{array}{ll}
		\sum_{x \in \Omega_X} x \, p_{X|Y}(x ~|~ y) & \text{ discrete case} \\
		\int_{\Omega_X} x \, p_{X|Y}(x ~|~ y) \, dx& \text{ continuous case} 
	\end{array} \right.  	$$
	%	
	
	%	
	\emph{Interpretation:} The expected value of $X$ under the condition that $Y=y$ holds.
	%	
	\lz
	
	Note that $\E(X|Y=y)$ induces a mapping from $\Omega_Y$ to $\R$ according to $y \mapsto \E(X|Y=y).$
	%	
	This function is called the \emph{conditional expectation of $X$ given $Y$} and simply denoted by $\E(X|Y).$
	%	
	Note that $\E(X|Y)$ is a random variable!
	%
	
	\lz
	
	$\E(X|Y)$ can be also interpreted as a prediction of $X$ under the information encoded by the random variable $Y.$
	
	\framebreak
	
	
	%
	Some important properties of the conditional expected value are the following. For any random variables $X,Y,Z$ it holds that
	%
	\begin{itemize}
		%		
		\item Linearity ---  For any constants $c_1,c_2 \in \R$  it holds that $ \E(c_1 X + c_2 Y | Z) = c_1 \E(X|Z) + c_2 \E(Y|Z).$
		%		
		\item Independence --- If $X$ and $Y$ are independent random variables,Im then $\E(X|Y)=\E(X).$
		%		
		%		
		\item Transformations --- If $f:\R \to \R$  is a (measurable) function, then the conditional expectation of $f(X)$ given $Y=y$ is 
		%		
		$$\E(f(X)|Y=y)  = \left\{ \begin{array}{ll}
			\sum_{x \in \Omega_X} f(x) \, p_{X|Y}(x ~|~ y) & \text{ discrete case} \\
			\int_{\Omega_X} f(x) \, p_{X|Y}(x ~|~ y) \, dx & \text{ continuous case} 
		\end{array} \right.  	$$
		%		 
		%
		\item Law of total expectation ---
		$\E(\E(X|Y)) = \E(X)$.  
%		{\tiny (Note: Here, the outer expectation on the left hand-side is taken with respect to the distribution of $Y$.)}
		%		 
		\item Tower property --- $ \E(\E(X|Y,Z)|Y) = \E(X|Y).$
		%		
	\end{itemize}
	
\end{vbframe}


\begin{frame}[t]{Conditional variance} 
	%	
	Let $X$ and $Y$ be random variables, then the conditional variance of $X$ given $Y=y$, denoted by $\var(X|Y=y),$ is given by
	%	
	$$	\var(X|Y=y)  = \E \left[ (X - \E[X~|~Y=y])^2  ~|~ Y=y  \right]	$$
	%	
	
	%	
	\emph{Interpretation:} Variance of the prediction $\E[X~|~Y]$ for $X.$
	%	
	\lz
	
	Note that $\var(X|Y=y)$ induces a mapping from $\Omega_Y$ to $\R$ according to $y \mapsto \var(X|Y=y).$
	%	
	This function is called the \emph{conditional variance of $X$ given $Y$} and simply denoted by $\var(X|Y).$
	%	
	Note that $\var(X|Y)$ is a random variable!
	%
	
	\lz
	
	An important property is the law of total variance:
	
	$$  \var(X) = \E(\var(X~|~Y)) + \var(\E(X~|~Y)).  $$

	
\end{frame}




\endlecture

\end{document} 
