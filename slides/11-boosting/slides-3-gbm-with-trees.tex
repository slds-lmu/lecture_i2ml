
\input{../../2021/style/preamble4tex}
% math spaces
\newcommand{\N}{\mathds{N}}                                                 % N, naturals
\newcommand{\Z}{\mathds{Z}}                                                 % Z, integers
\newcommand{\Q}{\mathds{Q}}                                                 % Q, rationals
\newcommand{\R}{\mathds{R}}                                                 % R, reals
\newcommand{\C}{\mathds{C}}                                                 % C, complex
\newcommand{\HS}{\mathcal{H}}                                               % H, hilbertspace
\newcommand{\continuous}{\mathcal{C}}                                       % C, space of continuous functions
\newcommand{\M}{\mathcal{M}} 												% machine numbers
\newcommand{\epsm}{\epsilon_m} 												% maximum error


% basic math stuff
\newcommand{\xt}{\tilde x}													% x tilde
\def\argmax{\mathop{\sf arg\,max}}                                          % argmax
\def\argmin{\mathop{\sf arg\,min}}                                          % argmin
\newcommand{\sign}{\operatorname{sign}}                                     % sign, signum
\newcommand{\I}{\mathbb{I}}                                                 % I, indicator
\newcommand{\order}{\mathcal{O}}                                            % O, order
\newcommand{\fp}[2]{\frac{\partial #1}{\partial #2}}                        % partial derivative
\newcommand{\pd}[2]{\frac{\partial{#1}}{\partial #2}}						% partial derivative

% sums and products
\newcommand{\sumin}{\sum_{i=1}^n}											% summation from i=1 to n
\newcommand{\sumkg}{\sum_{k=1}^g}											% summation from k=1 to g
\newcommand{\prodin}{\prod_{i=1}^n}											% product from i=1 to n
\newcommand{\prodkg}{\prod_{k=1}^g}											% product from k=1 to g

% linear algebra
\newcommand{\one}{\boldsymbol{1}}                                           % 1, unitvector
\newcommand{\id}{\mathrm{I}}                                                % I, identity
\newcommand{\diag}{\operatorname{diag}}                                     % diag, diagonal
\newcommand{\trace}{\operatorname{tr}}                                      % tr, trace
\newcommand{\spn}{\operatorname{span}}                                      % span
\newcommand{\scp}[2]{\left\langle #1, #2 \right\rangle}                     % <.,.>, scalarproduct
\newcommand{\mat}[1]{ 														% short pmatrix command
	\begin{pmatrix}
		#1
	\end{pmatrix}
}
\newcommand{\Amat}{\bm{A}}													% matrix A
\newcommand{\xv}{\bm{x}}													% vector x (bold)
\newcommand{\yv}{\bm{y}}														% vector y (bold)
\newcommand{\Deltab}{\bm{\Delta}}											% error term for vectors
															

% basic probability + stats
\renewcommand{\P}{\mathds{P}}                                               % P, probability
\newcommand{\E}{\mathds{E}}                                                 % E, expectation
\newcommand{\var}{\mathsf{Var}}                                             % Var, variance
\newcommand{\cov}{\mathsf{Cov}}                                             % Cov, covariance
\newcommand{\corr}{\mathsf{Corr}}                                           % Corr, correlation
\newcommand{\normal}{\mathcal{N}}                                           % N of the normal distribution
\newcommand{\iid}{\overset{i.i.d}{\sim}}                                    % dist with i.i.d superscript
\newcommand{\distas}[1]{\overset{#1}{\sim}}                                 % ... is distributed as ... 
% machine learning

%%%%%% ml - data
\newcommand{\Xspace}{\mathcal{X}}                                           % X, input space
\newcommand{\Yspace}{\mathcal{Y}}                                           % Y, output space
\newcommand{\nset}{\{1, \ldots, n\}}                                        % set from 1 to n
\newcommand{\pset}{\{1, \ldots, p\}}                                        % set from 1 to p
\newcommand{\gset}{\{1, \ldots, g\}}                                        % set from 1 to g
\newcommand{\Pxy}{\P_{xy}}                                                  % P_xy
\newcommand{\xy}{(x, y)}                                                    % observation (x, y)
\newcommand{\xvec}{(x_1, \ldots, x_p)^T}                                    % (x1, ..., xp) 
\newcommand{\D}{\mathcal{D}}                                                % D, data 
\newcommand{\Dset}{\{ (x^{(1)}, y^{(1)}), \ldots, (x^{(n)},  y^{(n)})\}}    % {(x1,y1)), ..., (xn,yn)}, data
\newcommand{\xdat}{\{ x^{(1)}, \ldots, x^{(n)}\}}   						 % {x1, ..., xn}, input data
\newcommand{\ydat}{\mathbf{y}}                                              % y (bold), vector of outcomes
\newcommand{\yvec}{(y^{(1)}, \hdots, y^{(n)})^T}                            % (y1, ..., yn), vector of outcomes
\renewcommand{\xi}[1][i]{x^{(#1)}}                                          % x^i, i-th observed value of x
\newcommand{\yi}[1][i]{y^{(#1)}}                                            % y^i, i-th observed value of y 
\newcommand{\xyi}{(\xi, \yi)}                                               % (x^i, y^i), i-th observation
\newcommand{\xivec}{(x^{(i)}_1, \ldots, x^{(i)}_p)^T}                       % (x1^i, ..., xp^i), i-th observation vector
\newcommand{\xj}{x_j}                                                       % x_j, j-th feature
\newcommand{\xjb}{\mathbf{x}_j}                                             % x_j (bold), j-th feature vecor
\newcommand{\xjvec}{(x^{(1)}_j, \ldots, x^{(n)}_j)^T}                       % (x^1_j, ..., x^n_j), j-th feature vector
\newcommand{\Dtrain}{\mathcal{D}_{\text{train}}}                            % D_train, training set
\newcommand{\Dtest}{\mathcal{D}_{\text{test}}}                              % D_test, test set

%%%%%% ml - models general

% continuous prediction function f
\newcommand{\fx}{f(x)}                                                      % f(x), continuous prediction function
\newcommand{\Hspace}{H}														% hypothesis space where f is from
\newcommand{\fh}{\hat{f}}                                                   % f hat, estimated prediction function
\newcommand{\fxh}{\fh(x)}                                                   % fhat(x)
\newcommand{\fxt}{f(x | \theta)}                                            % f(x | theta)
\newcommand{\fxi}{f(\xi)}                                                   % f(x^(i))
\newcommand{\fxih}{\hat{f}(\xi)}                                            % f(x^(i))
\newcommand{\fxit}{f(x^{(i)} | \theta)}                                     % f(x^(i) | theta)
\newcommand{\fhD}{\fh_{\D}}                                                 % fhat_D, estimate of f based on D
\newcommand{\fhDtrain}{\fh_{\Dtrain}}                                       % fhat_Dtrain, estimate of f based on D

% discrete prediction function h
\newcommand{\hx}{h(x)}                                                      % h(x), discrete prediction function
\newcommand{\hh}{\hat{h}}                                                   % h hat
\newcommand{\hxh}{\hat{h}(x)}                                               % hhat(x)
\newcommand{\hxt}{h(x | \theta)}                                            % h(x | theta)
\newcommand{\hxi}{h(\xi)}                                                   % h(x^(i))
\newcommand{\hxit}{h(x^{(i)} | \theta)}                                     % h(x^(i) | theta)

% yhat
\newcommand{\yh}{\hat{y}}                                                   % y hat for prediction of target
\newcommand{\yih}{\hat{y}}                                                  % y hat for prediction of target

% theta
\newcommand{\thetah}{\hat{\theta}}                                          % theta hat

% densities + probabilities
% pdf of x 
\newcommand{\pdf}{p}                                                        % p
\newcommand{\pdfx}{p(x)}                                                    % p(x)
\newcommand{\pixt}{\pi(x | \theta)}                                         % pi(x|theta), pdf of x given theta

% pdf of (x, y)
\newcommand{\pdfxy}{p(x,y)}                                                 % p(x, y)
\newcommand{\pdfxyt}{p(x, y | \theta)}                                      % p(x, y | theta)
\newcommand{\pdfxyit}{p(\xi, \yi | \theta)}                                 % p(x^(i), y^(i) | theta)

% pdf of x given y
\newcommand{\pdfxyk}{p(x | y=k)}                                            % p(x | y = k)
\newcommand{\lpdfxyk}{\log \pdfxyk}                                         % log p(x | y = k)
\newcommand{\pdfxiyk}{p(\xi | y=k)}                                         % p(x^i | y = k)

% prior probabilities
\newcommand{\pik}{\pi_k}                                                    % pi_k, prior
\newcommand{\lpik}{\log \pik}                                               % log pi_k, log of the prior

% posterior probabilities
\newcommand{\post}{\P(y = 1 | x)}                                           % P(y = 1 | x), post. prob for y=1
\newcommand{\pix}{\pi(x)}                                                   % pi(x), P(y = 1 | x)
\newcommand{\postk}{\P(y = k | x)}                                          % P(y = k | y), post. prob for y=k
\newcommand{\pikx}{\pi_k(x)}                                                % pi_k(x), P(y = k | x)
\newcommand{\pikxt}{\pi_k(x | \theta)}                                      % pi_k(x | theta), P(y = k | x, theta)
\newcommand{\pijx}{\pi_j(x)}                                                % pi_j(x), P(y = j | x)
\newcommand{\pdfygxt}{p(y |x, \theta)}                                      % p(y | x, theta)
\newcommand{\pdfyigxit}{p(\yi |\xi, \theta)}                                % p(y^i |x^i, theta)
\newcommand{\lpdfygxt}{\log \pdfygxt }                                      % log p(y | x, theta)
\newcommand{\lpdfyigxit}{\log \pdfyigxit}                                   % log p(y^i |x^i, theta)
\newcommand{\pixh}{\hat \pi(x)}                                             % pi(x) hat, P(y = 1 | x) hat
\newcommand{\pikxh}{\hat \pi_k(x)}                                          % pi_k(x) hat, P(y = k | x) hat

% residual and margin
\newcommand{\eps}{\epsilon}                                                 % residual, stochastic
\newcommand{\epsi}{\epsilon^{(i)}}                                          % epsilon^i, residual, stochastic
\newcommand{\epsh}{\hat{\epsilon}}                                          % residual, estimated
\newcommand{\yf}{y \fx}                                                     % y f(x), margin
\newcommand{\yfi}{\yi \fxi}                                                 % y^i f(x^i), margin
\newcommand{\Sigmah}{\hat \Sigma}											% estimated covariance matrix
\newcommand{\Sigmahj}{\hat \Sigma_j}										% estimated covariance matrix for the j-th class

% ml - loss, risk, likelihood
\newcommand{\Lxy}{L(y, f(x))}                                               % L(y, f(x)), loss function
\newcommand{\Lxyi}{L(\yi, \fxi)}                                            % L(y^i, f(x^i))
\newcommand{\Lxyt}{L(y, \fxt)}                                              % L(y, f(x | theta))
\newcommand{\Lxyit}{L(\yi, \fxit)}                                          % L(y^i, f(x^i | theta)
\newcommand{\risk}{\mathcal{R}}                                             % R, risk
\newcommand{\riskf}{\risk(f)}                                               % R(f), risk
\newcommand{\riske}{\mathcal{R}_{\text{emp}}}                               % R_emp, empirical risk
\newcommand{\riskef}{\riske(f)}                                             % R_emp(f)
\newcommand{\risket}{\mathcal{R}_{\text{emp}}(\theta)}                      % R_emp(theta)
\newcommand{\riskr}{\mathcal{R}_{\text{reg}}}                               % R_reg, regularized risk
\newcommand{\riskrt}{\mathcal{R}_{\text{reg}}(\theta)}                      % R_reg(theta)
\newcommand{\riskrf}{\riskr(f)}                                             % R_reg(f)
\newcommand{\LL}{\mathcal{L}}                                               % L, likelihood
\newcommand{\LLt}{\mathcal{L}(\theta)}                                      % L(theta), likelihood
\renewcommand{\ll}{\ell}                                                    % l, log-likelihood
\newcommand{\llt}{\ell(\theta)}                                             % l(theta), log-likelihood
\newcommand{\LS}{\mathfrak{L}}                                              % ????????????
\newcommand{\TS}{\mathfrak{T}}                                              % ??????????????
\newcommand{\errtrain}{\text{err}_{\text{train}}}                           % training error
\newcommand{\errtest}{\text{err}_{\text{test}}}                             % training error
\newcommand{\errexp}{\overline{\text{err}_{\text{test}}}}                   % training error

% resampling
\newcommand{\GE}[1]{GE(\fh_{#1})}                                           % Generalization error GE
\newcommand{\GEh}[1]{\widehat{GE}_{#1}}                                     % Estimated train error
\newcommand{\GED}{\GE{\D}}                                                  % Generalization error GE
\newcommand{\EGEn}{EGE_n}                                                   % Generalization error GE
\newcommand{\EDn}{\E_{|D| = n}}                                             % Generalization error GE


% ml - irace
\newcommand{\costs}{\mathcal{C}} % costs
\newcommand{\Celite}{\theta^*} % elite configurations
\newcommand{\instances}{\mathcal{I}} % sequence of instances
\newcommand{\budget}{\mathcal{B}} % computational budget
% ml - bagging, random forest

\newcommand{\bl}[1]{b^{[#1]}(x)}											% baselearner with argument for m
\newcommand{\blm}{\bl{m}}												    % baselearner without argument for m
\newcommand{\blmh}{\hat{b}^{[m]}(x)}										% estimated base learner 

\input{../../latex-math/ml-boosting.tex}
% ml - trees, extra trees

\newcommand{\Np}{\mathcal{N}}												% Parent node N
\newcommand{\Nl}{\Np_1}														% Left node N_1
\newcommand{\Nr}{\Np_2}														% Right node N_2


\begin{document}

\lecturechapter{11}{Gradient Boosting with Trees}
\lecture{Fortgeschrittene Computerintensive Methoden}


\begin{vbframe}{Gradient boosting with trees}

Trees are mainly used as base learners for gradient boosting in ML.
A great deal of research has been done on this combination so far, and it often provides the best results.

\begin{blocki}{Reminder: Advantages of trees}
\item No problems with categorical features.
\item No problems with outliers in feature values.
\item No problems with missing values.
\item No problems with monotone transformations of features.
\item Trees (and stumps!) can be fitted quickly, even for large $n$.
\item Trees have a simple built-in type of variable selection.
%\item Interpretation of Trees is rather easy.
\end{blocki}
The gradient-boosted trees method retains all of them, and strongly improves the trees' predictive power.
Furthermore, it is possible to adapt gradient boosting to tree learners in a targeted manner.

\framebreak

One can write a tree as: $ b(\xv) = \sum_{t=1}^{T} c_t \mathds{1}_{\{\xv \in R_t\}} $,
where $R_t$ are the terminal regions and $c_t$ the corresponding constant parameter.

\vspace*{0.2cm}

For a fitted tree with regions $R_t$, the special additive structure can be exploited in boosting:  %Finished here

\begin{align*}
  \fm(\xv) &= \fmd(\xv) +  \betam \bmm(\xv) \\
         &= \fmd(\xv) +  \betam \sum_{t=1}^{\Tm} \ctm \mathds{1}_{\{\xv \in \Rtm\}}.
\end{align*}

Actually, we do not have to find $\ctm$ and $\betam$ in two separate steps
(fitting against pseudo-residuals, then line search) but find optimal $\ctm$ (including $\betam$).
Also note that the $\ctm$ will not really be loss-optimal as we used squared error loss
to fit them against the pseudo-residuals.

\framebreak
\begin{footnotesize}
We do the same steps as before: (1) calculate the pseudo-residuals, (2) fit a tree against pseudo-residuals, \textbf{but now} we keep only the structure of the tree and optimize the $c$ parameter in a (further) post-hoc step.

$$
\fm(\xv) = \fmd(\xv) +  \sum_{t=1}^{\Tm} \ctmt \mathds{1}_{\{\xv \in \Rtm\}}. 
$$

We want to find the constant value $c$ that drives down risk the most w.r.t the squared error loss,
when added to the respective terminal region.
We can determine/change all $\ctmt$ individually and directly $L$-optimally:


%\vspace{-0.2cm}

$$ \ctmt = \argmin_{c} \sum_{\xi \in \Rtm} L(\yi, \fmd(\xi) + c). $$

\vspace{-0.5cm}

\begin{center}

\includegraphics[width=0.38\textwidth]{figure_man/gbm_leaf_adjustment.pdf}

\end{center}

\end{footnotesize}

\framebreak

\begin{algorithm}[H]
  \begin{footnotesize}
  \begin{center}
  \caption{Gradient Tree Boosting Algorithm.}
    \begin{algorithmic}[1]
      \State Initialize $\hat{f}^{[0]}(\xv) = \argmin_{\theta} \sumin L(\yi, \theta)$
      \For{$m = 1 \to M$}
          \State For all $i$: $\rmi = -\left[\pd{\Lxyi}{\fxi}\right]_{f=\fmdh}$
          \State Fit regr. tree to the $\rmi$, giving terminal regions $\Rtm,
          \ t = 1,\ldots,\Tm$
        \For{$t = 1 \to \Tm$}
          \State $\ctmh = \argmin_{c} \sum \limits_{\xi \in \Rtm} L(\yi, \fmdh(\xi) + c)$
        \EndFor
        \State $\bmmh(\xv) = \sum_{t=1}^{T} \ctmh \mathds{1}_{\{x \in R_t\}} $
        \State Update $\fmh(\xv) = \fmdh(\xv) + \bmmh(\xv)$
      \EndFor
      \State Output $\hat{f}(\xv) = \hat{f}^{[M]}(\xv)$
    \end{algorithmic}
    \end{center}
    \end{footnotesize}
\end{algorithm}


\end{vbframe}


\begin{vbframe}{Visualization 1}
\begin{footnotesize}
\textbf{Simulation Setting:}
\begin{itemize}
\item Given: One feature $\xv$ and one numeric target variable $y$ of 50 observations.
\item $\xv$ is uniformly distributed between 0 and 10.
\item $y$ depends on $\xv$ as follows: $y^{(i)} = \sin{(x^{(i)})} + \epsilon^{(i)}$ with $\epsilon^{(i)} \sim \mathcal{N}(0,0.2)$, $\forall i \in 1,\dots,50$.
\end{itemize}

\vspace*{0.2cm}

\begin{columns}

\column{5cm}
\begin{figure}
  \includegraphics{figure_man/gbm_anim_data.png}
\end{figure}

\column{5cm}
\textbf{Aim:} We want to fit a gradient boosting model to the data by using stumps as base learners.
\begin{itemize}
\item Since we are facing a regression problem, we use L2 loss.
%\item Self-implementation of GB where learning rate found by line search.
\end{itemize}


\end{columns}
\end{footnotesize}
\framebreak

\begin{footnotesize}
\textbf{Iteration 0:} Initialization by optimal constant (mean) prediction $\hat f^{[0](i)}(\xv) = \overline{y} \approx 0.2$

\vspace*{0.2cm}

\begin{columns}

\column{5cm}
\begin{center}
\begin{tabular}{c|c|c|c}
$i$ & $x^{(i)}$ & $y^{(i)}$ & $\hat{f}^{[0]}$ \\ \hline
1 & 0.03 &  0.03 & 0.20 \\
2 & 0.03 & -0.01 & 0.20 \\
3 & 0.06 &  0.08 & 0.20 \\
\vdots & \vdots & \vdots & \vdots \\
50 & 9.69 & -0.29 & 0.20\\
\end{tabular}
\end{center}

\column{5cm}
\begin{figure}
  \includegraphics[width=\textwidth]{figure_man/gbm_anim_init.png}
\end{figure}

\end{columns}

\end{footnotesize}

\framebreak
\begin{footnotesize}
\textbf{Iteration 1:} (1) Calculate residuals $\tilde{r}^{[m](i)}$ and (2) fit a regression stump $b^{[m]}$ %which is multiplied by the learning rate $\beta$% (here: $\hat \beta^{[1]} = 1$).

\begin{columns}

\column{5cm}
\begin{center}
\begin{tabular}{c|c|c|c|c|c}
$i$ & $x^{(i)}$ & $y^{(i)}$ & $\hat{f}^{[0]}$ & $\tilde{r}^{[1](i)}$ & $\hat{b}^{[1](i)}$\\ \hline
1 & 0.03 &  0.03 & 0.20 & -0.17 & -0.18\\
2 & 0.03 & -0.01 & 0.20 & -0.21 & -0.18 \\
3 & 0.06 &  0.08 & 0.20 & -0.12 & -0.18 \\
\vdots & \vdots & \vdots & \vdots  & \vdots & \vdots \\
50 & 9.69 & -0.29 & 0.20 & -0.49 & 0.34\\
\end{tabular}
\end{center}

\column{4.9cm}
\begin{figure}
  \includegraphics[width=\textwidth]{figure_man/gbm_anim_01.png}
\end{figure}

\end{columns}

\vspace*{0.2cm}

%(3) Update model by $\hat{f}^{[1]}(\xv) = \hat{f}^{[0]}(\xv) + \hat{\beta}^{[1]} \hat{b}^{[1]$
(3) Update model by $\hat{f}^{[1]}(\xv) = \hat{f}^{[0]}(\xv) + \hat{b}^{[1]$
\end{footnotesize}


\end{vbframe}

\begin{frame}{Visualization 1}
Repeat step (1) to (3):
\begin{center}
\only<1>{ \includegraphics[width=0.6\textwidth]{figure_man/gbm_anim_02.png} }
\only<2>{ \includegraphics[width=0.6\textwidth]{figure_man/gbm_anim_03.png} }
\only<3>{ \includegraphics[width=0.6\textwidth]{figure_man/gbm_anim_04.png} }
\only<4>{ \includegraphics[width=0.6\textwidth]{figure_man/gbm_anim_05.png} }
\only<5>{ \includegraphics[width=0.6\textwidth]{figure_man/gbm_anim_31.png} }
\end{center}

\end{frame}
% 
% \begin{columns}
% \begin{column}{0.6\textwidth}
% \begin{center}
% \only<1>{ \includegraphics[width=\textwidth]{figure_man/gbm_anim_01.png} }
% \only<2>{ \includegraphics[width=\textwidth]{figure_man/gbm_anim_02.png} }
% \only<3>{ \includegraphics[width=\textwidth]{figure_man/gbm_anim_03.png} }
% \only<4>{ \includegraphics[width=\textwidth]{figure_man/gbm_anim_04.png} }
% \only<5>{ \includegraphics[width=\textwidth]{figure_man/gbm_anim_05.png} }
% \only<6>{ \includegraphics[width=\textwidth]{figure_man/gbm_anim_31.png} }
% \end{center}
% \end{column}
% \begin{column}{0.4\textwidth}
% \begin{itemize}
% \item L2 loss and stumps.
% \item Left = current additive model.
% \item Right = Residuals and fitted base learner.
% \item Plot is created by a self-implementation of GB where step size found by line search.
% \end{itemize}
% \end{column}
% \end{columns}
% \end{frame}

\begin{vbframe}{Visualization 2}

\vspace{-0.6cm}
\begin{center}
\includegraphics[width=0.6\textwidth]{figure_man/gbm_sine.pdf}
\end{center}

\vspace{-0.8cm}
\footnotesize
\begin{itemize}

  \item
    Iterating this very simple base learner yields a rather nice approximation of a smooth model in the end.

  \item
    Severe overfitting apparent in the noisy case. We will discuss and solve this problem later.

\end{itemize}

\end{vbframe}

\begin{vbframe}{Visualization 3}
\begin{center}
\includegraphics[width=\textwidth]{figure_man/gbm_anim/gbm1.jpg}
\href{http://arogozhnikov.github.io/2016/06/24/gradient_boosting_explained.html}{\beamergotobutton{Open in browser.}}
\end{center}
\end{vbframe}

% \begin{vbframe}{Gradient Boosting Visualization}
% \begin{center}
% \includegraphics[width=\textwidth]{figure_man/gbm_anim/gbm2.jpg}
% \href{http://arogozhnikov.github.io/2016/06/24/gradient_boosting_explained.html}{\beamergotobutton{Open in browser.}}
% \end{center}
% \addtocounter{framenumber}{-1}
% \end{vbframe}

\begin{vbframe}{Visualization 3}
\begin{center}
\includegraphics[width=\textwidth]{figure_man/gbm_anim/gbm3.jpg}
\href{http://arogozhnikov.github.io/2016/06/24/gradient_boosting_explained.html}{\beamergotobutton{Open in browser.}}
\end{center}
\addtocounter{framenumber}{-1}
\end{vbframe}

% \begin{vbframe}{Gradient Boosting Visualization}
% \begin{center}
% \includegraphics[width=\textwidth]{figure_man/gbm_anim/gbm4.jpg}
% \href{http://arogozhnikov.github.io/2016/06/24/gradient_boosting_explained.html}{\beamergotobutton{Open in browser.}}
% \end{center}
% \addtocounter{framenumber}{-1}
% \end{vbframe}

\begin{vbframe}{Visualization 3}
\begin{center}
\includegraphics[width=\textwidth]{figure_man/gbm_anim/gbm5.jpg}
\href{http://arogozhnikov.github.io/2016/06/24/gradient_boosting_explained.html}{\beamergotobutton{Open in browser.}}
\end{center}
\addtocounter{framenumber}{-1}
\end{vbframe}

\begin{vbframe}{Visualization 3}
\begin{center}
\includegraphics[width=\textwidth]{figure_man/gbm_anim/gbm7.jpg}
\href{http://arogozhnikov.github.io/2016/06/24/gradient_boosting_explained.html}{\beamergotobutton{Open in browser.}}
\end{center}
\addtocounter{framenumber}{-1}
\end{vbframe}

% \begin{vbframe}{Gradient Boosting Visualization}
% \begin{center}
% \includegraphics[width=\textwidth]{figure_man/gbm_anim/gbm8.jpg}
% \href{http://arogozhnikov.github.io/2016/06/24/gradient_boosting_explained.html}{\beamergotobutton{Open in browser.}}
% \end{center}
% \addtocounter{framenumber}{-1}
% \end{vbframe}

% \begin{vbframe}{Gradient Boosting Visualization}
% \begin{center}
% \includegraphics[width=\textwidth]{figure_man/gbm_anim/gbm9.jpg}
% \href{http://arogozhnikov.github.io/2016/06/24/gradient_boosting_explained.html}{\beamergotobutton{Open in browser.}}
% \end{center}
% \addtocounter{framenumber}{-1}
% \end{vbframe}

\begin{vbframe}{Visualization 3}
\begin{center}
\includegraphics[width=\textwidth]{figure_man/gbm_anim/gbm10.jpg}
\href{http://arogozhnikov.github.io/2016/06/24/gradient_boosting_explained.html}{\beamergotobutton{Open in browser.}}
\end{center}
\addtocounter{framenumber}{-1}
\end{vbframe}


\endlecture
\end{document}
