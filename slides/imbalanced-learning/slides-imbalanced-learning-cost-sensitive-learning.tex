\usepackage[]{graphicx}
\usepackage[]{color}
% maxwidth is the original width if it is less than linewidth
% otherwise use linewidth (to make sure the graphics do not exceed the margin)
\makeatletter
\def\maxwidth{ %
  \ifdim\Gin@nat@width>\linewidth
    \linewidth
  \else
    \Gin@nat@width
  \fi
}
\makeatother

% ---------------------------------%
% latex-math dependencies, do not remove:
% - \usepackage{mathtools}
% - \usepackage{bm}
% - \usepackage{siunitx}
% - \usepackage{dsfont}
% - \usepackage{xspace}
% ---------------------------------%

%--------------------------------------------------------%
%       Language, encoding, typography
%--------------------------------------------------------%

\usepackage[english]{babel}
\usepackage[utf8]{inputenc} % Enables inputting UTF-8 symbols
% Standard AMS suite
\usepackage{amsmath,amsfonts,amssymb}

% Font four double-stroke / blackboard letters for sets of numbers (N, R, ...)
% Distribution name is "doublestroke"
% According to https://mirror.physik.tu-berlin.de/pub/CTAN/fonts/doublestroke/dsdoc.pdf
% the "bbm" package does a similar thing and may be superfluous.
% Required for latex-math
\usepackage{dsfont}

% bbm – "Blackboard-style" cm fonts (https://www.ctan.org/pkg/bbm)
% Used to be in common.tex, loaded directly after this file
% Maybe superfluous given dsfont is loaded
% TODO: Check if really unused?
% \usepackage{bbm}

% bm – Access bold symbols in maths mode - https://ctan.org/pkg/bm
% Required for latex-math
% https://tex.stackexchange.com/questions/3238/bm-package-versus-boldsymbol
\usepackage{bm}

% pifont – Access to PostScript standard Symbol and Dingbats fonts
% Used for \newcommand{\xmark}{\ding{55}, which is never used
% aside from lecture_advml/attic/xx-automl/slides.Rnw
% \usepackage{pifont}

% Quotes (inline and display), provdes \enquote
% https://ctan.org/pkg/csquotes
\usepackage{csquotes}

% Adds arg to enumerate env, technically superseded by enumitem according
% to https://ctan.org/pkg/enumerate
% Replace with https://ctan.org/pkg/enumitem ?
\usepackage{enumerate}

% Line spacing - provides \singlespacing \doublespacing \onehalfspacing
% https://ctan.org/pkg/setspace
% TODO: Check if really unused?
%\usepackage{setspace}

% mathtools – Mathematical tools to use with amsmath
% https://ctan.org/pkg/mathtools?lang=en
% latex-math dependency according to latex-math repo
\usepackage{mathtools}

%--------------------------------------------------------%
%       Displaying code and algorithms
%--------------------------------------------------------%
\usepackage{verbatim}
\usepackage{algorithm}
\usepackage{algpseudocode}

%--------------------------------------------------------%
%       Tables
%--------------------------------------------------------%

% multi-row table cells: https://www.namsu.de/Extra/pakete/Multirow.html
\usepackage{multirow}

% long/multi-page tables: https://texdoc.org/serve/longtable.pdf/0
% TODO: Check if really unused?

\usepackage{longtable}

% pretty table env: https://ctan.org/pkg/booktabs?lang=en
% TODO: Check if really unused?
\usepackage{booktabs}

%--------------------------------------------------------%
%       Figures: Creating, placing, verbing
%--------------------------------------------------------%

% wrapfig - Wrapping text around figures https://de.overleaf.com/learn/latex/Wrapping_text_around_figures
\usepackage{wrapfig}

% Sub figures in figures and tables
% https://ctan.org/pkg/subfig -- supersedes subfigure package
% TODO: Check if really unused?
\usepackage{subfig}

% Actually it's pronounced PGF https://en.wikibooks.org/wiki/LaTeX/PGF/TikZ
\usepackage{tikz}

\usetikzlibrary{shapes,arrows,automata,positioning,calc,chains,trees, shadows}
\tikzset{
  %Define standard arrow tip
  >=stealth',
  %Define style for boxes
  punkt/.style={
    rectangle,
    rounded corners,
    draw=black, very thick,
    text width=6.5em,
    minimum height=2em,
    text centered},
  % Define arrow style
  pil/.style={
    ->,
    thick,
    shorten <=2pt,
    shorten >=2pt,}
}


% Unsorted
% textpos – Place boxes at arbitrary positions on the LATEX page
% https://ctan.org/pkg/textpos?lang=en
% Provides \begin{textblock}
 % TODO: Check if really unused?
\usepackage[absolute,overlay]{textpos}

% psfrag – Replace strings in encapsulated PostScript figures
% https://www.overleaf.com/latex/examples/psfrag-example/tggxhgzwrzhn
% https://ftp.mpi-inf.mpg.de/pub/tex/mirror/ftp.dante.de/pub/tex/macros/latex/contrib/psfrag/pfgguide.pdf
% Can't tell if this is needed
% TODO: Check if really unused?
\usepackage{psfrag}

% Maybe not great to use this https://tex.stackexchange.com/a/197/19093
% Use align instead -- TODO: Global search & replace to check
\usepackage{eqnarray}

\usepackage{colortbl}

% arydshln – Draw dash-lines in array/tabular
% https://www.ctan.org/pkg/arydshln
% !! "arydshln has to be loaded after array, longtable, colortab and/or colortbl"
% Provides \hdashline and \cdashline
% TODO: Check if really unused?
% \usepackage{arydshln}

% tabularx – Tabulars with adjustable-width columns
% https://ctan.org/pkg/tabularx
% Provides \begin{tabularx}
% TODO: Check if really unused?
% \usepackage{tabularx}

% placeins – Control float placement
% https://ctan.org/pkg/placeins
% Defines a \FloatBarrier command
% TODO: Check if really unused?
% \usepackage{placeins}


% framed – Framed or shaded regions that can break across pages
% https://ctan.org/pkg/framed
% Provides \begin{framed} which uses \colorbox{shadecolor} relying on \definecolor{shadecolor}.
% TODO: Check if really unused?
% \usepackage{framed}

% Used often in conjunction with \definecolor{shadecolor}{rgb}{0.969, 0.969, 0.969}
% Might be able to be removed or at least redefined to only have shadecolor (if needed)
\definecolor{fgcolor}{rgb}{0.345, 0.345, 0.345}
\definecolor{shadecolor}{rgb}{0.969, 0.969, 0.969}
\newenvironment{knitrout}{}{} % an empty environment to be redefined in TeX


% Defines macros and environments
\usepackage{../../style/lmu-lecture}

\let\code=\texttt % Used regularly
\let\proglang=\textsf % Unused?

% Not sure what/why this does
\setkeys{Gin}{width=0.9\textwidth}

\setbeamertemplate{frametitle}{\expandafter\uppercase\expandafter\insertframetitle}

% Can't find a reason why common.tex is not just part of this file?

% basic latex stuff
\newcommand{\pkg}[1]{{\fontseries{b}\selectfont #1}} %fontstyle for R packages
\newcommand{\lz}{\vspace{0.5cm}} %vertical space
\newcommand{\dlz}{\vspace{1cm}} %double vertical space
\newcommand{\oneliner}[1] % Oneliner for important statements
{\begin{block}{}\begin{center}\begin{Large}#1\end{Large}\end{center}\end{block}}


%new environments
\newenvironment{vbframe}  %frame with breaks and verbatim
{
 \begin{frame}[containsverbatim,allowframebreaks]
}
{
\end{frame}
}

\newenvironment{vframe}  %frame with verbatim without breaks (to avoid numbering one slided frames)
{
 \begin{frame}[containsverbatim]
}
{
\end{frame}
}

\newenvironment{blocki}[1]   % itemize block
{
 \begin{block}{#1}\begin{itemize}
}
{
\end{itemize}\end{block}
}

\newenvironment{fragileframe}[2]{  %fragile frame with framebreaks
\begin{frame}[allowframebreaks, fragile, environment = fragileframe]
\frametitle{#1}
#2}
{\end{frame}}


\newcommand{\myframe}[2]{  %short for frame with framebreaks
\begin{frame}[allowframebreaks]
\frametitle{#1}
#2
\end{frame}}

\newcommand{\remark}[1]{
  \textbf{Remark:} #1
}


\newenvironment{deleteframe}
{
\begingroup
\usebackgroundtemplate{\includegraphics[width=\paperwidth,height=\paperheight]{../style/color/red.png}}
 \begin{frame}
}
{
\end{frame}
\endgroup
}
\newenvironment{simplifyframe}
{
\begingroup
\usebackgroundtemplate{\includegraphics[width=\paperwidth,height=\paperheight]{../style/color/yellow.png}}
 \begin{frame}
}
{
\end{frame}
\endgroup
}\newenvironment{draftframe}
{
\begingroup
\usebackgroundtemplate{\includegraphics[width=\paperwidth,height=\paperheight]{../style/color/green.jpg}}
 \begin{frame}
}
{
\end{frame}
\endgroup
}
% https://tex.stackexchange.com/a/261480: textcolor that works in mathmode
\makeatletter
\renewcommand*{\@textcolor}[3]{%
  \protect\leavevmode
  \begingroup
    \color#1{#2}#3%
  \endgroup
}
\makeatother


%-------------------------------------------------------------------------------------------------------%
%  Unused stuff that needs to go but is kept here currently juuuust in case it was important after all  %
%-------------------------------------------------------------------------------------------------------%

% \newcommand{\hlnum}[1]{\textcolor[rgb]{0.686,0.059,0.569}{#1}}%
% \newcommand{\hlstr}[1]{\textcolor[rgb]{0.192,0.494,0.8}{#1}}%
% \newcommand{\hlcom}[1]{\textcolor[rgb]{0.678,0.584,0.686}{\textit{#1}}}%
% \newcommand{\hlopt}[1]{\textcolor[rgb]{0,0,0}{#1}}%
% \newcommand{\hlstd}[1]{\textcolor[rgb]{0.345,0.345,0.345}{#1}}%
% \newcommand{\hlkwa}[1]{\textcolor[rgb]{0.161,0.373,0.58}{\textbf{#1}}}%
% \newcommand{\hlkwb}[1]{\textcolor[rgb]{0.69,0.353,0.396}{#1}}%
% \newcommand{\hlkwc}[1]{\textcolor[rgb]{0.333,0.667,0.333}{#1}}%
% \newcommand{\hlkwd}[1]{\textcolor[rgb]{0.737,0.353,0.396}{\textbf{#1}}}%
% \let\hlipl\hlkwb

% \makeatletter
% \newenvironment{kframe}{%
%  \def\at@end@of@kframe{}%
%  \ifinner\ifhmode%
%   \def\at@end@of@kframe{\end{minipage}}%
%   \begin{minipage}{\columnwidth}%
%  \fi\fi%
%  \def\FrameCommand##1{\hskip\@totalleftmargin \hskip-\fboxsep
%  \colorbox{shadecolor}{##1}\hskip-\fboxsep
%      % There is no \\@totalrightmargin, so:
%      \hskip-\linewidth \hskip-\@totalleftmargin \hskip\columnwidth}%
%  \MakeFramed {\advance\hsize-\width
%    \@totalleftmargin\z@ \linewidth\hsize
%    \@setminipage}}%
%  {\par\unskip\endMakeFramed%
%  \at@end@of@kframe}
% \makeatother

% \definecolor{shadecolor}{rgb}{.97, .97, .97}
% \definecolor{messagecolor}{rgb}{0, 0, 0}
% \definecolor{warningcolor}{rgb}{1, 0, 1}
% \definecolor{errorcolor}{rgb}{1, 0, 0}
% \newenvironment{knitrout}{}{} % an empty environment to be redefined in TeX

% \usepackage{alltt}
% \newcommand{\SweaveOpts}[1]{}  % do not interfere with LaTeX
% \newcommand{\SweaveInput}[1]{} % because they are not real TeX commands
% \newcommand{\Sexpr}[1]{}       % will only be parsed by R
% \newcommand{\xmark}{\ding{55}}%

% math spaces
\newcommand{\N}{\mathds{N}}                                                 % N, naturals
\newcommand{\Z}{\mathds{Z}}                                                 % Z, integers
\newcommand{\Q}{\mathds{Q}}                                                 % Q, rationals
\newcommand{\R}{\mathds{R}}                                                 % R, reals
\newcommand{\C}{\mathds{C}}                                                 % C, complex
\newcommand{\HS}{\mathcal{H}}                                               % H, hilbertspace
\newcommand{\continuous}{\mathcal{C}}                                       % C, space of continuous functions
\newcommand{\M}{\mathcal{M}} 												% machine numbers
\newcommand{\epsm}{\epsilon_m} 												% maximum error


% basic math stuff
\newcommand{\xt}{\tilde x}													% x tilde
\def\argmax{\mathop{\sf arg\,max}}                                          % argmax
\def\argmin{\mathop{\sf arg\,min}}                                          % argmin
\newcommand{\sign}{\operatorname{sign}}                                     % sign, signum
\newcommand{\I}{\mathbb{I}}                                                 % I, indicator
\newcommand{\order}{\mathcal{O}}                                            % O, order
\newcommand{\fp}[2]{\frac{\partial #1}{\partial #2}}                        % partial derivative
\newcommand{\pd}[2]{\frac{\partial{#1}}{\partial #2}}						% partial derivative

% sums and products
\newcommand{\sumin}{\sum_{i=1}^n}											% summation from i=1 to n
\newcommand{\sumkg}{\sum_{k=1}^g}											% summation from k=1 to g
\newcommand{\prodin}{\prod_{i=1}^n}											% product from i=1 to n
\newcommand{\prodkg}{\prod_{k=1}^g}											% product from k=1 to g

% linear algebra
\newcommand{\one}{\boldsymbol{1}}                                           % 1, unitvector
\newcommand{\id}{\mathrm{I}}                                                % I, identity
\newcommand{\diag}{\operatorname{diag}}                                     % diag, diagonal
\newcommand{\trace}{\operatorname{tr}}                                      % tr, trace
\newcommand{\spn}{\operatorname{span}}                                      % span
\newcommand{\scp}[2]{\left\langle #1, #2 \right\rangle}                     % <.,.>, scalarproduct
\newcommand{\mat}[1]{ 														% short pmatrix command
	\begin{pmatrix}
		#1
	\end{pmatrix}
}
\newcommand{\Amat}{\bm{A}}													% matrix A
\newcommand{\xv}{\bm{x}}													% vector x (bold)
\newcommand{\yv}{\bm{y}}														% vector y (bold)
\newcommand{\Deltab}{\bm{\Delta}}											% error term for vectors
															

% basic probability + stats
\renewcommand{\P}{\mathds{P}}                                               % P, probability
\newcommand{\E}{\mathds{E}}                                                 % E, expectation
\newcommand{\var}{\mathsf{Var}}                                             % Var, variance
\newcommand{\cov}{\mathsf{Cov}}                                             % Cov, covariance
\newcommand{\corr}{\mathsf{Corr}}                                           % Corr, correlation
\newcommand{\normal}{\mathcal{N}}                                           % N of the normal distribution
\newcommand{\iid}{\overset{i.i.d}{\sim}}                                    % dist with i.i.d superscript
\newcommand{\distas}[1]{\overset{#1}{\sim}}                                 % ... is distributed as ... 
% machine learning

%%%%%% ml - data
\newcommand{\Xspace}{\mathcal{X}}                                           % X, input space
\newcommand{\Yspace}{\mathcal{Y}}                                           % Y, output space
\newcommand{\nset}{\{1, \ldots, n\}}                                        % set from 1 to n
\newcommand{\pset}{\{1, \ldots, p\}}                                        % set from 1 to p
\newcommand{\gset}{\{1, \ldots, g\}}                                        % set from 1 to g
\newcommand{\Pxy}{\P_{xy}}                                                  % P_xy
\newcommand{\xy}{(x, y)}                                                    % observation (x, y)
\newcommand{\xvec}{(x_1, \ldots, x_p)^T}                                    % (x1, ..., xp) 
\newcommand{\D}{\mathcal{D}}                                                % D, data 
\newcommand{\Dset}{\{ (x^{(1)}, y^{(1)}), \ldots, (x^{(n)},  y^{(n)})\}}    % {(x1,y1)), ..., (xn,yn)}, data
\newcommand{\xdat}{\{ x^{(1)}, \ldots, x^{(n)}\}}   						 % {x1, ..., xn}, input data
\newcommand{\ydat}{\mathbf{y}}                                              % y (bold), vector of outcomes
\newcommand{\yvec}{(y^{(1)}, \hdots, y^{(n)})^T}                            % (y1, ..., yn), vector of outcomes
\renewcommand{\xi}[1][i]{x^{(#1)}}                                          % x^i, i-th observed value of x
\newcommand{\yi}[1][i]{y^{(#1)}}                                            % y^i, i-th observed value of y 
\newcommand{\xyi}{(\xi, \yi)}                                               % (x^i, y^i), i-th observation
\newcommand{\xivec}{(x^{(i)}_1, \ldots, x^{(i)}_p)^T}                       % (x1^i, ..., xp^i), i-th observation vector
\newcommand{\xj}{x_j}                                                       % x_j, j-th feature
\newcommand{\xjb}{\mathbf{x}_j}                                             % x_j (bold), j-th feature vecor
\newcommand{\xjvec}{(x^{(1)}_j, \ldots, x^{(n)}_j)^T}                       % (x^1_j, ..., x^n_j), j-th feature vector
\newcommand{\Dtrain}{\mathcal{D}_{\text{train}}}                            % D_train, training set
\newcommand{\Dtest}{\mathcal{D}_{\text{test}}}                              % D_test, test set

%%%%%% ml - models general

% continuous prediction function f
\newcommand{\fx}{f(x)}                                                      % f(x), continuous prediction function
\newcommand{\Hspace}{H}														% hypothesis space where f is from
\newcommand{\fh}{\hat{f}}                                                   % f hat, estimated prediction function
\newcommand{\fxh}{\fh(x)}                                                   % fhat(x)
\newcommand{\fxt}{f(x | \theta)}                                            % f(x | theta)
\newcommand{\fxi}{f(\xi)}                                                   % f(x^(i))
\newcommand{\fxih}{\hat{f}(\xi)}                                            % f(x^(i))
\newcommand{\fxit}{f(x^{(i)} | \theta)}                                     % f(x^(i) | theta)
\newcommand{\fhD}{\fh_{\D}}                                                 % fhat_D, estimate of f based on D
\newcommand{\fhDtrain}{\fh_{\Dtrain}}                                       % fhat_Dtrain, estimate of f based on D

% discrete prediction function h
\newcommand{\hx}{h(x)}                                                      % h(x), discrete prediction function
\newcommand{\hh}{\hat{h}}                                                   % h hat
\newcommand{\hxh}{\hat{h}(x)}                                               % hhat(x)
\newcommand{\hxt}{h(x | \theta)}                                            % h(x | theta)
\newcommand{\hxi}{h(\xi)}                                                   % h(x^(i))
\newcommand{\hxit}{h(x^{(i)} | \theta)}                                     % h(x^(i) | theta)

% yhat
\newcommand{\yh}{\hat{y}}                                                   % y hat for prediction of target
\newcommand{\yih}{\hat{y}}                                                  % y hat for prediction of target

% theta
\newcommand{\thetah}{\hat{\theta}}                                          % theta hat

% densities + probabilities
% pdf of x 
\newcommand{\pdf}{p}                                                        % p
\newcommand{\pdfx}{p(x)}                                                    % p(x)
\newcommand{\pixt}{\pi(x | \theta)}                                         % pi(x|theta), pdf of x given theta

% pdf of (x, y)
\newcommand{\pdfxy}{p(x,y)}                                                 % p(x, y)
\newcommand{\pdfxyt}{p(x, y | \theta)}                                      % p(x, y | theta)
\newcommand{\pdfxyit}{p(\xi, \yi | \theta)}                                 % p(x^(i), y^(i) | theta)

% pdf of x given y
\newcommand{\pdfxyk}{p(x | y=k)}                                            % p(x | y = k)
\newcommand{\lpdfxyk}{\log \pdfxyk}                                         % log p(x | y = k)
\newcommand{\pdfxiyk}{p(\xi | y=k)}                                         % p(x^i | y = k)

% prior probabilities
\newcommand{\pik}{\pi_k}                                                    % pi_k, prior
\newcommand{\lpik}{\log \pik}                                               % log pi_k, log of the prior

% posterior probabilities
\newcommand{\post}{\P(y = 1 | x)}                                           % P(y = 1 | x), post. prob for y=1
\newcommand{\pix}{\pi(x)}                                                   % pi(x), P(y = 1 | x)
\newcommand{\postk}{\P(y = k | x)}                                          % P(y = k | y), post. prob for y=k
\newcommand{\pikx}{\pi_k(x)}                                                % pi_k(x), P(y = k | x)
\newcommand{\pikxt}{\pi_k(x | \theta)}                                      % pi_k(x | theta), P(y = k | x, theta)
\newcommand{\pijx}{\pi_j(x)}                                                % pi_j(x), P(y = j | x)
\newcommand{\pdfygxt}{p(y |x, \theta)}                                      % p(y | x, theta)
\newcommand{\pdfyigxit}{p(\yi |\xi, \theta)}                                % p(y^i |x^i, theta)
\newcommand{\lpdfygxt}{\log \pdfygxt }                                      % log p(y | x, theta)
\newcommand{\lpdfyigxit}{\log \pdfyigxit}                                   % log p(y^i |x^i, theta)
\newcommand{\pixh}{\hat \pi(x)}                                             % pi(x) hat, P(y = 1 | x) hat
\newcommand{\pikxh}{\hat \pi_k(x)}                                          % pi_k(x) hat, P(y = k | x) hat

% residual and margin
\newcommand{\eps}{\epsilon}                                                 % residual, stochastic
\newcommand{\epsi}{\epsilon^{(i)}}                                          % epsilon^i, residual, stochastic
\newcommand{\epsh}{\hat{\epsilon}}                                          % residual, estimated
\newcommand{\yf}{y \fx}                                                     % y f(x), margin
\newcommand{\yfi}{\yi \fxi}                                                 % y^i f(x^i), margin
\newcommand{\Sigmah}{\hat \Sigma}											% estimated covariance matrix
\newcommand{\Sigmahj}{\hat \Sigma_j}										% estimated covariance matrix for the j-th class

% ml - loss, risk, likelihood
\newcommand{\Lxy}{L(y, f(x))}                                               % L(y, f(x)), loss function
\newcommand{\Lxyi}{L(\yi, \fxi)}                                            % L(y^i, f(x^i))
\newcommand{\Lxyt}{L(y, \fxt)}                                              % L(y, f(x | theta))
\newcommand{\Lxyit}{L(\yi, \fxit)}                                          % L(y^i, f(x^i | theta)
\newcommand{\risk}{\mathcal{R}}                                             % R, risk
\newcommand{\riskf}{\risk(f)}                                               % R(f), risk
\newcommand{\riske}{\mathcal{R}_{\text{emp}}}                               % R_emp, empirical risk
\newcommand{\riskef}{\riske(f)}                                             % R_emp(f)
\newcommand{\risket}{\mathcal{R}_{\text{emp}}(\theta)}                      % R_emp(theta)
\newcommand{\riskr}{\mathcal{R}_{\text{reg}}}                               % R_reg, regularized risk
\newcommand{\riskrt}{\mathcal{R}_{\text{reg}}(\theta)}                      % R_reg(theta)
\newcommand{\riskrf}{\riskr(f)}                                             % R_reg(f)
\newcommand{\LL}{\mathcal{L}}                                               % L, likelihood
\newcommand{\LLt}{\mathcal{L}(\theta)}                                      % L(theta), likelihood
\renewcommand{\ll}{\ell}                                                    % l, log-likelihood
\newcommand{\llt}{\ell(\theta)}                                             % l(theta), log-likelihood
\newcommand{\LS}{\mathfrak{L}}                                              % ????????????
\newcommand{\TS}{\mathfrak{T}}                                              % ??????????????
\newcommand{\errtrain}{\text{err}_{\text{train}}}                           % training error
\newcommand{\errtest}{\text{err}_{\text{test}}}                             % training error
\newcommand{\errexp}{\overline{\text{err}_{\text{test}}}}                   % training error

% resampling
\newcommand{\GE}[1]{GE(\fh_{#1})}                                           % Generalization error GE
\newcommand{\GEh}[1]{\widehat{GE}_{#1}}                                     % Estimated train error
\newcommand{\GED}{\GE{\D}}                                                  % Generalization error GE
\newcommand{\EGEn}{EGE_n}                                                   % Generalization error GE
\newcommand{\EDn}{\E_{|D| = n}}                                             % Generalization error GE


% ml - irace
\newcommand{\costs}{\mathcal{C}} % costs
\newcommand{\Celite}{\theta^*} % elite configurations
\newcommand{\instances}{\mathcal{I}} % sequence of instances
\newcommand{\budget}{\mathcal{B}} % computational budget

%\usepackage{algorithm}
%\usepackage{algorithmic}

\newcommand{\sens}{\mathbf{A}} % vector x (bold)
\newcommand{\ba}{\mathbf{a}}
\newcommand{\batilde}{\tilde{\mathbf{a}}}
\newcommand{\Px}{\mathbb{P}_{x}} % P_x
\newcommand{\Pxj}{\mathbb{P}_{x_j}} % P_{x_j}
\newcommand{\indep}{\perp \!\!\! \perp} % independence symbol
% ml - ROC
\newcommand{\np}{n_{+}} % no. of positive instances
\newcommand{\nn}{n_{-}} % no. of negative instances
\newcommand{\rn}{\pi_{-}} % proportion negative instances
\newcommand{\rp}{\pi_{+}} % proportion negative instances
% true/false pos/neg:
\newcommand{\tp}{\# \text{TP}} % true pos
\newcommand{\fap}{\# \text{FP}} % false pos (fp taken for partial derivs)
\newcommand{\tn}{\# \text{TN}} % true neg
\newcommand{\fan}{\# \text{FN}} % false neg

\usepackage{multicol}

\newcommand{\titlefigure}{figure/cost_matrix}
\newcommand{\learninggoals}{
  \item Learn the modelling approach via the cost matrix in cost-sensitive learning settings
  \item Understand the connection between cost-sensitive and imbalanced learning
  \item Get to know MetaCost as a general approach to make classifiers cost-sensitve
}

\title{Advanced Machine Learning}
\date{}

\begin{document}

\lecturechapter{Imbalanced Learning via Cost-Sensitive Learning}
\lecture{Advanced Machine Learning}



\sloppy


\begin{vbframe}{Cost-Sensitive learning: In a Nutshell}
	%	
	\scriptsize{
		%
		%
		\begin{itemize}
%			
		\item Cost-sensitive learning is a learning paradigm, where different (mis-)classification costs are taken into consideration and the learner seeks to minimize the total costs in expectation.
%
		\item Thus, the major difference to the ``classical'' cost-insensitive learning setting is that cost-sensitive learning deals differently with misclassifications. Most of the algorithms of the latter kind assume that the data sets are balanced, and all errors have the same cost.
		
%
		\item This is motivated by a plethora of real-world applications, where different costs between misclassifications are present:
%		
		\begin{itemize}
			\scriptsize
%			
			\item Medicine --- Misdiagnosing a cancer patient as healthy vs.\ misdiagnosing a healthy patient as having cancer (and then check again).
			%					
			\item Tracking criminals ---  Classify an innocent person as a terrorist vs.\ overlooking a terrorist.
			%			
			\item (Extreme) Weather prediction ---  Incorrectly predicting that no hurricane occurs vs.\ predicting a strong wind as a hurricane.
%			
			\item Credit granting --- Lending to a risky client vs. not lending to a trustworthy client.
			%			
			\item $\ldots$
%    	
%			
		\end{itemize}
%	
		\item In all these examples the costs of a false negative is much higher than the costs of a false positive.
%
		\end{itemize}
		
	%
	}
\end{vbframe}


\begin{vbframe}{Cost matrix}
%	
\scriptsize{
%	
	\begin{itemize}
%		
		\item In cost-sensitive learning we are provided with a cost matrix $\mathbf{C}$ of the form
%		
	\end{itemize}
	%	
	\begin{center}
		\tiny
		\begin{tabular}{cc|>{\centering\arraybackslash}p{8em}>{\centering\arraybackslash}p{8em}>{\centering\arraybackslash}p{5em}>{\centering\arraybackslash}p{8em}}
			& & \multicolumn{4}{c}{\bfseries True Class $y$} \\
			&  & $1$ & $2$ & $\ldots$ & $g$  \\
			\hline
			\bfseries Classification     & $1$ & $C(1,1)$  &  $C(1,2)$  & $\ldots$ &  $C(1,g)$ \\
			& & (True 1's) & (False 1's for 2's) & $\ldots$ &  (False 1's for $g$'s)  \\
			& $2$ &  $C(2,1)$  &  $C(2,2)$  & $\ldots$ & $C(2,g)$  \\
			$\yh$ & & (False 2's for 1's) & (True 2's) & $\ldots$ &  (False 2's for $g$'s)  \\
			& $\vdots$ & $\vdots$ & $\vdots$ & $\ldots$ & $\vdots$ \\
			& $g$ & $C(g,1)$ & $C(g,2)$  & $\ldots$ &  $C(g,g)$\\
			& & (False $g$'s for 1's) & (False $g$'s for 2's) & $\ldots$ &  (True $g$'s)  \\
		\end{tabular}
	\end{center}
	%	
	\begin{itemize}
		%		
		\item 	Here, $C(i,j)$ is the cost of classifying $j$ as $i,$ which in the cost-insensitive learning case is simply $C(i,j) = \mathds{1}_{[ i \neq j ]},$ i.e., each misclassification has the same cost of 1.
		%		
		
		\item Sometimes the cost matrix is provided by the application at hand, e.g.\ in the credit granting example, or provided by a domain expert. In many cases, however, the cost matrix needs to be estimated. This is usually done by using a heuristic or by learning a proper cost matrix from the training data.
%		
		\item The cost matrix is essential for the learning process, as
%		
		\begin{enumerate}
%			
			\scriptsize
%			
		\item too low costs might not change the decision boundaries significantly leading to (still) costly predictions,
%		
		\item too high costs might harm the generalization capability of the classifier on costly classes.
%			
		\end{enumerate}

%		
%		
	\end{itemize}

}
\end{vbframe}


\begin{vbframe}{Cost matrix for Imbalanced Learning}
	%	
	\footnotesize{
		%	
		\begin{itemize}
			%		
%			
			\item For imbalanced data sets biases towards majority classes can be enlarged or even too strong biases towards the minority can be created if the cost matrix is set inappropriately.
%			 
			\item A common heuristic for imbalanced data sets is to use 
%			
			\begin{itemize}
%				
				\footnotesize \item the imbalance ratio between majority and minority classes for misclassifying a minority class $j$ as a majority class $i$, i.e., $C(i,j) = \frac{n_i}{n_j}$ for classes $i$ and $j$ such that $n_j  \ll n_i,$  
%				
				\item and costs of 1 for misclassifying a majority class $j$ as a minority class $i$, i.e., $C(i,j) = 1$ for classes $i$ and $j$ such that $n_i \ll n_j.$ 
%				
				\item Usually, the costs of a correct classification is set to 0, which is justified also from a theoretical point of view as we will see later.
%				
			\end{itemize}
%
		\begin{minipage}{0.45\textwidth}    
					\item In an imbalanced binary classification problem we obtain the following cost matrix using this heuristic:
		\end{minipage}
%		
		\begin{minipage}{0.35\textwidth}    
						\hfill		
				\begin{tabular}{cc|cc}
					& &\multicolumn{2}{c}{True class} \\
					& & $y=1$ & $y=-1$  \\
					\hline
					\multirow{2}{*}{\parbox{0.3cm}{Pred.  class}}& $\hat y$ = 1     & $0$                & $ 1 $\\
					& $\hat y$ = -1 & $ \frac{n_-}{n_+} $              &  $0$   \\
				\end{tabular}
		\end{minipage}
			%		
	\item Thus, this heuristic is consistent with the general real case that the cost of false negatives is much higher than the cost of false positives.	
%	
%	\item Although this is a simple heuristic which provides a cost matrix very quickly, it has a couple of drawbacks.
%	
	\end{itemize}
		
	}
\end{vbframe}


\begin{vbframe}{Minimum expected Cost Principle}
	%	
	\footnotesize{
		%	

		\begin{itemize}\footnotesize
			%		
			\item Suppose we are provided with a cost matrix $\mathbf{C}$ and also we have knowledge of the true posterior distribution $p(\cdot ~|~ \xv).$ The most natural way to classify a given feature $\xv$ is then by following the \emph{minimum expected cost principle}.
%			
			\item Minimum expected cost principle: Use the class for prediction with the smallest expected costs, where the expected costs of a class $i\in\{1,\ldots,g\}$ is
%			
			$$ 	\E_{J \sim p(\cdot ~|~ \xv)}( C(i,J) ) = \sum_{j=1}^g 	p(j ~|~ \xv) C(i,j).	$$
			%
			\item Thus, if we have a classifier $f$ which uses a probabilistic score function $\pi:\Xspace \to [0,1]^g$ with $\pi(\xv) = (\pi(\xv)_1,\ldots,\pi(\xv)_g)^\top$ and $\sum_{j=1}^g \pi(\xv)_j = 1$ for the classification, then one can easily modify $f$ to take the expected costs into account:
%			
			$$  \tilde f (\xv) = \argmin_{i=1,\ldots,g} \sum_{j=1}^g 	\pi(\xv)_j C(i,j). $$
%					
		\end{itemize}
	}
\end{vbframe}


\begin{vbframe}{Minimum expected Cost Principle: Binary Case}
	%	
	\footnotesize{
		%		
		\begin{itemize}\footnotesize
			%		
			\item In the binary classification setting (i.e., $\Yspace = \{-1,1\}$) the minimum expected costs principle translates to predict the positive class if 
%			
			\begin{align*}
%				
					&\E_{J \sim p(\cdot ~|~ \xv)}( C(1,J) )  \leq \E_{J \sim p(\cdot ~|~ \xv)}( C(-1,J) ) \\
%
				 	&\Leftrightarrow p(-1 ~|~ \xv ) C(1,-1)  + 	p(1 ~|~ \xv ) C(1,1) \\ &
				 	\qquad \leq  p(-1 ~|~ \xv ) C(-1,-1)  + 	p(1 ~|~ \xv ) C(-1,1)  \\
%				 	
					&\Leftrightarrow p(-1 ~|~ \xv ) \left( C(1,-1) - C(-1,-1) \right)  \leq  	p(1 ~|~ \xv ) \left( C(-1,1) -C(1,1)\right)  
%				
			\end{align*}
			%			
			\item Note that the decision for predicting the positive class does not change if we use instead of $\mathbf{C}$ the simpler cost matrix $\mathbf{C}_{simple}$, where 
			\begin{itemize}
				\footnotesize
				 \item $C_{simple}(-1,-1)=C_{simple}(1,1) = 0$ 
				 \item  $C_{simple}(1,-1) =  C(1,-1) - C(-1,-1) $ 
				 \item $C_{simple}(-1,1) = C(-1,1) -C(1,1).$
			\end{itemize} 
%		
		\item Thus, one can assume without loss of generality that the cost matrix $\mathbf{C}$
		\lz
		
%		\centerline{Cost matrix }
		\begin{tabular}{cc|cc}
			& &\multicolumn{2}{c}{True class} \\
			& & $y=1$ & $y=-1$  \\
			\hline
			\multirow{2}{*}{\parbox{0.3cm}{Pred.  class}}& $\hat y$ = 1     & $C(1,1)$                & $C(1,-1)$\\
			& $\hat y$ = -1 & $C(-1,1)$              &  $C(-1,-1)$   \\
		\end{tabular}
%		

		\lz
		is of a simpler form  $\mathbf{C}_{simple}$:
		\lz
		\lz
		
%		
%		\centerline{Simple Cost matrix $\mathbf{C}_{simple}$}
		\begin{tabular}{cc|cc}
			& &\multicolumn{2}{c}{True class} \\
			& & $y=1$ & $y=-1$  \\
			\hline
			\multirow{2}{*}{\parbox{0.3cm}{Pred.  class}}& $\hat y$ = 1     & 0                 & $C(1,-1) - C(-1,-1) $\\
			& $\hat y$ = -1 & $C(-1,1) -C(1,1)$              & 0\\
		\end{tabular}

		\lz
		\item An analogous result can be shown for the multiclass setting.
		
		
		\item With this simpler cost matrix (relabeling $\mathbf{C}$ by $\mathbf{C}_{simple}$), the decision to predict the positive class boils down to 
%		
		\begin{align*}
%			 	
			& p(-1 ~|~ \xv ) C(1,-1)   \leq  	p(1 ~|~ \xv ) C(-1,1)  \\
%			
			&\Leftrightarrow (1- p(1 ~|~ \xv ) ) C(1,-1)   \leq  	p(1 ~|~ \xv ) C(-1,1) \\
%			
			&\Leftrightarrow \underbrace{\frac{C(1,-1)}{C(1,-1) + C(-1,1) } }_{=:c^*}  \leq  	p(1 ~|~ \xv ) \\
			%				
		\end{align*}
%	
		\item This yields the optimal threshold value $c^*$ for probabilistic score classifiers, so that any probabilistic classifier $f$ using a probabilistic score $\pi:\Xspace \to [0,1]$ can be modified to
%		
		$$   \tilde f(\xv) = 2 \cdot \mathds{1}_{[ \pi(\xv) \geq c^*]} -1. $$
							
		\end{itemize}
	}
\end{vbframe}

%%	\begin{itemize}
	%		
	%		
	%		%\begin{center}
	%		\centerline{Confusion matrix}
	%		\begin{tabular}{cc|cc}
		%			& &\multicolumn{2}{c}{True class} \\
		%			& & $y=1$ & $y=-1$  \\
		%			\hline
		%			\multirow{2}{*}{\parbox{0.3cm}{Pred.  class}}& $\hat y$ = 1     & TP                 & FP\\
		%			& $\hat y$ = -1 & FN              & TN\\
		%			%& & P(y = 1) & P(y = 0)
		%		\end{tabular}
	%		
	%		\lz
	%		
	%		\centerline{Cost matrix }
	%		\begin{tabular}{cc|cc}
		%			& &\multicolumn{2}{c}{True class} \\
		%			& & $y=1$ & $y=-1$  \\
		%			\hline
		%			\multirow{2}{*}{\parbox{0.3cm}{Pred.  class}}& $\hat y$ = 1     & $C(1,1)$                & $C(1,-1)$\\
		%			& $\hat y$ = -1 & $C(-1,1)$              &  $C(-1,-1)$   \\
		%		\end{tabular}
	%		%\end{center}
	%		
	
	
	%	\end{itemize}



\begin{vbframe}{MetaCost: Overview}
%	
	\small{
	\begin{itemize}
%		
		\item Substantial work has gone into making individual algorithms cost-sensitive. A better solution would be to have a procedure that can convert a broad variety of cost-insensitive classifiers into cost-sensitive ones.
%		
		\item MetaCost is a wrapper method which can be used for \emph{any} type of classifier to obtain a cost-sensitive classifier. It treats the underlying classifier as a black-box in the sense that neither knowledge about its mechanism is required nor changes to its mechanism is needed.
%		
		\item MetaCost needs only a cost-matrix $\mathbf{C}$ for the underlying learning setting, which is used to adapt the decision boundaries towards predictions with low expected costs. \\
		{\scriptsize (Some tuning parameters are also needed.)}
%		
		\item Roughly speaking, the procedure of MetaCost is: relabel the training examples with their ``optimal'' classes, i.e., the ones with low expected costs, and apply the classifier on the relabeled data set.
		
%		
	\end{itemize}
%
	}
%	
\end{vbframe}


\begin{vbframe}{MetaCost: Algorithm}
	
	\scriptsize{
%		
 	The procedure of MetaCost is divided into three phases:
			%			
		\begin{minipage}{0.53\textwidth} 
%				
				\begin{enumerate}
%					
					\scriptsize
					\item Bagging --- The underlying classifier is used (trained) $L$ times on different bootstrapped samples of the training data, respectively.
%					
					\item Relabeling --- These $L$ trained classifiers are used to relabel the original training data set by taking the cost-matrix into account.
%					
					\item Cost-sensitivity ---  The classifier is trained on the relabeled data set resulting in a cost-sensitive classifier.
%					
				\end{enumerate}
%
	\scriptsize
	\lz 
			Predictions of the classifier $f$ are converted (if necessary) into probabilistic prediction:
			\begin{center}
				\tiny
							ProbPrediction$(j,f,\xv) = \begin{cases}
					(f(\xv))_j & \mbox{$f$ is a prob.\ classifier,} \\
					\mbox{One-hot$(f(\xv))$}& \mbox{else,}
				\end{cases}$
			\end{center}
%		
		\scriptsize
		where One-hot$(f(\xv))$ uses a one-hot-encoding of the prediction to make it a probability, i.e., one for the predicted class and 0 else. 
				%		
		\end{minipage}
		\begin{minipage}{0.45\textwidth} 
			\begin{algorithmic}
				
				\tiny
%				
				\State \textbf{MetaCost}  
				\State \textbf{Input:} 
				$\D = \{(\xi,\yi)\}_{i=1}^n$ training data, \\
				$L \in \N$ number of bagging iterations, \\
				$B \in \N$ bootstrap size, \\
				$f$ (black-box) classifier, 
				$\mathbf{C}$ cost matrix, 
%				$g = |\Yspace|$ number of classes
				\State \# 1st phase:
				\For{$l=1,\ldots,L$}
					\State $\D_l  \leftarrow $ BootstrapSample ($\D,B$)
					\State $f_l  \leftarrow $ train $f$ on $\D_l$
				\EndFor
				\State \# 2nd phase:
				\For{$i=1,\ldots,n$}
					\If{$\xi \in \D_l$ for all $l=1,\ldots,L$}
						\State $\tilde L \leftarrow \{1,\ldots,L\}$
					\Else
						\State $\tilde L \leftarrow \bigcup_{l: \xi \notin \D_l} \{l\}$
					\EndIf
					\For{$j=1,\ldots,g$} (relabel for binary case)
						\State $p_j(\xi)  \leftarrow \frac{1}{|\tilde L| } \sum_{l \in \tilde L}   p_j(\xi~|~ f_l) $
						\State $p_j(\xi~|~ f_l) = $ ProbPrediction$(j,f_l,\xi)$
						\State $\tilde y^{(i)} \leftarrow \argmin_{i^*} \sum_{j=1}^g p_j(\xi) C(i^*,j) $
					\EndFor
					\State $\tilde D \leftarrow \tilde D \cup \{(\xi,\tilde y^{(i)})\} $
				\EndFor
				\State \# 3rd phase:
%				
				\State $f_{meta} \leftarrow$ train $f$ on $\tilde D$
%			
			\end{algorithmic}
		\end{minipage}
		%
	}
	%	
\end{vbframe}


%\begin{vbframe}{MetaCost: Example}
%	%	
%	\small{
%		\begin{itemize}
%			%		
%			\item We compare C4.5 (decision tree) with MetaCost using C4.5 for the \href{http://staffwww.itn.liu.se/~aidvi/courses/06/dm/
%				labs/heart-c.arff}{heart data set}  in \href{ http://www.cs.waikato.ac.nz/ml/weka/}{Weka}. 
%			%		
%			\item The cost matrix $\mathbf{C}$ is 
%			
%			\begin{center}
%				\begin{tabular}{cc|cc}
%					& &\multicolumn{2}{c}{True class} \\
%					& & $y=1$ & $y=-1$  \\
%					\hline
%					\multirow{2}{*}{\parbox{0.3cm}{Pred.  class}}& $\hat y$ = 1     & $0$                & $ 1 $\\
%					& $\hat y$ = -1 & $ 4 $              &  $0$   \\
%				\end{tabular}
%			\end{center}
%		
%			\item The resulting confusion matrices are 
%			
%						
%			\begin{center}
%				\begin{tabular}{cc|cc}
%					& &\multicolumn{2}{c}{True class} \\
%					& MetaCost & $y=1$ & $y=-1$  \\
%					\hline
%					\multirow{2}{*}{\parbox{0.3cm}{Pred.  class}}& $\hat y$ = 1     & $104$                & $ 21 $\\
%					& $\hat y$ = -1 & $ 61 $              &  $117$   \\
%				\end{tabular}
%							\begin{tabular}{cc|cc}
%				& &\multicolumn{2}{c}{True class} \\
%				& C4.5 & $y=1$ & $y=-1$  \\
%				\hline
%				\multirow{2}{*}{\parbox{0.3cm}{Pred.  class}}& $\hat y$ = 1     & $138$                & $ 40 $\\
%				& $\hat y$ = -1 & $ 27$              &  $98$   \\
%			\end{tabular}
%			\end{center}
%		
%		
%			%		
%			\item The total cost of MetaCost is 145, while C4.5 has total costs of 187. However, MetaCost has $0.729$ correct classifications and C4.5 has $0.779.$ 
%			
%			%		
%		\end{itemize}
%		%
%	}
%	%	
%\end{vbframe}



%\begin{vbframe}{Cost-Sensitive Decision Trees}
%	%	
%	\small{
%		\begin{itemize}
%			%		
%			\item 
%			%		
%		\end{itemize}
%		%
%	}
%	%	
%\end{vbframe}


%
\endlecture
\end{document}
