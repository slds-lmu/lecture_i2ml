\documentclass[11pt,compress,t,notes=noshow, xcolor=table]{beamer}
\usepackage[]{graphicx}\usepackage[]{color}
% maxwidth is the original width if it is less than linewidth
% otherwise use linewidth (to make sure the graphics do not exceed the margin)
\makeatletter
\def\maxwidth{ %
  \ifdim\Gin@nat@width>\linewidth
    \linewidth
  \else
    \Gin@nat@width
  \fi
}
\makeatother

\definecolor{fgcolor}{rgb}{0.345, 0.345, 0.345}
\newcommand{\hlnum}[1]{\textcolor[rgb]{0.686,0.059,0.569}{#1}}%
\newcommand{\hlstr}[1]{\textcolor[rgb]{0.192,0.494,0.8}{#1}}%
\newcommand{\hlcom}[1]{\textcolor[rgb]{0.678,0.584,0.686}{\textit{#1}}}%
\newcommand{\hlopt}[1]{\textcolor[rgb]{0,0,0}{#1}}%
\newcommand{\hlstd}[1]{\textcolor[rgb]{0.345,0.345,0.345}{#1}}%
\newcommand{\hlkwa}[1]{\textcolor[rgb]{0.161,0.373,0.58}{\textbf{#1}}}%
\newcommand{\hlkwb}[1]{\textcolor[rgb]{0.69,0.353,0.396}{#1}}%
\newcommand{\hlkwc}[1]{\textcolor[rgb]{0.333,0.667,0.333}{#1}}%
\newcommand{\hlkwd}[1]{\textcolor[rgb]{0.737,0.353,0.396}{\textbf{#1}}}%
\let\hlipl\hlkwb

\usepackage{framed}
\makeatletter
\newenvironment{kframe}{%
 \def\at@end@of@kframe{}%
 \ifinner\ifhmode%
  \def\at@end@of@kframe{\end{minipage}}%
  \begin{minipage}{\columnwidth}%
 \fi\fi%
 \def\FrameCommand##1{\hskip\@totalleftmargin \hskip-\fboxsep
 \colorbox{shadecolor}{##1}\hskip-\fboxsep
     % There is no \\@totalrightmargin, so:
     \hskip-\linewidth \hskip-\@totalleftmargin \hskip\columnwidth}%
 \MakeFramed {\advance\hsize-\width
   \@totalleftmargin\z@ \linewidth\hsize
   \@setminipage}}%
 {\par\unskip\endMakeFramed%
 \at@end@of@kframe}
\makeatother

\definecolor{shadecolor}{rgb}{.97, .97, .97}
\definecolor{messagecolor}{rgb}{0, 0, 0}
\definecolor{warningcolor}{rgb}{1, 0, 1}
\definecolor{errorcolor}{rgb}{1, 0, 0}
\newenvironment{knitrout}{}{} % an empty environment to be redefined in TeX

\usepackage{alltt}
\newcommand{\SweaveOpts}[1]{}  % do not interfere with LaTeX
\newcommand{\SweaveInput}[1]{} % because they are not real TeX commands
\newcommand{\Sexpr}[1]{}       % will only be parsed by R



\usepackage[english]{babel}
\usepackage[utf8]{inputenc}

\usepackage{dsfont}
\usepackage{verbatim}
\usepackage{amsmath}
\usepackage{amsfonts}
\usepackage{bm}
\usepackage{csquotes}
\usepackage{multirow}
\usepackage{longtable}
\usepackage{booktabs}
\usepackage{enumerate}
\usepackage[absolute,overlay]{textpos}
\usepackage{psfrag}
\usepackage{algorithm}
\usepackage{algpseudocode}
\usepackage{eqnarray}
\usepackage{arydshln}
\usepackage{tabularx}
\usepackage{placeins}
\usepackage{tikz}
\usepackage{setspace}
\usepackage{colortbl}
\usepackage{mathtools}
\usepackage{wrapfig}
\usepackage{bm}
\usetikzlibrary{shapes,arrows,automata,positioning,calc,chains,trees, shadows}
\tikzset{
  %Define standard arrow tip
  >=stealth',
  %Define style for boxes
  punkt/.style={
    rectangle,
    rounded corners,
    draw=black, very thick,
    text width=6.5em,
    minimum height=2em,
    text centered},
  % Define arrow style
  pil/.style={
    ->,
    thick,
    shorten <=2pt,
    shorten >=2pt,}
}
\usepackage{subfig}


% Defines macros and environments

% basic latex stuff
\newcommand{\pkg}[1]{{\fontseries{b}\selectfont #1}} %fontstyle for R packages
\newcommand{\lz}{\vspace{0.5cm}} %vertical space
\newcommand{\dlz}{\vspace{1cm}} %double vertical space
\newcommand{\oneliner}[1] % Oneliner for important statements
{\begin{block}{}\begin{center}\begin{Large}#1\end{Large}\end{center}\end{block}}


%new environments
\newenvironment{vbframe}  %frame with breaks and verbatim
{
 \begin{frame}[containsverbatim,allowframebreaks]
}
{
\end{frame}
}

\newenvironment{vframe}  %frame with verbatim without breaks (to avoid numbering one slided frames)
{
 \begin{frame}[containsverbatim]
}
{
\end{frame}
}

\newenvironment{blocki}[1]   % itemize block
{
 \begin{block}{#1}\begin{itemize}
}
{
\end{itemize}\end{block}
}

\newenvironment{fragileframe}[2]{  %fragile frame with framebreaks
\begin{frame}[allowframebreaks, fragile, environment = fragileframe]
\frametitle{#1}
#2}
{\end{frame}}


\newcommand{\myframe}[2]{  %short for frame with framebreaks
\begin{frame}[allowframebreaks]
\frametitle{#1}
#2
\end{frame}}

\newcommand{\remark}[1]{
  \textbf{Remark:} #1
}


\newenvironment{deleteframe}
{
\begingroup
\usebackgroundtemplate{\includegraphics[width=\paperwidth,height=\paperheight]{../style/color/red.png}}
 \begin{frame}
}
{
\end{frame}
\endgroup
}
\newenvironment{simplifyframe}
{
\begingroup
\usebackgroundtemplate{\includegraphics[width=\paperwidth,height=\paperheight]{../style/color/yellow.png}}
 \begin{frame}
}
{
\end{frame}
\endgroup
}\newenvironment{draftframe}
{
\begingroup
\usebackgroundtemplate{\includegraphics[width=\paperwidth,height=\paperheight]{../style/color/green.jpg}}
 \begin{frame}
}
{
\end{frame}
\endgroup
}
% https://tex.stackexchange.com/a/261480: textcolor that works in mathmode
\makeatletter
\renewcommand*{\@textcolor}[3]{%
  \protect\leavevmode
  \begingroup
    \color#1{#2}#3%
  \endgroup
}
\makeatother

% 
% basic latex stuff
\newcommand{\pkg}[1]{{\fontseries{b}\selectfont #1}} %fontstyle for R packages
\newcommand{\lz}{\vspace{0.5cm}} %vertical space
\newcommand{\dlz}{\vspace{1cm}} %double vertical space
\newcommand{\oneliner}[1] % Oneliner for important statements
{\begin{block}{}\begin{center}\begin{Large}#1\end{Large}\end{center}\end{block}}


%new environments
\newenvironment{vbframe}  %frame with breaks and verbatim
{
 \begin{frame}[containsverbatim,allowframebreaks]
}
{
\end{frame}
}

\newenvironment{vframe}  %frame with verbatim without breaks (to avoid numbering one slided frames)
{
 \begin{frame}[containsverbatim]
}
{
\end{frame}
}

\newenvironment{blocki}[1]   % itemize block
{
 \begin{block}{#1}\begin{itemize}
}
{
\end{itemize}\end{block}
}

\newenvironment{fragileframe}[2]{  %fragile frame with framebreaks
\begin{frame}[allowframebreaks, fragile, environment = fragileframe]
\frametitle{#1}
#2}
{\end{frame}}


\newcommand{\myframe}[2]{  %short for frame with framebreaks
\begin{frame}[allowframebreaks]
\frametitle{#1}
#2
\end{frame}}

\newcommand{\remark}[1]{
  \textbf{Remark:} #1
}


\newenvironment{deleteframe}
{
\begingroup
\usebackgroundtemplate{\includegraphics[width=\paperwidth,height=\paperheight]{../style/color/red.png}}
 \begin{frame}
}
{
\end{frame}
\endgroup
}
\newenvironment{simplifyframe}
{
\begingroup
\usebackgroundtemplate{\includegraphics[width=\paperwidth,height=\paperheight]{../style/color/yellow.png}}
 \begin{frame}
}
{
\end{frame}
\endgroup
}\newenvironment{draftframe}
{
\begingroup
\usebackgroundtemplate{\includegraphics[width=\paperwidth,height=\paperheight]{../style/color/green.jpg}}
 \begin{frame}
}
{
\end{frame}
\endgroup
}
% https://tex.stackexchange.com/a/261480: textcolor that works in mathmode
\makeatletter
\renewcommand*{\@textcolor}[3]{%
  \protect\leavevmode
  \begingroup
    \color#1{#2}#3%
  \endgroup
}
\makeatother


%\usetheme{lmu-lecture}
\newcommand{\titlefigure}{figure_man/roc_metrics.png}
\newcommand{\learninggoals}{
\item Understand why accuracy is not an optimal performance measure for 
imbalanced labels
\item Understand the different measures computable from a confusion matrix
\item Be aware that each of these measures has a variety of names }
\usepackage{../../style/lmu-lecture}

\let\code=\texttt
\let\proglang=\textsf

\setkeys{Gin}{width=0.9\textwidth}

\title{Introduction to Machine Learning}
% \author{Bernd Bischl, Christoph Molnar, Daniel Schalk, Fabian Scheipl}
\institute{\href{https://compstat-lmu.github.io/lecture_i2ml/}{compstat-lmu.github.io/lecture\_i2ml}}
\date{}

\setbeamertemplate{frametitle}{\expandafter\uppercase\expandafter\insertframetitle}

\begin{document}

% This file loads R packages, configures knitr options and sets preamble.Rnw as parent file
% IF YOU MODIFY THIS, PLZ ALSO MODIFY setup.Rmd ACCORDINGLY...

% Defines macros and environments
% math spaces
\newcommand{\N}{\mathds{N}}                                                 % N, naturals
\newcommand{\Z}{\mathds{Z}}                                                 % Z, integers
\newcommand{\Q}{\mathds{Q}}                                                 % Q, rationals
\newcommand{\R}{\mathds{R}}                                                 % R, reals
\newcommand{\C}{\mathds{C}}                                                 % C, complex
\newcommand{\HS}{\mathcal{H}}                                               % H, hilbertspace
\newcommand{\continuous}{\mathcal{C}}                                       % C, space of continuous functions
\newcommand{\M}{\mathcal{M}} 												% machine numbers
\newcommand{\epsm}{\epsilon_m} 												% maximum error


% basic math stuff
\newcommand{\xt}{\tilde x}													% x tilde
\def\argmax{\mathop{\sf arg\,max}}                                          % argmax
\def\argmin{\mathop{\sf arg\,min}}                                          % argmin
\newcommand{\sign}{\operatorname{sign}}                                     % sign, signum
\newcommand{\I}{\mathbb{I}}                                                 % I, indicator
\newcommand{\order}{\mathcal{O}}                                            % O, order
\newcommand{\fp}[2]{\frac{\partial #1}{\partial #2}}                        % partial derivative
\newcommand{\pd}[2]{\frac{\partial{#1}}{\partial #2}}						% partial derivative

% sums and products
\newcommand{\sumin}{\sum_{i=1}^n}											% summation from i=1 to n
\newcommand{\sumkg}{\sum_{k=1}^g}											% summation from k=1 to g
\newcommand{\prodin}{\prod_{i=1}^n}											% product from i=1 to n
\newcommand{\prodkg}{\prod_{k=1}^g}											% product from k=1 to g

% linear algebra
\newcommand{\one}{\boldsymbol{1}}                                           % 1, unitvector
\newcommand{\id}{\mathrm{I}}                                                % I, identity
\newcommand{\diag}{\operatorname{diag}}                                     % diag, diagonal
\newcommand{\trace}{\operatorname{tr}}                                      % tr, trace
\newcommand{\spn}{\operatorname{span}}                                      % span
\newcommand{\scp}[2]{\left\langle #1, #2 \right\rangle}                     % <.,.>, scalarproduct
\newcommand{\mat}[1]{ 														% short pmatrix command
	\begin{pmatrix}
		#1
	\end{pmatrix}
}
\newcommand{\Amat}{\bm{A}}													% matrix A
\newcommand{\xv}{\bm{x}}													% vector x (bold)
\newcommand{\yv}{\bm{y}}														% vector y (bold)
\newcommand{\Deltab}{\bm{\Delta}}											% error term for vectors
															

% basic probability + stats
\renewcommand{\P}{\mathds{P}}                                               % P, probability
\newcommand{\E}{\mathds{E}}                                                 % E, expectation
\newcommand{\var}{\mathsf{Var}}                                             % Var, variance
\newcommand{\cov}{\mathsf{Cov}}                                             % Cov, covariance
\newcommand{\corr}{\mathsf{Corr}}                                           % Corr, correlation
\newcommand{\normal}{\mathcal{N}}                                           % N of the normal distribution
\newcommand{\iid}{\overset{i.i.d}{\sim}}                                    % dist with i.i.d superscript
\newcommand{\distas}[1]{\overset{#1}{\sim}}                                 % ... is distributed as ... 
% machine learning

%%%%%% ml - data
\newcommand{\Xspace}{\mathcal{X}}                                           % X, input space
\newcommand{\Yspace}{\mathcal{Y}}                                           % Y, output space
\newcommand{\nset}{\{1, \ldots, n\}}                                        % set from 1 to n
\newcommand{\pset}{\{1, \ldots, p\}}                                        % set from 1 to p
\newcommand{\gset}{\{1, \ldots, g\}}                                        % set from 1 to g
\newcommand{\Pxy}{\P_{xy}}                                                  % P_xy
\newcommand{\xy}{(x, y)}                                                    % observation (x, y)
\newcommand{\xvec}{(x_1, \ldots, x_p)^T}                                    % (x1, ..., xp) 
\newcommand{\D}{\mathcal{D}}                                                % D, data 
\newcommand{\Dset}{\{ (x^{(1)}, y^{(1)}), \ldots, (x^{(n)},  y^{(n)})\}}    % {(x1,y1)), ..., (xn,yn)}, data
\newcommand{\xdat}{\{ x^{(1)}, \ldots, x^{(n)}\}}   						 % {x1, ..., xn}, input data
\newcommand{\ydat}{\mathbf{y}}                                              % y (bold), vector of outcomes
\newcommand{\yvec}{(y^{(1)}, \hdots, y^{(n)})^T}                            % (y1, ..., yn), vector of outcomes
\renewcommand{\xi}[1][i]{x^{(#1)}}                                          % x^i, i-th observed value of x
\newcommand{\yi}[1][i]{y^{(#1)}}                                            % y^i, i-th observed value of y 
\newcommand{\xyi}{(\xi, \yi)}                                               % (x^i, y^i), i-th observation
\newcommand{\xivec}{(x^{(i)}_1, \ldots, x^{(i)}_p)^T}                       % (x1^i, ..., xp^i), i-th observation vector
\newcommand{\xj}{x_j}                                                       % x_j, j-th feature
\newcommand{\xjb}{\mathbf{x}_j}                                             % x_j (bold), j-th feature vecor
\newcommand{\xjvec}{(x^{(1)}_j, \ldots, x^{(n)}_j)^T}                       % (x^1_j, ..., x^n_j), j-th feature vector
\newcommand{\Dtrain}{\mathcal{D}_{\text{train}}}                            % D_train, training set
\newcommand{\Dtest}{\mathcal{D}_{\text{test}}}                              % D_test, test set

%%%%%% ml - models general

% continuous prediction function f
\newcommand{\fx}{f(x)}                                                      % f(x), continuous prediction function
\newcommand{\Hspace}{H}														% hypothesis space where f is from
\newcommand{\fh}{\hat{f}}                                                   % f hat, estimated prediction function
\newcommand{\fxh}{\fh(x)}                                                   % fhat(x)
\newcommand{\fxt}{f(x | \theta)}                                            % f(x | theta)
\newcommand{\fxi}{f(\xi)}                                                   % f(x^(i))
\newcommand{\fxih}{\hat{f}(\xi)}                                            % f(x^(i))
\newcommand{\fxit}{f(x^{(i)} | \theta)}                                     % f(x^(i) | theta)
\newcommand{\fhD}{\fh_{\D}}                                                 % fhat_D, estimate of f based on D
\newcommand{\fhDtrain}{\fh_{\Dtrain}}                                       % fhat_Dtrain, estimate of f based on D

% discrete prediction function h
\newcommand{\hx}{h(x)}                                                      % h(x), discrete prediction function
\newcommand{\hh}{\hat{h}}                                                   % h hat
\newcommand{\hxh}{\hat{h}(x)}                                               % hhat(x)
\newcommand{\hxt}{h(x | \theta)}                                            % h(x | theta)
\newcommand{\hxi}{h(\xi)}                                                   % h(x^(i))
\newcommand{\hxit}{h(x^{(i)} | \theta)}                                     % h(x^(i) | theta)

% yhat
\newcommand{\yh}{\hat{y}}                                                   % y hat for prediction of target
\newcommand{\yih}{\hat{y}}                                                  % y hat for prediction of target

% theta
\newcommand{\thetah}{\hat{\theta}}                                          % theta hat

% densities + probabilities
% pdf of x 
\newcommand{\pdf}{p}                                                        % p
\newcommand{\pdfx}{p(x)}                                                    % p(x)
\newcommand{\pixt}{\pi(x | \theta)}                                         % pi(x|theta), pdf of x given theta

% pdf of (x, y)
\newcommand{\pdfxy}{p(x,y)}                                                 % p(x, y)
\newcommand{\pdfxyt}{p(x, y | \theta)}                                      % p(x, y | theta)
\newcommand{\pdfxyit}{p(\xi, \yi | \theta)}                                 % p(x^(i), y^(i) | theta)

% pdf of x given y
\newcommand{\pdfxyk}{p(x | y=k)}                                            % p(x | y = k)
\newcommand{\lpdfxyk}{\log \pdfxyk}                                         % log p(x | y = k)
\newcommand{\pdfxiyk}{p(\xi | y=k)}                                         % p(x^i | y = k)

% prior probabilities
\newcommand{\pik}{\pi_k}                                                    % pi_k, prior
\newcommand{\lpik}{\log \pik}                                               % log pi_k, log of the prior

% posterior probabilities
\newcommand{\post}{\P(y = 1 | x)}                                           % P(y = 1 | x), post. prob for y=1
\newcommand{\pix}{\pi(x)}                                                   % pi(x), P(y = 1 | x)
\newcommand{\postk}{\P(y = k | x)}                                          % P(y = k | y), post. prob for y=k
\newcommand{\pikx}{\pi_k(x)}                                                % pi_k(x), P(y = k | x)
\newcommand{\pikxt}{\pi_k(x | \theta)}                                      % pi_k(x | theta), P(y = k | x, theta)
\newcommand{\pijx}{\pi_j(x)}                                                % pi_j(x), P(y = j | x)
\newcommand{\pdfygxt}{p(y |x, \theta)}                                      % p(y | x, theta)
\newcommand{\pdfyigxit}{p(\yi |\xi, \theta)}                                % p(y^i |x^i, theta)
\newcommand{\lpdfygxt}{\log \pdfygxt }                                      % log p(y | x, theta)
\newcommand{\lpdfyigxit}{\log \pdfyigxit}                                   % log p(y^i |x^i, theta)
\newcommand{\pixh}{\hat \pi(x)}                                             % pi(x) hat, P(y = 1 | x) hat
\newcommand{\pikxh}{\hat \pi_k(x)}                                          % pi_k(x) hat, P(y = k | x) hat

% residual and margin
\newcommand{\eps}{\epsilon}                                                 % residual, stochastic
\newcommand{\epsi}{\epsilon^{(i)}}                                          % epsilon^i, residual, stochastic
\newcommand{\epsh}{\hat{\epsilon}}                                          % residual, estimated
\newcommand{\yf}{y \fx}                                                     % y f(x), margin
\newcommand{\yfi}{\yi \fxi}                                                 % y^i f(x^i), margin
\newcommand{\Sigmah}{\hat \Sigma}											% estimated covariance matrix
\newcommand{\Sigmahj}{\hat \Sigma_j}										% estimated covariance matrix for the j-th class

% ml - loss, risk, likelihood
\newcommand{\Lxy}{L(y, f(x))}                                               % L(y, f(x)), loss function
\newcommand{\Lxyi}{L(\yi, \fxi)}                                            % L(y^i, f(x^i))
\newcommand{\Lxyt}{L(y, \fxt)}                                              % L(y, f(x | theta))
\newcommand{\Lxyit}{L(\yi, \fxit)}                                          % L(y^i, f(x^i | theta)
\newcommand{\risk}{\mathcal{R}}                                             % R, risk
\newcommand{\riskf}{\risk(f)}                                               % R(f), risk
\newcommand{\riske}{\mathcal{R}_{\text{emp}}}                               % R_emp, empirical risk
\newcommand{\riskef}{\riske(f)}                                             % R_emp(f)
\newcommand{\risket}{\mathcal{R}_{\text{emp}}(\theta)}                      % R_emp(theta)
\newcommand{\riskr}{\mathcal{R}_{\text{reg}}}                               % R_reg, regularized risk
\newcommand{\riskrt}{\mathcal{R}_{\text{reg}}(\theta)}                      % R_reg(theta)
\newcommand{\riskrf}{\riskr(f)}                                             % R_reg(f)
\newcommand{\LL}{\mathcal{L}}                                               % L, likelihood
\newcommand{\LLt}{\mathcal{L}(\theta)}                                      % L(theta), likelihood
\renewcommand{\ll}{\ell}                                                    % l, log-likelihood
\newcommand{\llt}{\ell(\theta)}                                             % l(theta), log-likelihood
\newcommand{\LS}{\mathfrak{L}}                                              % ????????????
\newcommand{\TS}{\mathfrak{T}}                                              % ??????????????
\newcommand{\errtrain}{\text{err}_{\text{train}}}                           % training error
\newcommand{\errtest}{\text{err}_{\text{test}}}                             % training error
\newcommand{\errexp}{\overline{\text{err}_{\text{test}}}}                   % training error

% resampling
\newcommand{\GE}[1]{GE(\fh_{#1})}                                           % Generalization error GE
\newcommand{\GEh}[1]{\widehat{GE}_{#1}}                                     % Estimated train error
\newcommand{\GED}{\GE{\D}}                                                  % Generalization error GE
\newcommand{\EGEn}{EGE_n}                                                   % Generalization error GE
\newcommand{\EDn}{\E_{|D| = n}}                                             % Generalization error GE


% ml - irace
\newcommand{\costs}{\mathcal{C}} % costs
\newcommand{\Celite}{\theta^*} % elite configurations
\newcommand{\instances}{\mathcal{I}} % sequence of instances
\newcommand{\budget}{\mathcal{B}} % computational budget
%! includes: evaluation-measures-classification

\lecturechapter{Evaluation: Measures for Binary Classification: ROC Measures}
\lecture{Introduction to Machine Learning}

% ------------------------------------------------------------------------------

% \begin{vbframe}{Imbalanced Binary Labels}
% 
% \begin{center}
% % FIGURE SOURCE: https://docs.google.com/drawings/d/1WERS9WXwS4zla86fk6ESQkskNN1WZMI1YCPprnp0Ew0/edit?usp=sharing
% \includegraphics[width=.9\textwidth]{figure_man/imbalanced.pdf}\\
% Classify all as \enquote{no disease} (green) $\rightarrow$ high accuracy.
% 
% \lz
% 
% \textbf{Accuracy Paradox}
% \end{center}
% 
% \end{vbframe}

% ------------------------------------------------------------------------------

% \begin{vbframe}{Imbalanced Costs}
% 
% \begin{center}
% % FIGURE SOURCE: https://docs.google.com/drawings/d/1GlmMqzpeNHU_rtPFIrJMlY9Iz6XexvHEwTl3dNYKyQU/edit?usp=sharing
% \includegraphics[width=.3\textwidth]{figure_man/imbalanced-costs.pdf}\\
% Classify incorrectly as \enquote{no disease} $\rightarrow$ very high cost
% 
% \end{center}
% 
% \end{vbframe}

% ------------------------------------------------------------------------------

\begin{vbframe}{class imbalance}

\begin{itemize}
 \item Consider a binary classifier for diagnosing a serious medical 
 condition.
 \item Here, label distribution is often \textbf{imbalanced}, i.e, not many 
 people have the disease.
 \item Evaluating on error rates is often inappropriate for scenarios with 
 imbalanced labels:
 \begin{itemize}
   \item Assume that only 0.5\,\% of 1000 patients have the disease.
   \item Always returning \enquote{no disease} has an error rate of 0.5\,\%, 
   corresponding to very high accuracy.
   \item However, this sends all sick patients home, which is the worst 
   possible system -- even classifying everyone as \enquote{disease} might be 
   better (depending on the treatment).
 \end{itemize}
 \item This problem is known as the \textbf{accuracy paradox}.
\end{itemize}

Classifying all observations as \enquote{no disease} (green) yields top 
accuracy simply because the \enquote{disease} occurs so rarely 
$\rightarrow$ accuracy paradox.

\lz

\begin{center}
  % FIGURE SOURCE: https://docs.google.com/drawings/d  /1WERS9WXwS4zla86fk6ESQkskNN1WZMI1YCPprnp0Ew0/edit?usp=sharing
  \includegraphics[width=0.7\textwidth]{figure_man/imbalanced.pdf}
\end{center}

\end{vbframe}
 
% ------------------------------------------------------------------------------

\begin{vbframe}{imbalanced costs}
 
\begin{itemize}
  \item Another point of view is \textbf{imbalanced costs}.
  \item In our example, classifying a sick patient as healthy should incur a 
  much higher loss then classifying a healthy patient as sick.
  \item The costs depend a lot on what happens next: we can likely assume that 
  our system is some type of screening filter, and often the next step after 
  labeling someone as sick might be a more invasive, expensive, but also  more 
  reliable test for the disease.
  \item Erroneously subjecting someone to this step is undesirable 
  (psychological, economic, medical expense), but sending someone home to get 
  worse or die seems is much more so.
  \item Such situations not only arise under label imbalance, but also when 
  costs differ (even though classes might be balanced).
  \item We could see this as imbalanced costs of misclassification, rather than 
  imbalanced labels; both situations are tightly connected.
\end{itemize}

\framebreak

\lz

\begin{minipage}[c]{0.65\textwidth}
  \raggedright
  \textbf{Imbalanced costs: } classifying incorrectly as \enquote{no disease} 
  incurs very high cost.
\end{minipage}%
\begin{minipage}[c]{0.35\textwidth}
  \centering
  \includegraphics[trim = 0 0 0 10, clip, width=0.4\textwidth]
  {figure_man/imbalanced-costs.pdf}
\end{minipage}

\lz

\begin{itemize}
  \item Problem: if we were able to specify costs precisely, we could evaluate 
  or even optimize on them.
  \item This important subfield of ML is called \textbf{cost-sensitive 
  learning}, which we will not cover in this lecture unit.
  \item Unfortunately, users tend to find it notoriously hard to come up with 
  precise cost figures in imbalanced scenarios.
  \item Evaluating \enquote{from different perspectives}, with multiple metrics, 
  often helps to get a first impression of system quality.
\end{itemize}
 
\end{vbframe}
 
% ------------------------------------------------------------------------------
 
 % \begin{vbframe}{Binary Classifiers and Costs}
 % \begin{itemize}
 %   \item Problem is: If we could specify costs precisely, we could evaluate against them, we might even optimize our model for them
 %   \item This important subfield of ML is called \textbf{cost-sensitive learning}, which we will not cover in this lecture unit
 %   \item Unfortunately, users often have a notoriously hard time to come up with precise cost numbers in imbalanced scenarios
 %   \item Evaluating "from different perspectives", with multiple metrics, often helps, especially to get a first impression
 %     of the quality of the system
 % \end{itemize}
 % \end{vbframe}

% ------------------------------------------------------------------------------

\begin{frame}{ROC Analysis}

\begin{itemize}
  \item \textbf{ROC analysis} is a subfield of ML which studies the evaluation 
  of binary prediction systems.
  \item ROC stands for \enquote{receiver operating characteristics} and was 
  initially developed by electrical engineers and radar engineers during World 
  War II for detecting enemy objects in battlefields -- still has the funny 
  name.
\end{itemize}

\lz

\begin{center}
\includegraphics[width=.4\textwidth]{figure_man/receiver_operator.jpg}
{\tiny \url{http://media.iwm.org.uk/iwm/mediaLib//39/media-39665/large.jpg}}
\end{center}

\end{frame}

% ------------------------------------------------------------------------------

%  \begin{vbframe}{Confusion Matrix and ROC Metrics}
%  \begin{itemize}
%    \item From now on, we will call one class "positive", one "negative" and 
%    their respective sizes $n_+$ and $n_-$.
%    \item The positive class is the more important, often smaller one.
%    \item We represent all predictions in a confusion matrix and count correct 
%    and incorrect class assignments
%    \item False Positive means: We assigned "positive", but were wrong
%  \end{itemize}
% % % FIGURE SOURCE: No source
%  \includegraphics[width=0.7\textwidth]{figure_man/roc-confmatrix1.png}
%  \end{vbframe}

% ------------------------------------------------------------------------------

% \begin{vbframe}{Confusion Matrix}
% 
% \begin{center}
% \small
% \renewcommand{\arraystretch}{1.5}
% \begin{tabular}{cc||cc}
%     & & \multicolumn{2}{c}{\bfseries True Class $y$}  \\
%     & & $+$ & $-$  \\ 
%     \hline \hline
%     \bfseries Pred.     & $+$ & TP & FP\\
%               $\yh$ & $-$ & FN & TN\\ 
% \end{tabular}
% \renewcommand{\arraystretch}{1}
% \end{center}
% 
% \begin{itemize}
%   \item $+$: \enquote{positive} class
%   \item $-$: \enquote{negative} class
%   \item $\np$: number of observations in $+$
%   \item $\nn$: number of observations in $-$
% \end{itemize}
% \end{vbframe}

% ------------------------------------------------------------------------------

\begin{vbframe}{Labels: ROC Metrics}
From the confusion matrix (binary case), we can calculate "ROC" metrics.

% % FIGURE SOURCE: No source
% \includegraphics[width=0.7\textwidth]{figure_man/roc-confmatrix2.png}

% \begin{center}
% \small
% \begin{tabular}{cc|>{\centering\arraybackslash}p{7em}>{\centering\arraybackslash}p{8em}|>{\centering\arraybackslash}p{8em}}
%     & & \multicolumn{2}{c}{\bfseries True Class $y$} & \\
%     & & $+$ & $-$ & \\
%     \hline
%     \bfseries Pred.     & $+$ & True Positive (TP)  & False Positive (FP) & Positive Predictive Value (PPV) = $\frac{\text{TP}}{\text{TP} + \text{FP}}$\\
%               $\hat{y}$ & $-$ & False Negative (FN) & True Negative (TN) & Negative Predictive Value (NPV) = $\frac{\text{TN}}{\text{FN} + \text{TN}}$\\
%     \hline
%     & & TPR = $\frac{\text{TP}}{\text{TP} + \text{FN}}$ & TNR = $\frac{\text{TN}}{\text{FP} + \text{TN}}$ & Accuracy = $\frac{\text{TP}+ \text{TN}}{\text{TOTAL}}$
% \end{tabular}
% \end{center}

\begin{center}
\small
\renewcommand{\arraystretch}{1.5}
\begin{tabular}{cc||cc|c}
    & & \multicolumn{2}{c|}{\bfseries True Class $y$} & \\
    & & $+$ & $-$ & \\ 
    \hline \hline
    \bfseries Pred.     & $+$ & TP & FP & PPV = $\frac{\text{TP}}{\text{TP} + \text{FP}}$\\
              $\yh$ & $-$ & FN & TN & NPV = $\frac{\text{TN}}{\text{FN} + \text{TN}}$\\
    \hline
    & & TPR = $\frac{\text{TP}}{\text{TP} + \text{FN}}$ & TNR = $\frac{\text{TN}}{\text{FP} + \text{TN}}$ & Accuracy = $\frac{\text{TP}+ \text{TN}}{\text{TOTAL}}$
\end{tabular}
\renewcommand{\arraystretch}{1}
\end{center}

\begin{itemize}
  \item True Positive Rate: how many of the true 1s did we predict as 1?
  \item True Negative Rate: how many of the true 0s did we predict as 0?
  \item Positive Predictive Value: if we predict 1, how likely is it a true 1?
  \item Negative Predictive Value: if we predict 0, how likely is it a true 0?
\end{itemize}
\end{vbframe}

% ------------------------------------------------------------------------------

\begin{vbframe}{Labels: ROC Metrics}

Example:

\begin{center}
  % FIGURE SOURCE: No source
  \includegraphics[width=\textwidth]{figure_man/roc-confmatrix-example.png}
\end{center}

\end{vbframe}

% ------------------------------------------------------------------------------

\begin{vbframe}{More metrics and alternative terminology}

Unfortunately, for many concepts in ROC, 2-3 different terms exist.

\begin{center}
% FIGURE SOURCE: https://en.wikipedia.org/wiki/F1_score#Diagnostic_testing
\includegraphics[width=0.95\textwidth]{figure_man/roc-confmatrix-allterms.png}
\end{center}
\href{https://en.wikipedia.org/wiki/F1_score#Diagnostic_testing}{\beamergotobutton{Clickable version/picture source}} $\phantom{blablabla}$
\href{https://upload.wikimedia.org/wikipedia/commons/0/0e/DiagnosticTesting_Diagram.svg}{\beamergotobutton{Interactive diagram}}
\end{vbframe}

% ------------------------------------------------------------------------------

\begin{vbframe}{Labels: $F_1$ Measure}

\small

\begin{itemize}
  \item It is difficult to achieve a high \textbf{Positive Predictive Value} and 
  high \textbf{True Positive Rate} simultaneously.
   \item A classifier predicting more positive will be more 
   sensitive (higher TPR), but it will also tend to give more \textbf{false} 
   positives (lower TNR, lower PPV).
   \item A classifier that predicts more negatives will be more precise 
   (higher PPV), but it will also produce more \textbf{false} negatives 
   (lower TPR).
 \end{itemize}

The \textbf{$F_1$ score} balances two conflicting goals:\\%[.5em]
\begin{enumerate}
 \item Maximising Positive Predictive Value
 \item Maximising True Positive Rate\\[.5em]
\end{enumerate}

$F_1$ is the harmonic mean of PPV and TPR:
$$F_1 = 2 \cdot \cfrac{PPV \cdot TPR}{PPV + TPR}$$

Note that this measure still does not account for the number of true
negatives.

\framebreak

\normalsize

\begin{minipage}[c]{0.5\textwidth}
  % \small
  $F_1$ score for different combinations of TPR \& PPV. \\
  $\rightarrow$ Tends more towards the lower of the two combined values.
\end{minipage}%
\begin{minipage}[c]{0.5\textwidth}
  \centering
  \includegraphics[width=0.8\textwidth]{figure/eval_mclass_roc.pdf}
\end{minipage}

% Tabulated $F_1$-Score for different TPR (rows) and PPV (cols) combinations:
% \begin{knitrout}\scriptsize
% \definecolor{shadecolor}{rgb}{0.969, 0.969, 0.969}\color{fgcolor}
% 
% \includegraphics[width=0.5\textwidth]{figure/eval_mclass_roc.pdf}
% 
% \end{knitrout}
% $\rightarrow$ Tends more towards the lower of the two combined values.

\begin{itemize}
  \item A model with $TPR=0$ (no positive instance predicted as positive) or 
  $PPV=0$ (no true positives among the predicted) has $F_1 = 0$.
  \item Always predicting \enquote{negative}: $F_1 = 0$.
  \item Always predicting \enquote{positive}: $F_1 = 2 \cdot PPV / (PPV + 1) = 
  2 \cdot \np / (\np + n)$,\\ 
  which will be small when the size of the positive class $\np$ is small.
\end{itemize}

\end{vbframe}

% ------------------------------------------------------------------------------

\begin{vbframe}{which metric to use?}

\textcolor{blue}{WIP (inspo: chapter 3.3. in Japkowicz/Shah (2011)}

\begin{itemize}
  \item As we have seen, there is a plethora of methods applicable to evaluating 
  classifiers.
  \item Different metrics might lead to different conclusions.
\end{itemize}

\includegraphics[width=0.8\textwidth]{figure/eval_mclass_benchmark.pdf}

\end{vbframe}

% ------------------------------------------------------------------------------

\endlecture

\end{document}
