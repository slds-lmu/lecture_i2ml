\usepackage[]{graphicx}
\usepackage[]{color}
% maxwidth is the original width if it is less than linewidth
% otherwise use linewidth (to make sure the graphics do not exceed the margin)
\makeatletter
\def\maxwidth{ %
  \ifdim\Gin@nat@width>\linewidth
    \linewidth
  \else
    \Gin@nat@width
  \fi
}
\makeatother

% ---------------------------------%
% latex-math dependencies, do not remove:
% - \usepackage{mathtools}
% - \usepackage{bm}
% - \usepackage{siunitx}
% - \usepackage{dsfont}
% - \usepackage{xspace}
% ---------------------------------%

%--------------------------------------------------------%
%       Language, encoding, typography
%--------------------------------------------------------%

\usepackage[english]{babel}
\usepackage[utf8]{inputenc} % Enables inputting UTF-8 symbols
% Standard AMS suite
\usepackage{amsmath,amsfonts,amssymb}

% Font four double-stroke / blackboard letters for sets of numbers (N, R, ...)
% Distribution name is "doublestroke"
% According to https://mirror.physik.tu-berlin.de/pub/CTAN/fonts/doublestroke/dsdoc.pdf
% the "bbm" package does a similar thing and may be superfluous.
% Required for latex-math
\usepackage{dsfont}

% bbm – "Blackboard-style" cm fonts (https://www.ctan.org/pkg/bbm)
% Used to be in common.tex, loaded directly after this file
% Maybe superfluous given dsfont is loaded
% TODO: Check if really unused?
% \usepackage{bbm}

% bm – Access bold symbols in maths mode - https://ctan.org/pkg/bm
% Required for latex-math
% https://tex.stackexchange.com/questions/3238/bm-package-versus-boldsymbol
\usepackage{bm}

% pifont – Access to PostScript standard Symbol and Dingbats fonts
% Used for \newcommand{\xmark}{\ding{55}, which is never used
% aside from lecture_advml/attic/xx-automl/slides.Rnw
% \usepackage{pifont}

% Quotes (inline and display), provdes \enquote
% https://ctan.org/pkg/csquotes
\usepackage{csquotes}

% Adds arg to enumerate env, technically superseded by enumitem according
% to https://ctan.org/pkg/enumerate
% Replace with https://ctan.org/pkg/enumitem ?
\usepackage{enumerate}

% Line spacing - provides \singlespacing \doublespacing \onehalfspacing
% https://ctan.org/pkg/setspace
% TODO: Check if really unused?
%\usepackage{setspace}

% mathtools – Mathematical tools to use with amsmath
% https://ctan.org/pkg/mathtools?lang=en
% latex-math dependency according to latex-math repo
\usepackage{mathtools}

%--------------------------------------------------------%
%       Displaying code and algorithms
%--------------------------------------------------------%
\usepackage{verbatim}
\usepackage{algorithm}
\usepackage{algpseudocode}

%--------------------------------------------------------%
%       Tables
%--------------------------------------------------------%

% multi-row table cells: https://www.namsu.de/Extra/pakete/Multirow.html
\usepackage{multirow}

% long/multi-page tables: https://texdoc.org/serve/longtable.pdf/0
% TODO: Check if really unused?

\usepackage{longtable}

% pretty table env: https://ctan.org/pkg/booktabs?lang=en
% TODO: Check if really unused?
\usepackage{booktabs}

%--------------------------------------------------------%
%       Figures: Creating, placing, verbing
%--------------------------------------------------------%

% wrapfig - Wrapping text around figures https://de.overleaf.com/learn/latex/Wrapping_text_around_figures
\usepackage{wrapfig}

% Sub figures in figures and tables
% https://ctan.org/pkg/subfig -- supersedes subfigure package
% TODO: Check if really unused?
\usepackage{subfig}

% Actually it's pronounced PGF https://en.wikibooks.org/wiki/LaTeX/PGF/TikZ
\usepackage{tikz}

\usetikzlibrary{shapes,arrows,automata,positioning,calc,chains,trees, shadows}
\tikzset{
  %Define standard arrow tip
  >=stealth',
  %Define style for boxes
  punkt/.style={
    rectangle,
    rounded corners,
    draw=black, very thick,
    text width=6.5em,
    minimum height=2em,
    text centered},
  % Define arrow style
  pil/.style={
    ->,
    thick,
    shorten <=2pt,
    shorten >=2pt,}
}


% Unsorted
% textpos – Place boxes at arbitrary positions on the LATEX page
% https://ctan.org/pkg/textpos?lang=en
% Provides \begin{textblock}
 % TODO: Check if really unused?
\usepackage[absolute,overlay]{textpos}

% psfrag – Replace strings in encapsulated PostScript figures
% https://www.overleaf.com/latex/examples/psfrag-example/tggxhgzwrzhn
% https://ftp.mpi-inf.mpg.de/pub/tex/mirror/ftp.dante.de/pub/tex/macros/latex/contrib/psfrag/pfgguide.pdf
% Can't tell if this is needed
% TODO: Check if really unused?
\usepackage{psfrag}

% Maybe not great to use this https://tex.stackexchange.com/a/197/19093
% Use align instead -- TODO: Global search & replace to check
\usepackage{eqnarray}

\usepackage{colortbl}

% arydshln – Draw dash-lines in array/tabular
% https://www.ctan.org/pkg/arydshln
% !! "arydshln has to be loaded after array, longtable, colortab and/or colortbl"
% Provides \hdashline and \cdashline
% TODO: Check if really unused?
% \usepackage{arydshln}

% tabularx – Tabulars with adjustable-width columns
% https://ctan.org/pkg/tabularx
% Provides \begin{tabularx}
% TODO: Check if really unused?
% \usepackage{tabularx}

% placeins – Control float placement
% https://ctan.org/pkg/placeins
% Defines a \FloatBarrier command
% TODO: Check if really unused?
% \usepackage{placeins}


% framed – Framed or shaded regions that can break across pages
% https://ctan.org/pkg/framed
% Provides \begin{framed} which uses \colorbox{shadecolor} relying on \definecolor{shadecolor}.
% TODO: Check if really unused?
% \usepackage{framed}

% Used often in conjunction with \definecolor{shadecolor}{rgb}{0.969, 0.969, 0.969}
% Might be able to be removed or at least redefined to only have shadecolor (if needed)
\definecolor{fgcolor}{rgb}{0.345, 0.345, 0.345}
\definecolor{shadecolor}{rgb}{0.969, 0.969, 0.969}
\newenvironment{knitrout}{}{} % an empty environment to be redefined in TeX


% Defines macros and environments
\usepackage{../../style/lmu-lecture}

\let\code=\texttt % Used regularly
\let\proglang=\textsf % Unused?

% Not sure what/why this does
\setkeys{Gin}{width=0.9\textwidth}

\setbeamertemplate{frametitle}{\expandafter\uppercase\expandafter\insertframetitle}

% Can't find a reason why common.tex is not just part of this file?

% basic latex stuff
\newcommand{\pkg}[1]{{\fontseries{b}\selectfont #1}} %fontstyle for R packages
\newcommand{\lz}{\vspace{0.5cm}} %vertical space
\newcommand{\dlz}{\vspace{1cm}} %double vertical space
\newcommand{\oneliner}[1] % Oneliner for important statements
{\begin{block}{}\begin{center}\begin{Large}#1\end{Large}\end{center}\end{block}}


%new environments
\newenvironment{vbframe}  %frame with breaks and verbatim
{
 \begin{frame}[containsverbatim,allowframebreaks]
}
{
\end{frame}
}

\newenvironment{vframe}  %frame with verbatim without breaks (to avoid numbering one slided frames)
{
 \begin{frame}[containsverbatim]
}
{
\end{frame}
}

\newenvironment{blocki}[1]   % itemize block
{
 \begin{block}{#1}\begin{itemize}
}
{
\end{itemize}\end{block}
}

\newenvironment{fragileframe}[2]{  %fragile frame with framebreaks
\begin{frame}[allowframebreaks, fragile, environment = fragileframe]
\frametitle{#1}
#2}
{\end{frame}}


\newcommand{\myframe}[2]{  %short for frame with framebreaks
\begin{frame}[allowframebreaks]
\frametitle{#1}
#2
\end{frame}}

\newcommand{\remark}[1]{
  \textbf{Remark:} #1
}


\newenvironment{deleteframe}
{
\begingroup
\usebackgroundtemplate{\includegraphics[width=\paperwidth,height=\paperheight]{../style/color/red.png}}
 \begin{frame}
}
{
\end{frame}
\endgroup
}
\newenvironment{simplifyframe}
{
\begingroup
\usebackgroundtemplate{\includegraphics[width=\paperwidth,height=\paperheight]{../style/color/yellow.png}}
 \begin{frame}
}
{
\end{frame}
\endgroup
}\newenvironment{draftframe}
{
\begingroup
\usebackgroundtemplate{\includegraphics[width=\paperwidth,height=\paperheight]{../style/color/green.jpg}}
 \begin{frame}
}
{
\end{frame}
\endgroup
}
% https://tex.stackexchange.com/a/261480: textcolor that works in mathmode
\makeatletter
\renewcommand*{\@textcolor}[3]{%
  \protect\leavevmode
  \begingroup
    \color#1{#2}#3%
  \endgroup
}
\makeatother


%-------------------------------------------------------------------------------------------------------%
%  Unused stuff that needs to go but is kept here currently juuuust in case it was important after all  %
%-------------------------------------------------------------------------------------------------------%

% \newcommand{\hlnum}[1]{\textcolor[rgb]{0.686,0.059,0.569}{#1}}%
% \newcommand{\hlstr}[1]{\textcolor[rgb]{0.192,0.494,0.8}{#1}}%
% \newcommand{\hlcom}[1]{\textcolor[rgb]{0.678,0.584,0.686}{\textit{#1}}}%
% \newcommand{\hlopt}[1]{\textcolor[rgb]{0,0,0}{#1}}%
% \newcommand{\hlstd}[1]{\textcolor[rgb]{0.345,0.345,0.345}{#1}}%
% \newcommand{\hlkwa}[1]{\textcolor[rgb]{0.161,0.373,0.58}{\textbf{#1}}}%
% \newcommand{\hlkwb}[1]{\textcolor[rgb]{0.69,0.353,0.396}{#1}}%
% \newcommand{\hlkwc}[1]{\textcolor[rgb]{0.333,0.667,0.333}{#1}}%
% \newcommand{\hlkwd}[1]{\textcolor[rgb]{0.737,0.353,0.396}{\textbf{#1}}}%
% \let\hlipl\hlkwb

% \makeatletter
% \newenvironment{kframe}{%
%  \def\at@end@of@kframe{}%
%  \ifinner\ifhmode%
%   \def\at@end@of@kframe{\end{minipage}}%
%   \begin{minipage}{\columnwidth}%
%  \fi\fi%
%  \def\FrameCommand##1{\hskip\@totalleftmargin \hskip-\fboxsep
%  \colorbox{shadecolor}{##1}\hskip-\fboxsep
%      % There is no \\@totalrightmargin, so:
%      \hskip-\linewidth \hskip-\@totalleftmargin \hskip\columnwidth}%
%  \MakeFramed {\advance\hsize-\width
%    \@totalleftmargin\z@ \linewidth\hsize
%    \@setminipage}}%
%  {\par\unskip\endMakeFramed%
%  \at@end@of@kframe}
% \makeatother

% \definecolor{shadecolor}{rgb}{.97, .97, .97}
% \definecolor{messagecolor}{rgb}{0, 0, 0}
% \definecolor{warningcolor}{rgb}{1, 0, 1}
% \definecolor{errorcolor}{rgb}{1, 0, 0}
% \newenvironment{knitrout}{}{} % an empty environment to be redefined in TeX

% \usepackage{alltt}
% \newcommand{\SweaveOpts}[1]{}  % do not interfere with LaTeX
% \newcommand{\SweaveInput}[1]{} % because they are not real TeX commands
% \newcommand{\Sexpr}[1]{}       % will only be parsed by R
% \newcommand{\xmark}{\ding{55}}%

% math spaces
\newcommand{\N}{\mathds{N}}                                                 % N, naturals
\newcommand{\Z}{\mathds{Z}}                                                 % Z, integers
\newcommand{\Q}{\mathds{Q}}                                                 % Q, rationals
\newcommand{\R}{\mathds{R}}                                                 % R, reals
\newcommand{\C}{\mathds{C}}                                                 % C, complex
\newcommand{\HS}{\mathcal{H}}                                               % H, hilbertspace
\newcommand{\continuous}{\mathcal{C}}                                       % C, space of continuous functions
\newcommand{\M}{\mathcal{M}} 												% machine numbers
\newcommand{\epsm}{\epsilon_m} 												% maximum error


% basic math stuff
\newcommand{\xt}{\tilde x}													% x tilde
\def\argmax{\mathop{\sf arg\,max}}                                          % argmax
\def\argmin{\mathop{\sf arg\,min}}                                          % argmin
\newcommand{\sign}{\operatorname{sign}}                                     % sign, signum
\newcommand{\I}{\mathbb{I}}                                                 % I, indicator
\newcommand{\order}{\mathcal{O}}                                            % O, order
\newcommand{\fp}[2]{\frac{\partial #1}{\partial #2}}                        % partial derivative
\newcommand{\pd}[2]{\frac{\partial{#1}}{\partial #2}}						% partial derivative

% sums and products
\newcommand{\sumin}{\sum_{i=1}^n}											% summation from i=1 to n
\newcommand{\sumkg}{\sum_{k=1}^g}											% summation from k=1 to g
\newcommand{\prodin}{\prod_{i=1}^n}											% product from i=1 to n
\newcommand{\prodkg}{\prod_{k=1}^g}											% product from k=1 to g

% linear algebra
\newcommand{\one}{\boldsymbol{1}}                                           % 1, unitvector
\newcommand{\id}{\mathrm{I}}                                                % I, identity
\newcommand{\diag}{\operatorname{diag}}                                     % diag, diagonal
\newcommand{\trace}{\operatorname{tr}}                                      % tr, trace
\newcommand{\spn}{\operatorname{span}}                                      % span
\newcommand{\scp}[2]{\left\langle #1, #2 \right\rangle}                     % <.,.>, scalarproduct
\newcommand{\mat}[1]{ 														% short pmatrix command
	\begin{pmatrix}
		#1
	\end{pmatrix}
}
\newcommand{\Amat}{\bm{A}}													% matrix A
\newcommand{\xv}{\bm{x}}													% vector x (bold)
\newcommand{\yv}{\bm{y}}														% vector y (bold)
\newcommand{\Deltab}{\bm{\Delta}}											% error term for vectors
															

% basic probability + stats
\renewcommand{\P}{\mathds{P}}                                               % P, probability
\newcommand{\E}{\mathds{E}}                                                 % E, expectation
\newcommand{\var}{\mathsf{Var}}                                             % Var, variance
\newcommand{\cov}{\mathsf{Cov}}                                             % Cov, covariance
\newcommand{\corr}{\mathsf{Corr}}                                           % Corr, correlation
\newcommand{\normal}{\mathcal{N}}                                           % N of the normal distribution
\newcommand{\iid}{\overset{i.i.d}{\sim}}                                    % dist with i.i.d superscript
\newcommand{\distas}[1]{\overset{#1}{\sim}}                                 % ... is distributed as ... 
% machine learning

%%%%%% ml - data
\newcommand{\Xspace}{\mathcal{X}}                                           % X, input space
\newcommand{\Yspace}{\mathcal{Y}}                                           % Y, output space
\newcommand{\nset}{\{1, \ldots, n\}}                                        % set from 1 to n
\newcommand{\pset}{\{1, \ldots, p\}}                                        % set from 1 to p
\newcommand{\gset}{\{1, \ldots, g\}}                                        % set from 1 to g
\newcommand{\Pxy}{\P_{xy}}                                                  % P_xy
\newcommand{\xy}{(x, y)}                                                    % observation (x, y)
\newcommand{\xvec}{(x_1, \ldots, x_p)^T}                                    % (x1, ..., xp) 
\newcommand{\D}{\mathcal{D}}                                                % D, data 
\newcommand{\Dset}{\{ (x^{(1)}, y^{(1)}), \ldots, (x^{(n)},  y^{(n)})\}}    % {(x1,y1)), ..., (xn,yn)}, data
\newcommand{\xdat}{\{ x^{(1)}, \ldots, x^{(n)}\}}   						 % {x1, ..., xn}, input data
\newcommand{\ydat}{\mathbf{y}}                                              % y (bold), vector of outcomes
\newcommand{\yvec}{(y^{(1)}, \hdots, y^{(n)})^T}                            % (y1, ..., yn), vector of outcomes
\renewcommand{\xi}[1][i]{x^{(#1)}}                                          % x^i, i-th observed value of x
\newcommand{\yi}[1][i]{y^{(#1)}}                                            % y^i, i-th observed value of y 
\newcommand{\xyi}{(\xi, \yi)}                                               % (x^i, y^i), i-th observation
\newcommand{\xivec}{(x^{(i)}_1, \ldots, x^{(i)}_p)^T}                       % (x1^i, ..., xp^i), i-th observation vector
\newcommand{\xj}{x_j}                                                       % x_j, j-th feature
\newcommand{\xjb}{\mathbf{x}_j}                                             % x_j (bold), j-th feature vecor
\newcommand{\xjvec}{(x^{(1)}_j, \ldots, x^{(n)}_j)^T}                       % (x^1_j, ..., x^n_j), j-th feature vector
\newcommand{\Dtrain}{\mathcal{D}_{\text{train}}}                            % D_train, training set
\newcommand{\Dtest}{\mathcal{D}_{\text{test}}}                              % D_test, test set

%%%%%% ml - models general

% continuous prediction function f
\newcommand{\fx}{f(x)}                                                      % f(x), continuous prediction function
\newcommand{\Hspace}{H}														% hypothesis space where f is from
\newcommand{\fh}{\hat{f}}                                                   % f hat, estimated prediction function
\newcommand{\fxh}{\fh(x)}                                                   % fhat(x)
\newcommand{\fxt}{f(x | \theta)}                                            % f(x | theta)
\newcommand{\fxi}{f(\xi)}                                                   % f(x^(i))
\newcommand{\fxih}{\hat{f}(\xi)}                                            % f(x^(i))
\newcommand{\fxit}{f(x^{(i)} | \theta)}                                     % f(x^(i) | theta)
\newcommand{\fhD}{\fh_{\D}}                                                 % fhat_D, estimate of f based on D
\newcommand{\fhDtrain}{\fh_{\Dtrain}}                                       % fhat_Dtrain, estimate of f based on D

% discrete prediction function h
\newcommand{\hx}{h(x)}                                                      % h(x), discrete prediction function
\newcommand{\hh}{\hat{h}}                                                   % h hat
\newcommand{\hxh}{\hat{h}(x)}                                               % hhat(x)
\newcommand{\hxt}{h(x | \theta)}                                            % h(x | theta)
\newcommand{\hxi}{h(\xi)}                                                   % h(x^(i))
\newcommand{\hxit}{h(x^{(i)} | \theta)}                                     % h(x^(i) | theta)

% yhat
\newcommand{\yh}{\hat{y}}                                                   % y hat for prediction of target
\newcommand{\yih}{\hat{y}}                                                  % y hat for prediction of target

% theta
\newcommand{\thetah}{\hat{\theta}}                                          % theta hat

% densities + probabilities
% pdf of x 
\newcommand{\pdf}{p}                                                        % p
\newcommand{\pdfx}{p(x)}                                                    % p(x)
\newcommand{\pixt}{\pi(x | \theta)}                                         % pi(x|theta), pdf of x given theta

% pdf of (x, y)
\newcommand{\pdfxy}{p(x,y)}                                                 % p(x, y)
\newcommand{\pdfxyt}{p(x, y | \theta)}                                      % p(x, y | theta)
\newcommand{\pdfxyit}{p(\xi, \yi | \theta)}                                 % p(x^(i), y^(i) | theta)

% pdf of x given y
\newcommand{\pdfxyk}{p(x | y=k)}                                            % p(x | y = k)
\newcommand{\lpdfxyk}{\log \pdfxyk}                                         % log p(x | y = k)
\newcommand{\pdfxiyk}{p(\xi | y=k)}                                         % p(x^i | y = k)

% prior probabilities
\newcommand{\pik}{\pi_k}                                                    % pi_k, prior
\newcommand{\lpik}{\log \pik}                                               % log pi_k, log of the prior

% posterior probabilities
\newcommand{\post}{\P(y = 1 | x)}                                           % P(y = 1 | x), post. prob for y=1
\newcommand{\pix}{\pi(x)}                                                   % pi(x), P(y = 1 | x)
\newcommand{\postk}{\P(y = k | x)}                                          % P(y = k | y), post. prob for y=k
\newcommand{\pikx}{\pi_k(x)}                                                % pi_k(x), P(y = k | x)
\newcommand{\pikxt}{\pi_k(x | \theta)}                                      % pi_k(x | theta), P(y = k | x, theta)
\newcommand{\pijx}{\pi_j(x)}                                                % pi_j(x), P(y = j | x)
\newcommand{\pdfygxt}{p(y |x, \theta)}                                      % p(y | x, theta)
\newcommand{\pdfyigxit}{p(\yi |\xi, \theta)}                                % p(y^i |x^i, theta)
\newcommand{\lpdfygxt}{\log \pdfygxt }                                      % log p(y | x, theta)
\newcommand{\lpdfyigxit}{\log \pdfyigxit}                                   % log p(y^i |x^i, theta)
\newcommand{\pixh}{\hat \pi(x)}                                             % pi(x) hat, P(y = 1 | x) hat
\newcommand{\pikxh}{\hat \pi_k(x)}                                          % pi_k(x) hat, P(y = k | x) hat

% residual and margin
\newcommand{\eps}{\epsilon}                                                 % residual, stochastic
\newcommand{\epsi}{\epsilon^{(i)}}                                          % epsilon^i, residual, stochastic
\newcommand{\epsh}{\hat{\epsilon}}                                          % residual, estimated
\newcommand{\yf}{y \fx}                                                     % y f(x), margin
\newcommand{\yfi}{\yi \fxi}                                                 % y^i f(x^i), margin
\newcommand{\Sigmah}{\hat \Sigma}											% estimated covariance matrix
\newcommand{\Sigmahj}{\hat \Sigma_j}										% estimated covariance matrix for the j-th class

% ml - loss, risk, likelihood
\newcommand{\Lxy}{L(y, f(x))}                                               % L(y, f(x)), loss function
\newcommand{\Lxyi}{L(\yi, \fxi)}                                            % L(y^i, f(x^i))
\newcommand{\Lxyt}{L(y, \fxt)}                                              % L(y, f(x | theta))
\newcommand{\Lxyit}{L(\yi, \fxit)}                                          % L(y^i, f(x^i | theta)
\newcommand{\risk}{\mathcal{R}}                                             % R, risk
\newcommand{\riskf}{\risk(f)}                                               % R(f), risk
\newcommand{\riske}{\mathcal{R}_{\text{emp}}}                               % R_emp, empirical risk
\newcommand{\riskef}{\riske(f)}                                             % R_emp(f)
\newcommand{\risket}{\mathcal{R}_{\text{emp}}(\theta)}                      % R_emp(theta)
\newcommand{\riskr}{\mathcal{R}_{\text{reg}}}                               % R_reg, regularized risk
\newcommand{\riskrt}{\mathcal{R}_{\text{reg}}(\theta)}                      % R_reg(theta)
\newcommand{\riskrf}{\riskr(f)}                                             % R_reg(f)
\newcommand{\LL}{\mathcal{L}}                                               % L, likelihood
\newcommand{\LLt}{\mathcal{L}(\theta)}                                      % L(theta), likelihood
\renewcommand{\ll}{\ell}                                                    % l, log-likelihood
\newcommand{\llt}{\ell(\theta)}                                             % l(theta), log-likelihood
\newcommand{\LS}{\mathfrak{L}}                                              % ????????????
\newcommand{\TS}{\mathfrak{T}}                                              % ??????????????
\newcommand{\errtrain}{\text{err}_{\text{train}}}                           % training error
\newcommand{\errtest}{\text{err}_{\text{test}}}                             % training error
\newcommand{\errexp}{\overline{\text{err}_{\text{test}}}}                   % training error

% resampling
\newcommand{\GE}[1]{GE(\fh_{#1})}                                           % Generalization error GE
\newcommand{\GEh}[1]{\widehat{GE}_{#1}}                                     % Estimated train error
\newcommand{\GED}{\GE{\D}}                                                  % Generalization error GE
\newcommand{\EGEn}{EGE_n}                                                   % Generalization error GE
\newcommand{\EDn}{\E_{|D| = n}}                                             % Generalization error GE


% ml - irace
\newcommand{\costs}{\mathcal{C}} % costs
\newcommand{\Celite}{\theta^*} % elite configurations
\newcommand{\instances}{\mathcal{I}} % sequence of instances
\newcommand{\budget}{\mathcal{B}} % computational budget

\newcommand{\titlefigure}{figure_man/rect_upper_x.png}
\newcommand{\learninggoals}{
  \item Know PAC learning
  \item Know that there is no "universal" learner which works on every task (no free lunch)
  \item Know that complexity of a hypothesis space can be measured by VC dimension
  \item Know that a hypothesis space is PAC learnable iff it has finite VC dimension
}

\title{Introduction to Machine Learning}
\date{}

\begin{document}

\lecturechapter{PAC Learning and VC Dimension}
\lecture{Introduction to Machine Learning}


\begin{vbframe}{PAC Learning}
A hypothesis space $\Hspace$ over a data space $\Xspace \times \Yspace$ is agnostic PAC learnable, if there exists a function $n_\Hspace: (0,1)^2 \rightarrow \N$
and a learning algorithm with the following property: 
  
  For every $\epsilon, \delta$ and for \textbf{every data distribution} $\Pxy$, when running the algorithm on $n \geq n_\Hspace(\epsilon, \delta)$ i.i.d. examples from $\Pxy$, the learner returns a model $\fh$ such that, which probability at least $1-\delta$ (over the choice of the $n$ training examples), it holds
$$\risk(\fh) \leq \min_{f \in \Hspace} \risk(f) + \epsilon$$

\begin{itemize}
  \item PAC = Probably ($\delta$) Approximately ($\epsilon$) Correct learning.
  \item It implies that our learner, given enough samples, always returns an "approximately" correct function.
  \item $n_\Hspace(\epsilon, \delta)$ is the sample complexity of our learner, how many samples do we need to obtain a PAC solution.
  \item PAC gives us finite-sample bounds on \textbf{arbitrary} data distributions!
\end{itemize}

\end{vbframe}

\begin{vbframe}{Finite Spaces are agnostic PAC learnable}
Every finite hypothesis space is agnostic PAC learnable, using empirical risk minimization, with sample complexity

  $$ n_\Hspace(\epsilon, \delta) \leq \lceil \frac{2 \log (2|\Hspace| / \delta)}{\epsilon^2} \rceil$$ 

\begin{itemize}
  \item Proof: See "Understanding Machine Learning", chapter 4.
  \item While many spaces $\Hspace$ are infinite, we can discretize them to get a certain impression of their sample complexity.
\begin{itemize}
  \item Assume that parameters are floats on a 32bit machine, then there are at most $2^{32}$ different values for each parameter.
  \item If we have $d$ parameters, that's $2^{32d}$ functions in $\Hspace$.
  \item That gives us $n_\Hspace \leq \frac{64d + \log{2/\delta}}{\epsilon^2}$
  \item For 10 parameters and $\epsilon = \delta = 0.05$ that is ca. $n = 250.000$.
\end{itemize}
\end{itemize}

\end{vbframe}

\begin{vbframe}{No Free Lunch}
Let $\inducer$ be any learning algorithm for binary classification, with respect to 0-1 loss over domain $\Xspace$. Let the training set size $n \leq |\Xspace|/2$. Then a data distribution $\Pxy$ exists, such that

\begin{enumerate}
  \item There exists a function f with $\risk_{\P}(f) = 0$
  \item With probability at least $1/7$ (over the choice of $\D_n$) we have that $\risk_\P(I(\D_n)) \geq 1/8$.
\end{enumerate}

\lz

\begin{itemize}
  \item Proof: See "Understanding Machine Learning", chapter 5.
  \item This implies that for every learner, there is a task on which it fails, even though it could be learned by another learner. So there is no "universal" learner.
  \item This implies that if $\Xspace$ is infinite, the space $\Hspace$ of all functions is not PAC learnable. Learning without any assumptions does not work.
\end{itemize}

\end{vbframe}

\begin{vbframe}{VC dimension}

A general measure of the complexity of a function space is the \textbf{Vapnik-Chervonenkis (VC)} dimension.

\lz

The \textbf{VC dimension} of a class of binary-valued functions $\Hspace = \{h: \Xspace \to \{0, 1\}\}$ is defined to be the largest number of points in $\Xspace$ (in some configuration) that can be \enquote{shattered} by members of $\Hspace$. We write $VC_p(\Hspace)$, where $p$ denotes the dimension of the input space.

\lz

A set of points is said to be \textbf{shattered} by a class of functions if  a member of this class can perfectly separate them no matter how we assign binary labels to the points.

\lz

\textbf{Note:} If the VC dimension of a hypothesis class is $d$, it does not mean that \textbf{all} sets of size $d$ can be shattered. Rather, it simply means that there is at least \textbf{one} such set which can be shattered and \textbf{no} set of size $d+1$ which can be shattered.
\end{vbframe}


\begin{vbframe}{VC dimension of Hyperplanes}

\center
\includegraphics{figure_man/vcdim3.png}

\flushleft

For $\xv\in\R^2$, the class of linear indicator functions $\Hspace = \{h: \R^2 \to \{0, 1\} \text{ | } h(\xv~|~\theta_0, \bm{\theta}) = \I[\xv^T\bm{\theta} - \theta_0 > 0]\}$

\begin{itemize}
  \item can shatter $3$ points: No matter how we assign labels to the configuration of three points shown above, we can find a linear line separating them perfectly;
  \item cannot shatter a configuration of $4$ points. 
\end{itemize}

Hence, $VC_2(\Hspace) = 3$.

% \framebreak
% 
% \textbf{Claim:} The hypothesis space of separating hyperplanes $\Hspace = \{h(\xv) = \I[\bm{x}^\top\bm{\theta} - \theta_0 > 0]\}$ in a $p$-dimensional space $\Xspace \subset \R^p$ has VC-dimension $p + 1$. 
% 
% \lz 
% 
% % \
% 
% % In general: A hyperplane in a $p$-dimensional space can shatter $p+1$ points. So the VC dimension of $\I[x^T\theta - \theta_0 > 0]$ for $x\in\R^p$ is $p+1$.
% 
% 
% For a (brief) sketch of the proof we need to show that $\I[\bm{x}^\top\bm{\theta} - \theta_0 > 0]$ can shatter $p + 1$ points.
% 
% \lz
% 
% \begin{enumerate}
%   \item $p+1$ as a lower bound: 
%   \begin{itemize}
%     \item for any $p$, $p + 1$ points can be chosen such that they can be shattered by a $p$-dimensional hyperplane.\\
%     Example configuration: Axis-aligned unit vectors and the origin 
%   \end{itemize}
% 
% \item $p+1$ as an upper bound : For any $p$, no set of $p+2$ points can be shattered by a $p$-dimensional hyperplane.
% \begin{itemize}
%   \item \small{Fact \#1 : For any group of points belonging to a particular class, a separating hyperplane always assigns the same class to all points that lie in the convex hull formed by the points.
%   \item Fact \#2 : An interesting result known as Radon's Theorem shows that for any set of $p+2$ points in a $p$-dimensional space, it is possible to form a partition of the points into two disjoint sets whose convex hulls intersect.
%   \begin{figure}
%     \centering
%       \scalebox{0.15}{\includegraphics{figure_man/convexhull.png}}
%       \tiny{\\source: Wikipedia}
%       \caption{\footnotesize A convex hull of a set of points is the smallest convex set that contains the points. Here, the convex hull of the red set is the blue and red convex set.}
%   \end{figure}
%   \item Assume the two sets are associated with different classes. For any point in the feature space that lies in this intersection, a hyperplane would assign both classes to it simultaneously. This is, of course, a contradiction which means no such hyperplane exists and the labelling associated with that partition cannot be realized.}
% \end{itemize}
% \end{enumerate}
% 
% Therefore, the VC-dimension of a $p$-dimensional hyperplane is $p+1$.

\framebreak

\textbf{Theorem}: The VC dimension of the class of homogeneous halfspaces, $\Hspace = \{h: \R^p \to \{-1, 1\} \text{ | } h(\xv) =  \sign (\xv^T\bm{\theta})\}$, in $\R^p$ is $p$.

\lz

\textbf{Proof}: $p$ as a lower bound: 

Consider the set of standard basis vectors $\mathbf{e}^{(1)},\mathbf{e}^{(2)},\ldots,\mathbf{e}^{(p)}$ in $\R^p$. For every possible labeling $y^{(1)},y^{(2)},\ldots, y^{(p)} \in \{-1,+1\}$, if we set $\bm{\theta} = (y^{(1)},y^{(2)},\ldots, y^{(p)})^\top$ , then $h(\mathbf{e}^{(i)}) = \text{sgn}\left(\bm{\theta}^\top \mathbf{e}^{(i)}\right) = \text{sgn}\left(y^{(i)}\right) = y^{(i)}$ for all $i$. Therefore, the $p$ points are shattered.

\lz

$p$ as an upper bound: 
\begin{itemize}
  \item Let $\xv^{(1)},\xv^{(2)},\ldots,\xv^{(p+1)}$ be a set of $p+1$ vectors in $\R^p$.
  \item Because any set of $p+1$ vectors in $\R^p$ is linearly dependent, there must exist real numbers $a_1,a_2, \ldots,a_{p+1} \in \R$, not all of them zero, such that $\sum_{i=1}^{p+1} a_i\xv^{(i)} = 0$.
\end{itemize}

Let $I=\left\{i: a_{i}>0\right\} \text { and } J=\left\{j: a_{j}<0\right\}$. Either $I$ or $J$ is nonempty. If we assume both $I$ and $J$ are nonempty, then:
\begin{itemize}
  \item $\sum_{i \in I} a_i \xv^{(i)} = \sum_{j \in J} |a_j| \xv^{(j)}$
  \item Let us assume $\xv^{(1)},\xv^{(2)},\ldots,\xv^{(p + 1)}$ are shattered by $\Hspace$.
  \item There must exist a vector $\bm{\theta}\in \R^p$ such that 

  \begin{eqnarray*}
    h(\xv^{(i)}~|~\bm{\theta}) = 1 &\Leftrightarrow& \bm{\theta}^\top \xv^{(i)} > 0 \quad \text{ for all } i \in I \\
    h(\xv^{(j)}~|~\bm{\theta}) = - 1 &\Leftrightarrow& \bm{\theta}^\top \xv^{(j)} < 0 \quad \text{ for all } j \in J
  \end{eqnarray*}
  
  % \begin{equation*}
  %   \begin{split}
  % 0<\sum_{i \in I} a_{i}\left\langle \xv^{(i)}, \bm{\theta} \right\rangle & = \left\langle\sum_{i \in I} a_{i} \xv^{(i)}, \bm{\theta} \right\rangle \\ 
  % & = \left\langle\sum_{j \in J}\left|a_{j}\right| \xv^{(j)}, \bm{\theta} \right\rangle=\sum_{j \in J}\left|a_{j}\right| \xv^{(j)}, \bm{\theta}^\top  \right\rangle< 0
  %   \end{split}
  % \end{equation*} which is a contradiction.
  
  \item This implies
  
  \begin{equation*}
    \begin{split}
  0<\sum_{i \in I} a_{i} \cdot \bm{\theta}^\top \xv^{(i)} & = \left(\sum_{i \in I} a_{i} \xv^{(i)}\right)^\top \bm{\theta} \\ 
  & = \left(\sum_{j \in J}\left|a_{j}\right| \xv^{(j)}\right)^\top \bm{\theta} =\sum_{j \in J}\left|a_{j}\right| \cdot \bm{\theta}^\top \xv^{(j)} < 0
    \end{split}
  \end{equation*} which is a contradiction.
\end{itemize}

On the other hand, if we assume $J$ (respectively, $I$) is empty, then the rightmost (respectively, leftmost) inequality should be replaced by an equality, which is still a contradiction.\\
\hspace*{\fill}   $\qed$

\framebreak

\textbf{Theorem}: The VC dimension of the class of non-homogeneous halfspaces, $\Hspace = \{h: \R^p \to \{-1, 1\} \text{ | } h(\xv~|~\bm{\theta}) = \sign(\xv^T\bm{\theta} + \theta_0)\}$, in $\R^p$ is $p+1$.

\textbf{Proof}:
$p+1$ as a lower bound: Similar to the proof of the previous theorem, the set of basis vectors and the origin, that is, $0,\mathbf{e}^{(1)},\ldots,\mathbf{e}^{(p)}$ can be shattered by non-homogenous halfspaces.

\lz

$p+1$ as an upper bound: 
\begin{itemize}
  \item Assume that $p+2$ vectors $\xv^{(1)}, \ldots \xv^{(p+2)}$ are shattered. 
    % tby the class of non-homogeneous halfspaces.
  \item If we denote $\bm{\tilde \theta} = (\theta_0, \ldots, \theta_p)^\top \in \R^{p+1}$ , where $\theta_0$ is the bias/intercept, and $\bm{\tilde x} = (1,x_1,\ldots x_{p})^\top \in \R^{p+1}$, then $h(\xv~|~\thetab)=\text{sgn}\left(\xv^\top \thetab+ \theta_0\right)=\text{sgn}\left(\bm{\tilde x}^\top\bm{\tilde\theta}\right)$. Any affine function in $\R^p$ can be rewritten as a homogeneous linear function in $\R^{p+1}$.
  \item By the previous theorem, the set of homogeneous halfspaces in $\R^{p+1}$ cannot shatter any $p+2$ points. Contradiction.
\end{itemize}
\end{vbframe}

% \framebreak
% 
% 
% 
% The definition is extended to real-valued functions:
% 
% \lz
% 
% The \emph{VC dimension} of a class of of real-valued functions $\{f(x, \theta)\}$ is defined to be the VC dimension of the indicator class $\I[f(x, \theta) - \theta_0 > 0 ]$.
% 
% \lz
% 
% The \emph{VC dimension} of a finite classification model, which can return at most $2^d$ different classifiers, is at most $d$.

% \textbf{Example}: For $k$-nearest neighbors with $k = 1$, the VC dimension is infinite.

% \lz


\begin{vbframe}{VC dimension of Rectangles}

\textbf{Example}: Let $\Hspace$ be the class of axis-aligned rectangles in $\R^2$

$$\Hspace=\left\{h_{\left(a_{1}, a_{2}, b_{1}, b_{2}\right)}: \R^2 \to \{0, 1\}: a_{1} \leq a_{2} \text { and } b_{1} \leq b_{2}\right\}$$

where

% \begin{equation}
%    h_{\left(a_{1}, a_{2}, b_{1}, b_{2}\right)}\left(x_{1}, x_{2}\right) = \begin{cases}
%     1 & a_{1} \leq x_{1} \leq a_{2} \text { and } b_{1} \leq x_{2} \leq b_{2}
%     0 & \text{ otherwise }
%   \end{cases}
% \end{equation}

$$
   h_{\left(a_{1}, a_{2}, b_{1}, b_{2}\right)}\left(\bm{x}\right) = \begin{cases}
    1 & a_{1} \leq x_{1} \leq a_{2} \text { and } b_{1} \leq x_{2} \leq b_{2} \\
    0 & \text{ otherwise }
  \end{cases}
$$

%   h_{\left(a_{1}, a_{2}, b_{1}, b_{2}\right)}\left(x_{1}, x_{2}\right)=\left\{\begin{array}{ll}{1} & {\text { if } a_{1} \leq x_{1} \leq a_{2} \text { and } b_{1} \leq x_{2} \leq b_{2}} \\ {0} & {\text { otherwise }}\end{array}\right
% $$

% {\begin{array}{ll}{1} & {\text { if } a_{1} \leq x_{1} \leq a_{2} \text { and } b_{1} \leq x_{2} \leq b_{2}} \\ {0} & {\text { otherwise }}\end{array}

\lz

\textbf{Claim}: $VC_2(\Hspace) = 4$

\lz 

\textbf{Proof}: (next slide)

\framebreak

$4$ as a lower bound: There exists a set of $4$ points that can be shattered.

  \begin{figure}
      \centering
        \scalebox{0.8}{\includegraphics{figure_man/rect_lower.png}}
    \end{figure}

\framebreak

4 as an upper bound: For any set of $5$ points $\xv^{(1)}, \xv^{(2)}, \xv^{(3)}, \xv^{(4)}, \xv^{(5)} \in \R^2$:

\begin{itemize}
  \item Assign the leftmost point (lowest $x_1$), rightmost point (highest $x_1$), highest point (highest $x_2$), and lowest point (lowest $x_2$)  to class $1$.
  \item The point not chosen, $\xv^{(5)}$, is assigned to class $0$.
  \item The rectangle must contain $\xv^{(1)}$, $\xv^{(2)}$, $\xv^{(3)}$ and $\xv^{(4)}$.
  \item $\xv^{(5)}$ is classified as $1$ as well since its coordinates are within the intervals defined by the other $4$.
\end{itemize}

    \begin{figure}
      \centering
        \scalebox{0.2}{\includegraphics{figure_man/rect_upper_x.png}}
        \tiny{\\Credit: Shalev-Shwartz, Ben-David. Understanding Machine Learning.}
    \end{figure}
\vspace{-0.3cm}
Therefore, the VC dimension of axis-aligned rectangles is 4.
\hspace*{\fill}   $\qed$ \\[0.2cm]
\framebreak

\end{vbframe}

\begin{vbframe}{Infinite VC dimension}


\begin{itemize}
\item We can show that if $\Hspace$ has a VC dimension of $2n$, we cannot reliably learn $\Hspace$ from only $n$ examples. Similar to our first statement of the NFL theorem, we can now show that an adversarial data distribution exists, on which our learner fails. But also a function with risk 0 exists, but because of the shattering, this will be in $\Hspace$. 
\item This directly implies that spaces of infinite VC dimension are not PAC learnable.
\end{itemize}

\end{vbframe}

\begin{vbframe}{Fundamental theorem of PAC learning}
  Assume hypothesis space $\Hspace$, classification, and 0-1 loss. Then:

\lz

\begin{itemize}
    \item $\Hspace$ is agnostic PAC learnable if and only if $\Hspace$ has finite VC dimension.
    \item Any ERM algorithm is a successful agnostic PAC learner for $\Hspace$.
    \item For finite VC dimension $d$, the sample complexity is
      $$ C_1 \frac{d + log(1/\delta)}{\epsilon} \leq
      n_\Hspace(\epsilon, \delta) \leq 
         C_2 \frac{d \log(1/\epsilon) + log(1/\delta)}{\epsilon} 
      $$
      with positive, absolute constants $C_1$ and $C_2$.
\end{itemize}
\end{vbframe}


\begin{vbframe}{Probabilistic Bound on Test Error}

\small{Recall that the training error is an optimistic estimate of the generalization (or test) error. For a classification model with VC dimension $d$, 0-1-loss, and a training set of size $n$, the VC dimension can predict a probabilistic upper bound on the test error (with probability $1-\delta$): 

$$
\risk(f) \le \riske(f) + \sqrt{\frac{1}{n}\left[d\left(\log \frac{2n}{d}+ 1 \right)- \log \frac{\delta}{4}\right]}
$$

if the training data set is large enough ($d < n$ required).
}

\lz

\begin{itemize}
    \item So for finite VC dimension we could increase our sample size $n$ so much, that the training error
      would give a close estimate of the test error, with high probability.
    \item Usually such a bound is too loose for practical relevance, and we would have to use an enormous
      amount of data.
\end{itemize}
\end{vbframe}
  
% \begin{vbframe}{Probabilistic Bound on Test Error}

% \small{Recall that the training error is an optimistic estimate of the generalization (or test) error. For a classification model with VC dimension $d$, the VC dimension can predict a probabilistic upper bound on the test error.

% $$
% \P \left(\risk(f) \le \riske(f) + \underbrace{\sqrt{\frac{1}{|\Dtrain|}\left[d\left(\log \frac{2|\Dtrain|}{d}+ 1 \right)- \log \frac{\delta}{4}\right]}}_{\epsilon}\right) = 1 - \delta,
% $$

% for $\delta \in [0, 1]$ if the training data set is large enough ($d < |\Dtrain|$ required).
% }
  % \begin{itemize}
    % \item \small{As a corollary, if the training error is low, the test error is also (probably) low. 
    % \item In other words, an algorithm which minimizes training error reliably picks a "good" hypothesis from the hypothesis set.
    % \item Such an algorithm is known as a \textbf{Probably Approximately Correct} (or \textbf{PAC}) algorithm.
    % }
  % \end{itemize}

% \framebreak
% \end{vbframe}

% \textbf{Example 1:}
% A constant classifier $h$ (no parameters) has VC dimension $0$: it cannot shatter even a single point.

\begin{vbframe}{VC dimension vs Nr of Parameters}

Often, VC dimension of a hypothesis space increases with the number of learnable parameters. However, there are counterexamples. 

\lz

\textbf{Example:}
A single-parametric threshold classifier ($h(x) = \I[x \ge \theta]$) has VC dimension $1$:
\begin{itemize}
\item It can shatter a single point.
\item It cannot shatter any set of $2$ points (for every set of $2$ numbers, if the smaller is labeled $1$, the larger must also be labeled $1$).
\end{itemize}


\begin{center}
\includegraphics[width = 10cm ]{figure_man/VC-example.png} \\
\end{center}


\framebreak



A single-parametric sine classifier $h(x) = \I[\sin(\theta x) > 0]\text{, for }x \in \R $, however, has infinite VC dimension, since it can shatter any set of points if the frequency $\theta$ is chosen large enough.

\center
\includegraphics{figure_man/vcdim_sine.png}\\
\tiny{Credit: Trevor Hastie (2019). The Elements of Statistical Learning.}
\normalsize
\end{vbframe}

% \begin{vbframe}{Final comments}
%   \begin{itemize}
%     \item The bounds derived by the VC analysis are extremely loose and pessimistic.
%     \item Because they have to hold for an arbitrary $\Pxy$, tightening the bounds to a desired level requires the training sets to be extremely large.
%     \item In practice, complex models (such as neural networks) often perform much better than these bounds suggest.
%     \item Other measures of model capacity such as \textbf{Rademacher complexity} may be easier to compute or provide tighter bounds.
%     \item In addition to the hypothesis space, the effective capacity of a learning algorithm also depends, in a complicated way, on the optimizer.
%     \item A better estimate of the generalization error can be obtained simply by evaluating the learned hypothesis on the test set.
%   \end{itemize}
% \end{vbframe}



\endlecture
\end{document}



