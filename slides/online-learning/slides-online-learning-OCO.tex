\usepackage[]{graphicx}
\usepackage[]{color}
% maxwidth is the original width if it is less than linewidth
% otherwise use linewidth (to make sure the graphics do not exceed the margin)
\makeatletter
\def\maxwidth{ %
  \ifdim\Gin@nat@width>\linewidth
    \linewidth
  \else
    \Gin@nat@width
  \fi
}
\makeatother

% ---------------------------------%
% latex-math dependencies, do not remove:
% - \usepackage{mathtools}
% - \usepackage{bm}
% - \usepackage{siunitx}
% - \usepackage{dsfont}
% - \usepackage{xspace}
% ---------------------------------%

%--------------------------------------------------------%
%       Language, encoding, typography
%--------------------------------------------------------%

\usepackage[english]{babel}
\usepackage[utf8]{inputenc} % Enables inputting UTF-8 symbols
% Standard AMS suite
\usepackage{amsmath,amsfonts,amssymb}

% Font four double-stroke / blackboard letters for sets of numbers (N, R, ...)
% Distribution name is "doublestroke"
% According to https://mirror.physik.tu-berlin.de/pub/CTAN/fonts/doublestroke/dsdoc.pdf
% the "bbm" package does a similar thing and may be superfluous.
% Required for latex-math
\usepackage{dsfont}

% bbm – "Blackboard-style" cm fonts (https://www.ctan.org/pkg/bbm)
% Used to be in common.tex, loaded directly after this file
% Maybe superfluous given dsfont is loaded
% TODO: Check if really unused?
% \usepackage{bbm}

% bm – Access bold symbols in maths mode - https://ctan.org/pkg/bm
% Required for latex-math
% https://tex.stackexchange.com/questions/3238/bm-package-versus-boldsymbol
\usepackage{bm}

% pifont – Access to PostScript standard Symbol and Dingbats fonts
% Used for \newcommand{\xmark}{\ding{55}, which is never used
% aside from lecture_advml/attic/xx-automl/slides.Rnw
% \usepackage{pifont}

% Quotes (inline and display), provdes \enquote
% https://ctan.org/pkg/csquotes
\usepackage{csquotes}

% Adds arg to enumerate env, technically superseded by enumitem according
% to https://ctan.org/pkg/enumerate
% Replace with https://ctan.org/pkg/enumitem ?
\usepackage{enumerate}

% Line spacing - provides \singlespacing \doublespacing \onehalfspacing
% https://ctan.org/pkg/setspace
% TODO: Check if really unused?
%\usepackage{setspace}

% mathtools – Mathematical tools to use with amsmath
% https://ctan.org/pkg/mathtools?lang=en
% latex-math dependency according to latex-math repo
\usepackage{mathtools}

%--------------------------------------------------------%
%       Displaying code and algorithms
%--------------------------------------------------------%
\usepackage{verbatim}
\usepackage{algorithm}
\usepackage{algpseudocode}

%--------------------------------------------------------%
%       Tables
%--------------------------------------------------------%

% multi-row table cells: https://www.namsu.de/Extra/pakete/Multirow.html
\usepackage{multirow}

% long/multi-page tables: https://texdoc.org/serve/longtable.pdf/0
% TODO: Check if really unused?

\usepackage{longtable}

% pretty table env: https://ctan.org/pkg/booktabs?lang=en
% TODO: Check if really unused?
\usepackage{booktabs}

%--------------------------------------------------------%
%       Figures: Creating, placing, verbing
%--------------------------------------------------------%

% wrapfig - Wrapping text around figures https://de.overleaf.com/learn/latex/Wrapping_text_around_figures
\usepackage{wrapfig}

% Sub figures in figures and tables
% https://ctan.org/pkg/subfig -- supersedes subfigure package
% TODO: Check if really unused?
\usepackage{subfig}

% Actually it's pronounced PGF https://en.wikibooks.org/wiki/LaTeX/PGF/TikZ
\usepackage{tikz}

\usetikzlibrary{shapes,arrows,automata,positioning,calc,chains,trees, shadows}
\tikzset{
  %Define standard arrow tip
  >=stealth',
  %Define style for boxes
  punkt/.style={
    rectangle,
    rounded corners,
    draw=black, very thick,
    text width=6.5em,
    minimum height=2em,
    text centered},
  % Define arrow style
  pil/.style={
    ->,
    thick,
    shorten <=2pt,
    shorten >=2pt,}
}


% Unsorted
% textpos – Place boxes at arbitrary positions on the LATEX page
% https://ctan.org/pkg/textpos?lang=en
% Provides \begin{textblock}
 % TODO: Check if really unused?
\usepackage[absolute,overlay]{textpos}

% psfrag – Replace strings in encapsulated PostScript figures
% https://www.overleaf.com/latex/examples/psfrag-example/tggxhgzwrzhn
% https://ftp.mpi-inf.mpg.de/pub/tex/mirror/ftp.dante.de/pub/tex/macros/latex/contrib/psfrag/pfgguide.pdf
% Can't tell if this is needed
% TODO: Check if really unused?
\usepackage{psfrag}

% Maybe not great to use this https://tex.stackexchange.com/a/197/19093
% Use align instead -- TODO: Global search & replace to check
\usepackage{eqnarray}

\usepackage{colortbl}

% arydshln – Draw dash-lines in array/tabular
% https://www.ctan.org/pkg/arydshln
% !! "arydshln has to be loaded after array, longtable, colortab and/or colortbl"
% Provides \hdashline and \cdashline
% TODO: Check if really unused?
% \usepackage{arydshln}

% tabularx – Tabulars with adjustable-width columns
% https://ctan.org/pkg/tabularx
% Provides \begin{tabularx}
% TODO: Check if really unused?
% \usepackage{tabularx}

% placeins – Control float placement
% https://ctan.org/pkg/placeins
% Defines a \FloatBarrier command
% TODO: Check if really unused?
% \usepackage{placeins}


% framed – Framed or shaded regions that can break across pages
% https://ctan.org/pkg/framed
% Provides \begin{framed} which uses \colorbox{shadecolor} relying on \definecolor{shadecolor}.
% TODO: Check if really unused?
% \usepackage{framed}

% Used often in conjunction with \definecolor{shadecolor}{rgb}{0.969, 0.969, 0.969}
% Might be able to be removed or at least redefined to only have shadecolor (if needed)
\definecolor{fgcolor}{rgb}{0.345, 0.345, 0.345}
\definecolor{shadecolor}{rgb}{0.969, 0.969, 0.969}
\newenvironment{knitrout}{}{} % an empty environment to be redefined in TeX


% Defines macros and environments
\usepackage{../../style/lmu-lecture}

\let\code=\texttt % Used regularly
\let\proglang=\textsf % Unused?

% Not sure what/why this does
\setkeys{Gin}{width=0.9\textwidth}

\setbeamertemplate{frametitle}{\expandafter\uppercase\expandafter\insertframetitle}

% Can't find a reason why common.tex is not just part of this file?

% basic latex stuff
\newcommand{\pkg}[1]{{\fontseries{b}\selectfont #1}} %fontstyle for R packages
\newcommand{\lz}{\vspace{0.5cm}} %vertical space
\newcommand{\dlz}{\vspace{1cm}} %double vertical space
\newcommand{\oneliner}[1] % Oneliner for important statements
{\begin{block}{}\begin{center}\begin{Large}#1\end{Large}\end{center}\end{block}}


%new environments
\newenvironment{vbframe}  %frame with breaks and verbatim
{
 \begin{frame}[containsverbatim,allowframebreaks]
}
{
\end{frame}
}

\newenvironment{vframe}  %frame with verbatim without breaks (to avoid numbering one slided frames)
{
 \begin{frame}[containsverbatim]
}
{
\end{frame}
}

\newenvironment{blocki}[1]   % itemize block
{
 \begin{block}{#1}\begin{itemize}
}
{
\end{itemize}\end{block}
}

\newenvironment{fragileframe}[2]{  %fragile frame with framebreaks
\begin{frame}[allowframebreaks, fragile, environment = fragileframe]
\frametitle{#1}
#2}
{\end{frame}}


\newcommand{\myframe}[2]{  %short for frame with framebreaks
\begin{frame}[allowframebreaks]
\frametitle{#1}
#2
\end{frame}}

\newcommand{\remark}[1]{
  \textbf{Remark:} #1
}


\newenvironment{deleteframe}
{
\begingroup
\usebackgroundtemplate{\includegraphics[width=\paperwidth,height=\paperheight]{../style/color/red.png}}
 \begin{frame}
}
{
\end{frame}
\endgroup
}
\newenvironment{simplifyframe}
{
\begingroup
\usebackgroundtemplate{\includegraphics[width=\paperwidth,height=\paperheight]{../style/color/yellow.png}}
 \begin{frame}
}
{
\end{frame}
\endgroup
}\newenvironment{draftframe}
{
\begingroup
\usebackgroundtemplate{\includegraphics[width=\paperwidth,height=\paperheight]{../style/color/green.jpg}}
 \begin{frame}
}
{
\end{frame}
\endgroup
}
% https://tex.stackexchange.com/a/261480: textcolor that works in mathmode
\makeatletter
\renewcommand*{\@textcolor}[3]{%
  \protect\leavevmode
  \begingroup
    \color#1{#2}#3%
  \endgroup
}
\makeatother


%-------------------------------------------------------------------------------------------------------%
%  Unused stuff that needs to go but is kept here currently juuuust in case it was important after all  %
%-------------------------------------------------------------------------------------------------------%

% \newcommand{\hlnum}[1]{\textcolor[rgb]{0.686,0.059,0.569}{#1}}%
% \newcommand{\hlstr}[1]{\textcolor[rgb]{0.192,0.494,0.8}{#1}}%
% \newcommand{\hlcom}[1]{\textcolor[rgb]{0.678,0.584,0.686}{\textit{#1}}}%
% \newcommand{\hlopt}[1]{\textcolor[rgb]{0,0,0}{#1}}%
% \newcommand{\hlstd}[1]{\textcolor[rgb]{0.345,0.345,0.345}{#1}}%
% \newcommand{\hlkwa}[1]{\textcolor[rgb]{0.161,0.373,0.58}{\textbf{#1}}}%
% \newcommand{\hlkwb}[1]{\textcolor[rgb]{0.69,0.353,0.396}{#1}}%
% \newcommand{\hlkwc}[1]{\textcolor[rgb]{0.333,0.667,0.333}{#1}}%
% \newcommand{\hlkwd}[1]{\textcolor[rgb]{0.737,0.353,0.396}{\textbf{#1}}}%
% \let\hlipl\hlkwb

% \makeatletter
% \newenvironment{kframe}{%
%  \def\at@end@of@kframe{}%
%  \ifinner\ifhmode%
%   \def\at@end@of@kframe{\end{minipage}}%
%   \begin{minipage}{\columnwidth}%
%  \fi\fi%
%  \def\FrameCommand##1{\hskip\@totalleftmargin \hskip-\fboxsep
%  \colorbox{shadecolor}{##1}\hskip-\fboxsep
%      % There is no \\@totalrightmargin, so:
%      \hskip-\linewidth \hskip-\@totalleftmargin \hskip\columnwidth}%
%  \MakeFramed {\advance\hsize-\width
%    \@totalleftmargin\z@ \linewidth\hsize
%    \@setminipage}}%
%  {\par\unskip\endMakeFramed%
%  \at@end@of@kframe}
% \makeatother

% \definecolor{shadecolor}{rgb}{.97, .97, .97}
% \definecolor{messagecolor}{rgb}{0, 0, 0}
% \definecolor{warningcolor}{rgb}{1, 0, 1}
% \definecolor{errorcolor}{rgb}{1, 0, 0}
% \newenvironment{knitrout}{}{} % an empty environment to be redefined in TeX

% \usepackage{alltt}
% \newcommand{\SweaveOpts}[1]{}  % do not interfere with LaTeX
% \newcommand{\SweaveInput}[1]{} % because they are not real TeX commands
% \newcommand{\Sexpr}[1]{}       % will only be parsed by R
% \newcommand{\xmark}{\ding{55}}%

% math spaces
\newcommand{\N}{\mathds{N}}                                                 % N, naturals
\newcommand{\Z}{\mathds{Z}}                                                 % Z, integers
\newcommand{\Q}{\mathds{Q}}                                                 % Q, rationals
\newcommand{\R}{\mathds{R}}                                                 % R, reals
\newcommand{\C}{\mathds{C}}                                                 % C, complex
\newcommand{\HS}{\mathcal{H}}                                               % H, hilbertspace
\newcommand{\continuous}{\mathcal{C}}                                       % C, space of continuous functions
\newcommand{\M}{\mathcal{M}} 												% machine numbers
\newcommand{\epsm}{\epsilon_m} 												% maximum error


% basic math stuff
\newcommand{\xt}{\tilde x}													% x tilde
\def\argmax{\mathop{\sf arg\,max}}                                          % argmax
\def\argmin{\mathop{\sf arg\,min}}                                          % argmin
\newcommand{\sign}{\operatorname{sign}}                                     % sign, signum
\newcommand{\I}{\mathbb{I}}                                                 % I, indicator
\newcommand{\order}{\mathcal{O}}                                            % O, order
\newcommand{\fp}[2]{\frac{\partial #1}{\partial #2}}                        % partial derivative
\newcommand{\pd}[2]{\frac{\partial{#1}}{\partial #2}}						% partial derivative

% sums and products
\newcommand{\sumin}{\sum_{i=1}^n}											% summation from i=1 to n
\newcommand{\sumkg}{\sum_{k=1}^g}											% summation from k=1 to g
\newcommand{\prodin}{\prod_{i=1}^n}											% product from i=1 to n
\newcommand{\prodkg}{\prod_{k=1}^g}											% product from k=1 to g

% linear algebra
\newcommand{\one}{\boldsymbol{1}}                                           % 1, unitvector
\newcommand{\id}{\mathrm{I}}                                                % I, identity
\newcommand{\diag}{\operatorname{diag}}                                     % diag, diagonal
\newcommand{\trace}{\operatorname{tr}}                                      % tr, trace
\newcommand{\spn}{\operatorname{span}}                                      % span
\newcommand{\scp}[2]{\left\langle #1, #2 \right\rangle}                     % <.,.>, scalarproduct
\newcommand{\mat}[1]{ 														% short pmatrix command
	\begin{pmatrix}
		#1
	\end{pmatrix}
}
\newcommand{\Amat}{\bm{A}}													% matrix A
\newcommand{\xv}{\bm{x}}													% vector x (bold)
\newcommand{\yv}{\bm{y}}														% vector y (bold)
\newcommand{\Deltab}{\bm{\Delta}}											% error term for vectors
															

% basic probability + stats
\renewcommand{\P}{\mathds{P}}                                               % P, probability
\newcommand{\E}{\mathds{E}}                                                 % E, expectation
\newcommand{\var}{\mathsf{Var}}                                             % Var, variance
\newcommand{\cov}{\mathsf{Cov}}                                             % Cov, covariance
\newcommand{\corr}{\mathsf{Corr}}                                           % Corr, correlation
\newcommand{\normal}{\mathcal{N}}                                           % N of the normal distribution
\newcommand{\iid}{\overset{i.i.d}{\sim}}                                    % dist with i.i.d superscript
\newcommand{\distas}[1]{\overset{#1}{\sim}}                                 % ... is distributed as ... 
% machine learning

%%%%%% ml - data
\newcommand{\Xspace}{\mathcal{X}}                                           % X, input space
\newcommand{\Yspace}{\mathcal{Y}}                                           % Y, output space
\newcommand{\nset}{\{1, \ldots, n\}}                                        % set from 1 to n
\newcommand{\pset}{\{1, \ldots, p\}}                                        % set from 1 to p
\newcommand{\gset}{\{1, \ldots, g\}}                                        % set from 1 to g
\newcommand{\Pxy}{\P_{xy}}                                                  % P_xy
\newcommand{\xy}{(x, y)}                                                    % observation (x, y)
\newcommand{\xvec}{(x_1, \ldots, x_p)^T}                                    % (x1, ..., xp) 
\newcommand{\D}{\mathcal{D}}                                                % D, data 
\newcommand{\Dset}{\{ (x^{(1)}, y^{(1)}), \ldots, (x^{(n)},  y^{(n)})\}}    % {(x1,y1)), ..., (xn,yn)}, data
\newcommand{\xdat}{\{ x^{(1)}, \ldots, x^{(n)}\}}   						 % {x1, ..., xn}, input data
\newcommand{\ydat}{\mathbf{y}}                                              % y (bold), vector of outcomes
\newcommand{\yvec}{(y^{(1)}, \hdots, y^{(n)})^T}                            % (y1, ..., yn), vector of outcomes
\renewcommand{\xi}[1][i]{x^{(#1)}}                                          % x^i, i-th observed value of x
\newcommand{\yi}[1][i]{y^{(#1)}}                                            % y^i, i-th observed value of y 
\newcommand{\xyi}{(\xi, \yi)}                                               % (x^i, y^i), i-th observation
\newcommand{\xivec}{(x^{(i)}_1, \ldots, x^{(i)}_p)^T}                       % (x1^i, ..., xp^i), i-th observation vector
\newcommand{\xj}{x_j}                                                       % x_j, j-th feature
\newcommand{\xjb}{\mathbf{x}_j}                                             % x_j (bold), j-th feature vecor
\newcommand{\xjvec}{(x^{(1)}_j, \ldots, x^{(n)}_j)^T}                       % (x^1_j, ..., x^n_j), j-th feature vector
\newcommand{\Dtrain}{\mathcal{D}_{\text{train}}}                            % D_train, training set
\newcommand{\Dtest}{\mathcal{D}_{\text{test}}}                              % D_test, test set

%%%%%% ml - models general

% continuous prediction function f
\newcommand{\fx}{f(x)}                                                      % f(x), continuous prediction function
\newcommand{\Hspace}{H}														% hypothesis space where f is from
\newcommand{\fh}{\hat{f}}                                                   % f hat, estimated prediction function
\newcommand{\fxh}{\fh(x)}                                                   % fhat(x)
\newcommand{\fxt}{f(x | \theta)}                                            % f(x | theta)
\newcommand{\fxi}{f(\xi)}                                                   % f(x^(i))
\newcommand{\fxih}{\hat{f}(\xi)}                                            % f(x^(i))
\newcommand{\fxit}{f(x^{(i)} | \theta)}                                     % f(x^(i) | theta)
\newcommand{\fhD}{\fh_{\D}}                                                 % fhat_D, estimate of f based on D
\newcommand{\fhDtrain}{\fh_{\Dtrain}}                                       % fhat_Dtrain, estimate of f based on D

% discrete prediction function h
\newcommand{\hx}{h(x)}                                                      % h(x), discrete prediction function
\newcommand{\hh}{\hat{h}}                                                   % h hat
\newcommand{\hxh}{\hat{h}(x)}                                               % hhat(x)
\newcommand{\hxt}{h(x | \theta)}                                            % h(x | theta)
\newcommand{\hxi}{h(\xi)}                                                   % h(x^(i))
\newcommand{\hxit}{h(x^{(i)} | \theta)}                                     % h(x^(i) | theta)

% yhat
\newcommand{\yh}{\hat{y}}                                                   % y hat for prediction of target
\newcommand{\yih}{\hat{y}}                                                  % y hat for prediction of target

% theta
\newcommand{\thetah}{\hat{\theta}}                                          % theta hat

% densities + probabilities
% pdf of x 
\newcommand{\pdf}{p}                                                        % p
\newcommand{\pdfx}{p(x)}                                                    % p(x)
\newcommand{\pixt}{\pi(x | \theta)}                                         % pi(x|theta), pdf of x given theta

% pdf of (x, y)
\newcommand{\pdfxy}{p(x,y)}                                                 % p(x, y)
\newcommand{\pdfxyt}{p(x, y | \theta)}                                      % p(x, y | theta)
\newcommand{\pdfxyit}{p(\xi, \yi | \theta)}                                 % p(x^(i), y^(i) | theta)

% pdf of x given y
\newcommand{\pdfxyk}{p(x | y=k)}                                            % p(x | y = k)
\newcommand{\lpdfxyk}{\log \pdfxyk}                                         % log p(x | y = k)
\newcommand{\pdfxiyk}{p(\xi | y=k)}                                         % p(x^i | y = k)

% prior probabilities
\newcommand{\pik}{\pi_k}                                                    % pi_k, prior
\newcommand{\lpik}{\log \pik}                                               % log pi_k, log of the prior

% posterior probabilities
\newcommand{\post}{\P(y = 1 | x)}                                           % P(y = 1 | x), post. prob for y=1
\newcommand{\pix}{\pi(x)}                                                   % pi(x), P(y = 1 | x)
\newcommand{\postk}{\P(y = k | x)}                                          % P(y = k | y), post. prob for y=k
\newcommand{\pikx}{\pi_k(x)}                                                % pi_k(x), P(y = k | x)
\newcommand{\pikxt}{\pi_k(x | \theta)}                                      % pi_k(x | theta), P(y = k | x, theta)
\newcommand{\pijx}{\pi_j(x)}                                                % pi_j(x), P(y = j | x)
\newcommand{\pdfygxt}{p(y |x, \theta)}                                      % p(y | x, theta)
\newcommand{\pdfyigxit}{p(\yi |\xi, \theta)}                                % p(y^i |x^i, theta)
\newcommand{\lpdfygxt}{\log \pdfygxt }                                      % log p(y | x, theta)
\newcommand{\lpdfyigxit}{\log \pdfyigxit}                                   % log p(y^i |x^i, theta)
\newcommand{\pixh}{\hat \pi(x)}                                             % pi(x) hat, P(y = 1 | x) hat
\newcommand{\pikxh}{\hat \pi_k(x)}                                          % pi_k(x) hat, P(y = k | x) hat

% residual and margin
\newcommand{\eps}{\epsilon}                                                 % residual, stochastic
\newcommand{\epsi}{\epsilon^{(i)}}                                          % epsilon^i, residual, stochastic
\newcommand{\epsh}{\hat{\epsilon}}                                          % residual, estimated
\newcommand{\yf}{y \fx}                                                     % y f(x), margin
\newcommand{\yfi}{\yi \fxi}                                                 % y^i f(x^i), margin
\newcommand{\Sigmah}{\hat \Sigma}											% estimated covariance matrix
\newcommand{\Sigmahj}{\hat \Sigma_j}										% estimated covariance matrix for the j-th class

% ml - loss, risk, likelihood
\newcommand{\Lxy}{L(y, f(x))}                                               % L(y, f(x)), loss function
\newcommand{\Lxyi}{L(\yi, \fxi)}                                            % L(y^i, f(x^i))
\newcommand{\Lxyt}{L(y, \fxt)}                                              % L(y, f(x | theta))
\newcommand{\Lxyit}{L(\yi, \fxit)}                                          % L(y^i, f(x^i | theta)
\newcommand{\risk}{\mathcal{R}}                                             % R, risk
\newcommand{\riskf}{\risk(f)}                                               % R(f), risk
\newcommand{\riske}{\mathcal{R}_{\text{emp}}}                               % R_emp, empirical risk
\newcommand{\riskef}{\riske(f)}                                             % R_emp(f)
\newcommand{\risket}{\mathcal{R}_{\text{emp}}(\theta)}                      % R_emp(theta)
\newcommand{\riskr}{\mathcal{R}_{\text{reg}}}                               % R_reg, regularized risk
\newcommand{\riskrt}{\mathcal{R}_{\text{reg}}(\theta)}                      % R_reg(theta)
\newcommand{\riskrf}{\riskr(f)}                                             % R_reg(f)
\newcommand{\LL}{\mathcal{L}}                                               % L, likelihood
\newcommand{\LLt}{\mathcal{L}(\theta)}                                      % L(theta), likelihood
\renewcommand{\ll}{\ell}                                                    % l, log-likelihood
\newcommand{\llt}{\ell(\theta)}                                             % l(theta), log-likelihood
\newcommand{\LS}{\mathfrak{L}}                                              % ????????????
\newcommand{\TS}{\mathfrak{T}}                                              % ??????????????
\newcommand{\errtrain}{\text{err}_{\text{train}}}                           % training error
\newcommand{\errtest}{\text{err}_{\text{test}}}                             % training error
\newcommand{\errexp}{\overline{\text{err}_{\text{test}}}}                   % training error

% resampling
\newcommand{\GE}[1]{GE(\fh_{#1})}                                           % Generalization error GE
\newcommand{\GEh}[1]{\widehat{GE}_{#1}}                                     % Estimated train error
\newcommand{\GED}{\GE{\D}}                                                  % Generalization error GE
\newcommand{\EGEn}{EGE_n}                                                   % Generalization error GE
\newcommand{\EDn}{\E_{|D| = n}}                                             % Generalization error GE


% ml - irace
\newcommand{\costs}{\mathcal{C}} % costs
\newcommand{\Celite}{\theta^*} % elite configurations
\newcommand{\instances}{\mathcal{I}} % sequence of instances
\newcommand{\budget}{\mathcal{B}} % computational budget

\newcommand{\sens}{\mathbf{A}} % vector x (bold)
\newcommand{\ba}{\mathbf{a}}
\newcommand{\batilde}{\tilde{\mathbf{a}}}
\newcommand{\Px}{\mathbb{P}_{x}} % P_x
\newcommand{\Pxj}{\mathbb{P}_{x_j}} % P_{x_j}
\newcommand{\indep}{\perp \!\!\! \perp} % independence symbol
% ml - ROC
\newcommand{\np}{n_{+}} % no. of positive instances
\newcommand{\nn}{n_{-}} % no. of negative instances
\newcommand{\rn}{\pi_{-}} % proportion negative instances
\newcommand{\rp}{\pi_{+}} % proportion negative instances
% true/false pos/neg:
\newcommand{\tp}{\# \text{TP}} % true pos
\newcommand{\fap}{\# \text{FP}} % false pos (fp taken for partial derivs)
\newcommand{\tn}{\# \text{TN}} % true neg
\newcommand{\fan}{\# \text{FN}} % false neg

\newcommand{\Tspace}{\mathcal{T}}
\newcommand{\tv}{\mathbf{t}}
\newcommand{\tj}{\mathbf{t}_j}
\renewcommand{\l}{L}
\newcommand{\Aspace}{\mathcal{A}}
\newcommand{\Zspace}{\mathcal{Z}}
\newcommand{\norm}[1]{\left|\left|#1\right|\right|_2}


\newcommand{\llin}{\l^{\texttt{lin}}}
\newcommand{\lzeroone}{\l^{0-1}}
\newcommand{\lhinge}{\l^{\texttt{hinge}}}
\newcommand{\lexphinge}{\widetilde{\l^{\texttt{hinge}}}}
\newcommand{\lconv}{\l^{\texttt{conv}}}
\newcommand{\FTL}{\texttt{FTL}}
\newcommand{\FTRL}{\texttt{FTRL}}
\newcommand{\OGD}{{\texttt{OGD}}}
\newcommand{\EWA}{{\texttt{EWA}}} 
\newcommand{\REWA}{{\texttt{REWA}}} 
\newcommand{\EXPthree}{{\texttt{EXP3}}}
\newcommand{\EXPthreep}{{\texttt{EXP3P}}}
\newcommand{\reg}{\psi}
\newcommand{\Algo}{\texttt{Algo}}

\usepackage{multicol}

\newcommand{\titlefigure}{figure/gradienten_verfahren}
\newcommand{\learninggoals}{
  \item Get to know the class of online convex optimization problems
  \item See the online gradient descent as a satisfactory learning algorithm for such cases
  \item Know its connection to the FTRL via linearization of convex functions
}

\title{Advanced Machine Learning}
\date{}

\begin{document}

\lecturechapter{Online Convex Optimization}
\lecture{Advanced Machine Learning}



\sloppy

\begin{frame} 
	\frametitle{Online Convex Optimization}
	%	
	\small
	\begin{itemize}
%		
		\item 	One of the most relevant instantiations of the online learning problem is the problem of \emph{online convex optimization} (OCO), which is characterized by a loss function $\l:\Aspace \times \Zspace \to \R,$ which is convex w.r.t.\ the action, i.e., 
		$a \mapsto \l(a,z)$ is convex for any $z \in \Zspace.$ 
%		
	 	 {\visible<2->{ \item Note that both OLO and OQO belong to the class of online convex optimization problems:
		%	
		\begin{itemize} \small
			%		
			%		
			\item \emph{Online linear optimization (OLO) with convex action spaces:}  $\l(a,z)=a^\top z$ is a convex function in $a\in \Aspace,$ provided $\Aspace$ is convex.
			%		
			\item \emph{Online quadratic optimization (OQO) with convex action spaces:}  $\l(a,z)=\frac12 \norm{a-z}^2$ is a convex function in $a\in \Aspace,$ provided $\Aspace$ is convex.
			%	{\tiny (We will show the convexity for specific choices in an exercise on the third assignment sheet.)}
			%
		\end{itemize} }}
		%		
	\end{itemize}  
	%	
\end{frame}



\begin{vbframe} 
	\frametitle{Online Gradient Descent: Motivation}
	%	
	\small
	\begin{itemize}
		%		
		\item We have seen that the FTRL algorithm with the $\l_2$ norm regularization $\reg(a) = \frac{1}{2 \eta}  \norm{a}^2$ achieves satisfactory results for online linear optimization (OLO) problems, that is, if $\l(a,z)=\llin(a,z):=a^\top z,$ then we have
		%		
		\begin{itemize} \small
			%			
			\item \emph{Fast updates ---} If $\Aspace = \R^d,$ then 
			$$	a_{t+1}^{\FTRL} = a_t^{\FTRL} - \eta \, z_t, \qquad t=1,\ldots,T; $$ 
			%			
			\item \emph{Regret bounds ---} By an appropriate choice of $\eta$ and some (mild) assumptions on $\Aspace$ and $\Zspace,$ we have $$	R_T^{\FTRL} = o(T).$$
			%			
		\end{itemize}
		%	
	\end{itemize}
%
\end{vbframe}
%
	\begin{frame} 
			\frametitle{Online Gradient Descent: Motivation}
			%	
		\small 
		Apparently, the nice form of the loss function $\llin$ is responsible for the appealing properties of FTRL in this case. 
		Indeed, 
		since $\nabla_a \llin(a,z) = z$   note that the update rule can be written as 
		%		
		\begin{equation*}
%			\label{eq:forel_update}
			%	
			a_{t+1}^{\FTRL} = a_t^{\FTRL} - \eta \, z_t =  a_t^{\FTRL} - \eta \, \nabla_a \llin(a_t^{\FTRL},z_t).
			%	
		\end{equation*}
		%		
		%since $\nabla \llin(a,z) = z.$  
		%		
		\begin{minipage}{.6\textwidth}
		{\visible<2->{ \emph{Interpretation}: In each time step $t+1,$  we are following the direction with the steepest decrease of the loss (represented by $- \nabla \llin(a_t^{\FTRL},z_t)$) from the current ''position'' $a_t^{\FTRL}$ with the step size $\eta$ }}
		\lz 
		
		{\visible<3>{ $\Rightarrow $ Gradient Descent. }}
		
		\end{minipage}
		\begin{minipage}{.3\textwidth}{\visible<3>{ 
			\begin{figure}
				\centering
				\includegraphics[width=0.8\linewidth]{figure/gradienten_verfahren}
			\end{figure}}}
		\end{minipage}
		%		
	%
\end{frame}


\begin{frame} 
	\frametitle{Online Gradient Descent: Motivation}
	%	
	\small
	\begin{itemize}
		%	
		\item \textbf{Question:} How to transfer this idea of the Gradient Descent for the update formula to other loss functions, while still preserving the regret bounds?
		%	
		{\visible<2->{  \item \textbf{Solution (for convex losses):} Recall the equivalent characterization of convexity of differentiable convex functions:
		%	
		\begin{align*}
			%		
			f:S\to \R \mbox{ is convex } 
			&\Leftrightarrow  f(y) \geq f(x)+(y-x)^\top  \nabla f(x)  \mbox{ for any } x,y\in S \\
			 &\Leftrightarrow  f(x) - f(y) \leq (x-y)^\top  \nabla f(x)  \mbox{ for any } x,y\in S.	
			%		
		\end{align*}}
		}
		%		
		{\visible<3>{  \item This means if we are dealing with a loss function $\l:\Aspace \times \Zspace \to \R,$ which is convex and differentiable in its first argument ($\Aspace$ has also to be convex), then 
		%	
		$$		\l(a,z) - \l(\tilde a, z) \leq 	(a-\tilde a )^\top \, \nabla_a \l(a,z), \quad \forall a,\tilde a\in \Aspace, z\in \Zspace. $$}}
		%
	\end{itemize}
	%
\end{frame}


\begin{frame} 
	\frametitle{Online Gradient Descent: Motivation}
	%	
	\footnotesize
	\begin{itemize}
		%
		\item \textbf{Reminder:} $		\l(a,z) - \l(\tilde a, z) \leq 	(a-\tilde a )^\top \, \nabla_a \l(a,z), \quad \forall a,\tilde a\in \Aspace, z\in \Zspace. $
		\item {\visible<2->{  Let  $z_1,\ldots,z_T$ arbitrary environmental data and $a_1,\ldots,a_T$ be some arbitrary action sequence. Substitute $\tilde z_t := \nabla_a \l(a_t,z_t)$ and note that }}
		%	
		%
		{\visible<3->{		
		\begin{align*}
			%		
			 	{\color{blue} R_T(\tilde a) } 
					%		 
					&= \sum_{t=1}^T \l(a_t,z_t) -  \l(\tilde a ,z_t) 
			%		
			   \leq  \sum_{t=1}^T  (a_t -\tilde a )^\top \, \nabla_a \l(a_t,z_t)	\\
			%		
			  &= \sum_{t=1}^T  (a_t -\tilde a )^\top \, \tilde z_t 
			%			
			  = \sum_{t=1}^T  a_t^\top \, \tilde z_t  - \tilde a^\top \, \tilde z_t 
			%			
			  =  {\color{orange} \sum_{t=1}^T \llin(a_t,\tilde z_t) - \llin(\tilde a,\tilde z_t)}. 
			%		
		\end{align*}  }}
		%
		{\visible<4->{  \emph{Conclusion:} {\color{blue}The regret of a learner with respect to a differentiable and convex loss function $\l$} is bounded by {\color{orange} the regret corresponding to an online linear optimization problem with environmental data $\tilde z_t = \nabla_a \l(a_t,z_t).$} }}
		%	
		\item {\visible<5->{ \textbf{We know:} Online linear optimization problems can be tackled by means of the FTRL algorithm!}}
		%		 
		\item [$\leadsto$] {\visible<6->{ Incorporate the substitution $\tilde z_t = \nabla_a \l(a_t,z_t)$ into the update formula of FTRL with squared L2-norm regularization.}}
		%	
		%
	\end{itemize}
\end{frame}
%
\begin{frame} 
	\frametitle{Online Gradient Descent: Definition and properties}
	\small
	\begin{itemize}
		%	
		\item 
		%	
		The corresponding algorithm which chooses its action according to these considerations is called the \emph{Online Gradient Descent} (OGD) algorithm with step size $\eta>0.$ It holds in particular,
		%	
		%	More specifically, the action are chosen as 
		%	
		\begin{equation}
			\label{eq:OGD_update}
			%		
			a_{t+1}^\OGD = a_{t}^\OGD - \eta \nabla_a \l(a_{t}^\OGD,z_t ), \quad t=1,\ldots T.
			%		
		\end{equation}
		{\tiny (Technical side note:  For this update formula we assume that $\Aspace=\R^d.$ Moreover, the first action $a_1^\OGD$ is arbitrary. )}
		%
		 {\visible<2->{ \item We have the following connection between FTRL and OGD:
		\begin{itemize}\small
			%		
			\item Let $\tilde z_t^\OGD :=  \nabla_a \l(a_{t}^\OGD,z_t )$ for any $t=1,\ldots,T.$ 
			%		be the substituted environmental data.
			%		
			\item The update formula for FTRL with $\l_2$ norm regularization for the linear loss $\llin$ and the environmental data $\tilde z_t^\OGD$ is
			{\footnotesize		$$a_{t+1}^{\FTRL} = a_{t}^{\FTRL} - \eta \tilde z_t^\OGD =  a_{t}^{\FTRL} - \eta  \nabla_a \l(a_{t}^\OGD,z_t ). $$}
			%		
			\item If we have that $a_{1}^{\FTRL} = a_{1}^\OGD,$ then it iteratively follows that $a_{t+1}^{\FTRL} = a_{t+1}^\OGD$ for any $t=1,\ldots,T$ in this case.
			%
		\end{itemize} }}
		%
	\end{itemize}
\end{frame}
%
\begin{frame} 
	\frametitle{Online Gradient Descent: Definition and properties}
	\footnotesize
	\begin{itemize}
		\small
		%
		\item With the deliberations above we can infer that
		%	
		\begin{align*}
%			
		 R_{T,\l}^\OGD(\tilde a ~|~ (z_t)_{t} ) 
%			
			&= \sum\nolimits_{t=1}^T \l(a_t^\OGD,z_t) -  \l(\tilde a ,z_t)  \\
%			
		 	&\leq \sum\nolimits_{t=1}^T \llin(a_t^\OGD,\tilde z_t^\OGD) - \llin(\tilde a,\tilde z_t^\OGD) \\ 
%			
			&\stackrel{\mbox{(if $a_1^\OGD=a_1^{\FTRL}$})}{=} 
			\sum\nolimits_{t=1}^T \llin(a_t^{\FTRL},\tilde z_t^\OGD) - \llin(\tilde a,\tilde z_t^\OGD) \\
			%
			&= R_{T,\llin}^{\FTRL}(\tilde a ~|~ (\tilde z_t^\OGD)_{t}), 
		\end{align*}
		%	
		where we write in the subscripts of the regret the corresponding loss function and also include the corresponding environmental data as a second argument in order to emphasize the connections.
		%	
		\item 
		%	It holds for any $\tilde a \in \Aspace$ that $R_{T,\l}^\OGD(\tilde a| (z_t)_{t})\leq R_{T,\llin}^{\FTRL}(\tilde a| (\tilde z_t)_{t}).$ \\
		{\visible<2->{  \emph{Interpretation:} The regret of the FTRL algorithm (with $\l_2$ norm regularization) for the online linear optimization problem (characterized by the linear loss $\llin$) with environmental data $\tilde z_t^\OGD$ is an upper bound for the OGD algorithm for the online convex problem (characterized by a differentiable convex loss $\l$) with the original environmental data $z_t.$ }}
		%	
	\end{itemize}
\end{frame}
%
\begin{frame} 
	\frametitle{Online Gradient Descent:  Regret}
	\footnotesize
	\begin{itemize}
		%	
		%	
		\item Due to this connection we immediately obtain a similar decomposition of the regret upper bound into a bias term and a variance term as for the FTRL algorithm for OLO problems.
		% 
		\item \textbf{Corollary.} Using the OGD algorithm on any online convex optimization problem (with differentiable loss function $\l$)  leads to a regret of OGD with respect to any action $\tilde a \in \Aspace$  of
		%	 
		\begin{align*}
			%	
			R_T^{\OGD}(\tilde a)  
%			
			&\leq \frac{1}{2\eta}  \norm{\tilde a}^2 +   \eta  \sum\nolimits_{t=1}^T \norm{\tilde z_t^\OGD}^2 \\
%			
			&= \frac{1}{2\eta}  \norm{\tilde a}^2 +   \eta  \sum\nolimits_{t=1}^T \norm{ \nabla_a \l(a_t^\OGD,z_t) }^2.
			%		
		\end{align*}	
		%
		 {\visible<2->{ \item Note that the step size $\eta>0$ of OGD has the same role as the regularization magnitude of FTRL: It should balance the trade-off between the bias- and the variance-term. }}
		%
	\end{itemize}
\end{frame}


\begin{frame} 
	\frametitle{Online Gradient Descent:  Regret}
	\small
	\begin{itemize}
		%	
		%	
		%
		\item As a consequence, we can also derive a similar order of the regret for the OGD algorithm on OCO problems as for the FTRL on OLO problems by imposing a slightly different assumption on the (new) ``variance'' term $\sum\nolimits_{t=1}^T \norm{ \nabla_a \l(a_t^\OGD,z_t) }^2.$
		%	
		{\visible<2->{\item  \textbf{Corollary:}
		%	
		Suppose we use the OGD algorithm on an online convex optimization problem with a convex action space  $\Aspace \subset \mathbb{R}^d$ such that 
%		
		\begin{itemize}\small
%			
			\item $\sup_{\tilde a \in \Aspace}\norm{\tilde a} \leq B$ for some finite constant $B>0$
%			
			\item $\sup_{a\in \Aspace, z \in \Zspace}\norm{ \nabla_a \l(a,z) } \leq V$ for some finite constant $V>0.$
%			
		\end{itemize}
		%	
		Then, by choosing the step size $\eta$ for OGD as $\eta = \frac{B}{V\sqrt{2\, T}}$ we get
		%	
		$$	R_T^\OGD \leq   BV\sqrt{2\, T}.		$$ }}
		%	
		%	
	\end{itemize}
\end{frame}

\begin{frame}
%	
	\frametitle{Regret lower bounds for OCO}
	%     
	\small
	%	
	\begin{itemize}
		%    		
		\item   \textbf{Theorem.} For any online learning algorithm there exists an online convex optimization problem characterized by a convex loss function $\l,$ a  bounded (convex) action space  $\Aspace= [-B,B]^d $ and bounded gradients $\sup_{a\in \Aspace, z \in \Zspace}\norm{ \nabla_a \l(a,z) } \leq V$  for some finite constants $B,V>0,$	such that the algorithm incurs a regret of $\Omega(\sqrt{T})$ in the worst case.
		%		
		{\visible<2->{\item Recall that  under (almost) the same assumptions as the theorem above, we have $R_T^\OGD \leq   BV\sqrt{2\, T}.$ }}
		%    		
		{\visible<3->{	\item [$\leadsto$] This result shows that the Online Gradient Descent 
		%		and the Exponentiated Gradient Descent algorithm are 
		is \emph{optimal} regarding its order of its regret with respect to the time horizon $T.$  }}
%		
	\end{itemize}
	%    
\end{frame}




\section{Outlook}

\begin{frame} {Online Machine Learning: Outlook}
%	
	Online machine learning is a very large field of research.	
	%    
	\begin{figure}
		\centering
		\includegraphics[width=0.99\linewidth]{figure/online_learning_overview}
		\caption{Hoi et al. (2018), ''Online Learning: A Comprehensive Survey''.}
	\end{figure}
\end{frame}

%
\endlecture
\end{document}
