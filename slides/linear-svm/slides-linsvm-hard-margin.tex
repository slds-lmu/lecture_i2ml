\usepackage[]{graphicx}
\usepackage[]{color}
% maxwidth is the original width if it is less than linewidth
% otherwise use linewidth (to make sure the graphics do not exceed the margin)
\makeatletter
\def\maxwidth{ %
  \ifdim\Gin@nat@width>\linewidth
    \linewidth
  \else
    \Gin@nat@width
  \fi
}
\makeatother

% ---------------------------------%
% latex-math dependencies, do not remove:
% - \usepackage{mathtools}
% - \usepackage{bm}
% - \usepackage{siunitx}
% - \usepackage{dsfont}
% - \usepackage{xspace}
% ---------------------------------%

%--------------------------------------------------------%
%       Language, encoding, typography
%--------------------------------------------------------%

\usepackage[english]{babel}
\usepackage[utf8]{inputenc} % Enables inputting UTF-8 symbols
% Standard AMS suite
\usepackage{amsmath,amsfonts,amssymb}

% Font four double-stroke / blackboard letters for sets of numbers (N, R, ...)
% Distribution name is "doublestroke"
% According to https://mirror.physik.tu-berlin.de/pub/CTAN/fonts/doublestroke/dsdoc.pdf
% the "bbm" package does a similar thing and may be superfluous.
% Required for latex-math
\usepackage{dsfont}

% bbm – "Blackboard-style" cm fonts (https://www.ctan.org/pkg/bbm)
% Used to be in common.tex, loaded directly after this file
% Maybe superfluous given dsfont is loaded
% TODO: Check if really unused?
% \usepackage{bbm}

% bm – Access bold symbols in maths mode - https://ctan.org/pkg/bm
% Required for latex-math
% https://tex.stackexchange.com/questions/3238/bm-package-versus-boldsymbol
\usepackage{bm}

% pifont – Access to PostScript standard Symbol and Dingbats fonts
% Used for \newcommand{\xmark}{\ding{55}, which is never used
% aside from lecture_advml/attic/xx-automl/slides.Rnw
% \usepackage{pifont}

% Quotes (inline and display), provdes \enquote
% https://ctan.org/pkg/csquotes
\usepackage{csquotes}

% Adds arg to enumerate env, technically superseded by enumitem according
% to https://ctan.org/pkg/enumerate
% Replace with https://ctan.org/pkg/enumitem ?
\usepackage{enumerate}

% Line spacing - provides \singlespacing \doublespacing \onehalfspacing
% https://ctan.org/pkg/setspace
% TODO: Check if really unused?
%\usepackage{setspace}

% mathtools – Mathematical tools to use with amsmath
% https://ctan.org/pkg/mathtools?lang=en
% latex-math dependency according to latex-math repo
\usepackage{mathtools}

%--------------------------------------------------------%
%       Displaying code and algorithms
%--------------------------------------------------------%
\usepackage{verbatim}
\usepackage{algorithm}
\usepackage{algpseudocode}

%--------------------------------------------------------%
%       Tables
%--------------------------------------------------------%

% multi-row table cells: https://www.namsu.de/Extra/pakete/Multirow.html
\usepackage{multirow}

% long/multi-page tables: https://texdoc.org/serve/longtable.pdf/0
% TODO: Check if really unused?

\usepackage{longtable}

% pretty table env: https://ctan.org/pkg/booktabs?lang=en
% TODO: Check if really unused?
\usepackage{booktabs}

%--------------------------------------------------------%
%       Figures: Creating, placing, verbing
%--------------------------------------------------------%

% wrapfig - Wrapping text around figures https://de.overleaf.com/learn/latex/Wrapping_text_around_figures
\usepackage{wrapfig}

% Sub figures in figures and tables
% https://ctan.org/pkg/subfig -- supersedes subfigure package
% TODO: Check if really unused?
\usepackage{subfig}

% Actually it's pronounced PGF https://en.wikibooks.org/wiki/LaTeX/PGF/TikZ
\usepackage{tikz}

\usetikzlibrary{shapes,arrows,automata,positioning,calc,chains,trees, shadows}
\tikzset{
  %Define standard arrow tip
  >=stealth',
  %Define style for boxes
  punkt/.style={
    rectangle,
    rounded corners,
    draw=black, very thick,
    text width=6.5em,
    minimum height=2em,
    text centered},
  % Define arrow style
  pil/.style={
    ->,
    thick,
    shorten <=2pt,
    shorten >=2pt,}
}


% Unsorted
% textpos – Place boxes at arbitrary positions on the LATEX page
% https://ctan.org/pkg/textpos?lang=en
% Provides \begin{textblock}
 % TODO: Check if really unused?
\usepackage[absolute,overlay]{textpos}

% psfrag – Replace strings in encapsulated PostScript figures
% https://www.overleaf.com/latex/examples/psfrag-example/tggxhgzwrzhn
% https://ftp.mpi-inf.mpg.de/pub/tex/mirror/ftp.dante.de/pub/tex/macros/latex/contrib/psfrag/pfgguide.pdf
% Can't tell if this is needed
% TODO: Check if really unused?
\usepackage{psfrag}

% Maybe not great to use this https://tex.stackexchange.com/a/197/19093
% Use align instead -- TODO: Global search & replace to check
\usepackage{eqnarray}

\usepackage{colortbl}

% arydshln – Draw dash-lines in array/tabular
% https://www.ctan.org/pkg/arydshln
% !! "arydshln has to be loaded after array, longtable, colortab and/or colortbl"
% Provides \hdashline and \cdashline
% TODO: Check if really unused?
% \usepackage{arydshln}

% tabularx – Tabulars with adjustable-width columns
% https://ctan.org/pkg/tabularx
% Provides \begin{tabularx}
% TODO: Check if really unused?
% \usepackage{tabularx}

% placeins – Control float placement
% https://ctan.org/pkg/placeins
% Defines a \FloatBarrier command
% TODO: Check if really unused?
% \usepackage{placeins}


% framed – Framed or shaded regions that can break across pages
% https://ctan.org/pkg/framed
% Provides \begin{framed} which uses \colorbox{shadecolor} relying on \definecolor{shadecolor}.
% TODO: Check if really unused?
% \usepackage{framed}

% Used often in conjunction with \definecolor{shadecolor}{rgb}{0.969, 0.969, 0.969}
% Might be able to be removed or at least redefined to only have shadecolor (if needed)
\definecolor{fgcolor}{rgb}{0.345, 0.345, 0.345}
\definecolor{shadecolor}{rgb}{0.969, 0.969, 0.969}
\newenvironment{knitrout}{}{} % an empty environment to be redefined in TeX


% Defines macros and environments
\usepackage{../../style/lmu-lecture}

\let\code=\texttt % Used regularly
\let\proglang=\textsf % Unused?

% Not sure what/why this does
\setkeys{Gin}{width=0.9\textwidth}

\setbeamertemplate{frametitle}{\expandafter\uppercase\expandafter\insertframetitle}

% Can't find a reason why common.tex is not just part of this file?

% basic latex stuff
\newcommand{\pkg}[1]{{\fontseries{b}\selectfont #1}} %fontstyle for R packages
\newcommand{\lz}{\vspace{0.5cm}} %vertical space
\newcommand{\dlz}{\vspace{1cm}} %double vertical space
\newcommand{\oneliner}[1] % Oneliner for important statements
{\begin{block}{}\begin{center}\begin{Large}#1\end{Large}\end{center}\end{block}}


%new environments
\newenvironment{vbframe}  %frame with breaks and verbatim
{
 \begin{frame}[containsverbatim,allowframebreaks]
}
{
\end{frame}
}

\newenvironment{vframe}  %frame with verbatim without breaks (to avoid numbering one slided frames)
{
 \begin{frame}[containsverbatim]
}
{
\end{frame}
}

\newenvironment{blocki}[1]   % itemize block
{
 \begin{block}{#1}\begin{itemize}
}
{
\end{itemize}\end{block}
}

\newenvironment{fragileframe}[2]{  %fragile frame with framebreaks
\begin{frame}[allowframebreaks, fragile, environment = fragileframe]
\frametitle{#1}
#2}
{\end{frame}}


\newcommand{\myframe}[2]{  %short for frame with framebreaks
\begin{frame}[allowframebreaks]
\frametitle{#1}
#2
\end{frame}}

\newcommand{\remark}[1]{
  \textbf{Remark:} #1
}


\newenvironment{deleteframe}
{
\begingroup
\usebackgroundtemplate{\includegraphics[width=\paperwidth,height=\paperheight]{../style/color/red.png}}
 \begin{frame}
}
{
\end{frame}
\endgroup
}
\newenvironment{simplifyframe}
{
\begingroup
\usebackgroundtemplate{\includegraphics[width=\paperwidth,height=\paperheight]{../style/color/yellow.png}}
 \begin{frame}
}
{
\end{frame}
\endgroup
}\newenvironment{draftframe}
{
\begingroup
\usebackgroundtemplate{\includegraphics[width=\paperwidth,height=\paperheight]{../style/color/green.jpg}}
 \begin{frame}
}
{
\end{frame}
\endgroup
}
% https://tex.stackexchange.com/a/261480: textcolor that works in mathmode
\makeatletter
\renewcommand*{\@textcolor}[3]{%
  \protect\leavevmode
  \begingroup
    \color#1{#2}#3%
  \endgroup
}
\makeatother


%-------------------------------------------------------------------------------------------------------%
%  Unused stuff that needs to go but is kept here currently juuuust in case it was important after all  %
%-------------------------------------------------------------------------------------------------------%

% \newcommand{\hlnum}[1]{\textcolor[rgb]{0.686,0.059,0.569}{#1}}%
% \newcommand{\hlstr}[1]{\textcolor[rgb]{0.192,0.494,0.8}{#1}}%
% \newcommand{\hlcom}[1]{\textcolor[rgb]{0.678,0.584,0.686}{\textit{#1}}}%
% \newcommand{\hlopt}[1]{\textcolor[rgb]{0,0,0}{#1}}%
% \newcommand{\hlstd}[1]{\textcolor[rgb]{0.345,0.345,0.345}{#1}}%
% \newcommand{\hlkwa}[1]{\textcolor[rgb]{0.161,0.373,0.58}{\textbf{#1}}}%
% \newcommand{\hlkwb}[1]{\textcolor[rgb]{0.69,0.353,0.396}{#1}}%
% \newcommand{\hlkwc}[1]{\textcolor[rgb]{0.333,0.667,0.333}{#1}}%
% \newcommand{\hlkwd}[1]{\textcolor[rgb]{0.737,0.353,0.396}{\textbf{#1}}}%
% \let\hlipl\hlkwb

% \makeatletter
% \newenvironment{kframe}{%
%  \def\at@end@of@kframe{}%
%  \ifinner\ifhmode%
%   \def\at@end@of@kframe{\end{minipage}}%
%   \begin{minipage}{\columnwidth}%
%  \fi\fi%
%  \def\FrameCommand##1{\hskip\@totalleftmargin \hskip-\fboxsep
%  \colorbox{shadecolor}{##1}\hskip-\fboxsep
%      % There is no \\@totalrightmargin, so:
%      \hskip-\linewidth \hskip-\@totalleftmargin \hskip\columnwidth}%
%  \MakeFramed {\advance\hsize-\width
%    \@totalleftmargin\z@ \linewidth\hsize
%    \@setminipage}}%
%  {\par\unskip\endMakeFramed%
%  \at@end@of@kframe}
% \makeatother

% \definecolor{shadecolor}{rgb}{.97, .97, .97}
% \definecolor{messagecolor}{rgb}{0, 0, 0}
% \definecolor{warningcolor}{rgb}{1, 0, 1}
% \definecolor{errorcolor}{rgb}{1, 0, 0}
% \newenvironment{knitrout}{}{} % an empty environment to be redefined in TeX

% \usepackage{alltt}
% \newcommand{\SweaveOpts}[1]{}  % do not interfere with LaTeX
% \newcommand{\SweaveInput}[1]{} % because they are not real TeX commands
% \newcommand{\Sexpr}[1]{}       % will only be parsed by R
% \newcommand{\xmark}{\ding{55}}%

% math spaces
\newcommand{\N}{\mathds{N}}                                                 % N, naturals
\newcommand{\Z}{\mathds{Z}}                                                 % Z, integers
\newcommand{\Q}{\mathds{Q}}                                                 % Q, rationals
\newcommand{\R}{\mathds{R}}                                                 % R, reals
\newcommand{\C}{\mathds{C}}                                                 % C, complex
\newcommand{\HS}{\mathcal{H}}                                               % H, hilbertspace
\newcommand{\continuous}{\mathcal{C}}                                       % C, space of continuous functions
\newcommand{\M}{\mathcal{M}} 												% machine numbers
\newcommand{\epsm}{\epsilon_m} 												% maximum error


% basic math stuff
\newcommand{\xt}{\tilde x}													% x tilde
\def\argmax{\mathop{\sf arg\,max}}                                          % argmax
\def\argmin{\mathop{\sf arg\,min}}                                          % argmin
\newcommand{\sign}{\operatorname{sign}}                                     % sign, signum
\newcommand{\I}{\mathbb{I}}                                                 % I, indicator
\newcommand{\order}{\mathcal{O}}                                            % O, order
\newcommand{\fp}[2]{\frac{\partial #1}{\partial #2}}                        % partial derivative
\newcommand{\pd}[2]{\frac{\partial{#1}}{\partial #2}}						% partial derivative

% sums and products
\newcommand{\sumin}{\sum_{i=1}^n}											% summation from i=1 to n
\newcommand{\sumkg}{\sum_{k=1}^g}											% summation from k=1 to g
\newcommand{\prodin}{\prod_{i=1}^n}											% product from i=1 to n
\newcommand{\prodkg}{\prod_{k=1}^g}											% product from k=1 to g

% linear algebra
\newcommand{\one}{\boldsymbol{1}}                                           % 1, unitvector
\newcommand{\id}{\mathrm{I}}                                                % I, identity
\newcommand{\diag}{\operatorname{diag}}                                     % diag, diagonal
\newcommand{\trace}{\operatorname{tr}}                                      % tr, trace
\newcommand{\spn}{\operatorname{span}}                                      % span
\newcommand{\scp}[2]{\left\langle #1, #2 \right\rangle}                     % <.,.>, scalarproduct
\newcommand{\mat}[1]{ 														% short pmatrix command
	\begin{pmatrix}
		#1
	\end{pmatrix}
}
\newcommand{\Amat}{\bm{A}}													% matrix A
\newcommand{\xv}{\bm{x}}													% vector x (bold)
\newcommand{\yv}{\bm{y}}														% vector y (bold)
\newcommand{\Deltab}{\bm{\Delta}}											% error term for vectors
															

% basic probability + stats
\renewcommand{\P}{\mathds{P}}                                               % P, probability
\newcommand{\E}{\mathds{E}}                                                 % E, expectation
\newcommand{\var}{\mathsf{Var}}                                             % Var, variance
\newcommand{\cov}{\mathsf{Cov}}                                             % Cov, covariance
\newcommand{\corr}{\mathsf{Corr}}                                           % Corr, correlation
\newcommand{\normal}{\mathcal{N}}                                           % N of the normal distribution
\newcommand{\iid}{\overset{i.i.d}{\sim}}                                    % dist with i.i.d superscript
\newcommand{\distas}[1]{\overset{#1}{\sim}}                                 % ... is distributed as ... 
% machine learning

%%%%%% ml - data
\newcommand{\Xspace}{\mathcal{X}}                                           % X, input space
\newcommand{\Yspace}{\mathcal{Y}}                                           % Y, output space
\newcommand{\nset}{\{1, \ldots, n\}}                                        % set from 1 to n
\newcommand{\pset}{\{1, \ldots, p\}}                                        % set from 1 to p
\newcommand{\gset}{\{1, \ldots, g\}}                                        % set from 1 to g
\newcommand{\Pxy}{\P_{xy}}                                                  % P_xy
\newcommand{\xy}{(x, y)}                                                    % observation (x, y)
\newcommand{\xvec}{(x_1, \ldots, x_p)^T}                                    % (x1, ..., xp) 
\newcommand{\D}{\mathcal{D}}                                                % D, data 
\newcommand{\Dset}{\{ (x^{(1)}, y^{(1)}), \ldots, (x^{(n)},  y^{(n)})\}}    % {(x1,y1)), ..., (xn,yn)}, data
\newcommand{\xdat}{\{ x^{(1)}, \ldots, x^{(n)}\}}   						 % {x1, ..., xn}, input data
\newcommand{\ydat}{\mathbf{y}}                                              % y (bold), vector of outcomes
\newcommand{\yvec}{(y^{(1)}, \hdots, y^{(n)})^T}                            % (y1, ..., yn), vector of outcomes
\renewcommand{\xi}[1][i]{x^{(#1)}}                                          % x^i, i-th observed value of x
\newcommand{\yi}[1][i]{y^{(#1)}}                                            % y^i, i-th observed value of y 
\newcommand{\xyi}{(\xi, \yi)}                                               % (x^i, y^i), i-th observation
\newcommand{\xivec}{(x^{(i)}_1, \ldots, x^{(i)}_p)^T}                       % (x1^i, ..., xp^i), i-th observation vector
\newcommand{\xj}{x_j}                                                       % x_j, j-th feature
\newcommand{\xjb}{\mathbf{x}_j}                                             % x_j (bold), j-th feature vecor
\newcommand{\xjvec}{(x^{(1)}_j, \ldots, x^{(n)}_j)^T}                       % (x^1_j, ..., x^n_j), j-th feature vector
\newcommand{\Dtrain}{\mathcal{D}_{\text{train}}}                            % D_train, training set
\newcommand{\Dtest}{\mathcal{D}_{\text{test}}}                              % D_test, test set

%%%%%% ml - models general

% continuous prediction function f
\newcommand{\fx}{f(x)}                                                      % f(x), continuous prediction function
\newcommand{\Hspace}{H}														% hypothesis space where f is from
\newcommand{\fh}{\hat{f}}                                                   % f hat, estimated prediction function
\newcommand{\fxh}{\fh(x)}                                                   % fhat(x)
\newcommand{\fxt}{f(x | \theta)}                                            % f(x | theta)
\newcommand{\fxi}{f(\xi)}                                                   % f(x^(i))
\newcommand{\fxih}{\hat{f}(\xi)}                                            % f(x^(i))
\newcommand{\fxit}{f(x^{(i)} | \theta)}                                     % f(x^(i) | theta)
\newcommand{\fhD}{\fh_{\D}}                                                 % fhat_D, estimate of f based on D
\newcommand{\fhDtrain}{\fh_{\Dtrain}}                                       % fhat_Dtrain, estimate of f based on D

% discrete prediction function h
\newcommand{\hx}{h(x)}                                                      % h(x), discrete prediction function
\newcommand{\hh}{\hat{h}}                                                   % h hat
\newcommand{\hxh}{\hat{h}(x)}                                               % hhat(x)
\newcommand{\hxt}{h(x | \theta)}                                            % h(x | theta)
\newcommand{\hxi}{h(\xi)}                                                   % h(x^(i))
\newcommand{\hxit}{h(x^{(i)} | \theta)}                                     % h(x^(i) | theta)

% yhat
\newcommand{\yh}{\hat{y}}                                                   % y hat for prediction of target
\newcommand{\yih}{\hat{y}}                                                  % y hat for prediction of target

% theta
\newcommand{\thetah}{\hat{\theta}}                                          % theta hat

% densities + probabilities
% pdf of x 
\newcommand{\pdf}{p}                                                        % p
\newcommand{\pdfx}{p(x)}                                                    % p(x)
\newcommand{\pixt}{\pi(x | \theta)}                                         % pi(x|theta), pdf of x given theta

% pdf of (x, y)
\newcommand{\pdfxy}{p(x,y)}                                                 % p(x, y)
\newcommand{\pdfxyt}{p(x, y | \theta)}                                      % p(x, y | theta)
\newcommand{\pdfxyit}{p(\xi, \yi | \theta)}                                 % p(x^(i), y^(i) | theta)

% pdf of x given y
\newcommand{\pdfxyk}{p(x | y=k)}                                            % p(x | y = k)
\newcommand{\lpdfxyk}{\log \pdfxyk}                                         % log p(x | y = k)
\newcommand{\pdfxiyk}{p(\xi | y=k)}                                         % p(x^i | y = k)

% prior probabilities
\newcommand{\pik}{\pi_k}                                                    % pi_k, prior
\newcommand{\lpik}{\log \pik}                                               % log pi_k, log of the prior

% posterior probabilities
\newcommand{\post}{\P(y = 1 | x)}                                           % P(y = 1 | x), post. prob for y=1
\newcommand{\pix}{\pi(x)}                                                   % pi(x), P(y = 1 | x)
\newcommand{\postk}{\P(y = k | x)}                                          % P(y = k | y), post. prob for y=k
\newcommand{\pikx}{\pi_k(x)}                                                % pi_k(x), P(y = k | x)
\newcommand{\pikxt}{\pi_k(x | \theta)}                                      % pi_k(x | theta), P(y = k | x, theta)
\newcommand{\pijx}{\pi_j(x)}                                                % pi_j(x), P(y = j | x)
\newcommand{\pdfygxt}{p(y |x, \theta)}                                      % p(y | x, theta)
\newcommand{\pdfyigxit}{p(\yi |\xi, \theta)}                                % p(y^i |x^i, theta)
\newcommand{\lpdfygxt}{\log \pdfygxt }                                      % log p(y | x, theta)
\newcommand{\lpdfyigxit}{\log \pdfyigxit}                                   % log p(y^i |x^i, theta)
\newcommand{\pixh}{\hat \pi(x)}                                             % pi(x) hat, P(y = 1 | x) hat
\newcommand{\pikxh}{\hat \pi_k(x)}                                          % pi_k(x) hat, P(y = k | x) hat

% residual and margin
\newcommand{\eps}{\epsilon}                                                 % residual, stochastic
\newcommand{\epsi}{\epsilon^{(i)}}                                          % epsilon^i, residual, stochastic
\newcommand{\epsh}{\hat{\epsilon}}                                          % residual, estimated
\newcommand{\yf}{y \fx}                                                     % y f(x), margin
\newcommand{\yfi}{\yi \fxi}                                                 % y^i f(x^i), margin
\newcommand{\Sigmah}{\hat \Sigma}											% estimated covariance matrix
\newcommand{\Sigmahj}{\hat \Sigma_j}										% estimated covariance matrix for the j-th class

% ml - loss, risk, likelihood
\newcommand{\Lxy}{L(y, f(x))}                                               % L(y, f(x)), loss function
\newcommand{\Lxyi}{L(\yi, \fxi)}                                            % L(y^i, f(x^i))
\newcommand{\Lxyt}{L(y, \fxt)}                                              % L(y, f(x | theta))
\newcommand{\Lxyit}{L(\yi, \fxit)}                                          % L(y^i, f(x^i | theta)
\newcommand{\risk}{\mathcal{R}}                                             % R, risk
\newcommand{\riskf}{\risk(f)}                                               % R(f), risk
\newcommand{\riske}{\mathcal{R}_{\text{emp}}}                               % R_emp, empirical risk
\newcommand{\riskef}{\riske(f)}                                             % R_emp(f)
\newcommand{\risket}{\mathcal{R}_{\text{emp}}(\theta)}                      % R_emp(theta)
\newcommand{\riskr}{\mathcal{R}_{\text{reg}}}                               % R_reg, regularized risk
\newcommand{\riskrt}{\mathcal{R}_{\text{reg}}(\theta)}                      % R_reg(theta)
\newcommand{\riskrf}{\riskr(f)}                                             % R_reg(f)
\newcommand{\LL}{\mathcal{L}}                                               % L, likelihood
\newcommand{\LLt}{\mathcal{L}(\theta)}                                      % L(theta), likelihood
\renewcommand{\ll}{\ell}                                                    % l, log-likelihood
\newcommand{\llt}{\ell(\theta)}                                             % l(theta), log-likelihood
\newcommand{\LS}{\mathfrak{L}}                                              % ????????????
\newcommand{\TS}{\mathfrak{T}}                                              % ??????????????
\newcommand{\errtrain}{\text{err}_{\text{train}}}                           % training error
\newcommand{\errtest}{\text{err}_{\text{test}}}                             % training error
\newcommand{\errexp}{\overline{\text{err}_{\text{test}}}}                   % training error

% resampling
\newcommand{\GE}[1]{GE(\fh_{#1})}                                           % Generalization error GE
\newcommand{\GEh}[1]{\widehat{GE}_{#1}}                                     % Estimated train error
\newcommand{\GED}{\GE{\D}}                                                  % Generalization error GE
\newcommand{\EGEn}{EGE_n}                                                   % Generalization error GE
\newcommand{\EDn}{\E_{|D| = n}}                                             % Generalization error GE


% ml - irace
\newcommand{\costs}{\mathcal{C}} % costs
\newcommand{\Celite}{\theta^*} % elite configurations
\newcommand{\instances}{\mathcal{I}} % sequence of instances
\newcommand{\budget}{\mathcal{B}} % computational budget

\newcommand{\titlefigure}{figure_man/svm-geometry02.png}
\newcommand{\learninggoals}{
  \item Know that the hard-margin SVM maximizes the margin between data points and hyperplane 
  \item Know that this is a quadratic program
  \item Know that support vectors are the data points closest to the separating hyperplane
}

\title{Introduction to Machine Learning}
\date{}

\begin{document}

\lecturechapter{Linear Hard Margin SVM}
\lecture{Introduction to Machine Learning}

\sloppy


\begin{vbframe}{Linear classifiers}


\begin{center}
\includegraphics[width =9cm]{figure_man/linear-classifiers01.png} \\
\end{center}


\vspace{-0.5em}

\begin{itemize}
    \item We want study how to build a binary, linear classifier 
      from solid geometrical principles.  
    \item Which of these two classifiers is \enquote{better}?
\end{itemize}
  
  \framebreak
  

\begin{center}
\includegraphics[width =9cm]{figure_man/linear-classifiers02.png} \\
\end{center}


\vspace{-0.5em}

\begin{itemize}
    \item We want study how to build a binary, linear classifier 
      from solid geometrical principles.  
    \item Which of these two classifiers is \enquote{better}?
\end{itemize}
  
    $\quad \rightarrow$ The decision boundary on the right has a larger \textbf{safety margin.}


\end{vbframe}


% \begin{vbframe}{Recall: Hyperplanes}

%   A hyperplane in $\Xspace = \R^p$ is a $p-1$ dimensional linear subspace defined by a normal vector $\thetab$ (usually with $||\thetab|| = 1$), perpendicular to the hyperplane, and an offset $\theta_0$.
%   \lz
% For $\fx := \scp{\thetab}{\xv} + \theta_0$, the hyperplane is defined as
% \vspace{-0.3cm}
%   $$
%   \{\xv \in \Xspace: \scp{\thetab}{\xv} + \theta_0 = 0 \} = \{\xv \in \Xspace ~|~ \fx = 0 \}
%   $$
%   \begin{center}
%   \includegraphics[width=3cm]{figure_man/introduction/hyperplane2d.pdf} ~~~~~
%   \includegraphics[width=3cm]{figure_man/introduction/hyperplane3d.pdf} ~~~~~
%   \includegraphics[width=2cm]{figure_man/introduction/hyperplane_posneg.pdf} \\
%   % \footnotesize{Hyperplane in a 2-/3-dimensional space}
% \end{center}

% % \vspace{-0.5cm}

% \framebreak

% Positive halfspace: $\phantom{i}\{x \in \Xspace : f(x) > 0\}$ (in direction of $\thetab$)\\
% Negative halfspace: $\{x \in \Xspace : f(x) < 0\}$

% % \vspace{-0.5cm}

% The distance between point $x$ and hyperplane is
% $$d(f, x) = \frac{|\scp{\thetab}{x} + \theta_0|}{\|\thetab\|} = \frac{|f(x)|}{||\thetab||},$$\\
% i.e., $d(f, 0) = |\theta_0| / ||\thetab||$.

% For unit length $\thetab$, these simplify to $$d(f, x) = |f(x)|$$ and $d(f, 0) = |\theta_0|$ .

%   % \frac{|\scp{\thetab}{x} + \theta_0|}{\|\thetab\|}.

%   % $f(x)$

% % Consider now labeled data, e. g. $\xyi$ with $\yi \in \{-1, +1\}$.

% % \vspace{1.5cm}

% \end{vbframe}

\begin{vbframe}{Support Vector Machines: Geometry}

For labeled data $\D = \Dset$, with $\yi \in \{-1, +1\}$:
\begin{itemize}
  \item Assume linear separation by $\fx = \thetab^\top \xv + \theta_0$, such that all $+$-observations are in the positive halfspace

  $$
  \phantom{i}\{\xi \in \Xspace: \fx > 0\}
  $$

  and all $-$-observations are in the negative halfspace

  $$
  \phantom{i}\{\xv \in \Xspace : \fx < 0\}.
  $$

  \item For a linear separating hyperplane, we have
  $$
    \yi \underbrace{\left(\thetab^\top \xi + \theta_0\right)}_{= \fxi} > 0 \quad \forall i \in \{1, 2, ..., n\}.
  $$

  \item 
    % For correctly classified points $\left(\xi, \yi\right)$,
  $$
    d \left(f, \xi \right) = \frac{\yi \fxi}{\|\thetab\|} = \yi \frac{\thetab^T \xi + \theta_0}{\|\thetab\|}
  $$
  computes the (signed) distance to the separating hyperplane $\fx$,
    positive for correct classifications, negative for incorrect.
  \item This expression becomes negative for misclassified points.
\end{itemize}


\begin{center}
\includegraphics[width =7cm]{figure_man/svm-geometry01.png} \\
\end{center}


\framebreak

\begin{itemize}
    \item The distance of $f$ to the whole dataset $\D$
    is the smallest distance
    $$
    \gamma = \min\limits_i \Big\{ d \left(f, \xi \right) \Big\}.
    $$
    \item This represents the \enquote{safety margin}, it is positive if $f$ separates and we want to maximize it.
\end{itemize}


\begin{center}
\includegraphics[width =7cm]{figure_man/svm-geometry02.png} \\
\end{center}


\end{vbframe}

\begin{vbframe}{Maximum margin separation}

  We formulate the desired property of a large \enquote{safety margin} as an optimization problem:
  \begin{eqnarray*}
    & \max\limits_{\thetab, \theta_0} & \gamma \\
    & \text{s.t.} & \,\, d \left(f, \xi \right) \geq \gamma \quad \forall\, i \in \nset.
    \end{eqnarray*}

    \begin{itemize}
      \item The constraints mean: We require that any instance $i$ should have a \enquote{safety} distance of at least $\gamma$ from the decision boundary defined by $f = \thetab^T \xv + \theta_0 $.
      \item Our objective is to maximize the \enquote{safety} distance.
    \end{itemize}

\end{vbframe}


\begin{vbframe}{Maximum margin separation}

We reformulate the problem:

\begin{eqnarray*}
   & \max \limits_{\thetab, \theta_0} & \gamma \\
   & \text{s.t.} & \,\, \frac{\yi \left( \scp{\thetab}{\xi} + \theta_0 \right)}{\|\thetab\|} \geq \gamma \quad \forall\, i \in \nset.
\end{eqnarray*}

\begin{itemize}
  \item The inequality is rearranged by multiplying both sides with $\|\thetab\|$:
\end{itemize}

\begin{eqnarray*}
   & \max \limits_{\thetab, \theta_0} & \gamma \\
   & \text{s.t.} & \,\,\yi  \left( \scp{\thetab}{\xi} + \theta_0 \right) \geq \|\thetab\| \gamma \quad \forall\, i \in \nset.
\end{eqnarray*}

\framebreak

  \begin{itemize}
    \item Note that the same hyperplane does not have a unique representation:
    $$
      \{\xv \in \Xspace ~|~ \thetab^\top \xv = 0\} = \{\xv \in \Xspace ~|~ c \cdot \thetab^\top \xv = 0\}
    $$
    for arbitrary $c \ne 0$.
    \item To ensure uniqueness of the solution, we make a reference choice -- we only consider hyperplanes with $\|\thetab\| = 1 / \gamma$:
  \end{itemize}

  \begin{eqnarray*}
     & \max \limits_{\thetab, \theta_0} & \gamma \\
     & \text{s.t.} & \,\,\yi  \left( \scp{\thetab}{\xi} + \theta_0 \right) \geq 1 \quad \forall\, i \in \nset.
  \end{eqnarray*}

\framebreak

  \begin{itemize}
    \item Substituting $\gamma = 1 / \|\thetab\|$ in the objective yields:
  \end{itemize}

  \begin{eqnarray*}
     & \max \limits_{\thetab, \theta_0} & \frac{1}{\|\thetab\|} \\
     & \text{s.t.} & \,\,\yi  \left( \scp{\thetab}{\xi} + \theta_0 \right) \geq 1 \quad \forall\, i \in \nset.
  \end{eqnarray*}

 \begin{itemize}
  \item Maximizing $1 / \|\thetab\|$ is the same as minimizing $\|\thetab\|$, which is the same as minimizing $\frac{1}{2}\|\thetab\|^2$:
  \end{itemize}

\begin{eqnarray*}
  & \min\limits_{\thetab, \theta_0} \quad & \frac{1}{2} \|\thetab\|^2 \\
   & \text{s.t.} & \,\,\yi  \left( \scp{\thetab}{\xi} + \theta_0 \right) \geq 1 \quad \forall\, i \in \nset.
\end{eqnarray*}

\end{vbframe}

\begin{vbframe}{Quadratic Program}

We derived the following optimization problem:

  \begin{eqnarray*}
  & \min\limits_{\thetab, \theta_0} \quad & \frac{1}{2} \|\thetab\|^2 \\
  & \text{s.t.} & \,\,\yi  \left( \scp{\thetab}{\xi} + \theta_0 \right) \geq 1 \quad \forall\, i \in \nset.
\end{eqnarray*}

This turns out to be a \textbf{convex optimization problem} -- particularly, a \textbf{quadratic program}: The objective function is quadratic, and the constraints are linear inequalities.

\lz

This is called the \textbf{primal} problem. We will later show that we can also derive a dual problem from it.

\lz

We will call this the \textbf{linear hard-margin SVM}.
\end{vbframe}

% \frame{
% \frametitle{Maximum margin separation}
%    \begin{eqnarray*}
%     \only<1-5>  {& \max \limits_{\thetab, \theta_0} & \gamma}
%     \only<6->{& \min\limits_{\thetab, \theta_0} \quad & \frac{1}{2} \|\thetab\|^2} \\
%     \only<1>{& \text{s.t.} & \,\, d(f, \xi) \geq \gamma \quad \forall\, i \in \nset}
%     \only<2>{& \text{s.t.} & \,\, \frac{\yi \left( \scp{\thetab}{\xi} + \theta_0 \right)}{\|\thetab\|} \geq \gamma \quad \forall\, i \in \nset} \\
%     \only<3-4> {& \text{s.t.} & \,\,\yi  \left( \scp{\thetab}{\xi} + \theta_0 \right) \geq \|\thetab\| \gamma \quad \forall\, i \in \nset} \\
%     \only<5-> {& \text{s.t.} & \,\,\yi  \left( \scp{\thetab}{\xi} + \theta_0 \right) \geq 1 \quad \forall\, i \in \nset}
%    \end{eqnarray*}
%    \vspace{-1em}
%   \pause
%
%    \begin{itemize}
%     % \item<1-> We can get rid of the $\|\thetab\|=1$ constraint by dividing the inequality constraint by $\|\thetab\|$
%     \pause
%     \item<3-> The inequality is rearranged by multiplying both sides with $\|\thetab\|$
%     \pause
%     \item<4-> Remember: As we assume linear separability, any positively scaled $\thetab, \theta_0$ satisfies the constraint, too \\
%     \item<5-> We substitute $\|\thetab\| = \frac{1}{\gamma} \Leftrightarrow \gamma = 1/\|\thetab\|$
%     \item<6-> Maximizing $\gamma$ is the same as minimizing $\|\thetab\|$ which is the same as minimizing $\frac{1}{2}\|\thetab\|^2$
%     % \item There are efficient ``off-the-shelf'' algorithms for solving
%     % such problems.
%    \end{itemize}
%
% }


% \frame{
% \frametitle{Maximum margin separation}
%
%   Now we will reformulate the optimization problem:
%   \begin{eqnarray*}
%     \only<1>  {& \max \limits_{\thetab, \theta_0} & \gamma}
%     \only<2-3>  {& \min\limits_{\thetab, \theta_0} & 1 / (2 \gamma^2)}
%     \only<4->{& \min\limits_{\thetab, \theta_0} \quad & \frac{1}{2} \|\thetab\|^2} \\
%     \only<1-2>{& \text{s.t.} & \,\, \yi  \left( \scp{\thetab}{\xi} + \theta_0 \right) \geq \gamma \quad \forall\, i \in \nset}
%     \only<3->{& \text{s.t.} & \,\, \yi  \left( \scp{\thetab}{\xi} + \theta_0 \right) \geq 1 \quad \forall\, i \in \nset} \\
%     \only<-2>{& \quad & \|\thetab\| = 1}
%     \only<3>{& \quad & \|\thetab\| = 1/\gamma}
%     \only<4->{& \enspace & \enspace}
%   \end{eqnarray*}
%   \vspace{-1em}
%   \pause
%   \begin{itemize}
%     \item Maximizing $\gamma$ is the same as minimizing $1 / (2 \gamma^2)$.
%     \pause
%     \item The solution $(\thetab, \theta_0)$ can be scaled without           changing the classifier. We scale it with the factor $1 / \gamma$.
%     \pause
%     \item The second constraint can be used to eliminate $\gamma$ from
%     the optimization problem.
%     It still holds $\gamma = 1 / \|\thetab\|$.
%     \pause
%     \item This turns out to be a convex optimization problem.
%     This particular form is called a \textbf{quadratic program}:
%     the objective function is quadratic, and the constraints are
%     linear (equalities and) inequalities.
%     % \item There are efficient ``off-the-shelf'' algorithms for solving
%     % such problems.
%   \end{itemize}
%
% }


% --- ADDED FOR MUDS: THIS SHOULD BE REDUNDANT AS IT IS IN SLIDES-2-HARD-MARGIN SVM

% Because we do not cover duality, slides-2-hard-margin-svm would only contain 2 slides. Thus I added it here.

\begin{vbframe}{Support Vectors}
% \vspace{1cm}

% \begin{center}
%   \includegraphics[width=5cm]{figure_man/kernels/separable-f.pdf}
% \end{center}

% \framebreak

  \begin{itemize}
    \item There exist instances $(\xi, \yi)$ with minimal margin
    $\yi  \fxi = 1$, fulfilling the inequality constraints with equality.
    \item They are called
    \textbf{support vectors (SVs)}. They are located exactly at
    a distance of $\gamma = 1 / \|\thetab\|$ from the separating hyperplane.
    \item It is already geometrically obvious 
      that the solution does not depend on the non-SVs! We could delete them from the data and would arrive at the same solution.
    \vspace{0.5cm}
    

\begin{center}
\includegraphics[width =6cm]{figure_man/support-vectors.png} \\
\end{center}


  \end{itemize}


\end{vbframe}


% \begin{vbframe}{Dual Hard-Margin SVM}
% 
% It can be shown that \textbf{dual} representation is an equivalent representation of the problem:
% 
% \vspace*{-0.5cm}
% \begin{eqnarray*}
%     & \max\limits_{\alpha \in \R^n} & \sum_{i=1}^n \alpha_i - \frac{1}{2}\sum_{i=1}^n\sum_{j=1}^n\alpha_i\alpha_j\yi y^{(j)} \scp{\xi}{\xv^{(j)}} \\
%     & \text{s.t.} & \sum_{i=1}^n \alpha_i \yi = 0, \\
%     & \quad & \alpha_i \ge 0~\forall i \in \nset,
% \end{eqnarray*}
% 
% or, defining $\bm{K}:= \Xmat \Xmat^\top$,   equivalently in matrix notation:
% 
% \vspace*{-.5cm}
% \begin{eqnarray*}
%   & \max\limits_{\alpha \in \R^n} & \one^\top \alpha - \frac{1}{2} \alpha^\top \diag(\ydat)\bm{K} \diag(\ydat) \alpha \\
%   & \text{s. t.} & \alpha^\top \ydat = 0, \\
%   & \quad & \alpha \geq 0.
% \end{eqnarray*}
% 
% 
% \framebreak
% 
% It can be shown that the solution of a hard-margin support vector machine can be written as follows:
% 
%   $$
%     \thetah = \sumin \alpha_i \yi \xi \quad \text{ and }\quad \theta_0 = y^{(i^\star)} - \scp{\thetab}{\xv^{(i^\star)}},
%   $$
% 
%   where $(\xv^{(i^\star)}, y^{(i^\star)})$ can be any support vector.
% 
% \lz
% 
% Class predictions for a new observation $\xv$ are constructed by
% $$h(\xv) = \text{sgn}(f(\xv)) = \text{sgn}\left(\sumin \alpha_i \yi {\xi}^{\top} \xv + \theta_0 \right),$$
% 
% which is essentially a weighted sum of the
% \begin{itemize}
% \item dot products ${\xi}^{\top} \xv$ (i.e., the similarity of $\xi$ and $\xv$), with
% \item weights $\alpha_i \yi$, where $\alpha_i = 0$ for all non-SVs.
% \end{itemize}
% \end{vbframe}


\endlecture
\end{document}
