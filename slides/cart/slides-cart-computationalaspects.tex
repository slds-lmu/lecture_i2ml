\documentclass[11pt,compress,t,notes=noshow, xcolor=table]{beamer}
% graphicx and color are loaded via lmu-lecture.sty
% maxwidth is the original width if it is less than linewidth
% otherwise use linewidth (to make sure the graphics do not exceed the margin)
% TODO: Remove once cleared to be superfluous
% \makeatletter
% \def\maxwidth{ %
%   \ifdim\Gin@nat@width>\linewidth
%     \linewidth
%   \else
%     \Gin@nat@width
%   \fi
% }
% \makeatother

% ---------------------------------%
% latex-math dependencies, do not remove:
% - mathtools
% - bm
% - siunitx
% - dsfont
% - xspace
% ---------------------------------%

%--------------------------------------------------------%
%       Language, encoding, typography
%--------------------------------------------------------%

\usepackage[english]{babel}
\usepackage[utf8]{inputenc} % Enables inputting UTF-8 symbols
% Standard AMS suite (loaded via lmu-lecture.sty)

% Font for double-stroke / blackboard letters for sets of numbers (N, R, ...)
% Distribution name is "doublestroke"
% According to https://mirror.physik.tu-berlin.de/pub/CTAN/fonts/doublestroke/dsdoc.pdf
% the "bbm" package does a similar thing and may be superfluous.
% Required for latex-math
\usepackage{dsfont}

% bbm – "Blackboard-style" cm fonts (https://www.ctan.org/pkg/bbm)
% Used to be in common.tex, loaded directly after this file
% Maybe superfluous given dsfont is loaded
% TODO: Check if really unused?
% \usepackage{bbm}

% bm – Access bold symbols in maths mode - https://ctan.org/pkg/bm
% Required for latex-math, preferred over \boldsymbol
% https://tex.stackexchange.com/questions/3238/bm-package-versus-boldsymbol
\usepackage{bm}

% pifont – Access to PostScript standard Symbol and Dingbats fonts
% Used for \newcommand{\xmark}{\ding{55}, which is never used
% aside from lecture_advml/attic/xx-automl/slides.Rnw
% \usepackage{pifont}

% Quotes (inline and display), provdes \enquote
% https://ctan.org/pkg/csquotes
\usepackage{csquotes}

% Adds arg to enumerate env, technically superseded by enumitem according
% to https://ctan.org/pkg/enumerate
% Replace with https://ctan.org/pkg/enumitem ?
% Even better: enumitem is not really compatible with beamer and breaks all sorts of things
% particularly the enumerate environment. The enumerate package also just isn't required
% from what I can tell so... don't re-add it I guess?
% \usepackage{enumerate}

% Line spacing - provides \singlespacing \doublespacing \onehalfspacing
% https://ctan.org/pkg/setspace
% \usepackage{setspace}

% mathtools – Mathematical tools to use with amsmath
% https://ctan.org/pkg/mathtools?lang=en
% latex-math dependency according to latex-math repo
\usepackage{mathtools}

% Maybe not great to use this https://tex.stackexchange.com/a/197/19093
% Use align instead -- TODO: Global search & replace to check, eqnarray is used a lot
% $ rg -f -u "\begin{eqnarray" -l | grep -v attic | awk -F '/' '{print $1}' | sort | uniq -c
%   13 lecture_advml
%   14 lecture_i2ml
%    2 lecture_iml
%   27 lecture_optimization
%   45 lecture_sl
\usepackage{eqnarray}
% For shaded regions / boxes
% Used sometimes in optim
% https://www.ctan.org/pkg/framed
\usepackage{framed}

%--------------------------------------------------------%
%       Cite button (version 2024-05)
%--------------------------------------------------------%

% Superseded by style/ref-buttons.sty, kept just in case these don't work out somehow.

% Note this requires biber to be in $PATH when running,
% telltale error in log would be e.g. Package biblatex Info: ... file 'authoryear.dbx' not found
% aside from obvious "biber: command not found" or similar.
% Tried moving this to lmu-lecture.sty but had issues I didn't quite understood,
% so it's here for now.

\usepackage{hyperref}

% Only try adding a references file if it exists, otherwise
% this would compile error when references.bib is not found
% NOTE: Bibliography packages (usebib, biblatex) are now loaded by ref-buttons.sty when needed
% This keeps all bibliography-related setup in one place

% Legacy \citelink command removed - superseded by ref-buttons.sty

%--------------------------------------------------------%
%       Displaying code and algorithms
%--------------------------------------------------------%

% Reimplements verbatim environments: https://ctan.org/pkg/verbatim
% verbatim used sed at least once in
% supervised-classification/slides-classification-tasks.tex
% Removed since code should not be put on slides anyway
% \usepackage{verbatim}

% Both used together for algorithm typesetting, see also overleaf: https://www.overleaf.com/learn/latex/Algorithms
% algorithmic env is also used, but part of the bundle:
%   "algpseudocode is part of the algorithmicx bundle, it gives you an improved version of algorithmic besides providing some other features"
% According to https://tex.stackexchange.com/questions/229355/algorithm-algorithmic-algorithmicx-algorithm2e-algpseudocode-confused
\usepackage{algorithm}
\usepackage{algpseudocode}

%--------------------------------------------------------%
%       Tables
%--------------------------------------------------------%

% multi-row table cells: https://www.namsu.de/Extra/pakete/Multirow.html
% Provides \multirow
% Used e.g. in evaluation/slides-evaluation-measures-classification.tex
\usepackage{multirow}

% colortbl: https://ctan.org/pkg/colortbl
% "The package allows rows and columns to be coloured, and even individual cells." well.
% Provides \columncolor and \rowcolor
% \rowcolor is used multiple times, e.g. in knn/slides-knn.tex
\usepackage{colortbl}

% long/multi-page tables: https://texdoc.org/serve/longtable.pdf/0
% Not used in slides
% \usepackage{longtable}

% pretty table env: https://ctan.org/pkg/booktabs
% Is used
% Defines \toprule
\usepackage{booktabs}

%--------------------------------------------------------%
%       Figures: Creating, placing, verbing
%--------------------------------------------------------%

% wrapfig - Wrapping text around figures https://de.overleaf.com/learn/latex/Wrapping_text_around_figures
% Provides wrapfigure environment -used in lecture_optimization
\usepackage{wrapfig}

% Sub figures in figures and tables
% https://ctan.org/pkg/subfig -- supersedes subfigure package
% Provides \subfigure
% \subfigure not used in slides but slides-tuning-practical.pdf errors without this pkg, error due to \captionsetup undefined
\usepackage{subfig}

% Actually it's pronounced PGF https://en.wikibooks.org/wiki/LaTeX/PGF/TikZ
\usepackage{tikz}

% No idea what/why these settings are what they are but I assume they're there on purpose
\usetikzlibrary{shapes,arrows,automata,positioning,calc,chains,trees, shadows}
\tikzset{
  %Define standard arrow tip
  >=stealth',
  %Define style for boxes
  punkt/.style={
    rectangle,
    rounded corners,
    draw=black, very thick,
    text width=6.5em,
    minimum height=2em,
    text centered},
  % Define arrow style
  pil/.style={
    ->,
    thick,
    shorten <=2pt,
    shorten >=2pt,}
}

%--------------------------------------------------------%
%       Beamer setup and custom macros & environments
%--------------------------------------------------------%

% Main sty file for beamer setup (layout, style, lecture page numbering, etc.)
% For long-term maintenance, this may me refactored into a more modular set of .sty files
\usepackage{../../style/lmu-lecture}
% Custom itemize wrappers, itemizeS, itemizeL, etc
\usepackage{../../style/customitemize}
% Custom framei environment, uses custom itemize!
\usepackage{../../style/framei}
% Custom frame2 environment, allows specifying font size for all content
\usepackage{../../style/frame2}
% Column layout macros
\usepackage{../../style/splitV}
% \image and derivatives
\usepackage{../../style/image}
% New generation of reference button macros
\usepackage{../../style/ref-buttons}

% Used regularly
\let\code=\texttt

% Not sure what/why this does
\setkeys{Gin}{width=0.9\textwidth}

% -- knitr leftovers --
% These may be used by knitr/R Markdown workflows in other lectures
\makeatletter
\def\maxwidth{ %
  \ifdim\Gin@nat@width>\linewidth
    \linewidth
  \else
    \Gin@nat@width
  \fi
}
\makeatother

% Define colors for syntax highlighting (may be used by knitr)
\definecolor{fgcolor}{rgb}{0.345, 0.345, 0.345}
\definecolor{shadecolor}{rgb}{.97, .97, .97}

% knitr code output environment
\newenvironment{knitrout}{}{}


% Can't find a reason why common.tex is not just part of this file?
% This file is included in slides and exercises

% Rarely used fontstyle for R packages, used only in 
% - forests/slides-forests-benchmark.tex
% - exercises/single-exercises/methods_l_1.Rnw
% - slides/cart/attic/slides_extra_trees.Rnw
\newcommand{\pkg}[1]{{\fontseries{b}\selectfont #1}}

% Spacing helpers, used often (mostly in exercises for \dlz)
\newcommand{\lz}{\vspace{0.5cm}} % vertical space (used often in slides)
\newcommand{\dlz}{\vspace{1cm}}  % double vertical space (used often in exercises, never in slides)
\newcommand{\oneliner}[1] % Oneliner for important statements, used e.g. in iml, algods
{\begin{block}{}\begin{center}\begin{Large}#1\end{Large}\end{center}\end{block}}

% Don't know if this is used or needed, remove?
% textcolor that works in mathmode
% https://tex.stackexchange.com/a/261480
% Used e.g. in forests/slides-forests-bagging.tex
% [...] \textcolor{blue}{\tfrac{1}{M}\sum^M_{m} [...]
% \makeatletter
% \renewcommand*{\@textcolor}[3]{%
%   \protect\leavevmode
%   \begingroup
%     \color#1{#2}#3%
%   \endgroup
% }
% \makeatother


\input{../../latex-math/ml-trees.tex}

\title{Introduction to Machine Learning}

\begin{document}

\titlemeta{% Chunk title (example: CART, Forests, Boosting, ...), can be empty
  CART 
  }{% Lecture title  
  Computational Aspects of Finding Splits
  }{% Relative path to title page image: Can be empty but must not start with slides/
  figure/categoryplot-binarysmall
  }{
  \item Know how monotone feature transformations affect the tree
  \item Understand how categorical features can be treated effectively while growing a CART
  \item Understand how missing values can be treated in a CART
}

\begin{frame2}{Monotone feature transformations}

Monotone transformations of one or several features will neither change the value of the splitting criterion nor the structure of the tree,  only the numerical value of the split point.
\vspace{0.5cm}
\begin{columns}[T]
\column{0.49\textwidth}
Original data
\begin{knitrout}\scriptsize
\definecolor{shadecolor}{rgb}{0.969, 0.969, 0.969}\color{fgcolor}
\begin{tabular}{l|r|r|r|r|r}
\hline
x & 1.0 & 2.0 & 7.0 & 10.0 & 20.0\\
\hline
y & 1.0 & 1.0 & 0.5 & 9.0 & 11.0\\
\hline
\end{tabular}


\end{knitrout}
% FIGURE SOURCE: Use picture created in rsrc/monotone_trafo.R
\includegraphics[width = \textwidth]{figure/cart_splitcomp_1}
\column{0.49\textwidth}
Data with log-transformed $x$
\begin{knitrout}\scriptsize
\definecolor{shadecolor}{rgb}{0.969, 0.969, 0.969}\color{fgcolor}
\begin{tabular}{l|r|r|r|r|r}
\hline
log(x) & 0.0 & 0.7 & 1.9 & 2.3 & 3.0\\
\hline
y & 1.0 & 1.0 & 0.5 & 9.0 & 11.0\\
\hline
\end{tabular}


\end{knitrout}
% FIGURE SOURCE: Use picture created in rsrc/monotone_trafo.R
\includegraphics[width = \textwidth]{figure/cart_splitcomp_2}
\end{columns}
\vspace{0.5cm}
\centering
\end{frame2}

\begin{frame2}{Categorical Features}
  \begin{itemize}
  \item A split on a categorical feature partitions the feature levels:
    $$x_j \in \{a,b,c\} \leftarrow \Np \rightarrow x_j \in \{d,e\} $$
  \end{itemize}
  \begin{figure}
   \includegraphics[width=0.8\textwidth]{figure/tree-categorical.pdf} 
  \end{figure}
  \end{frame2}
  
  \begin{frame2}{Categorical Features}
  \begin{itemize}
  \item A split on a categorical feature partitions the feature levels:
    $$x_j \in \{a,b,c\} \leftarrow \Np \rightarrow x_j \in \{d,e\} $$
  \item For a feature with $m$ levels,
  there are about $2^m$ different possible partitions of the $m$ values into two groups\\ ($2^{m-1} - 1$ because of symmetry and empty groups).
  \item Searching over all these becomes prohibitive for large values of $m$.
  \item For regression with L2 loss and for binary classification, we can define clever shortcuts.
  \end{itemize}

  \end{frame2}
  
  \begin{frame}{Categorical Features}

For $0-1$ responses, in each node:
  \begin{enumerate}
  \item Calculate the proportion of $1$-outcomes for each category of the feature in $\Np$.

  \end{enumerate}
  \begin{columns}
  \begin{column}{0.33\textwidth}
  \begin{figure}
  \includegraphics[width=0.8\textwidth]{figure/categoryplot-binary1.pdf} 
  \end{figure}
  \end{column}
  \begin{column}{0.33\textwidth}
  \lz
  \end{column}
  \begin{column}{0.33\textwidth}
  \lz
  \end{column}
  \end{columns}

\end{frame}

  \begin{frame}[noframenumbering]{Categorical Features}

For $0-1$ responses, in each node:
  \begin{enumerate}
  \item Calculate the proportion of $1$-outcomes for each category of the feature in $\Np$.
  \item Sort the categories according to these proportions.
  \end{enumerate}
  \begin{columns}
  \begin{column}{0.33\textwidth}
  \begin{figure}
  \includegraphics[width=0.8\textwidth]{figure/categoryplot-binary1.pdf} 
  \end{figure}
  \end{column}
  \begin{column}{0.33\textwidth}
  \begin{figure}
  \includegraphics[width=0.8\textwidth]{figure/categoryplot-binary2.pdf} 
  \end{figure}
  \end{column}
  \begin{column}{0.33\textwidth}
  \end{column}
  \end{columns}

\end{frame}

  \begin{frame}[noframenumbering]{Categorical Features}

For $0-1$ responses, in each node:
  \begin{enumerate}
  \item Calculate the proportion of $1$-outcomes for each category of the feature in $\Np$.
  \item Sort the categories according to these proportions.
  \item The feature can then be treated as if it was ordinal, so we only have to investigate at most $m-1$ splits.
  \end{enumerate}
  \begin{columns}
  \begin{column}{0.33\textwidth}
  \begin{figure}
  \includegraphics[width=0.8\textwidth]{figure/categoryplot-binary1.pdf} 
  \end{figure}
  \end{column}
  \begin{column}{0.33\textwidth}
  \begin{figure}
  \includegraphics[width=0.8\textwidth]{figure/categoryplot-binary2.pdf} 
  \end{figure}
  \end{column}
  \begin{column}{0.33\textwidth}
  \begin{figure}
  \includegraphics[width=0.8\textwidth]{figure/categoryplot-binary3.pdf} 
  \end{figure}
  \end{column}
  \end{columns}

\end{frame}


\begin{frame2}{Categorical Features}

  \begin{itemize}
  \item This procedure finds the optimal split.
  \item This result also holds for regression trees (with L2 loss) if the levels of the feature are ordered by increasing mean of the target
  \item The proofs are not trivial and can be found here:
    \begin{itemize}
    \item for 0-1 responses:
      \begin{itemize}
      \item \furtherreading{BREIMAN1984}
      \item  \furtherreading{RIPLEY1996}
      \end{itemize}
    \item for continuous responses:
      \begin{itemize}
      \item \furtherreading{FISHER1958}
      \end{itemize}
    \end{itemize}
  \item There are only heuristics for the multiclass case \furtherreading{WRIGHT2019}
  %\item The Algorithm prefers categorical variables with a large value
  %of categories $Q$
  \end{itemize}

\end{frame2}

\begin{frame2}{Categorical Features}

For continuous responses, in each node:
  \begin{enumerate}
  \item Calculate the mean of the outcome in each category
  \item Sort the categories by increasing mean of the outcome
  \end{enumerate}

  \begin{columns}
  \begin{column}{0.33\textwidth}
  \begin{figure}
  \includegraphics[width=0.8\textwidth]{figure/categoryplot-cont1.pdf} 
  \end{figure}
  \end{column}
  \begin{column}{0.33\textwidth}
  \begin{figure}
  \includegraphics[width=0.8\textwidth]{figure/categoryplot-cont2.pdf} 
  \end{figure}
  \end{column}
  \begin{column}{0.33\textwidth}
  \begin{figure}
  \includegraphics[width=0.8\textwidth]{figure/categoryplot-cont3.pdf} 
  \end{figure}
  \end{column}
  \end{columns}


\end{frame2}

\begin{frame2}{Missing feature values}
  \begin{itemize}
    \item When splits are evaluated, only observations for which the used feature is not missing are used. (This can actually bias splits towards using features with lots of missing values.) 
  \item \textbf{Surrogate splits} can deal with missing values during prediction.
  \item Surrogate splits are created during training. They define replacement splitting rules, using a different feature, that result in almost the same child nodes as the original split.
   \item When observations are passed down the tree, % (in training or prediction), 
   and the feature value used in a split is missing, we use the surrogate split instead to decide to which child the data should be assigned. 
  \end{itemize}
\end{frame2}

\begin{frame2}{Surrogate Splits}
\begin{itemize}
\item Each surrogate split is a decision stump that tries to learn the actual splitting rule
\item Consider this tree with the primary split w.r.t. \texttt{Sepal.Length} where we perform binary classification (\texttt{setosa} vs. \texttt{virginica}):
\begin{figure}
\includegraphics[width=0.75\textwidth]{figure/tree-binary.pdf} 
\end{figure}
\item Our surrogate split should optimize a splitting criterion w.r.t. \texttt{Sepal.Length < 5.8}
\end{itemize}



\end{frame2}

\begin{frame2}{Surrogate Splits}
\begin{itemize}
\item Consider this subsample of the data used to fit the tree:
\begin{table}[ht]
\tiny
\centering
\begin{tabular}{rrrrrll}
  \hline
 & Sepal.Length & ... & Petal.Width & Species & Sepal.Length $<$ 5.8 \\ 
  \hline
1 & 5.10 & ... & 0.20 & setosa & TRUE \\ 
  4 & 4.60 & ... & 0.20 & setosa & TRUE \\ 
  9 & 4.40 & ... & 0.20 & setosa & TRUE \\ 
  15 & 5.80 & ... & 0.20 & setosa & FALSE \\ 
  18 & 5.10 & ... & 0.30 & setosa & TRUE \\ 
  52 & 5.80 & ... & 1.90 & virginica & FALSE \\ 
  57 & 4.90 & ... & 1.70 & virginica & TRUE \\ 
  62 & 6.40 & ... & 1.90 & virginica & FALSE \\ 
  77 & 6.20 & ... & 1.80 & virginica & FALSE \\ 
  99 & 6.20 & ... & 2.30 & virginica & FALSE \\ 
   \hline
\end{tabular}
\end{table}
\item Add column that indicates whether \texttt{Sepal.Length < 5.8}
%\item As this splitting rule is very good, we will have many instances where \texttt{Sepal.Length < 5.8} is \texttt{TRUE} and \texttt{Species} is \texttt{setosa}
\item Fit tree of depth 1 using all features but \texttt{Sepal.Length} %used 
to derive a split that explains 
\texttt{Sepal.Length < 5.8} best $\Rightarrow$ surrogate split
\item Typically, software stores the best and a few more surrogate splits
%\item A good surrogate tries to mimic the primary split this way

\end{itemize}


\end{frame2}


\endlecture
\end{document}
