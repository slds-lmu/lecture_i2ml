\documentclass[11pt,compress,t,notes=noshow, xcolor=table]{beamer}
\usepackage[]{graphicx}
\usepackage[]{color}
% maxwidth is the original width if it is less than linewidth
% otherwise use linewidth (to make sure the graphics do not exceed the margin)
\makeatletter
\def\maxwidth{ %
  \ifdim\Gin@nat@width>\linewidth
    \linewidth
  \else
    \Gin@nat@width
  \fi
}
\makeatother

% ---------------------------------%
% latex-math dependencies, do not remove:
% - \usepackage{mathtools}
% - \usepackage{bm}
% - \usepackage{siunitx}
% - \usepackage{dsfont}
% - \usepackage{xspace}
% ---------------------------------%

%--------------------------------------------------------%
%       Language, encoding, typography
%--------------------------------------------------------%

\usepackage[english]{babel}
\usepackage[utf8]{inputenc} % Enables inputting UTF-8 symbols
% Standard AMS suite
\usepackage{amsmath,amsfonts,amssymb}

% Font four double-stroke / blackboard letters for sets of numbers (N, R, ...)
% Distribution name is "doublestroke"
% According to https://mirror.physik.tu-berlin.de/pub/CTAN/fonts/doublestroke/dsdoc.pdf
% the "bbm" package does a similar thing and may be superfluous.
% Required for latex-math
\usepackage{dsfont}

% bbm – "Blackboard-style" cm fonts (https://www.ctan.org/pkg/bbm)
% Used to be in common.tex, loaded directly after this file
% Maybe superfluous given dsfont is loaded
% TODO: Check if really unused?
% \usepackage{bbm}

% bm – Access bold symbols in maths mode - https://ctan.org/pkg/bm
% Required for latex-math
% https://tex.stackexchange.com/questions/3238/bm-package-versus-boldsymbol
\usepackage{bm}

% pifont – Access to PostScript standard Symbol and Dingbats fonts
% Used for \newcommand{\xmark}{\ding{55}, which is never used
% aside from lecture_advml/attic/xx-automl/slides.Rnw
% \usepackage{pifont}

% Quotes (inline and display), provdes \enquote
% https://ctan.org/pkg/csquotes
\usepackage{csquotes}

% Adds arg to enumerate env, technically superseded by enumitem according
% to https://ctan.org/pkg/enumerate
% Replace with https://ctan.org/pkg/enumitem ?
\usepackage{enumerate}

% Line spacing - provides \singlespacing \doublespacing \onehalfspacing
% https://ctan.org/pkg/setspace
% TODO: Check if really unused?
%\usepackage{setspace}

% mathtools – Mathematical tools to use with amsmath
% https://ctan.org/pkg/mathtools?lang=en
% latex-math dependency according to latex-math repo
\usepackage{mathtools}

%--------------------------------------------------------%
%       Displaying code and algorithms
%--------------------------------------------------------%
\usepackage{verbatim}
\usepackage{algorithm}
\usepackage{algpseudocode}

%--------------------------------------------------------%
%       Tables
%--------------------------------------------------------%

% multi-row table cells: https://www.namsu.de/Extra/pakete/Multirow.html
\usepackage{multirow}

% long/multi-page tables: https://texdoc.org/serve/longtable.pdf/0
% TODO: Check if really unused?

\usepackage{longtable}

% pretty table env: https://ctan.org/pkg/booktabs?lang=en
% TODO: Check if really unused?
\usepackage{booktabs}

%--------------------------------------------------------%
%       Figures: Creating, placing, verbing
%--------------------------------------------------------%

% wrapfig - Wrapping text around figures https://de.overleaf.com/learn/latex/Wrapping_text_around_figures
\usepackage{wrapfig}

% Sub figures in figures and tables
% https://ctan.org/pkg/subfig -- supersedes subfigure package
% TODO: Check if really unused?
\usepackage{subfig}

% Actually it's pronounced PGF https://en.wikibooks.org/wiki/LaTeX/PGF/TikZ
\usepackage{tikz}

\usetikzlibrary{shapes,arrows,automata,positioning,calc,chains,trees, shadows}
\tikzset{
  %Define standard arrow tip
  >=stealth',
  %Define style for boxes
  punkt/.style={
    rectangle,
    rounded corners,
    draw=black, very thick,
    text width=6.5em,
    minimum height=2em,
    text centered},
  % Define arrow style
  pil/.style={
    ->,
    thick,
    shorten <=2pt,
    shorten >=2pt,}
}


% Unsorted
% textpos – Place boxes at arbitrary positions on the LATEX page
% https://ctan.org/pkg/textpos?lang=en
% Provides \begin{textblock}
 % TODO: Check if really unused?
\usepackage[absolute,overlay]{textpos}

% psfrag – Replace strings in encapsulated PostScript figures
% https://www.overleaf.com/latex/examples/psfrag-example/tggxhgzwrzhn
% https://ftp.mpi-inf.mpg.de/pub/tex/mirror/ftp.dante.de/pub/tex/macros/latex/contrib/psfrag/pfgguide.pdf
% Can't tell if this is needed
% TODO: Check if really unused?
\usepackage{psfrag}

% Maybe not great to use this https://tex.stackexchange.com/a/197/19093
% Use align instead -- TODO: Global search & replace to check
\usepackage{eqnarray}

\usepackage{colortbl}

% arydshln – Draw dash-lines in array/tabular
% https://www.ctan.org/pkg/arydshln
% !! "arydshln has to be loaded after array, longtable, colortab and/or colortbl"
% Provides \hdashline and \cdashline
% TODO: Check if really unused?
% \usepackage{arydshln}

% tabularx – Tabulars with adjustable-width columns
% https://ctan.org/pkg/tabularx
% Provides \begin{tabularx}
% TODO: Check if really unused?
% \usepackage{tabularx}

% placeins – Control float placement
% https://ctan.org/pkg/placeins
% Defines a \FloatBarrier command
% TODO: Check if really unused?
% \usepackage{placeins}


% framed – Framed or shaded regions that can break across pages
% https://ctan.org/pkg/framed
% Provides \begin{framed} which uses \colorbox{shadecolor} relying on \definecolor{shadecolor}.
% TODO: Check if really unused?
% \usepackage{framed}

% Used often in conjunction with \definecolor{shadecolor}{rgb}{0.969, 0.969, 0.969}
% Might be able to be removed or at least redefined to only have shadecolor (if needed)
\definecolor{fgcolor}{rgb}{0.345, 0.345, 0.345}
\definecolor{shadecolor}{rgb}{0.969, 0.969, 0.969}
\newenvironment{knitrout}{}{} % an empty environment to be redefined in TeX


% Defines macros and environments
\usepackage{../../style/lmu-lecture}

\let\code=\texttt % Used regularly
\let\proglang=\textsf % Unused?

% Not sure what/why this does
\setkeys{Gin}{width=0.9\textwidth}

\setbeamertemplate{frametitle}{\expandafter\uppercase\expandafter\insertframetitle}

% Can't find a reason why common.tex is not just part of this file?

% basic latex stuff
\newcommand{\pkg}[1]{{\fontseries{b}\selectfont #1}} %fontstyle for R packages
\newcommand{\lz}{\vspace{0.5cm}} %vertical space
\newcommand{\dlz}{\vspace{1cm}} %double vertical space
\newcommand{\oneliner}[1] % Oneliner for important statements
{\begin{block}{}\begin{center}\begin{Large}#1\end{Large}\end{center}\end{block}}


%new environments
\newenvironment{vbframe}  %frame with breaks and verbatim
{
 \begin{frame}[containsverbatim,allowframebreaks]
}
{
\end{frame}
}

\newenvironment{vframe}  %frame with verbatim without breaks (to avoid numbering one slided frames)
{
 \begin{frame}[containsverbatim]
}
{
\end{frame}
}

\newenvironment{blocki}[1]   % itemize block
{
 \begin{block}{#1}\begin{itemize}
}
{
\end{itemize}\end{block}
}

\newenvironment{fragileframe}[2]{  %fragile frame with framebreaks
\begin{frame}[allowframebreaks, fragile, environment = fragileframe]
\frametitle{#1}
#2}
{\end{frame}}


\newcommand{\myframe}[2]{  %short for frame with framebreaks
\begin{frame}[allowframebreaks]
\frametitle{#1}
#2
\end{frame}}

\newcommand{\remark}[1]{
  \textbf{Remark:} #1
}


\newenvironment{deleteframe}
{
\begingroup
\usebackgroundtemplate{\includegraphics[width=\paperwidth,height=\paperheight]{../style/color/red.png}}
 \begin{frame}
}
{
\end{frame}
\endgroup
}
\newenvironment{simplifyframe}
{
\begingroup
\usebackgroundtemplate{\includegraphics[width=\paperwidth,height=\paperheight]{../style/color/yellow.png}}
 \begin{frame}
}
{
\end{frame}
\endgroup
}\newenvironment{draftframe}
{
\begingroup
\usebackgroundtemplate{\includegraphics[width=\paperwidth,height=\paperheight]{../style/color/green.jpg}}
 \begin{frame}
}
{
\end{frame}
\endgroup
}
% https://tex.stackexchange.com/a/261480: textcolor that works in mathmode
\makeatletter
\renewcommand*{\@textcolor}[3]{%
  \protect\leavevmode
  \begingroup
    \color#1{#2}#3%
  \endgroup
}
\makeatother


%-------------------------------------------------------------------------------------------------------%
%  Unused stuff that needs to go but is kept here currently juuuust in case it was important after all  %
%-------------------------------------------------------------------------------------------------------%

% \newcommand{\hlnum}[1]{\textcolor[rgb]{0.686,0.059,0.569}{#1}}%
% \newcommand{\hlstr}[1]{\textcolor[rgb]{0.192,0.494,0.8}{#1}}%
% \newcommand{\hlcom}[1]{\textcolor[rgb]{0.678,0.584,0.686}{\textit{#1}}}%
% \newcommand{\hlopt}[1]{\textcolor[rgb]{0,0,0}{#1}}%
% \newcommand{\hlstd}[1]{\textcolor[rgb]{0.345,0.345,0.345}{#1}}%
% \newcommand{\hlkwa}[1]{\textcolor[rgb]{0.161,0.373,0.58}{\textbf{#1}}}%
% \newcommand{\hlkwb}[1]{\textcolor[rgb]{0.69,0.353,0.396}{#1}}%
% \newcommand{\hlkwc}[1]{\textcolor[rgb]{0.333,0.667,0.333}{#1}}%
% \newcommand{\hlkwd}[1]{\textcolor[rgb]{0.737,0.353,0.396}{\textbf{#1}}}%
% \let\hlipl\hlkwb

% \makeatletter
% \newenvironment{kframe}{%
%  \def\at@end@of@kframe{}%
%  \ifinner\ifhmode%
%   \def\at@end@of@kframe{\end{minipage}}%
%   \begin{minipage}{\columnwidth}%
%  \fi\fi%
%  \def\FrameCommand##1{\hskip\@totalleftmargin \hskip-\fboxsep
%  \colorbox{shadecolor}{##1}\hskip-\fboxsep
%      % There is no \\@totalrightmargin, so:
%      \hskip-\linewidth \hskip-\@totalleftmargin \hskip\columnwidth}%
%  \MakeFramed {\advance\hsize-\width
%    \@totalleftmargin\z@ \linewidth\hsize
%    \@setminipage}}%
%  {\par\unskip\endMakeFramed%
%  \at@end@of@kframe}
% \makeatother

% \definecolor{shadecolor}{rgb}{.97, .97, .97}
% \definecolor{messagecolor}{rgb}{0, 0, 0}
% \definecolor{warningcolor}{rgb}{1, 0, 1}
% \definecolor{errorcolor}{rgb}{1, 0, 0}
% \newenvironment{knitrout}{}{} % an empty environment to be redefined in TeX

% \usepackage{alltt}
% \newcommand{\SweaveOpts}[1]{}  % do not interfere with LaTeX
% \newcommand{\SweaveInput}[1]{} % because they are not real TeX commands
% \newcommand{\Sexpr}[1]{}       % will only be parsed by R
% \newcommand{\xmark}{\ding{55}}%

% ml - trees, extra trees

\newcommand{\Np}{\mathcal{N}}												% Parent node N
\newcommand{\Nl}{\Np_1}														% Left node N_1
\newcommand{\Nr}{\Np_2}														% Right node N_2


\title{Introduction to Machine Learning}

\begin{document}

\titlemeta{% Chunk title (example: CART, Forests, Boosting, ...), can be empty
  CART 
  }{% Lecture title  
  Computational Aspects of Finding Splits
  }{% Relative path to title page image: Can be empty but must not start with slides/
  figure/categoryplot-binarysmall
  }{
  \item Know how monotone feature transformations affect the tree
  \item Understand how categorical features can be treated effectively while growing a CART
  \item Understand how missing values can be treated in a CART
}

\begin{framev}{Monotone feature transformations}

Monotone transformations of one or several features will neither change the value of the splitting criterion nor the structure of the tree,  only the numerical value of the split point.
\vspace{0.5cm}
\begin{columns}[T]
\column{0.49\textwidth}
Original data
\begin{knitrout}\scriptsize
\definecolor{shadecolor}{rgb}{0.969, 0.969, 0.969}\color{fgcolor}
\begin{tabular}{l|r|r|r|r|r}
\hline
x & 1.0 & 2.0 & 7.0 & 10.0 & 20.0\\
\hline
y & 1.0 & 1.0 & 0.5 & 9.0 & 11.0\\
\hline
\end{tabular}


\end{knitrout}
% FIGURE SOURCE: Use picture created in rsrc/monotone_trafo.R
\includegraphics[width = \textwidth]{figure/cart_splitcomp_1}
\column{0.49\textwidth}
Data with log-transformed $x$
\begin{knitrout}\scriptsize
\definecolor{shadecolor}{rgb}{0.969, 0.969, 0.969}\color{fgcolor}
\begin{tabular}{l|r|r|r|r|r}
\hline
log(x) & 0.0 & 0.7 & 1.9 & 2.3 & 3.0\\
\hline
y & 1.0 & 1.0 & 0.5 & 9.0 & 11.0\\
\hline
\end{tabular}


\end{knitrout}
% FIGURE SOURCE: Use picture created in rsrc/monotone_trafo.R
\includegraphics[width = \textwidth]{figure/cart_splitcomp_2}
\end{columns}
\vspace{0.5cm}
\centering
\end{framev}

\begin{framev}{Categorical Features}
  \begin{itemize}
  \item A split on a categorical feature partitions the feature levels:
    $$x_j \in \{a,b,c\} \leftarrow \Np \rightarrow x_j \in \{d,e\} $$
  \end{itemize}
  \begin{figure}
   \includegraphics[width=0.8\textwidth]{figure/tree-categorical.pdf} 
  \end{figure}
  \end{framev}
  
  \begin{framev}{Categorical Features}
  \begin{itemize}
  \item A split on a categorical feature partitions the feature levels:
    $$x_j \in \{a,b,c\} \leftarrow \Np \rightarrow x_j \in \{d,e\} $$
  \item For a feature with $m$ levels,
  there are about $2^m$ different possible partitions of the $m$ values into two groups\\ ($2^{m-1} - 1$ because of symmetry and empty groups).
  \item Searching over all these becomes prohibitive for large values of $m$.
  \item For regression with L2 loss and for binary classification, we can define clever shortcuts.
  \end{itemize}

  \end{framev}
  
  \begin{frame}{Categorical Features}

For $0-1$ responses, in each node:
  \begin{enumerate}
  \item Calculate the proportion of $1$-outcomes for each category of the feature in $\Np$.

  \end{enumerate}
  \begin{columns}
  \begin{column}{0.33\textwidth}
  \begin{figure}
  \includegraphics[width=0.8\textwidth]{figure/categoryplot-binary1.pdf} 
  \end{figure}
  \end{column}
  \begin{column}{0.33\textwidth}
  \lz
  \end{column}
  \begin{column}{0.33\textwidth}
  \lz
  \end{column}
  \end{columns}

\end{frame}

  \begin{frame}[noframenumbering]{Categorical Features}

For $0-1$ responses, in each node:
  \begin{enumerate}
  \item Calculate the proportion of $1$-outcomes for each category of the feature in $\Np$.
  \item Sort the categories according to these proportions.
  \end{enumerate}
  \begin{columns}
  \begin{column}{0.33\textwidth}
  \begin{figure}
  \includegraphics[width=0.8\textwidth]{figure/categoryplot-binary1.pdf} 
  \end{figure}
  \end{column}
  \begin{column}{0.33\textwidth}
  \begin{figure}
  \includegraphics[width=0.8\textwidth]{figure/categoryplot-binary2.pdf} 
  \end{figure}
  \end{column}
  \begin{column}{0.33\textwidth}
  \end{column}
  \end{columns}

\end{frame}

  \begin{frame}[noframenumbering]{Categorical Features}

For $0-1$ responses, in each node:
  \begin{enumerate}
  \item Calculate the proportion of $1$-outcomes for each category of the feature in $\Np$.
  \item Sort the categories according to these proportions.
  \item The feature can then be treated as if it was ordinal, so we only have to investigate at most $m-1$ splits.
  \end{enumerate}
  \begin{columns}
  \begin{column}{0.33\textwidth}
  \begin{figure}
  \includegraphics[width=0.8\textwidth]{figure/categoryplot-binary1.pdf} 
  \end{figure}
  \end{column}
  \begin{column}{0.33\textwidth}
  \begin{figure}
  \includegraphics[width=0.8\textwidth]{figure/categoryplot-binary2.pdf} 
  \end{figure}
  \end{column}
  \begin{column}{0.33\textwidth}
  \begin{figure}
  \includegraphics[width=0.8\textwidth]{figure/categoryplot-binary3.pdf} 
  \end{figure}
  \end{column}
  \end{columns}

\end{frame}


\begin{framev}{Categorical Features}

  \begin{itemize}
  \item This procedure finds the optimal split.
  \item This result also holds for regression trees (with L2 loss) if the levels of the feature are ordered by increasing mean of the target
  \item The proofs are not trivial and can be found here:
    \begin{itemize}
    \item for 0-1 responses:
      \begin{itemize}
      \item \furtherreading{BREIMAN1984}
      \item  \furtherreading{RIPLEY1996}
      \end{itemize}
    \item for continuous responses:
      \begin{itemize}
      \item \furtherreading{FISHER1958}
      \end{itemize}
    \end{itemize}
  \item There are only heuristics for the multiclass case \furtherreading{WRIGHT2019}
  %\item The Algorithm prefers categorical variables with a large value
  %of categories $Q$
  \end{itemize}

\end{framev}

\begin{framev}{Categorical Features}

For continuous responses, in each node:
  \begin{enumerate}
  \item Calculate the mean of the outcome in each category
  \item Sort the categories by increasing mean of the outcome
  \end{enumerate}

  \begin{columns}
  \begin{column}{0.33\textwidth}
  \begin{figure}
  \includegraphics[width=0.8\textwidth]{figure/categoryplot-cont1.pdf} 
  \end{figure}
  \end{column}
  \begin{column}{0.33\textwidth}
  \begin{figure}
  \includegraphics[width=0.8\textwidth]{figure/categoryplot-cont2.pdf} 
  \end{figure}
  \end{column}
  \begin{column}{0.33\textwidth}
  \begin{figure}
  \includegraphics[width=0.8\textwidth]{figure/categoryplot-cont3.pdf} 
  \end{figure}
  \end{column}
  \end{columns}


\end{framev}

\begin{framev}{Missing feature values}
  \begin{itemize}
    \item When splits are evaluated, only observations for which the used feature is not missing are used. (This can actually bias splits towards using features with lots of missing values.) 
  \item \textbf{Surrogate splits} can deal with missing values during prediction.
  \item Surrogate splits are created during training. They define replacement splitting rules, using a different feature, that result in almost the same child nodes as the original split.
   \item When observations are passed down the tree, % (in training or prediction), 
   and the feature value used in a split is missing, we use the surrogate split instead to decide to which child the data should be assigned. 
  \end{itemize}
\end{framev}

\begin{framev}{Surrogate Splits}
\begin{itemize}
\item Each surrogate split is a decision stump that tries to learn the actual splitting rule
\item Consider this tree with the primary split w.r.t. \texttt{Sepal.Length} where we perform binary classification (\texttt{setosa} vs. \texttt{virginica}):
\begin{figure}
\includegraphics[width=0.75\textwidth]{figure/tree-binary.pdf} 
\end{figure}
\item Our surrogate split should optimize a splitting criterion w.r.t. \texttt{Sepal.Length < 5.8}
\end{itemize}



\end{framev}

\begin{framev}{Surrogate Splits}
\begin{itemize}
\item Consider this subsample of the data used to fit the tree:
\begin{table}[ht]
\tiny
\centering
\begin{tabular}{rrrrrll}
  \hline
 & Sepal.Length & ... & Petal.Width & Species & Sepal.Length $<$ 5.8 \\ 
  \hline
1 & 5.10 & ... & 0.20 & setosa & TRUE \\ 
  4 & 4.60 & ... & 0.20 & setosa & TRUE \\ 
  9 & 4.40 & ... & 0.20 & setosa & TRUE \\ 
  15 & 5.80 & ... & 0.20 & setosa & FALSE \\ 
  18 & 5.10 & ... & 0.30 & setosa & TRUE \\ 
  52 & 5.80 & ... & 1.90 & virginica & FALSE \\ 
  57 & 4.90 & ... & 1.70 & virginica & TRUE \\ 
  62 & 6.40 & ... & 1.90 & virginica & FALSE \\ 
  77 & 6.20 & ... & 1.80 & virginica & FALSE \\ 
  99 & 6.20 & ... & 2.30 & virginica & FALSE \\ 
   \hline
\end{tabular}
\end{table}
\item Add column that indicates whether \texttt{Sepal.Length < 5.8}
%\item As this splitting rule is very good, we will have many instances where \texttt{Sepal.Length < 5.8} is \texttt{TRUE} and \texttt{Species} is \texttt{setosa}
\item Fit tree of depth 1 using all features but \texttt{Sepal.Length} %used 
to derive a split that explains 
\texttt{Sepal.Length < 5.8} best $\Rightarrow$ surrogate split
\item Typically, software stores the best and a few more surrogate splits
%\item A good surrogate tries to mimic the primary split this way

\end{itemize}


\end{framev}


\endlecture
\end{document}
