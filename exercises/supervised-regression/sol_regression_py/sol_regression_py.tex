\documentclass[11pt]{article}

    \usepackage[breakable]{tcolorbox}
    \usepackage{parskip} % Stop auto-indenting (to mimic markdown behaviour)
    

    % Basic figure setup, for now with no caption control since it's done
    % automatically by Pandoc (which extracts ![](path) syntax from Markdown).
    \usepackage{graphicx}
    % Maintain compatibility with old templates. Remove in nbconvert 6.0
    \let\Oldincludegraphics\includegraphics
    % Ensure that by default, figures have no caption (until we provide a
    % proper Figure object with a Caption API and a way to capture that
    % in the conversion process - todo).
    \usepackage{caption}
    \DeclareCaptionFormat{nocaption}{}
    \captionsetup{format=nocaption,aboveskip=0pt,belowskip=0pt}

    \usepackage{float}
    \floatplacement{figure}{H} % forces figures to be placed at the correct location
    \usepackage{xcolor} % Allow colors to be defined
    \usepackage{enumerate} % Needed for markdown enumerations to work
    \usepackage{geometry} % Used to adjust the document margins
    \usepackage{amsmath} % Equations
    \usepackage{amssymb} % Equations
    \usepackage{textcomp} % defines textquotesingle
    % Hack from http://tex.stackexchange.com/a/47451/13684:
    \AtBeginDocument{%
        \def\PYZsq{\textquotesingle}% Upright quotes in Pygmentized code
    }
    \usepackage{upquote} % Upright quotes for verbatim code
    \usepackage{eurosym} % defines \euro

    \usepackage{iftex}
    \ifPDFTeX
        \usepackage[T1]{fontenc}
        \IfFileExists{alphabeta.sty}{
              \usepackage{alphabeta}
          }{
              \usepackage[mathletters]{ucs}
              \usepackage[utf8x]{inputenc}
          }
    \else
        \usepackage{fontspec}
        \usepackage{unicode-math}
    \fi

    \usepackage{fancyvrb} % verbatim replacement that allows latex
    \usepackage{grffile} % extends the file name processing of package graphics
                         % to support a larger range
    \makeatletter % fix for old versions of grffile with XeLaTeX
    \@ifpackagelater{grffile}{2019/11/01}
    {
      % Do nothing on new versions
    }
    {
      \def\Gread@@xetex#1{%
        \IfFileExists{"\Gin@base".bb}%
        {\Gread@eps{\Gin@base.bb}}%
        {\Gread@@xetex@aux#1}%
      }
    }
    \makeatother
    \usepackage[Export]{adjustbox} % Used to constrain images to a maximum size
    \adjustboxset{max size={0.9\linewidth}{0.9\paperheight}}

    % The hyperref package gives us a pdf with properly built
    % internal navigation ('pdf bookmarks' for the table of contents,
    % internal cross-reference links, web links for URLs, etc.)
    \usepackage{hyperref}
    % The default LaTeX title has an obnoxious amount of whitespace. By default,
    % titling removes some of it. It also provides customization options.
    \usepackage{titling}
    \usepackage{longtable} % longtable support required by pandoc >1.10
    \usepackage{booktabs}  % table support for pandoc > 1.12.2
    \usepackage{array}     % table support for pandoc >= 2.11.3
    \usepackage{calc}      % table minipage width calculation for pandoc >= 2.11.1
    \usepackage[inline]{enumitem} % IRkernel/repr support (it uses the enumerate* environment)
    \usepackage[normalem]{ulem} % ulem is needed to support strikethroughs (\sout)
                                % normalem makes italics be italics, not underlines
    \usepackage{mathrsfs}
    

    
    % Colors for the hyperref package
    \definecolor{urlcolor}{rgb}{0,.145,.698}
    \definecolor{linkcolor}{rgb}{.71,0.21,0.01}
    \definecolor{citecolor}{rgb}{.12,.54,.11}

    % ANSI colors
    \definecolor{ansi-black}{HTML}{3E424D}
    \definecolor{ansi-black-intense}{HTML}{282C36}
    \definecolor{ansi-red}{HTML}{E75C58}
    \definecolor{ansi-red-intense}{HTML}{B22B31}
    \definecolor{ansi-green}{HTML}{00A250}
    \definecolor{ansi-green-intense}{HTML}{007427}
    \definecolor{ansi-yellow}{HTML}{DDB62B}
    \definecolor{ansi-yellow-intense}{HTML}{B27D12}
    \definecolor{ansi-blue}{HTML}{208FFB}
    \definecolor{ansi-blue-intense}{HTML}{0065CA}
    \definecolor{ansi-magenta}{HTML}{D160C4}
    \definecolor{ansi-magenta-intense}{HTML}{A03196}
    \definecolor{ansi-cyan}{HTML}{60C6C8}
    \definecolor{ansi-cyan-intense}{HTML}{258F8F}
    \definecolor{ansi-white}{HTML}{C5C1B4}
    \definecolor{ansi-white-intense}{HTML}{A1A6B2}
    \definecolor{ansi-default-inverse-fg}{HTML}{FFFFFF}
    \definecolor{ansi-default-inverse-bg}{HTML}{000000}

    % common color for the border for error outputs.
    \definecolor{outerrorbackground}{HTML}{FFDFDF}

    % commands and environments needed by pandoc snippets
    % extracted from the output of `pandoc -s`
    \providecommand{\tightlist}{%
      \setlength{\itemsep}{0pt}\setlength{\parskip}{0pt}}
    \DefineVerbatimEnvironment{Highlighting}{Verbatim}{commandchars=\\\{\}}
    % Add ',fontsize=\small' for more characters per line
    \newenvironment{Shaded}{}{}
    \newcommand{\KeywordTok}[1]{\textcolor[rgb]{0.00,0.44,0.13}{\textbf{{#1}}}}
    \newcommand{\DataTypeTok}[1]{\textcolor[rgb]{0.56,0.13,0.00}{{#1}}}
    \newcommand{\DecValTok}[1]{\textcolor[rgb]{0.25,0.63,0.44}{{#1}}}
    \newcommand{\BaseNTok}[1]{\textcolor[rgb]{0.25,0.63,0.44}{{#1}}}
    \newcommand{\FloatTok}[1]{\textcolor[rgb]{0.25,0.63,0.44}{{#1}}}
    \newcommand{\CharTok}[1]{\textcolor[rgb]{0.25,0.44,0.63}{{#1}}}
    \newcommand{\StringTok}[1]{\textcolor[rgb]{0.25,0.44,0.63}{{#1}}}
    \newcommand{\CommentTok}[1]{\textcolor[rgb]{0.38,0.63,0.69}{\textit{{#1}}}}
    \newcommand{\OtherTok}[1]{\textcolor[rgb]{0.00,0.44,0.13}{{#1}}}
    \newcommand{\AlertTok}[1]{\textcolor[rgb]{1.00,0.00,0.00}{\textbf{{#1}}}}
    \newcommand{\FunctionTok}[1]{\textcolor[rgb]{0.02,0.16,0.49}{{#1}}}
    \newcommand{\RegionMarkerTok}[1]{{#1}}
    \newcommand{\ErrorTok}[1]{\textcolor[rgb]{1.00,0.00,0.00}{\textbf{{#1}}}}
    \newcommand{\NormalTok}[1]{{#1}}

    % Additional commands for more recent versions of Pandoc
    \newcommand{\ConstantTok}[1]{\textcolor[rgb]{0.53,0.00,0.00}{{#1}}}
    \newcommand{\SpecialCharTok}[1]{\textcolor[rgb]{0.25,0.44,0.63}{{#1}}}
    \newcommand{\VerbatimStringTok}[1]{\textcolor[rgb]{0.25,0.44,0.63}{{#1}}}
    \newcommand{\SpecialStringTok}[1]{\textcolor[rgb]{0.73,0.40,0.53}{{#1}}}
    \newcommand{\ImportTok}[1]{{#1}}
    \newcommand{\DocumentationTok}[1]{\textcolor[rgb]{0.73,0.13,0.13}{\textit{{#1}}}}
    \newcommand{\AnnotationTok}[1]{\textcolor[rgb]{0.38,0.63,0.69}{\textbf{\textit{{#1}}}}}
    \newcommand{\CommentVarTok}[1]{\textcolor[rgb]{0.38,0.63,0.69}{\textbf{\textit{{#1}}}}}
    \newcommand{\VariableTok}[1]{\textcolor[rgb]{0.10,0.09,0.49}{{#1}}}
    \newcommand{\ControlFlowTok}[1]{\textcolor[rgb]{0.00,0.44,0.13}{\textbf{{#1}}}}
    \newcommand{\OperatorTok}[1]{\textcolor[rgb]{0.40,0.40,0.40}{{#1}}}
    \newcommand{\BuiltInTok}[1]{{#1}}
    \newcommand{\ExtensionTok}[1]{{#1}}
    \newcommand{\PreprocessorTok}[1]{\textcolor[rgb]{0.74,0.48,0.00}{{#1}}}
    \newcommand{\AttributeTok}[1]{\textcolor[rgb]{0.49,0.56,0.16}{{#1}}}
    \newcommand{\InformationTok}[1]{\textcolor[rgb]{0.38,0.63,0.69}{\textbf{\textit{{#1}}}}}
    \newcommand{\WarningTok}[1]{\textcolor[rgb]{0.38,0.63,0.69}{\textbf{\textit{{#1}}}}}


    % Define a nice break command that doesn't care if a line doesn't already
    % exist.
    \def\br{\hspace*{\fill} \\* }
    % Math Jax compatibility definitions
    \def\gt{>}
    \def\lt{<}
    \let\Oldtex\TeX
    \let\Oldlatex\LaTeX
    \renewcommand{\TeX}{\textrm{\Oldtex}}
    \renewcommand{\LaTeX}{\textrm{\Oldlatex}}
    % Document parameters
    % Document title
    \title{Exercise\_2\_Regression}
    
    
    
    
    
% Pygments definitions
\makeatletter
\def\PY@reset{\let\PY@it=\relax \let\PY@bf=\relax%
    \let\PY@ul=\relax \let\PY@tc=\relax%
    \let\PY@bc=\relax \let\PY@ff=\relax}
\def\PY@tok#1{\csname PY@tok@#1\endcsname}
\def\PY@toks#1+{\ifx\relax#1\empty\else%
    \PY@tok{#1}\expandafter\PY@toks\fi}
\def\PY@do#1{\PY@bc{\PY@tc{\PY@ul{%
    \PY@it{\PY@bf{\PY@ff{#1}}}}}}}
\def\PY#1#2{\PY@reset\PY@toks#1+\relax+\PY@do{#2}}

\@namedef{PY@tok@w}{\def\PY@tc##1{\textcolor[rgb]{0.73,0.73,0.73}{##1}}}
\@namedef{PY@tok@c}{\let\PY@it=\textit\def\PY@tc##1{\textcolor[rgb]{0.24,0.48,0.48}{##1}}}
\@namedef{PY@tok@cp}{\def\PY@tc##1{\textcolor[rgb]{0.61,0.40,0.00}{##1}}}
\@namedef{PY@tok@k}{\let\PY@bf=\textbf\def\PY@tc##1{\textcolor[rgb]{0.00,0.50,0.00}{##1}}}
\@namedef{PY@tok@kp}{\def\PY@tc##1{\textcolor[rgb]{0.00,0.50,0.00}{##1}}}
\@namedef{PY@tok@kt}{\def\PY@tc##1{\textcolor[rgb]{0.69,0.00,0.25}{##1}}}
\@namedef{PY@tok@o}{\def\PY@tc##1{\textcolor[rgb]{0.40,0.40,0.40}{##1}}}
\@namedef{PY@tok@ow}{\let\PY@bf=\textbf\def\PY@tc##1{\textcolor[rgb]{0.67,0.13,1.00}{##1}}}
\@namedef{PY@tok@nb}{\def\PY@tc##1{\textcolor[rgb]{0.00,0.50,0.00}{##1}}}
\@namedef{PY@tok@nf}{\def\PY@tc##1{\textcolor[rgb]{0.00,0.00,1.00}{##1}}}
\@namedef{PY@tok@nc}{\let\PY@bf=\textbf\def\PY@tc##1{\textcolor[rgb]{0.00,0.00,1.00}{##1}}}
\@namedef{PY@tok@nn}{\let\PY@bf=\textbf\def\PY@tc##1{\textcolor[rgb]{0.00,0.00,1.00}{##1}}}
\@namedef{PY@tok@ne}{\let\PY@bf=\textbf\def\PY@tc##1{\textcolor[rgb]{0.80,0.25,0.22}{##1}}}
\@namedef{PY@tok@nv}{\def\PY@tc##1{\textcolor[rgb]{0.10,0.09,0.49}{##1}}}
\@namedef{PY@tok@no}{\def\PY@tc##1{\textcolor[rgb]{0.53,0.00,0.00}{##1}}}
\@namedef{PY@tok@nl}{\def\PY@tc##1{\textcolor[rgb]{0.46,0.46,0.00}{##1}}}
\@namedef{PY@tok@ni}{\let\PY@bf=\textbf\def\PY@tc##1{\textcolor[rgb]{0.44,0.44,0.44}{##1}}}
\@namedef{PY@tok@na}{\def\PY@tc##1{\textcolor[rgb]{0.41,0.47,0.13}{##1}}}
\@namedef{PY@tok@nt}{\let\PY@bf=\textbf\def\PY@tc##1{\textcolor[rgb]{0.00,0.50,0.00}{##1}}}
\@namedef{PY@tok@nd}{\def\PY@tc##1{\textcolor[rgb]{0.67,0.13,1.00}{##1}}}
\@namedef{PY@tok@s}{\def\PY@tc##1{\textcolor[rgb]{0.73,0.13,0.13}{##1}}}
\@namedef{PY@tok@sd}{\let\PY@it=\textit\def\PY@tc##1{\textcolor[rgb]{0.73,0.13,0.13}{##1}}}
\@namedef{PY@tok@si}{\let\PY@bf=\textbf\def\PY@tc##1{\textcolor[rgb]{0.64,0.35,0.47}{##1}}}
\@namedef{PY@tok@se}{\let\PY@bf=\textbf\def\PY@tc##1{\textcolor[rgb]{0.67,0.36,0.12}{##1}}}
\@namedef{PY@tok@sr}{\def\PY@tc##1{\textcolor[rgb]{0.64,0.35,0.47}{##1}}}
\@namedef{PY@tok@ss}{\def\PY@tc##1{\textcolor[rgb]{0.10,0.09,0.49}{##1}}}
\@namedef{PY@tok@sx}{\def\PY@tc##1{\textcolor[rgb]{0.00,0.50,0.00}{##1}}}
\@namedef{PY@tok@m}{\def\PY@tc##1{\textcolor[rgb]{0.40,0.40,0.40}{##1}}}
\@namedef{PY@tok@gh}{\let\PY@bf=\textbf\def\PY@tc##1{\textcolor[rgb]{0.00,0.00,0.50}{##1}}}
\@namedef{PY@tok@gu}{\let\PY@bf=\textbf\def\PY@tc##1{\textcolor[rgb]{0.50,0.00,0.50}{##1}}}
\@namedef{PY@tok@gd}{\def\PY@tc##1{\textcolor[rgb]{0.63,0.00,0.00}{##1}}}
\@namedef{PY@tok@gi}{\def\PY@tc##1{\textcolor[rgb]{0.00,0.52,0.00}{##1}}}
\@namedef{PY@tok@gr}{\def\PY@tc##1{\textcolor[rgb]{0.89,0.00,0.00}{##1}}}
\@namedef{PY@tok@ge}{\let\PY@it=\textit}
\@namedef{PY@tok@gs}{\let\PY@bf=\textbf}
\@namedef{PY@tok@gp}{\let\PY@bf=\textbf\def\PY@tc##1{\textcolor[rgb]{0.00,0.00,0.50}{##1}}}
\@namedef{PY@tok@go}{\def\PY@tc##1{\textcolor[rgb]{0.44,0.44,0.44}{##1}}}
\@namedef{PY@tok@gt}{\def\PY@tc##1{\textcolor[rgb]{0.00,0.27,0.87}{##1}}}
\@namedef{PY@tok@err}{\def\PY@bc##1{{\setlength{\fboxsep}{\string -\fboxrule}\fcolorbox[rgb]{1.00,0.00,0.00}{1,1,1}{\strut ##1}}}}
\@namedef{PY@tok@kc}{\let\PY@bf=\textbf\def\PY@tc##1{\textcolor[rgb]{0.00,0.50,0.00}{##1}}}
\@namedef{PY@tok@kd}{\let\PY@bf=\textbf\def\PY@tc##1{\textcolor[rgb]{0.00,0.50,0.00}{##1}}}
\@namedef{PY@tok@kn}{\let\PY@bf=\textbf\def\PY@tc##1{\textcolor[rgb]{0.00,0.50,0.00}{##1}}}
\@namedef{PY@tok@kr}{\let\PY@bf=\textbf\def\PY@tc##1{\textcolor[rgb]{0.00,0.50,0.00}{##1}}}
\@namedef{PY@tok@bp}{\def\PY@tc##1{\textcolor[rgb]{0.00,0.50,0.00}{##1}}}
\@namedef{PY@tok@fm}{\def\PY@tc##1{\textcolor[rgb]{0.00,0.00,1.00}{##1}}}
\@namedef{PY@tok@vc}{\def\PY@tc##1{\textcolor[rgb]{0.10,0.09,0.49}{##1}}}
\@namedef{PY@tok@vg}{\def\PY@tc##1{\textcolor[rgb]{0.10,0.09,0.49}{##1}}}
\@namedef{PY@tok@vi}{\def\PY@tc##1{\textcolor[rgb]{0.10,0.09,0.49}{##1}}}
\@namedef{PY@tok@vm}{\def\PY@tc##1{\textcolor[rgb]{0.10,0.09,0.49}{##1}}}
\@namedef{PY@tok@sa}{\def\PY@tc##1{\textcolor[rgb]{0.73,0.13,0.13}{##1}}}
\@namedef{PY@tok@sb}{\def\PY@tc##1{\textcolor[rgb]{0.73,0.13,0.13}{##1}}}
\@namedef{PY@tok@sc}{\def\PY@tc##1{\textcolor[rgb]{0.73,0.13,0.13}{##1}}}
\@namedef{PY@tok@dl}{\def\PY@tc##1{\textcolor[rgb]{0.73,0.13,0.13}{##1}}}
\@namedef{PY@tok@s2}{\def\PY@tc##1{\textcolor[rgb]{0.73,0.13,0.13}{##1}}}
\@namedef{PY@tok@sh}{\def\PY@tc##1{\textcolor[rgb]{0.73,0.13,0.13}{##1}}}
\@namedef{PY@tok@s1}{\def\PY@tc##1{\textcolor[rgb]{0.73,0.13,0.13}{##1}}}
\@namedef{PY@tok@mb}{\def\PY@tc##1{\textcolor[rgb]{0.40,0.40,0.40}{##1}}}
\@namedef{PY@tok@mf}{\def\PY@tc##1{\textcolor[rgb]{0.40,0.40,0.40}{##1}}}
\@namedef{PY@tok@mh}{\def\PY@tc##1{\textcolor[rgb]{0.40,0.40,0.40}{##1}}}
\@namedef{PY@tok@mi}{\def\PY@tc##1{\textcolor[rgb]{0.40,0.40,0.40}{##1}}}
\@namedef{PY@tok@il}{\def\PY@tc##1{\textcolor[rgb]{0.40,0.40,0.40}{##1}}}
\@namedef{PY@tok@mo}{\def\PY@tc##1{\textcolor[rgb]{0.40,0.40,0.40}{##1}}}
\@namedef{PY@tok@ch}{\let\PY@it=\textit\def\PY@tc##1{\textcolor[rgb]{0.24,0.48,0.48}{##1}}}
\@namedef{PY@tok@cm}{\let\PY@it=\textit\def\PY@tc##1{\textcolor[rgb]{0.24,0.48,0.48}{##1}}}
\@namedef{PY@tok@cpf}{\let\PY@it=\textit\def\PY@tc##1{\textcolor[rgb]{0.24,0.48,0.48}{##1}}}
\@namedef{PY@tok@c1}{\let\PY@it=\textit\def\PY@tc##1{\textcolor[rgb]{0.24,0.48,0.48}{##1}}}
\@namedef{PY@tok@cs}{\let\PY@it=\textit\def\PY@tc##1{\textcolor[rgb]{0.24,0.48,0.48}{##1}}}

\def\PYZbs{\char`\\}
\def\PYZus{\char`\_}
\def\PYZob{\char`\{}
\def\PYZcb{\char`\}}
\def\PYZca{\char`\^}
\def\PYZam{\char`\&}
\def\PYZlt{\char`\<}
\def\PYZgt{\char`\>}
\def\PYZsh{\char`\#}
\def\PYZpc{\char`\%}
\def\PYZdl{\char`\$}
\def\PYZhy{\char`\-}
\def\PYZsq{\char`\'}
\def\PYZdq{\char`\"}
\def\PYZti{\char`\~}
% for compatibility with earlier versions
\def\PYZat{@}
\def\PYZlb{[}
\def\PYZrb{]}
\makeatother


    % For linebreaks inside Verbatim environment from package fancyvrb.
    \makeatletter
        \newbox\Wrappedcontinuationbox
        \newbox\Wrappedvisiblespacebox
        \newcommand*\Wrappedvisiblespace {\textcolor{red}{\textvisiblespace}}
        \newcommand*\Wrappedcontinuationsymbol {\textcolor{red}{\llap{\tiny$\m@th\hookrightarrow$}}}
        \newcommand*\Wrappedcontinuationindent {3ex }
        \newcommand*\Wrappedafterbreak {\kern\Wrappedcontinuationindent\copy\Wrappedcontinuationbox}
        % Take advantage of the already applied Pygments mark-up to insert
        % potential linebreaks for TeX processing.
        %        {, <, #, %, $, ' and ": go to next line.
        %        _, }, ^, &, >, - and ~: stay at end of broken line.
        % Use of \textquotesingle for straight quote.
        \newcommand*\Wrappedbreaksatspecials {%
            \def\PYGZus{\discretionary{\char`\_}{\Wrappedafterbreak}{\char`\_}}%
            \def\PYGZob{\discretionary{}{\Wrappedafterbreak\char`\{}{\char`\{}}%
            \def\PYGZcb{\discretionary{\char`\}}{\Wrappedafterbreak}{\char`\}}}%
            \def\PYGZca{\discretionary{\char`\^}{\Wrappedafterbreak}{\char`\^}}%
            \def\PYGZam{\discretionary{\char`\&}{\Wrappedafterbreak}{\char`\&}}%
            \def\PYGZlt{\discretionary{}{\Wrappedafterbreak\char`\<}{\char`\<}}%
            \def\PYGZgt{\discretionary{\char`\>}{\Wrappedafterbreak}{\char`\>}}%
            \def\PYGZsh{\discretionary{}{\Wrappedafterbreak\char`\#}{\char`\#}}%
            \def\PYGZpc{\discretionary{}{\Wrappedafterbreak\char`\%}{\char`\%}}%
            \def\PYGZdl{\discretionary{}{\Wrappedafterbreak\char`\$}{\char`\$}}%
            \def\PYGZhy{\discretionary{\char`\-}{\Wrappedafterbreak}{\char`\-}}%
            \def\PYGZsq{\discretionary{}{\Wrappedafterbreak\textquotesingle}{\textquotesingle}}%
            \def\PYGZdq{\discretionary{}{\Wrappedafterbreak\char`\"}{\char`\"}}%
            \def\PYGZti{\discretionary{\char`\~}{\Wrappedafterbreak}{\char`\~}}%
        }
        % Some characters . , ; ? ! / are not pygmentized.
        % This macro makes them "active" and they will insert potential linebreaks
        \newcommand*\Wrappedbreaksatpunct {%
            \lccode`\~`\.\lowercase{\def~}{\discretionary{\hbox{\char`\.}}{\Wrappedafterbreak}{\hbox{\char`\.}}}%
            \lccode`\~`\,\lowercase{\def~}{\discretionary{\hbox{\char`\,}}{\Wrappedafterbreak}{\hbox{\char`\,}}}%
            \lccode`\~`\;\lowercase{\def~}{\discretionary{\hbox{\char`\;}}{\Wrappedafterbreak}{\hbox{\char`\;}}}%
            \lccode`\~`\:\lowercase{\def~}{\discretionary{\hbox{\char`\:}}{\Wrappedafterbreak}{\hbox{\char`\:}}}%
            \lccode`\~`\?\lowercase{\def~}{\discretionary{\hbox{\char`\?}}{\Wrappedafterbreak}{\hbox{\char`\?}}}%
            \lccode`\~`\!\lowercase{\def~}{\discretionary{\hbox{\char`\!}}{\Wrappedafterbreak}{\hbox{\char`\!}}}%
            \lccode`\~`\/\lowercase{\def~}{\discretionary{\hbox{\char`\/}}{\Wrappedafterbreak}{\hbox{\char`\/}}}%
            \catcode`\.\active
            \catcode`\,\active
            \catcode`\;\active
            \catcode`\:\active
            \catcode`\?\active
            \catcode`\!\active
            \catcode`\/\active
            \lccode`\~`\~
        }
    \makeatother

    \let\OriginalVerbatim=\Verbatim
    \makeatletter
    \renewcommand{\Verbatim}[1][1]{%
        %\parskip\z@skip
        \sbox\Wrappedcontinuationbox {\Wrappedcontinuationsymbol}%
        \sbox\Wrappedvisiblespacebox {\FV@SetupFont\Wrappedvisiblespace}%
        \def\FancyVerbFormatLine ##1{\hsize\linewidth
            \vtop{\raggedright\hyphenpenalty\z@\exhyphenpenalty\z@
                \doublehyphendemerits\z@\finalhyphendemerits\z@
                \strut ##1\strut}%
        }%
        % If the linebreak is at a space, the latter will be displayed as visible
        % space at end of first line, and a continuation symbol starts next line.
        % Stretch/shrink are however usually zero for typewriter font.
        \def\FV@Space {%
            \nobreak\hskip\z@ plus\fontdimen3\font minus\fontdimen4\font
            \discretionary{\copy\Wrappedvisiblespacebox}{\Wrappedafterbreak}
            {\kern\fontdimen2\font}%
        }%

        % Allow breaks at special characters using \PYG... macros.
        \Wrappedbreaksatspecials
        % Breaks at punctuation characters . , ; ? ! and / need catcode=\active
        \OriginalVerbatim[#1,codes*=\Wrappedbreaksatpunct]%
    }
    \makeatother

    % Exact colors from NB
    \definecolor{incolor}{HTML}{303F9F}
    \definecolor{outcolor}{HTML}{D84315}
    \definecolor{cellborder}{HTML}{CFCFCF}
    \definecolor{cellbackground}{HTML}{F7F7F7}

    % prompt
    \makeatletter
    \newcommand{\boxspacing}{\kern\kvtcb@left@rule\kern\kvtcb@boxsep}
    \makeatother
    \newcommand{\prompt}[4]{
        {\ttfamily\llap{{\color{#2}[#3]:\hspace{3pt}#4}}\vspace{-\baselineskip}}
    }

	\newcommand{\kopf}[2]{
		\hrule
		\vspace{.15cm}
		\begin{minipage}{\textwidth}
			%akwardly i had to put \" here to make it compile correctly
			{\sf\bf Introduction to Machine Learning \hfill Exercise sheet #1\\
				\url{https://slds-lmu.github.io/i2ml/} \hfill #2}
		\end{minipage}
		\vspace{.05cm}
		\hrule
		\vspace{1cm}}
    

    
    % Prevent overflowing lines due to hard-to-break entities
    \sloppy
    % Setup hyperref package
    \hypersetup{
      breaklinks=true,  % so long urls are correctly broken across lines
      colorlinks=true,
      urlcolor=urlcolor,
      linkcolor=linkcolor,
      citecolor=citecolor,
      }
    % Slightly bigger margins than the latex defaults
    
    \geometry{verbose,tmargin=0.5in,bmargin=1in,lmargin=1in,rmargin=1in}
    
    

\begin{document}
    
    	\kopf{2}{Supervised Regression}
    
    

    
    \hypertarget{solution-1-hro-in-sklearn}{%
\subsection*{Solution 1: HRO in
sklearn}\label{solution-1-hro-in-sklearn}}

    \hypertarget{a-how-are-hypothesis-space-risk-and-optimization-in-sklearn-implemented}{%
\subsubsection*{a) }\label{a-how-are-hypothesis-space-risk-and-optimization-in-sklearn-implemented}}

    Model classes representing a certain \textbf{hypothesis} are stored in
subpackages of sklearn. You can reach it with importing the desired
class with e.g.

    \begin{tcolorbox}[breakable, size=fbox, boxrule=1pt, pad at break*=1mm,colback=cellbackground, colframe=cellborder]
\prompt{In}{incolor}{1}{\boxspacing}
\begin{Verbatim}[commandchars=\\\{\}]
\PY{k+kn}{import} \PY{n+nn}{numpy} \PY{k}{as} \PY{n+nn}{np}
\PY{k+kn}{import} \PY{n+nn}{pandas} \PY{k}{as} \PY{n+nn}{pd}
\PY{k+kn}{import} \PY{n+nn}{matplotlib}\PY{n+nn}{.}\PY{n+nn}{pyplot} \PY{k}{as} \PY{n+nn}{plt}
\PY{k+kn}{from} \PY{n+nn}{sklearn}\PY{n+nn}{.}\PY{n+nn}{datasets} \PY{k+kn}{import} \PY{n}{load\PYZus{}iris}
\PY{k+kn}{import} \PY{n+nn}{seaborn} \PY{k}{as} \PY{n+nn}{sns}
\PY{k+kn}{from} \PY{n+nn}{sklearn}\PY{n+nn}{.}\PY{n+nn}{tree} \PY{k+kn}{import} \PY{n}{DecisionTreeRegressor} 
\PY{k+kn}{import} \PY{n+nn}{math}
\PY{k+kn}{from} \PY{n+nn}{sklearn}\PY{n+nn}{.}\PY{n+nn}{linear\PYZus{}model} \PY{k+kn}{import} \PY{n}{LinearRegression}
\PY{k+kn}{from} \PY{n+nn}{sklearn}\PY{n+nn}{.}\PY{n+nn}{metrics} \PY{k+kn}{import} \PY{n}{mean\PYZus{}absolute\PYZus{}error}
\PY{k+kn}{import} \PY{n+nn}{sklearn}\PY{n+nn}{.}\PY{n+nn}{metrics} \PY{k}{as} \PY{n+nn}{metrics}
\end{Verbatim}
\end{tcolorbox}

    It is good practice to import everything in the beginning of your code.

    You initialize your ``learner'' or model with its properties, defined by
the parameters, e.g.

    \begin{tcolorbox}[breakable, size=fbox, boxrule=1pt, pad at break*=1mm,colback=cellbackground, colframe=cellborder]
\prompt{In}{incolor}{2}{\boxspacing}
\begin{Verbatim}[commandchars=\\\{\}]
\PY{n}{model} \PY{o}{=} \PY{n}{LinearRegression}\PY{p}{(}\PY{n}{fit\PYZus{}intercept} \PY{o}{=} \PY{k+kc}{True}\PY{p}{)}
\PY{n}{model}
\end{Verbatim}
\end{tcolorbox}

            \begin{tcolorbox}[breakable, size=fbox, boxrule=.5pt, pad at break*=1mm, opacityfill=0]
\prompt{Out}{outcolor}{2}{\boxspacing}
\begin{Verbatim}[commandchars=\\\{\}]
LinearRegression()
\end{Verbatim}
\end{tcolorbox}
        
    Before training them on actual data, they just contain information on
the functional form of f.~Once a learner has been trained we can examine
the parameters of the resulting model. The empirical \textbf{risk} can be assessed after training by several performance measures (e.g., based on L2 loss). \textbf{Optimization} happens rather implicitly as sklearn only acts as a wrapper for existing implementations and calls package-specific optimization procedures.

    \hypertarget{b-example-iris-data-set}{%
\subsubsection*{b) }\label{b-example-iris-data-set}}

    \begin{tcolorbox}[breakable, size=fbox, boxrule=1pt, pad at break*=1mm,colback=cellbackground, colframe=cellborder]
\prompt{In}{incolor}{6}{\boxspacing}
\begin{Verbatim}[commandchars=\\\{\}]
	\PY{n}{iris} \PY{o}{=} \PY{n}{load\PYZus{}iris}\PY{p}{(}\PY{p}{)} \PY{c+c1}{\PYZsh{} function to import iris data set as type \PYZdq{}utils.Bunch\PYZdq{} with sklearn}
	\PY{n}{X} \PY{o}{=} \PY{n}{iris}\PY{o}{.}\PY{n}{data}
	\PY{n}{y} \PY{o}{=} \PY{n}{iris}\PY{o}{.}\PY{n}{target}
	\PY{n}{feature\PYZus{}names} \PY{o}{=} \PY{n}{iris}\PY{o}{.}\PY{n}{feature\PYZus{}names}
	\PY{n}{target\PYZus{}names} \PY{o}{=} \PY{n}{iris}\PY{o}{.}\PY{n}{target\PYZus{}names}
	\PY{n+nb}{print}\PY{p}{(}\PY{l+s+s2}{\PYZdq{}}\PY{l+s+s2}{Feature names:}\PY{l+s+s2}{\PYZdq{}}\PY{p}{,} \PY{n}{feature\PYZus{}names}\PY{p}{)}
	\PY{n+nb}{print}\PY{p}{(}\PY{l+s+s2}{\PYZdq{}}\PY{l+s+s2}{Target names:}\PY{l+s+s2}{\PYZdq{}}\PY{p}{,} \PY{n}{target\PYZus{}names}\PY{p}{)}
	\PY{n+nb}{print}\PY{p}{(}\PY{l+s+s2}{\PYZdq{}}\PY{l+s+se}{\PYZbs{}n}\PY{l+s+s2}{Shape of X and y}\PY{l+s+se}{\PYZbs{}n}\PY{l+s+s2}{\PYZdq{}}\PY{p}{,} \PY{n}{X}\PY{o}{.}\PY{n}{shape}\PY{p}{,} \PY{n}{y}\PY{o}{.}\PY{n}{shape}\PY{p}{)}
	\PY{n+nb}{print}\PY{p}{(}\PY{l+s+s2}{\PYZdq{}}\PY{l+s+se}{\PYZbs{}n}\PY{l+s+s2}{Type of X and y}\PY{l+s+se}{\PYZbs{}n}\PY{l+s+s2}{\PYZdq{}}\PY{p}{,} \PY{n+nb}{type}\PY{p}{(}\PY{n}{X}\PY{p}{)}\PY{p}{,} \PY{n+nb}{type}\PY{p}{(}\PY{n}{y}\PY{p}{)}\PY{p}{)}
\end{Verbatim}
\end{tcolorbox}

\begin{Verbatim}[commandchars=\\\{\}]
Feature names: ['sepal length (cm)', 'sepal width (cm)', 'petal length (cm)',
'petal width (cm)']
Target names: ['setosa' 'versicolor' 'virginica']

Shape of X and y
(150, 4) (150,)

Type of X and y
<class 'numpy.ndarray'> <class 'numpy.ndarray'>
\end{Verbatim}

We obtain the following information:
\begin{itemize}
	\item \texttt{iris} has categorical targets \texttt{['setosa' 'versicolor' 'virginica']}
	\item It has 150 observations of 5 variables, one of which is the 
	target.
	\item both, X and y, are of type \texttt{'numpy.ndarray'}, thus numerical. The 3 classes in target y are stored as numbers 0, 1, 2.
\end{itemize}

    
    \hypertarget{c-learner-of-your-choice}{%
\subsubsection*{c) }\label{c-learner-of-your-choice}}

    \texttt{sklearn} offers many different models. Let's look at \textbf{regression trees}:

    Roughly speaking, regression trees create small, homogeneous subsets
(``nodes'') by repeatedly splitting the data at some cut-off (e.g., for
iris: partition into observations with Sepal.Width $\leq$ 3 and
\textgreater{} 3), and predict the mean target value within each final
group.

    \begin{tcolorbox}[breakable, size=fbox, boxrule=1pt, pad at break*=1mm,colback=cellbackground, colframe=cellborder]
\prompt{In}{incolor}{8}{\boxspacing}
\begin{Verbatim}[commandchars=\\\{\}]
\PY{c+c1}{\PYZsh{} help(DecisionTreeRegressor) }
\end{Verbatim}
\end{tcolorbox}

    Prints documentary in console, or visit scikit-learn.org
--\textgreater{} select right version --\textgreater{} go to right
class, here sklearn.tree.DecisionTreeRegressor.

    \begin{tcolorbox}[breakable, size=fbox, boxrule=1pt, pad at break*=1mm,colback=cellbackground, colframe=cellborder]
\prompt{In}{incolor}{9}{\boxspacing}
\begin{Verbatim}[commandchars=\\\{\}]
\PY{n}{rtree} \PY{o}{=} \PY{n}{DecisionTreeRegressor}\PY{p}{(}\PY{p}{)} \PY{c+c1}{\PYZsh{}default setting}
\PY{n+nb}{print}\PY{p}{(}\PY{n}{rtree}\PY{o}{.}\PY{n}{get\PYZus{}params}\PY{p}{(}\PY{p}{)}\PY{p}{)}
\PY{n+nb}{print}\PY{p}{(}\PY{n}{rtree}\PY{o}{.}\PY{n}{get\PYZus{}depth}\PY{p}{(}\PY{p}{)}\PY{p}{)} \PY{c+c1}{\PYZsh{} not working because no tree was fitted yet}
\PY{n+nb}{print}\PY{p}{(}\PY{n}{rtree}\PY{o}{.}\PY{n}{get\PYZus{}n\PYZus{}leaves}\PY{p}{(}\PY{p}{)}\PY{p}{)} \PY{c+c1}{\PYZsh{} not working because no tree was fitted yet}
\end{Verbatim}
\end{tcolorbox}

    \begin{Verbatim}[commandchars=\\\{\}]
\{'ccp\_alpha': 0.0, 'criterion': 'squared\_error', 'max\_depth': None,
'max\_features': None, 'max\_leaf\_nodes': None, 'min\_impurity\_decrease': 0.0,
'min\_samples\_leaf': 1, 'min\_samples\_split': 2, 'min\_weight\_fraction\_leaf': 0.0,
'random\_state': None, 'splitter': 'best'\}
    \end{Verbatim}

    \begin{Verbatim}[commandchars=\\\{\}, frame=single, framerule=2mm, rulecolor=\color{outerrorbackground}]
\textcolor{ansi-red-intense}{\textbf{---------------------------------------------------------------------------}}
\textcolor{ansi-red-intense}{\textbf{NotFittedError}}                            Traceback (most recent call last)
\textcolor{ansi-green-intense}{\textbf{\textasciitilde{}\textbackslash{}AppData\textbackslash{}Local\textbackslash{}Temp\textbackslash{}ipykernel\_16280\textbackslash{}3357154144.py}} in \textcolor{ansi-cyan}{<module>}
\textcolor{ansi-green}{      1} rtree \textcolor{ansi-yellow-intense}{\textbf{=}} DecisionTreeRegressor\textcolor{ansi-yellow-intense}{\textbf{(}}\textcolor{ansi-yellow-intense}{\textbf{)}} \textcolor{ansi-red-intense}{\textbf{\#default setting}}
\textcolor{ansi-green}{      2} print\textcolor{ansi-yellow-intense}{\textbf{(}}rtree\textcolor{ansi-yellow-intense}{\textbf{.}}get\_params\textcolor{ansi-yellow-intense}{\textbf{(}}\textcolor{ansi-yellow-intense}{\textbf{)}}\textcolor{ansi-yellow-intense}{\textbf{)}}
\textcolor{ansi-green-intense}{\textbf{----> 3}}\textcolor{ansi-yellow-intense}{\textbf{ }}print\textcolor{ansi-yellow-intense}{\textbf{(}}rtree\textcolor{ansi-yellow-intense}{\textbf{.}}get\_depth\textcolor{ansi-yellow-intense}{\textbf{(}}\textcolor{ansi-yellow-intense}{\textbf{)}}\textcolor{ansi-yellow-intense}{\textbf{)}} \textcolor{ansi-red-intense}{\textbf{\# not working because no tree was fitted yet}}
\textcolor{ansi-green}{      4} print\textcolor{ansi-yellow-intense}{\textbf{(}}rtree\textcolor{ansi-yellow-intense}{\textbf{.}}get\_n\_leaves\textcolor{ansi-yellow-intense}{\textbf{(}}\textcolor{ansi-yellow-intense}{\textbf{)}}\textcolor{ansi-yellow-intense}{\textbf{)}} \textcolor{ansi-red-intense}{\textbf{\# not working because no tree was fitted yet}}

\textcolor{ansi-green-intense}{\textbf{\textasciitilde{}\textbackslash{}anaconda3\textbackslash{}envs\textbackslash{}I2ML\_env\textbackslash{}lib\textbackslash{}site-packages\textbackslash{}sklearn\textbackslash{}tree\textbackslash{}\_classes.py}} in \textcolor{ansi-cyan}{get\_depth}\textcolor{ansi-blue-intense}{\textbf{(self)}}
\textcolor{ansi-green}{    134}             The maximum depth of the tree\textcolor{ansi-yellow-intense}{\textbf{.}}
\textcolor{ansi-green}{    135}         """
\textcolor{ansi-green-intense}{\textbf{--> 136}}\textcolor{ansi-yellow-intense}{\textbf{         }}check\_is\_fitted\textcolor{ansi-yellow-intense}{\textbf{(}}self\textcolor{ansi-yellow-intense}{\textbf{)}}
\textcolor{ansi-green}{    137}         \textcolor{ansi-green-intense}{\textbf{return}} self\textcolor{ansi-yellow-intense}{\textbf{.}}tree\_\textcolor{ansi-yellow-intense}{\textbf{.}}max\_depth
\textcolor{ansi-green}{    138} 

\textcolor{ansi-green-intense}{\textbf{\textasciitilde{}\textbackslash{}anaconda3\textbackslash{}envs\textbackslash{}I2ML\_env\textbackslash{}lib\textbackslash{}site-packages\textbackslash{}sklearn\textbackslash{}utils\textbackslash{}validation.py}} in \textcolor{ansi-cyan}{check\_is\_fitted}\textcolor{ansi-blue-intense}{\textbf{(estimator, attributes, msg, all\_or\_any)}}
\textcolor{ansi-green}{   1220} 
\textcolor{ansi-green}{   1221}     \textcolor{ansi-green-intense}{\textbf{if}} \textcolor{ansi-green-intense}{\textbf{not}} fitted\textcolor{ansi-yellow-intense}{\textbf{:}}
\textcolor{ansi-green-intense}{\textbf{-> 1222}}\textcolor{ansi-yellow-intense}{\textbf{         }}\textcolor{ansi-green-intense}{\textbf{raise}} NotFittedError\textcolor{ansi-yellow-intense}{\textbf{(}}msg \textcolor{ansi-yellow-intense}{\textbf{\%}} \textcolor{ansi-yellow-intense}{\textbf{\{}}\textcolor{ansi-blue-intense}{\textbf{"name"}}\textcolor{ansi-yellow-intense}{\textbf{:}} type\textcolor{ansi-yellow-intense}{\textbf{(}}estimator\textcolor{ansi-yellow-intense}{\textbf{)}}\textcolor{ansi-yellow-intense}{\textbf{.}}\_\_name\_\_\textcolor{ansi-yellow-intense}{\textbf{\}}}\textcolor{ansi-yellow-intense}{\textbf{)}}
\textcolor{ansi-green}{   1223} 
\textcolor{ansi-green}{   1224} 

\textcolor{ansi-red-intense}{\textbf{NotFittedError}}: This DecisionTreeRegressor instance is not fitted yet. Call 'fit' with appropriate arguments before using this estimator.
    \end{Verbatim}

    In general: DecisionTreeRegressor inherits from class sklearn.tree as it
is used for regression, it predicts regression value for input X.

    Important parameters\\
    criteria: choose between L2, L1, and others as Loss function \\
	splitter: strategy for choosing the split, default ``best'' \\
	max-depth: The maximum depth of the tree other complexity related params \\
	random\_state: Controls the randomness of the estimator. To obtain a deterministic behaviour during fitting, random\_state has to be fixed to an integer.
	
\subsection*{Solution 2: Loss Function for Regression Task}
See R solution sheet.


    \hypertarget{solution-3-polynomial-regression}{%
\subsection*{Solution 3: Polynomial
Regression}\label{solution-3-polynomial-regression}}
\subsubsection*{a)}
See R solution sheet.

    \hypertarget{b-choose-3-different-parameterizations-and-plot-the-resulting-polynomials}{%
\subsubsection*{b) }\label{b-choose-3-different-parameterizations-and-plot-the-resulting-polynomials}}
Choose 3 different parameterizations and plot the resulting polynomials:

\begin{tcolorbox}[breakable, size=fbox, boxrule=1pt, pad at break*=1mm,colback=cellbackground, colframe=cellborder]
\prompt{In}{incolor}{7}{\boxspacing}
\begin{Verbatim}[commandchars=\\\{\}]
	\PY{n}{x} \PY{o}{=} \PY{n}{np}\PY{o}{.}\PY{n}{linspace}\PY{p}{(}\PY{o}{\PYZhy{}}\PY{l+m+mi}{3}\PY{p}{,}\PY{l+m+mi}{3}\PY{p}{,} \PY{n}{num} \PY{o}{=} \PY{l+m+mi}{60}\PY{p}{)} \PY{c+c1}{\PYZsh{} 3 included}
\end{Verbatim}
\end{tcolorbox}

\begin{tcolorbox}[breakable, size=fbox, boxrule=1pt, pad at break*=1mm,colback=cellbackground, colframe=cellborder]
\prompt{In}{incolor}{11}{\boxspacing}
\begin{Verbatim}[commandchars=\\\{\}]
	\PY{k}{def} \PY{n+nf}{fun\PYZus{}y}\PY{p}{(}\PY{n}{x\PYZus{}in}\PY{p}{)}\PY{p}{:}
	\PY{l+s+sd}{\PYZdq{}\PYZdq{}\PYZdq{} }
	\PY{l+s+sd}{    Method to produce named sinus function}
	\PY{l+s+sd}{    Input: data as numpy array}
	\PY{l+s+sd}{    Output: Function \PYZhy{}3 + 5*sin(0.4 * pi * x\PYZus{}in) as numpy array}
	\PY{l+s+sd}{    \PYZdq{}\PYZdq{}\PYZdq{}}
	\PY{n}{erg} \PY{o}{=} \PY{o}{\PYZhy{}}\PY{l+m+mi}{3} \PY{o}{+} \PY{l+m+mi}{5}\PY{o}{*} \PY{n}{np}\PY{o}{.}\PY{n}{sin}\PY{p}{(}\PY{l+m+mf}{0.4} \PY{o}{*} \PY{n}{math}\PY{o}{.}\PY{n}{pi} \PY{o}{*} \PY{n}{x\PYZus{}in}\PY{p}{)}
	\PY{k}{return} \PY{n}{erg}
	
	\PY{k}{def} \PY{n+nf}{fun\PYZus{}poly3}\PY{p}{(}\PY{n}{x\PYZus{}in}\PY{p}{,} \PY{n}{beta}\PY{p}{)}\PY{p}{:}
	\PY{l+s+sd}{\PYZdq{}\PYZdq{}\PYZdq{} }
	\PY{l+s+sd}{    Method to produce poynomial degree 3 with coefficents in numpy array beta for input data x\PYZus{}in}
	\PY{l+s+sd}{    Input: x\PYZus{}in as numpy array, beta as numpy array}
	\PY{l+s+sd}{    Output: function value as numpy array}
	\PY{l+s+sd}{    \PYZdq{}\PYZdq{}\PYZdq{}}  
	\PY{n}{erg} \PY{o}{=} \PY{n}{beta}\PY{p}{[}\PY{l+m+mi}{0}\PY{p}{]} \PY{o}{+} \PY{n}{beta}\PY{p}{[}\PY{l+m+mi}{1}\PY{p}{]} \PY{o}{*} \PY{n}{x\PYZus{}in} \PY{o}{+} \PY{n}{beta}\PY{p}{[}\PY{l+m+mi}{2}\PY{p}{]} \PY{o}{*} \PY{n}{x\PYZus{}in}\PY{o}{*}\PY{n}{x\PYZus{}in} \PY{o}{+} \PY{n}{beta}\PY{p}{[}\PY{l+m+mi}{3}\PY{p}{]} \PY{o}{*} \PY{n}{x\PYZus{}in}\PY{o}{*}\PY{n}{x\PYZus{}in}\PY{o}{*}\PY{n}{x\PYZus{}in}
	\PY{k}{return} \PY{n}{erg}
\end{Verbatim}
\end{tcolorbox}

\begin{tcolorbox}[breakable, size=fbox, boxrule=1pt, pad at break*=1mm,colback=cellbackground, colframe=cellborder]
\prompt{In}{incolor}{12}{\boxspacing}
\begin{Verbatim}[commandchars=\\\{\}]
	\PY{n}{np}\PY{o}{.}\PY{n}{random}\PY{o}{.}\PY{n}{seed}\PY{p}{(}\PY{l+m+mi}{43}\PY{p}{)}
	\PY{n}{y} \PY{o}{=} \PY{n}{fun\PYZus{}y}\PY{p}{(}\PY{n}{x}\PY{p}{)} \PY{o}{+} \PY{n}{np}\PY{o}{.}\PY{n}{random}\PY{o}{.}\PY{n}{normal}\PY{p}{(}\PY{n}{size} \PY{o}{=} \PY{l+m+mi}{60}\PY{p}{)}
\end{Verbatim}
\end{tcolorbox}

\begin{tcolorbox}[breakable, size=fbox, boxrule=1pt, pad at break*=1mm,colback=cellbackground, colframe=cellborder]
\prompt{In}{incolor}{13}{\boxspacing}
\begin{Verbatim}[commandchars=\\\{\}]
	\PY{n}{poly3d} \PY{o}{=} \PY{n}{np}\PY{o}{.}\PY{n}{poly1d}\PY{p}{(}\PY{n}{np}\PY{o}{.}\PY{n}{polyfit}\PY{p}{(}\PY{n}{x}\PY{p}{,}\PY{n}{y}\PY{p}{,}\PY{l+m+mi}{3}\PY{p}{)}\PY{p}{)} \PY{c+c1}{\PYZsh{} Polyfit function for polynomial functions}
	\PY{n+nb}{print}\PY{p}{(}\PY{n}{np}\PY{o}{.}\PY{n}{polyfit}\PY{p}{(}\PY{n}{x}\PY{p}{,}\PY{n}{y}\PY{p}{,}\PY{l+m+mi}{3}\PY{p}{)}\PY{p}{)} \PY{c+c1}{\PYZsh{} coefficients}
\end{Verbatim}
\end{tcolorbox}

\begin{Verbatim}[commandchars=\\\{\}]
[-0.68708783 -0.04208217  4.60056399 -2.76629061]
\end{Verbatim}

Create plot from mathplotlib.pyplot

\begin{tcolorbox}[breakable, size=fbox, boxrule=1pt, pad at break*=1mm,colback=cellbackground, colframe=cellborder]
\prompt{In}{incolor}{14}{\boxspacing}
\begin{Verbatim}[commandchars=\\\{\}]
	\PY{n}{plt}\PY{o}{.}\PY{n}{figure}\PY{p}{(}\PY{n}{figsize}\PY{o}{=}\PY{p}{(}\PY{l+m+mi}{12}\PY{p}{,} \PY{l+m+mi}{6}\PY{p}{)}\PY{p}{)}
	\PY{n}{plt}\PY{o}{.}\PY{n}{grid}\PY{p}{(}\PY{k+kc}{True}\PY{p}{)}
	\PY{n}{plt}\PY{o}{.}\PY{n}{plot}\PY{p}{(}\PY{n}{x}\PY{p}{,} \PY{n}{y}\PY{p}{,} \PY{n}{color}\PY{o}{=}\PY{l+s+s1}{\PYZsq{}}\PY{l+s+s1}{blue}\PY{l+s+s1}{\PYZsq{}}\PY{p}{,} \PY{n}{linestyle}\PY{o}{=}\PY{l+s+s1}{\PYZsq{}}\PY{l+s+s1}{none}\PY{l+s+s1}{\PYZsq{}}\PY{p}{,}\PY{n}{marker}\PY{o}{=}\PY{l+s+s1}{\PYZsq{}}\PY{l+s+s1}{o}\PY{l+s+s1}{\PYZsq{}}\PY{p}{,}\PY{n}{markersize}\PY{o}{=}\PY{l+m+mi}{10}\PY{p}{,} \PY{n}{label} \PY{o}{=} \PY{l+s+s1}{\PYZsq{}}\PY{l+s+s1}{random numbers}\PY{l+s+s1}{\PYZsq{}}\PY{p}{)}
	\PY{n}{plt}\PY{o}{.}\PY{n}{plot}\PY{p}{(}\PY{n}{x}\PY{p}{,} \PY{n}{fun\PYZus{}y}\PY{p}{(}\PY{n}{x}\PY{p}{)}\PY{p}{,} \PY{n}{color}\PY{o}{=}\PY{l+s+s1}{\PYZsq{}}\PY{l+s+s1}{black}\PY{l+s+s1}{\PYZsq{}}\PY{p}{,} \PY{n}{linestyle}\PY{o}{=}\PY{l+s+s1}{\PYZsq{}}\PY{l+s+s1}{solid}\PY{l+s+s1}{\PYZsq{}}\PY{p}{,} \PY{n}{label} \PY{o}{=} \PY{l+s+s1}{\PYZsq{}}\PY{l+s+s1}{Correct function}\PY{l+s+s1}{\PYZsq{}}\PY{p}{)}
	\PY{n}{plt}\PY{o}{.}\PY{n}{plot}\PY{p}{(}\PY{n}{x}\PY{p}{,} \PY{n}{poly3d}\PY{p}{(}\PY{n}{x}\PY{p}{)}\PY{p}{,} \PY{n}{color}\PY{o}{=}\PY{l+s+s1}{\PYZsq{}}\PY{l+s+s1}{red}\PY{l+s+s1}{\PYZsq{}}\PY{p}{,} \PY{n}{linestyle}\PY{o}{=}\PY{l+s+s1}{\PYZsq{}}\PY{l+s+s1}{solid}\PY{l+s+s1}{\PYZsq{}}\PY{p}{,} \PY{n}{label} \PY{o}{=} \PY{l+s+s1}{\PYZsq{}}\PY{l+s+s1}{Fitted Function degree 3}\PY{l+s+s1}{\PYZsq{}}\PY{p}{)}
	\PY{n}{plt}\PY{o}{.}\PY{n}{plot}\PY{p}{(}\PY{n}{x}\PY{p}{,} \PY{n}{fun\PYZus{}poly3}\PY{p}{(}\PY{n}{x}\PY{p}{,}\PY{n}{np}\PY{o}{.}\PY{n}{array}\PY{p}{(}\PY{p}{[}\PY{l+m+mf}{1.5}\PY{p}{,}\PY{l+m+mi}{2}\PY{p}{,}\PY{o}{\PYZhy{}}\PY{l+m+mf}{0.7}\PY{p}{,}\PY{l+m+mi}{0}\PY{p}{]}\PY{p}{)}\PY{p}{)}\PY{p}{,} \PY{n}{color}\PY{o}{=}\PY{l+s+s1}{\PYZsq{}}\PY{l+s+s1}{green}\PY{l+s+s1}{\PYZsq{}}\PY{p}{,} \PY{n}{linestyle}\PY{o}{=}\PY{l+s+s1}{\PYZsq{}}\PY{l+s+s1}{solid}\PY{l+s+s1}{\PYZsq{}}\PY{p}{,} \PY{n}{label} \PY{o}{=} \PY{l+s+s1}{\PYZsq{}}\PY{l+s+s1}{Function degree 2}\PY{l+s+s1}{\PYZsq{}}\PY{p}{)}
	\PY{n}{plt}\PY{o}{.}\PY{n}{plot}\PY{p}{(}\PY{n}{x}\PY{p}{,} \PY{n}{fun\PYZus{}poly3}\PY{p}{(}\PY{n}{x}\PY{p}{,}\PY{n}{np}\PY{o}{.}\PY{n}{array}\PY{p}{(}\PY{p}{[}\PY{l+m+mi}{1}\PY{p}{,}\PY{o}{\PYZhy{}}\PY{l+m+mf}{1.6}\PY{p}{,}\PY{o}{\PYZhy{}}\PY{l+m+mf}{0.3}\PY{p}{,}\PY{l+m+mf}{0.2}\PY{p}{]}\PY{p}{)}\PY{p}{)}\PY{p}{,} \PY{n}{color}\PY{o}{=}\PY{l+s+s1}{\PYZsq{}}\PY{l+s+s1}{turquoise}\PY{l+s+s1}{\PYZsq{}}\PY{p}{,} \PY{n}{linestyle}\PY{o}{=}\PY{l+s+s1}{\PYZsq{}}\PY{l+s+s1}{solid}\PY{l+s+s1}{\PYZsq{}}\PY{p}{,} \PY{n}{label} \PY{o}{=} \PY{l+s+s1}{\PYZsq{}}\PY{l+s+s1}{Function degree 3}\PY{l+s+s1}{\PYZsq{}}\PY{p}{)}
	\PY{c+c1}{\PYZsh{} title \PYZam{} label axes}
	\PY{n}{plt}\PY{o}{.}\PY{n}{title}\PY{p}{(}\PY{l+s+s1}{\PYZsq{}}\PY{l+s+s1}{Polynomial Fit:}\PY{l+s+se}{\PYZbs{}n}\PY{l+s+s1}{Different curves for Sinus curve}\PY{l+s+s1}{\PYZsq{}}\PY{p}{,} \PY{n}{size}\PY{o}{=}\PY{l+m+mi}{18}\PY{p}{)}
	\PY{n}{plt}\PY{o}{.}\PY{n}{xlabel}\PY{p}{(}\PY{l+s+s1}{\PYZsq{}}\PY{l+s+s1}{x}\PY{l+s+s1}{\PYZsq{}}\PY{p}{,} \PY{n}{size}\PY{o}{=}\PY{l+m+mi}{16}\PY{p}{)}
	\PY{n}{plt}\PY{o}{.}\PY{n}{ylabel}\PY{p}{(}\PY{l+s+s1}{\PYZsq{}}\PY{l+s+s1}{y}\PY{l+s+s1}{\PYZsq{}}\PY{p}{,} \PY{n}{size}\PY{o}{=}\PY{l+m+mi}{16}\PY{p}{)}
	\PY{n}{plt}\PY{o}{.}\PY{n}{legend}\PY{p}{(}\PY{n}{loc}\PY{o}{=}\PY{l+s+s1}{\PYZsq{}}\PY{l+s+s1}{upper left}\PY{l+s+s1}{\PYZsq{}}\PY{p}{,} \PY{n}{prop}\PY{o}{=}\PY{p}{\PYZob{}}\PY{l+s+s1}{\PYZsq{}}\PY{l+s+s1}{size}\PY{l+s+s1}{\PYZsq{}}\PY{p}{:} \PY{l+m+mi}{10}\PY{p}{\PYZcb{}}\PY{p}{)}
	\PY{n}{plt}\PY{o}{.}\PY{n}{show}\PY{p}{(}\PY{p}{)}
\end{Verbatim}
\end{tcolorbox}

    \begin{center}
    \adjustimage{max size={0.9\linewidth}{0.9\paperheight}}{output_33_0.png}
    \end{center}
    { \hspace*{\fill} \\}
    
    We see that our hypothesis space is simply a family of curves. The 3
examples plotted here already hint at the amount of flexibility
third-degree polynomials offer over simple linear functions.

\subsubsection*{c, d, e, f)}
See R solution sheet.

    \hypertarget{solution-4-predicting-abalone}{%
\subsection*{Solution 4: Predicting
abalone}\label{solution-4-predicting-abalone}}

    \begin{tcolorbox}[breakable, size=fbox, boxrule=1pt, pad at break*=1mm,colback=cellbackground, colframe=cellborder]
\prompt{In}{incolor}{15}{\boxspacing}
\begin{Verbatim}[commandchars=\\\{\}]
\PY{c+c1}{\PYZsh{} Exercise 4 Prep: load data from url}

\PY{n}{url} \PY{o}{=} \PY{l+s+s2}{\PYZdq{}}\PY{l+s+s2}{https://archive.ics.uci.edu/ml/machine\PYZhy{}learning\PYZhy{}databases/abalone/abalone.data}\PY{l+s+s2}{\PYZdq{}}
\PY{n}{abalone} \PY{o}{=} \PY{n}{pd}\PY{o}{.}\PY{n}{read\PYZus{}csv}\PY{p}{(}\PY{n}{url}\PY{p}{,} \PY{n}{sep}\PY{o}{=}\PY{l+s+s1}{\PYZsq{}}\PY{l+s+s1}{,}\PY{l+s+s1}{\PYZsq{}}\PY{p}{,} \PY{n}{names}\PY{o}{=}\PY{p}{[}\PY{l+s+s1}{\PYZsq{}}\PY{l+s+s1}{sex}\PY{l+s+s1}{\PYZsq{}}\PY{p}{,} \PY{l+s+s2}{\PYZdq{}}\PY{l+s+s2}{longest\PYZus{}shell}\PY{l+s+s2}{\PYZdq{}}\PY{p}{,} \PY{l+s+s2}{\PYZdq{}}\PY{l+s+s2}{diameter}\PY{l+s+s2}{\PYZdq{}}\PY{p}{,} \PY{l+s+s2}{\PYZdq{}}\PY{l+s+s2}{height}\PY{l+s+s2}{\PYZdq{}}\PY{p}{,} \PY{l+s+s2}{\PYZdq{}}\PY{l+s+s2}{whole\PYZus{}weight}\PY{l+s+s2}{\PYZdq{}}\PY{p}{,} 
  \PY{l+s+s2}{\PYZdq{}}\PY{l+s+s2}{shucked\PYZus{}weight}\PY{l+s+s2}{\PYZdq{}}\PY{p}{,} \PY{l+s+s2}{\PYZdq{}}\PY{l+s+s2}{visceral\PYZus{}weight}\PY{l+s+s2}{\PYZdq{}}\PY{p}{,} \PY{l+s+s2}{\PYZdq{}}\PY{l+s+s2}{shell\PYZus{}weight}\PY{l+s+s2}{\PYZdq{}}\PY{p}{,} \PY{l+s+s2}{\PYZdq{}}\PY{l+s+s2}{rings}\PY{l+s+s2}{\PYZdq{}}\PY{p}{]}\PY{p}{)}

\PY{n}{abalone} \PY{o}{=} \PY{n}{abalone}\PY{p}{[}\PY{p}{[}\PY{l+s+s1}{\PYZsq{}}\PY{l+s+s1}{longest\PYZus{}shell}\PY{l+s+s1}{\PYZsq{}}\PY{p}{,} \PY{l+s+s1}{\PYZsq{}}\PY{l+s+s1}{whole\PYZus{}weight}\PY{l+s+s1}{\PYZsq{}}\PY{p}{,} \PY{l+s+s1}{\PYZsq{}}\PY{l+s+s1}{rings}\PY{l+s+s1}{\PYZsq{}}\PY{p}{]}\PY{p}{]}
\PY{n+nb}{print}\PY{p}{(}\PY{n}{abalone}\PY{o}{.}\PY{n}{head}\PY{p}{)}
\end{Verbatim}
\end{tcolorbox}

    \begin{Verbatim}[commandchars=\\\{\}]
<bound method NDFrame.head of       longest\_shell  whole\_weight  rings
0             0.455        0.5140     15
1             0.350        0.2255      7
2             0.530        0.6770      9
3             0.440        0.5160     10
4             0.330        0.2050      7
{\ldots}             {\ldots}           {\ldots}    {\ldots}
4172          0.565        0.8870     11
4173          0.590        0.9660     10
4174          0.600        1.1760      9
4175          0.625        1.0945     10
4176          0.710        1.9485     12

[4177 rows x 3 columns]>
    \end{Verbatim}

    \hypertarget{a-plot-longestshell-and-wholeweight-with-color-for-rings}{%
\subsubsection*{a) }\label{a-plot-longestshell-and-wholeweight-with-color-for-rings}}

    \begin{tcolorbox}[breakable, size=fbox, boxrule=1pt, pad at break*=1mm,colback=cellbackground, colframe=cellborder]
\prompt{In}{incolor}{16}{\boxspacing}
\begin{Verbatim}[commandchars=\\\{\}]
\PY{n}{plt}\PY{o}{.}\PY{n}{grid}\PY{p}{(}\PY{k+kc}{True}\PY{p}{)}
\PY{n}{plt}\PY{o}{.}\PY{n}{scatter}\PY{p}{(}\PY{n}{abalone}\PY{o}{.}\PY{n}{longest\PYZus{}shell}\PY{p}{,} \PY{n}{abalone}\PY{o}{.}\PY{n}{whole\PYZus{}weight}\PY{p}{,} \PY{n}{s}\PY{o}{=}\PY{l+m+mi}{10}\PY{p}{,} \PY{n}{c}\PY{o}{=} \PY{n}{abalone}\PY{o}{.}\PY{n}{rings}\PY{p}{,} \PY{n}{cmap} \PY{o}{=} \PY{l+s+s1}{\PYZsq{}}\PY{l+s+s1}{jet}\PY{l+s+s1}{\PYZsq{}}\PY{p}{)} \PY{c+c1}{\PYZsh{}choose appropriate color map}
\PY{n}{plt}\PY{o}{.}\PY{n}{colorbar}\PY{p}{(}\PY{n}{label} \PY{o}{=} \PY{l+s+s1}{\PYZsq{}}\PY{l+s+s1}{Rings}\PY{l+s+s1}{\PYZsq{}}\PY{p}{)} \PY{c+c1}{\PYZsh{} add color bar}
\PY{c+c1}{\PYZsh{} title \PYZam{} label axes}
\PY{n}{plt}\PY{o}{.}\PY{n}{title}\PY{p}{(}\PY{l+s+s1}{\PYZsq{}}\PY{l+s+s1}{Scatter Plot:}\PY{l+s+se}{\PYZbs{}n}\PY{l+s+s1}{Longest Shell and Whole Weight, Color by rings}\PY{l+s+s1}{\PYZsq{}}\PY{p}{,} \PY{n}{size}\PY{o}{=}\PY{l+m+mi}{15}\PY{p}{)}
\PY{n}{plt}\PY{o}{.}\PY{n}{xlabel}\PY{p}{(}\PY{l+s+s1}{\PYZsq{}}\PY{l+s+s1}{Longest Shell}\PY{l+s+s1}{\PYZsq{}}\PY{p}{,} \PY{n}{size}\PY{o}{=}\PY{l+m+mi}{11}\PY{p}{)}
\PY{n}{plt}\PY{o}{.}\PY{n}{ylabel}\PY{p}{(}\PY{l+s+s1}{\PYZsq{}}\PY{l+s+s1}{Whole Weight}\PY{l+s+s1}{\PYZsq{}}\PY{p}{,} \PY{n}{size}\PY{o}{=}\PY{l+m+mi}{11}\PY{p}{)}
\PY{n}{plt}\PY{o}{.}\PY{n}{show}\PY{p}{(}\PY{p}{)}
\end{Verbatim}
\end{tcolorbox}

    \begin{center}
    \adjustimage{max size={0.9\linewidth}{0.9\paperheight}}{output_38_0.png}
    \end{center}
    { \hspace*{\fill} \\}
    
    We see that weight scales exponentially with shell length and that
larger/heavier animals tend to have more rings.


    \hypertarget{c-fit-linear-regression}{%
\subsubsection*{b) }\label{c-fit-linear-regression}}

    \begin{tcolorbox}[breakable, size=fbox, boxrule=1pt, pad at break*=1mm,colback=cellbackground, colframe=cellborder]
\prompt{In}{incolor}{17}{\boxspacing}
\begin{Verbatim}[commandchars=\\\{\}]
\PY{n}{X\PYZus{}lm} \PY{o}{=} \PY{n}{abalone}\PY{o}{.}\PY{n}{iloc}\PY{p}{[}\PY{p}{:}\PY{p}{,} \PY{l+m+mi}{0}\PY{p}{:}\PY{l+m+mi}{2}\PY{p}{]}\PY{o}{.}\PY{n}{values}
\PY{n}{y\PYZus{}lm} \PY{o}{=} \PY{n}{abalone}\PY{o}{.}\PY{n}{rings}

\PY{n}{lm} \PY{o}{=} \PY{n}{LinearRegression}\PY{p}{(}\PY{p}{)}\PY{o}{.}\PY{n}{fit}\PY{p}{(}\PY{n}{X\PYZus{}lm}\PY{p}{,}\PY{n}{y\PYZus{}lm}\PY{p}{)}

\PY{n}{pred\PYZus{}lm} \PY{o}{=} \PY{n}{lm}\PY{o}{.}\PY{n}{predict}\PY{p}{(}\PY{n}{X\PYZus{}lm}\PY{p}{)}

\PY{n}{results\PYZus{}dic} \PY{o}{=} \PY{p}{\PYZob{}}\PY{l+s+s1}{\PYZsq{}}\PY{l+s+s1}{prediction}\PY{l+s+s1}{\PYZsq{}} \PY{p}{:} \PY{n}{pred\PYZus{}lm}\PY{p}{,}
           \PY{l+s+s1}{\PYZsq{}}\PY{l+s+s1}{truth}\PY{l+s+s1}{\PYZsq{}}\PY{p}{:} \PY{n}{y\PYZus{}lm}\PY{p}{\PYZcb{}}

\PY{n}{results} \PY{o}{=} \PY{n}{pd}\PY{o}{.}\PY{n}{DataFrame}\PY{p}{(}\PY{n}{results\PYZus{}dic}\PY{p}{)}

\PY{n}{results}\PY{o}{.}\PY{n}{head}\PY{p}{(}\PY{p}{)}
\end{Verbatim}
\end{tcolorbox}

            \begin{tcolorbox}[breakable, size=fbox, boxrule=.5pt, pad at break*=1mm, opacityfill=0]
\prompt{Out}{outcolor}{17}{\boxspacing}
\begin{Verbatim}[commandchars=\\\{\}]
   prediction  truth
0    8.840042     15
1    7.395659      7
2    9.821995      9
3    8.683616     10
4    7.160333      7
\end{Verbatim}
\end{tcolorbox}
        
    \hypertarget{d-compare-the-fitted-and-observed-targets-visually}{%
\subsubsection*{c) 
}\label{d-compare-the-fitted-and-observed-targets-visually}}

    \begin{tcolorbox}[breakable, size=fbox, boxrule=1pt, pad at break*=1mm,colback=cellbackground, colframe=cellborder]
\prompt{In}{incolor}{18}{\boxspacing}
\begin{Verbatim}[commandchars=\\\{\}]
\PY{n}{plt}\PY{o}{.}\PY{n}{grid}\PY{p}{(}\PY{k+kc}{True}\PY{p}{)}
\PY{c+c1}{\PYZsh{}plt.scatter(pred\PYZus{}lm, y\PYZus{}lm, s=5) }
\PY{n}{sns}\PY{o}{.}\PY{n}{regplot}\PY{p}{(}\PY{n}{x} \PY{o}{=} \PY{n}{pred\PYZus{}lm}\PY{p}{,} \PY{n}{y} \PY{o}{=} \PY{n}{y\PYZus{}lm}\PY{p}{,} \PY{n}{ci} \PY{o}{=} \PY{l+m+mi}{95}\PY{p}{,} \PY{n}{scatter\PYZus{}kws}\PY{o}{=}\PY{p}{\PYZob{}}\PY{l+s+s1}{\PYZsq{}}\PY{l+s+s1}{s}\PY{l+s+s1}{\PYZsq{}}\PY{p}{:}\PY{l+m+mi}{5}\PY{p}{\PYZcb{}}\PY{p}{,} \PY{n}{line\PYZus{}kws}\PY{o}{=}\PY{p}{\PYZob{}}\PY{l+s+s2}{\PYZdq{}}\PY{l+s+s2}{color}\PY{l+s+s2}{\PYZdq{}}\PY{p}{:} \PY{l+s+s2}{\PYZdq{}}\PY{l+s+s2}{black}\PY{l+s+s2}{\PYZdq{}}\PY{p}{,} \PY{l+s+s1}{\PYZsq{}}\PY{l+s+s1}{linewidth}\PY{l+s+s1}{\PYZsq{}}\PY{p}{:}\PY{l+m+mi}{1}\PY{p}{\PYZcb{}}\PY{p}{)}
\PY{n}{sns}\PY{o}{.}\PY{n}{rugplot}\PY{p}{(}\PY{n}{x} \PY{o}{=} \PY{n}{pred\PYZus{}lm}\PY{p}{,} \PY{n}{y} \PY{o}{=} \PY{n}{y\PYZus{}lm}\PY{p}{,} \PY{n}{height}\PY{o}{=}\PY{l+m+mf}{0.025}\PY{p}{,}  \PY{n}{color}\PY{o}{=}\PY{l+s+s1}{\PYZsq{}}\PY{l+s+s1}{k}\PY{l+s+s1}{\PYZsq{}}\PY{p}{)}
\PY{c+c1}{\PYZsh{} title \PYZam{} label axes}
\PY{n}{plt}\PY{o}{.}\PY{n}{title}\PY{p}{(}\PY{l+s+s1}{\PYZsq{}}\PY{l+s+s1}{Scatter Plot:}\PY{l+s+se}{\PYZbs{}n}\PY{l+s+s1}{Prediction vs. Truth}\PY{l+s+s1}{\PYZsq{}}\PY{p}{,} \PY{n}{size}\PY{o}{=}\PY{l+m+mi}{15}\PY{p}{)}
\PY{n}{plt}\PY{o}{.}\PY{n}{xlabel}\PY{p}{(}\PY{l+s+s1}{\PYZsq{}}\PY{l+s+s1}{Prediction}\PY{l+s+s1}{\PYZsq{}}\PY{p}{,} \PY{n}{size}\PY{o}{=}\PY{l+m+mi}{11}\PY{p}{)}
\PY{n}{plt}\PY{o}{.}\PY{n}{ylabel}\PY{p}{(}\PY{l+s+s1}{\PYZsq{}}\PY{l+s+s1}{Truth}\PY{l+s+s1}{\PYZsq{}}\PY{p}{,} \PY{n}{size}\PY{o}{=}\PY{l+m+mi}{11}\PY{p}{)}
\PY{n}{plt}\PY{o}{.}\PY{n}{show}\PY{p}{(}\PY{p}{)}
\end{Verbatim}
\end{tcolorbox}

    \begin{center}
    \adjustimage{max size={0.9\linewidth}{0.9\paperheight}}{output_45_0.png}
    \end{center}
    { \hspace*{\fill} \\}
    
    We see a scatterplot of prediction vs true values, where the small bars
along the axes (a so-called rugplot) indicate the number of observations
that fall into this area. As we might have suspected from the first
plot, the underlying relationship is not exactly linear (ideally, all
points and the resulting line should lie on the diagonal). With a linear
model we tend to underestimate the response.

    \hypertarget{e-assess-the-models-training-loss-in-terms-of-mae}{%
\subsubsection*{d) }\label{e-assess-the-models-training-loss-in-terms-of-mae}}

    \begin{tcolorbox}[breakable, size=fbox, boxrule=1pt, pad at break*=1mm,colback=cellbackground, colframe=cellborder]
\prompt{In}{incolor}{19}{\boxspacing}
\begin{Verbatim}[commandchars=\\\{\}]
\PY{c+c1}{\PYZsh{}import function from sklearn}
\PY{n}{MAE} \PY{o}{=} \PY{n}{mean\PYZus{}absolute\PYZus{}error}\PY{p}{(}\PY{n}{pred\PYZus{}lm}\PY{p}{,} \PY{n}{y\PYZus{}lm}\PY{p}{)}
\PY{n+nb}{print}\PY{p}{(}\PY{n}{MAE}\PY{p}{)}
\end{Verbatim}
\end{tcolorbox}

    \begin{Verbatim}[commandchars=\\\{\}]
1.9506602873468448
    \end{Verbatim}

\newpage
    \hypertarget{additional-model-assessing}{%
\subsubsection*{*) Additional model
assessing}\label{additional-model-assessing}}

    There exists no R type regression summary report in sklearn. The main
reason is that sklearn is used for predictive modelling / machine
learning and the evaluation criteria are based on performance on
previously unseen data (such as predictive r\^{}2 for regression).

    For the statistical view on Linear Regression you can use the package

    \begin{tcolorbox}[breakable, size=fbox, boxrule=1pt, pad at break*=1mm,colback=cellbackground, colframe=cellborder]
\prompt{In}{incolor}{20}{\boxspacing}
\begin{Verbatim}[commandchars=\\\{\}]
\PY{k+kn}{import} \PY{n+nn}{statsmodels}\PY{n+nn}{.}\PY{n+nn}{formula}\PY{n+nn}{.}\PY{n+nn}{api} \PY{k}{as} \PY{n+nn}{smf}
\end{Verbatim}
\end{tcolorbox}

The function OLS performs Ordinary least square fit (Linear regression) and
has a summary() function.\\
Nevertheless, self-defined functions can be used for assessing models from sklearn:

    \begin{tcolorbox}[breakable, size=fbox, boxrule=1pt, pad at break*=1mm,colback=cellbackground, colframe=cellborder]
\prompt{In}{incolor}{21}{\boxspacing}
\begin{Verbatim}[commandchars=\\\{\}]
\PY{k}{def} \PY{n+nf}{regression\PYZus{}results}\PY{p}{(}\PY{n}{y\PYZus{}true}\PY{p}{,} \PY{n}{y\PYZus{}pred}\PY{p}{)}\PY{p}{:}  
    \PY{l+s+sd}{\PYZdq{}\PYZdq{}\PYZdq{} }
\PY{l+s+sd}{    Method to produce model metrics for training data}
\PY{l+s+sd}{    Input: training response vector as array, prediction vector as array}
\PY{l+s+sd}{    Output: \PYZhy{}}
\PY{l+s+sd}{    \PYZdq{}\PYZdq{}\PYZdq{}}
    \PY{c+c1}{\PYZsh{} Regression metrics}
    \PY{n}{explained\PYZus{}variance}\PY{o}{=}\PY{n}{metrics}\PY{o}{.}\PY{n}{explained\PYZus{}variance\PYZus{}score}\PY{p}{(}\PY{n}{y\PYZus{}true}\PY{p}{,} \PY{n}{y\PYZus{}pred}\PY{p}{)}
    \PY{n}{mean\PYZus{}absolute\PYZus{}error}\PY{o}{=}\PY{n}{metrics}\PY{o}{.}\PY{n}{mean\PYZus{}absolute\PYZus{}error}\PY{p}{(}\PY{n}{y\PYZus{}true}\PY{p}{,} \PY{n}{y\PYZus{}pred}\PY{p}{)} 
    \PY{n}{mse}\PY{o}{=}\PY{n}{metrics}\PY{o}{.}\PY{n}{mean\PYZus{}squared\PYZus{}error}\PY{p}{(}\PY{n}{y\PYZus{}true}\PY{p}{,} \PY{n}{y\PYZus{}pred}\PY{p}{)} 
    \PY{n}{mean\PYZus{}squared\PYZus{}log\PYZus{}error}\PY{o}{=}\PY{n}{metrics}\PY{o}{.}\PY{n}{mean\PYZus{}squared\PYZus{}log\PYZus{}error}\PY{p}{(}\PY{n}{y\PYZus{}true}\PY{p}{,} \PY{n}{y\PYZus{}pred}\PY{p}{)}
    \PY{n}{median\PYZus{}absolute\PYZus{}error}\PY{o}{=}\PY{n}{metrics}\PY{o}{.}\PY{n}{median\PYZus{}absolute\PYZus{}error}\PY{p}{(}\PY{n}{y\PYZus{}true}\PY{p}{,} \PY{n}{y\PYZus{}pred}\PY{p}{)}
    \PY{n}{r2}\PY{o}{=}\PY{n}{metrics}\PY{o}{.}\PY{n}{r2\PYZus{}score}\PY{p}{(}\PY{n}{y\PYZus{}true}\PY{p}{,} \PY{n}{y\PYZus{}pred}\PY{p}{)}

    \PY{n+nb}{print}\PY{p}{(}\PY{l+s+s1}{\PYZsq{}}\PY{l+s+s1}{explained\PYZus{}variance: }\PY{l+s+s1}{\PYZsq{}}\PY{p}{,} \PY{n+nb}{round}\PY{p}{(}\PY{n}{explained\PYZus{}variance}\PY{p}{,}\PY{l+m+mi}{4}\PY{p}{)}\PY{p}{)}    
    \PY{n+nb}{print}\PY{p}{(}\PY{l+s+s1}{\PYZsq{}}\PY{l+s+s1}{mean\PYZus{}squared\PYZus{}log\PYZus{}error: }\PY{l+s+s1}{\PYZsq{}}\PY{p}{,} \PY{n+nb}{round}\PY{p}{(}\PY{n}{mean\PYZus{}squared\PYZus{}log\PYZus{}error}\PY{p}{,}\PY{l+m+mi}{4}\PY{p}{)}\PY{p}{)}
    \PY{n+nb}{print}\PY{p}{(}\PY{l+s+s1}{\PYZsq{}}\PY{l+s+s1}{r2: }\PY{l+s+s1}{\PYZsq{}}\PY{p}{,} \PY{n+nb}{round}\PY{p}{(}\PY{n}{r2}\PY{p}{,}\PY{l+m+mi}{4}\PY{p}{)}\PY{p}{)}
    \PY{n+nb}{print}\PY{p}{(}\PY{l+s+s1}{\PYZsq{}}\PY{l+s+s1}{MAE: }\PY{l+s+s1}{\PYZsq{}}\PY{p}{,} \PY{n+nb}{round}\PY{p}{(}\PY{n}{mean\PYZus{}absolute\PYZus{}error}\PY{p}{,}\PY{l+m+mi}{4}\PY{p}{)}\PY{p}{)}
    \PY{n+nb}{print}\PY{p}{(}\PY{l+s+s1}{\PYZsq{}}\PY{l+s+s1}{MSE: }\PY{l+s+s1}{\PYZsq{}}\PY{p}{,} \PY{n+nb}{round}\PY{p}{(}\PY{n}{mse}\PY{p}{,}\PY{l+m+mi}{4}\PY{p}{)}\PY{p}{)}
    \PY{n+nb}{print}\PY{p}{(}\PY{l+s+s1}{\PYZsq{}}\PY{l+s+s1}{RMSE: }\PY{l+s+s1}{\PYZsq{}}\PY{p}{,} \PY{n+nb}{round}\PY{p}{(}\PY{n}{np}\PY{o}{.}\PY{n}{sqrt}\PY{p}{(}\PY{n}{mse}\PY{p}{)}\PY{p}{,}\PY{l+m+mi}{4}\PY{p}{)}\PY{p}{)}
    \PY{n+nb}{print}\PY{p}{(}\PY{l+s+s1}{\PYZsq{}}\PY{l+s+s1}{Median Absolut Error: }\PY{l+s+s1}{\PYZsq{}}\PY{p}{,} \PY{n+nb}{round}\PY{p}{(}\PY{n}{median\PYZus{}absolute\PYZus{}error}\PY{p}{,}\PY{l+m+mi}{4}\PY{p}{)}\PY{p}{)}
\end{Verbatim}
\end{tcolorbox}

    \begin{tcolorbox}[breakable, size=fbox, boxrule=1pt, pad at break*=1mm,colback=cellbackground, colframe=cellborder]
\prompt{In}{incolor}{22}{\boxspacing}
\begin{Verbatim}[commandchars=\\\{\}]
\PY{n}{regression\PYZus{}results}\PY{p}{(}\PY{n}{y\PYZus{}lm}\PY{p}{,} \PY{n}{pred\PYZus{}lm}\PY{p}{)}
\end{Verbatim}
\end{tcolorbox}

    \begin{Verbatim}[commandchars=\\\{\}]
explained\_variance:  0.3144
mean\_squared\_log\_error:  0.0471
r2:  0.3144
MAE:  1.9507
MSE:  7.1255
RMSE:  2.6694
Median Absolut Error:  1.5254
    \end{Verbatim}


    % Add a bibliography block to the postdoc
    
    
    
\end{document}
